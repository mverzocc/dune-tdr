\section{Safety}
\label{sec:dp-pds-safety}

The \dword{dp} \dword{pd} consortium will observe and comply with the institutional and national safety regulations of all of its production and assembly sites. The relevant safety documents for these sites will be reviewed by the consortium, and the regulations will be implemented by those responsible for local operations.

The production/assembly site is where \dword{hv} cables, \dword{hv} splitter boxes, calibration fibers, and the \dword{pmt} mechanics are prepared for final installation or assembly. The \dwords{pmt} are received, and the bases and mounting structures are connected to the \dwords{pmt}. The outputs of the production/assembly procedure are the \dwords{pmt} in their final mechanical and electronic structure. The assembly procedure requires several steps followed by specific \dword{qc} tests. The main safety risks during production are excessive heat from electrical and mechanical processing, chemicals for cleaning components, electrocution, and to some extent, heavy lifting and tripping hazards. Each production site host institution will develop dedicated handling, cleaning, and equipment procedures. The main safety risks during assembly are mechanical hazards such as sharp edges, heavy tools, and small parts. Procedures for using safety equipment, e.g.,  safety goggles, gloves, and safety shoes, will be developed by representatives of the production/assembly site institution. Delicate material handling and transportation instructions will also be developed by the institution. These instructions will be incorporated into the assembled structure for any downstream operation site.

The testing locations at the \dword{ctsf} and underground cleanroom  are responsible for the \dword{qc} of the assembled \dword{pmt} structure. These stations are also responsible for testing the \dword{hv} cables, \dword{hv} splitters, and the calibration fibers. The previously developed handling instructions will be respected during the testing procedures. Possible safety concerns are electrocution and heavy lifting. The functional tests of the \dwords{pmt} involve powering up the \dwords{pmt} for a predefined period. The testing procedures and the relevant safety regulations will be developed for and incorporated into all the future \dword{pmt} tests.

The production and assembly sites and the \dword{ctsf} also serve as transportation sites. The \dwords{pmt} are packed for transportation to the \dword{ctsf} and \dword{surf}. The contents of the transportation boxes are the individual cartons for the \dwords{pmt} with their bases, mounting assemblies, and short cables, placed into larger plastic pallet boxes in \num{4} $\times$ \num{3} $\times$ \num{3} arrays. % for transportation from the production/assembly sites to the \dword{ctsf}. 
Three stacked custom-design structures that can hold an array of 4$\times$3
\dwords{pmt}  will serve to transport \dwords{pmt}  from the \dwords{ctsf}  to \dwords{surf} in the same transportation boxes. 
At this stage, the main safety concern is the heavy lifting combined with the delicate detector handling procedures. 
Personnel to move the \num{36}-\dword{pmt} plastic pallet boxes both indoors and outside for truck loading will have special training. The loading and unloading procedures developed for the \dword{pmt} transportation boxes must be followed at the production and assembly sites, \dword{ctsf}, and at the ground and underground stations of the experiment site. Similar procedures will be developed for transporting the \dword{hv} cables and splitters, and calibration fibers.

The \dword{tpb} coating of the \dword{pmt} windows is done at the \dword{ctsf}, as described in Section~\ref{subsec:dp-pds-itf}. %The \dwords{pmt} are placed on shelves before they are removed from the transportation boxes, the windows will be cleaned, and \dword{tpb} evaporation will be done before the \dwords{pmt} are stored on shelves after the coating. The \dwords{pmt} will then go through functional testing and placed in their transportation assembly to be taken to the underground hall. 
The main safety risks at the \dword{ctsf} are electrocution, exposure to excessive heat and chemicals, and heavy lifting. Workplace safety regulations will be developed for the \dword{ctsf}. These will include general electrical and mechanical safety rules, as well as special handling instructions for the \dword{tpb} and other chemicals. Intense and continued exposure to \dword{tpb} and chemicals can cause temporary incapacitation e.g., eye, skin or respiratory irritation. We will implement carefully all the necessary safety measures. % from \dword{ppe}, to working schedule. % will be taken and implemented. 
Transporting the \dword{pmt} boxes in the \dword{ctsf} must follow the general safety regulations, and the gantry crane operator and others in the area must have the necessary training. The \dword{ctsf} will have proper interlocks, administrative and personnel control procedures, and monitoring at all times of its operation.

The underground operation and installation safety rules will follow general facility rules. The common risks at the underground cleanrooms, cryostat roof, and inside the cryostat are working in confined spaces and oxygen deficiency hazard. The \dword{dp} \dword{pds}-specific risks include electrocution for pre-installation testing, heavy lifting, and tripping.

Safety is the highest priority at all stages of  \dword{dp} \dword{pds} operations. The safety procedures are developed by the particular institutions involved in specific operations and are discussed and approved by the consortium. The guidelines and procedures for handling and transportation of the  \dword{dp} \dword{pds} materials will be made part of the \dword{ctsf} and underground facility safety regulations.
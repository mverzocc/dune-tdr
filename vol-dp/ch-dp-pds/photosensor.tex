\section{Photosensor System}
\label{sec:dp-pds-photosensors}

The baseline photodetector for the light readout is the Hamamatsu R5912-MOD20 \dword{pmt}. This is the same model used in \dword{pddp}. The Hamamatsu R5912-MOD20, depicted in  Figure~\ref{fig:dppd_2_1}, is a \SI{20}{cm} (8 inch) diameter, 14-stage, high gain \dword{pmt} (nominal gain of \num{e9}). The maximum quantum efficiency of the R5912-MOD20 \dword{pmt} is approximately \SI{20}{\%} at \SI{400}{\nano\m}. In addition, this \dword{pmt} was designed to work at cryogenic temperatures via the addition of a thin platinum layer between the photocathode and the borosilicate glass envelope to preserve the conductance of the photocathode at low temperatures. This particular \dword{pmt} has proved reliable in other cryogenic detectors. The same or similar \dwords{pmt} have successfully operated in other \lar experiments like \dword{microboone}~\cite{microboone}, MiniCLEAN \cite{miniclean}, ArDM, \dword{icarus} T600 \cite{icarus}, and \dword{pddp}~\cite{protoDUNDP-tdr}. Discussions with other manufacturers like Electron Tubes Limited (UK) \cite{electrontubeslim} and HZC (China) \cite{hzc} are on-going to include them in the program.

\begin{dunefigure}[Hamamatsu R5912-MOD20 \dshort{pmt}]{fig:dppd_2_1}
{Picture of the Hamamatsu R5912-MOD20 \dword{pmt} \cite{hamamatsu-5912}.}
\includegraphics[width=0.3\textwidth]{dppd_2_1}
\end{dunefigure}

The baseline number of \dwords{pmt} is \dpnumpmtch with \num{80} spares.  Several operations and tests must be performed with the \dwords{pmt} before they are installed. The \dwords{pmt} must be ordered with sufficient lead time to complete the following planned operations: assembling the voltage divider circuit, mounting on the support structure, testing at room and cryogenic temperatures, packing and shipping to the \dword{ctsf}. The \dword{tpb} coating of the \dword{pmt} windows will be done at the \dword{ctsf}. The \dwords{pmt} will be re-tested for validation of basic functionality at the \dword{ctsf} and at \surf before installation (see Section~\ref{sec:dp-pds-installation}). Considering the large number of \dwords{pmt} required by the \dual \dword{pds}, the purchase order must be completed at least two years before installation. A staged or staggered order with a steady supply of \dwords{pmt} would be most convenient and will be negotiated with the manufacturer.

%%%%%%%%%%%%%%%%%%%%%%%%%%%%%%%%%
\subsection{Photodetector Characterization}
\label{sec:dp-pds-selection-characterization}

Before installation, the most important characteristics of the \dword{pmt} response must be determined with two goals: to possibly reject under-performing \dwords{pmt} and to store the characterization information in a database for later use during the \dword{dpmod} commissioning and operation.

The basic and most important parameters to characterize are the dark count rate versus \dword{hv} and the gain versus \dword{hv}. Both parameters must be measured at room and cryogenic temperatures. As with the baseline \dword{pmt} model, the rates of pre-pulsing and after-pulsing should be negligible, but they will be measured as part of testing. 

From the mechanical point of view, the test set up requires a light-tight dark vessel filled with cryogenic liquid (argon or nitrogen) and an infrastructure for filling and operating the vessel with temperature and liquid-level controls. For \dword{pddp}, \num{10} \dwords{pmt} were tested at a time over a week (cryogenic tests of \dwords{pmt} require several days for \dword{pmt} thermalization~\cite{Belver:2018erf}). Figure~\ref{fig:dppd_2_2a} shows the  \dword{pddp} \dwords{pmt} being installed in the testing vessel.
Increasing the capacity of the vessel, and thus the number of \dwords{pmt} that can be tested simultaneously,
could reduce the duration of the characterization test per \dword{pmt}.

\begin{dunefigure}[\dshort{pmt} being installed in the testing vessel]{fig:dppd_2_2a}
{Picture of the \dwords{pmt} being installed in the testing vessel used for the \dword{pddp} \dwords{pmt}.}
\includegraphics[width=0.3\textwidth]{dppd_2_2a}
\end{dunefigure}

Figure~\ref{fig:dppd_2_2b} shows a sketch of the proposed setup for \dword{pmt} characterization tests. For the electronics, the test set up requires an \dword{hv} power supply, a discriminator, a counter for the dark rate measurements, a pulsed light source, and a charge-to-digital or analog-to-digital converter for the \dword{pmt} gain versus voltage measurements. All instruments must allow computer control to automate data acquisition.

\begin{dunefigure}[Setup for \dshort{pmt} characterization tests]{fig:dppd_2_2b}
{Sketch of the setup for \dword{pmt} characterization tests.}
\includegraphics[width=0.7\textwidth]{dppd_2_2b}
\end{dunefigure}


%%%%%%%%%%%%%%%%%%%%%%%%%%%%%%%%%
\subsection{High Voltage System}
\label{sec:dp-pds-HV}

Based on the experience with the \dword{wa105}, we selected the A7030 power supply modules from CAEN~\footnote{CAEN\texttrademark{}, \url{http://www.caen.it/csite/CaenFlyer.jsp?parent=222}}  as the baseline power supply for the \dword{pmt} \dword{hv} system. 
These modules provide up to \SI{3}{kV} with a maximal output current of \SI{1}{mA} and a common floating ground to minimize noise. Module versions with \num{12}, \num{24}, \num{36}, or \num{48} \dword{hv} channels are available. The \dword{hv} polarity can be chosen for each module. Using the baseline \dword{pmt} powering scheme, modules with positive \dword{hv} polarity will be acquired for the experiment. Modules with \num{36} \dword{hv} channels and Radiall \num{52}\footnote{Radiall\texttrademark{}, \url{https://www.radiall.com/}.}  connectors are under consideration. The corresponding \dword{hv} cable connects the modules with the \dword{hv} splitters, described in Section~\ref{sec:fddp-pd-4.2}. For \dpnumpmtch \dwords{pmt}, \num{20} A7030 modules (plus \num{2} spares) are needed. These \num{20} \dword{hv} modules will be installed in mainframes from CAEN.

Each \dword{pmt} is powered individually.  This allows the gain of all \dwords{pmt} to be set individually by adjusting their operating high voltage.
This is controlled by software. The software interfaces to the \dword{pmt} calibration system and its database to extract the gain curves needed to set and/or equalize gains.

%%%%%%%%%%%%%%%%%%%%%%%%%%%%%%%%%
\subsection{Wavelength Shifting}
\label{sec:dppd-wls}

The \dual \dword{pds} requires wavelength-shifting of the \SI{127}{nm} scintillation photons toward visible wavelengths that can overlap with the photocathode luminous sensitivity. Coating the \dword{pmt} glass bulbs over the photocathode area with a thin film of \dword{tpb} has already been validated~\cite{Francini:2013lua} and adopted as the baseline plan. 

\dword{tpb} is a wavelength shifter that has high efficiency for converting \dword{lar} scintillation \dword{vuv} light toward the %light 
wavelength for which the \dword{pmt} photocathode is more sensitive. 

A thin layer of \dword{tpb} is deposited on the \dword{pmt} glass by means of a thermal evaporator that consists of a vacuum chamber with two copper crucibles (Knudsen cells) placed at the bottom of the chamber, see Figure~\ref{fig:dppd_11_4} in Section~\ref{sec:dp-pds-installation}. A \dword{pmt} is mounted on a rotating support that ensures a uniform coating layer and placed at the top of the evaporator with the  \dword{pmt} window pointing downward. The crucibles, filled with the \dword{tpb}, are heated to \SI{220}{\degreeCelsius}. At this temperature, the \dword{tpb} evaporates through a split in the crucible lid into the vacuum chamber, eventually reaching the \dword{pmt} window.

Several tests were performed to tune the evaporator's parameters, e.g., the coating thickness (\dword{tpb} surface density) and the deposition rate. A \dword{pmt} mock-up covered with mylar foils was used for these tests. A \dword{tpb} surface density of \SI{0.2}{mg/cm^2}, the value for which the \dword{pmt} efficiency is stable as a function of the surface density, was chosen for \dword{pddp}. Efficiency measurements were performed by using a \dword{vuv} monochromator and by comparing the cathode current of a coated \dword{pmt} with the current value of a calibrated photodiode. From these efficiency tests, we concluded that approximately \SI{0.8}{g} of \dword{tpb} must be placed in the crucible for each evaporation to achieve the desired \dword{pmt} coating surface density. %The best deposition rate was fixed to about 6.5\,\AA/s. 
This value optimizes the quantity of \dword{tpb} used per evaporation while keeping the coating surface density fluctuations below \num{5}$\%$.  
Two to four \dwords{pmt}  with these specifications can be coated per day at a single coating station. 
Several coating stations will be required to keep the installation and testing schedule (see Section~\ref{sec:dp-pds-installation} for details). An average quantum efficiency of \SI{12}{\%} at \SI{127}{\nano\m} has been measured for \dword{tpb}-coated \dwords{pmt} \cite{Bonesini:2018ubd}.

In order to enhance the light collection and to improve the \dword{pd} response uniformity throughout the entire \dword{tpc} active volume, \dword{tpb} coated reflector/\dword{wls} panels will be installed on the \dword{fc} inner surfaces. The impact on the light yield and physics measurements was evaluated for two particular cases: the full coverage of the \dword{fc} inner walls with the panels and the coverage of only the upper half of the \dword{fc}. The conclusion from the simulation studies is that half coverage is sufficiently performant as shown in Figure~\ref{fig:dppd_fd_light_yield_comparisons}. In order to reduce the cost but still keep the \dword{pds} response uniformity, the baseline design calls for equipping only the top half of the \dword{fc} walls. The \dword{tpb} coating on the \dword{pmt} windows has \num{90} \% transmittance at \SI{425}{nm} (peak of the emission spectrum of \dword{tpb}) \cite{Francini:2013lua}, therefore the baseline design of the reflector/\dword{wls} panels is optimal. 

The reflector/\dword{wls} panel is constructed with \SI{1}{\mm} thick \SI{93}{\cm} $\times$ \SI{93}{\cm} G10/\frfour plates. The central \SI{91}{\cm} (H) $\times$ \SI{93}{\cm} (W) area only on one side of the panel is laminated with a reflective foil, which is then evaporated with \dword{tpb}. The top and bottom \SI{1}{\cm} portion of the panel without coated foils are  sandwiched between horizontal support bars. Sketches of the panel are shown in Figure~\ref{fig:dppd_reflective_panel}. The support structure of the panels is described in Section~\ref{sec:dp-pds-mechanics}. 
\begin{dunefigure}[Front (left) and side (right) views of the reflective foil/\dshort{wls} panel.]{fig:dppd_reflective_panel}
{Sketches of the front (left) and side (right) views of the reflector/\dword{wls} panel (not to scale).}
\includegraphics[width=0.7\textwidth]{dppd_reflective_panel}
\end{dunefigure}

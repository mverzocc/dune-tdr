\section{Project Management}
\label{sec:dp-pds-management}

The  \dword{dp} \dword{pds} consortium was formed in 2017;  it comprises eleven institutions from France, Peru, Spain, UK, and USA, and is charged %. The charge of the \dword{dp} \dword{pds} consortium is to 
with planning and executing the construction, installation, and commissioning of the  \dword{dp} \dword{pds}.
% I think we should check with Peru and UK if they are still interested in being included in the consortium

%%%%%%%%%%%%%%%%%%%%%%%%%%%%%%%%%
\subsection{Consortium Organization}

The current \dword{dp} \dword{pds} consortium leader is %In\'{e}s Gil-Botella
 from CIEMAT (Spain) and the technical lead is %Dominique Duchesneau 
 from LAPP (France). They are members of the \dword{dune} technical board and represent the consortium in the overall \dword{dune} collaboration. The consortium leader is responsible for the subsystem deliverables and for effectively managing the consortium. The technical lead acts as the overall project manager and is the interface to the international project office; this person is responsible for monitoring and reporting on progress according to the agreed-upon schedule, and for interface documentation.

The institutions participating in the consortium are responsible for the design and/or construction of a particular subsystem. The national groups within the consortium plan to approach relevant funding agencies with a specific construction-phase proposal to obtain a likely funding line in 2019. The \dword{dp} \dword{pds} consortium %is open to any 
welcomes new institutions willing to join the current effort.

The current participating institutions %in the \dual \dword{pd} consortium 
are listed in Table \ref{tab:dppd_instit}.

\begin{dunetable}
[Participating institutes]
{ll}
{tab:dppd_instit}
{Institutions participating in the \dual \dword{pds} consortium.}   
Institution & Country \\ \toprowrule
LAPP & France \\ \colhline 
PUCP & Peru \\ \colhline
IFAE & Spain \\ \colhline
CIEMAT & Spain \\ \colhline
IFIC & Spain \\ \colhline%
UCL & United KIngdom \\ \colhline
Argonne National Laboratory & USA \\ \colhline
Duke University & USA \\ \colhline
University of Iowa & USA \\ \colhline
SDSMT & USA \\ \colhline
University of Texas at Austin & USA \\
\end{dunetable}

The % \dual \dword{pd} 
consortium is divided into four working groups: photosensors and electronics, calibration system, mechanics and integration, and simulation and physics. The corresponding current working group convener institutions are
\begin{itemize}
% AH REMOVING NAMES
\item WG1: Photosensors and Electronics -  CIEMAT, %A. Verdugo
\item WG2: Calibration System -  CIEMAT, %C. Cuesta
\item WG3: Mechanics and Integration - University of Iowa%, and %B. Bilki 
\item WG4: Simulation and Physics - %K. Scholberg (
Duke University, %M. Sorel (
IFIC, %L. Zambelli (
LAPP.
\end{itemize}

\subsection{Planning Assumptions}
\label{sec:fddp-pd-12.2}

The baseline design of the \dword{dp} \dword{pds} is a complex optimization based on the results from and experience with the first tonne \fixme{metric ton or kiloton?} scale \dword{dp} \dword{lartpc} %demonstrator 
(\dword{wa105}), critical evaluation of the design and construction of this demonstrator and \dword{pddp}, physics objectives of the \dword{dune} experiment, and the detailed simulation studies of the \dword{dpmod} as well as \dword{wa105} and \dword{pddp}.

The previous design and construction of the \dword{dp} modules have enabled us to critically evaluate several scenarios and develop the \dword{dpmod} \dword{pds} design optimized for maximum physics performance; simple construction, transportation, handling, and installation; and easy and robust operation.

Simulations are effectively used in designing and optimizing this \dword{pds}  to meet the physics requirements:
\begin{itemize}
\item light collection efficiency,
\item number of channels,
\item photosensor requirements,
\item dynamic range of readout electronics and timing resolution, and 
\item trigger strategy on non-beam events.
\end{itemize}

Although no major design modifications are foreseen, the \dword{pddp} operations will be closely monitored to fine-tune the \dword{dpmod} design. The baseline design will be validated with \dword{pddp} data in March \num{2020}. The final components of the system will be installed, tested and validated during the second \dword{pddp} run at \dword{cern} in March \num{2023}. Internal design reviews will be conducted after accomplishing these milestones.  

%%%%%%%%%%%%%%%%%%%%%%%%%%%%%%%%%
%\subsection{\dword{wbs} and Institutional Responsibilities}
\subsection{Work Breakdown Structure and Institutional Responsibilities}

The \dword{dp} \dword{pds} consortium has developed a detailed breakdown of deliverables and responsibilities \citedocdb{5606} included in the overall \dword{dune} collaboration \dword{wbs} \citedocdb{5594}%\cite{bib:docdb5594} 
coordinated by the international project office. The main deliverables are %based on the \dword{pddp} \dword{pds}  and are 
divided into seven topics: 
%These are listed along with the participating institutions below: 

%\dword{pddp} \dword{pds} 
\begin{enumerate}
\item management \dword{dp} \dword{pds} (includes milestones and review dates);
\item physics and simulations;
\item design, engineering, R\&D, and validation tests;
\item production set up (includes tooling);
\item production (includes component production, assembly, testing, and \dword{qc});
\item integration (contributions to activities at a global integration facility); and
\item installation (contributions to activities at \dword{surf}).

%\item Management \dual \dword{pds} (includes milestones and review dates) \textit{- LAPP, CIEMAT }
%\item Physics and Simulations \textit{- Duke, LAPP, IFIC, SDSMT, CIEMAT, PUCP, UCL, Texas-Austin}
%\item Design, Engineering, R\&D and validation tests \textit{- Iowa, CIEMAT, IFIC, UCL, Texas-Austin, IFAE, SDSMT}
%\item Production Setup (includes tooling) \textit{- UCL}
%\item Production (includes component production, assembly, testing, and \dword{qc}) \textit{- Iowa, CIEMAT, IFAE, IFIC, UCL, Texas-Austin, Duke, SDSMT, LAPP}
%\item Integration (contributions to activities at global integration facility) \textit{- SDSMT}
%\item Installation (contributions to activities at \surf) \textit{- CIEMAT, IFIC, SDSMT, Iowa}
\end{enumerate}

\subsection{High Level Schedule}


The main high-level milestones of the \dword{dp} \dword{pds}  and the dates by which they will be accomplished are summarized in Table~\ref{tab:Xsched}.

\begin{dunetable}
[\dual \dshort{pd} consortium schedule]
{p{0.65\textwidth}p{0.25\textwidth}}
{tab:Xsched}
{\dual \dword{pd} Consortium Schedule}   
Milestone & Date (Month YYYY)   \\ \toprowrule
Initial design validation with \dword{pddp} data & March 2020 \\ \colhline
\rowcolor{dunepeach} Start of \dword{pdsp}-II installation& \startpduneiispinstall      \\ \colhline
\rowcolor{dunepeach} Start of \dword{pddp}-II installation& \startpduneiidpinstall      \\ \colhline
\rowcolor{dunepeach}South Dakota Logistics Warehouse available& \sdlwavailable      \\ \colhline
\rowcolor{dunepeach}Beneficial occupancy of cavern 1 and \dword{cuc}& \cucbenocc      \\ \colhline
Final design validation with \dword{pddp} II data & March 2023 \\ \colhline
\rowcolor{dunepeach} \dword{cuc} counting room accessible& \accesscuccountrm      \\ \colhline

\rowcolor{dunepeach}Top of \dword{detmodule} \#1 cryostat accessible& \accesstopfirstcryo      \\ \colhline

\dword{pmt} procurement procedure and production & June 2024 \\ \colhline
\dword{pmt} base design and manufacturing &  June 2024 \\ \colhline
\dword{pmt} support structure production and assembly & July 2024 \\ \colhline

\rowcolor{dunepeach}Start of \dword{detmodule} \#1 TPC installation& \startfirsttpcinstall      \\ \colhline


\rowcolor{dunepeach}Top of \dword{detmodule} \#2 accessible& \accesstopsecondcryo      \\ \colhline

\dword{pmt} characterization & April 2025 \\ \colhline
Fibers, light source tests and procurement & April 2025 \\ \colhline
\rowcolor{dunepeach}End of \dword{detmodule} \#1 TPC installation& \firsttpcinstallend      \\ \colhline
Splitter production and tests & June 2025 \\ \colhline
\dword{tpb} coating of the \dwords{pmt} & July 2025 \\ \colhline
\dword{tpb} coating of the reflector/\dword{wls} panels & July 2025 \\ \colhline

 \rowcolor{dunepeach}Start of \dword{detmodule} \#2 TPC installation& \startsecondtpcinstall      \\ \colhline
 
\dword{pmt} cable and fiber routing in cryostat from flange to bottom & October 2025 \\ \colhline
\dword{pmt} testing, installation in cryostat and cabling & April 2026 \\ \colhline
\dword{pmt} support installation on the membrane & April 2026 \\ \colhline
Splitter installation & April 2026 \\ \colhline
Fiber calibration system installation & April 2026 \\ 
 
\rowcolor{dunepeach}End of \dword{detmodule} \#2 TPC installation& \secondtpcinstallend      \\ 
\end{dunetable}

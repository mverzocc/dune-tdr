\section{Calibration System}
\label{sec:dp-pds-calibration}

A photon calibration system is required in the \dword{dpmod} to calibrate the \dwords{pmt} installed in the \dword{lar} volume. The goal is to determine the \dword{pmt} gain and monitor the stability of the \dword{pmt} response. One of the main goals of the \dword{pds} is to provide a trigger for non-beam physics. The trigger is based on the amplitude of \dword{pmt} signals. The amplitudes of the \dword{pmt} signals are summed for groups of certain \dwords{pmt} and/or for all \dwords{pmt} and then, these input signals are discriminated according to the trigger logic. An equalized \dword{pmt} response allows using the same threshold definition for all \dword{pmt} groups, simplifying the determination of trigger efficiency. In addition to measuring the \dword{pmt} gain, the calibration system is also designed to monitor the stability of the \dword{pmt} response. % and its quantum efficiency.
Therefore, the calibration system must be able to illuminate all \dwords{pmt} at \dword{spe} level and at larger light intensities to ensure that the gain calibration and linearity can be determined from the data. Given the necessary \dword{pmt} characterization at cryogenic temperature before installation~\cite{Belver:2018erf}, and based on past experience with \dword{wa105} %detector 
operations~\cite{Aimard:2018yxp}, we conclude that a light calibration system is needed during experiment data taking.

%%%%%%%%%%%%%%%%%%%%%%%%%%%%%%%%%%%%%%%%%%%%%%%%%%%%%%%%%%%%%%%%%%%%

%\subsection{Calibration System Design}

An \dword{led}-driven fiber calibration system~\cite{Cuesta:2017nrs,Conrad:2015xta,Caccianiga:2003fm,ADAMSON2002325,Belver:2019lqm} is designed such that a configurable amount of light reaches each \dword{pmt}. The calibration light is provided by a blue \dword{led} of \SI{465}{\nm} using a Kapustinsky~\cite{KAPUSTINSKY1985612} circuit as \dword{led} driver and transmitted by a fiber system ending with an optical fiber installed at each \dword{pmt} (see Figure~\ref{dppd-6-LCS}). Twenty groups of six \dwords{led} placed in a hexagonal geometry and a reference sensor check the performance of the \dword{led} in the center of each group. The direct light goes to the fiber, and the stray light to the \dword{sipm} used as reference sensor. Each \dword{led} is connected to an external fiber going to one feedthrough. Then, fibers are connected inside the cryostat, and each fiber is attached to a 1-to-7 fiber bundle, so that one fiber is finally installed pointing at each \dword{pmt}. The \dwords{pmt} are oriented with the first dynode perpendicular to the Earth magnetic field and the fiber parallel to the first dynode to have a similar gain to the one obtained with diffuse light. The components placed outside the cryostat at room temperature form the ``external system,'' and the ones installed inside it at cryogenic temperature form the ``inner system.'' 

\begin{dunefigure}[\dshort{pds} light calibration system]{dppd-6-LCS}
{Sketch of the \dword{dune} \dword{fd} \dword{pds} light calibration system. The external system at room temperature is shown in blue and the inner system at cryogenic temperature in black.}
\includegraphics[width=0.7\textwidth]{dppd_LCSdiagram_DUNE}
\end{dunefigure}

%%%%%%%%%%%%%%%%%%%%%%%%%%%%%%%%%%%%%%%%%%%
\subsection{External System Design}

The design of the external components is driven by the need for a cost-effective system. An additional requirement is that a reference light sensor monitor the amount of injected light. The light is injected in the form of several \si{\ns}-long pulses provided by a Kapustinsky circuit. The setup consists of a commercial black box in which a light guide structure is mounted. There are \num{20} structures, and each has six arms and a central part, as shown in Figure~\ref{fig_source}. On each of the six arms, an electronics board containing a Kapustinsky circuit is mounted. The \dword{led}, NSPB300B from Nichia Corp.\footnote{www.nichia.com}, with a peak wavelength of \SI{465}{nm} is placed  on the \dword{pcb} in front of an optical \dword{sma}-to-\dword{sma} feedthrough. On the other side of each feedthrough, an optical fiber, FG105LCA-CUSTOM-MUC from Thorlabs\footnote{www.thorlabs.com}, is connected. The fiber transports the light to one of the \num{120} feedthroughs in the instrumentation flange on top of the cryostat. While a large fraction of the \dword{led} light is emitted forward, a small fraction, the stray light, is emitted under a large angle and reaches by reflection to the central region of the light guide structure where it is detected by a \dword{sipm}, MicroFJ-30035-TSV-TA from SensL\footnote{www.sensl.com}.


\begin{dunefigure}[Light guide structure during  development and light path]{fig_source}
{(Left) Picture of the light guide structure during the development phase. The six arms are visible, and on one of them a prototype \dword{led} driver is mounted. (Right) Schematics of the light way from the \dword{led} to the reference sensor.}
\includegraphics[width=0.25\textwidth]{dppd_LightGuide.jpg}
\includegraphics[width=0.6\textwidth]{dppd_Stray.png}
\end{dunefigure}




%%%%%%%%%%%%%%%%%%%%%%%%%%%%%%%%%%%%%%%%%%%
\subsection{Internal System Design}

The inner system is designed to minimize light losses at cryogenic temperatures. The external fibers are connected to 120 female optical feedthroughs from Allectra\footnote{www.allectra.com} installed at \num{20} flanges. Inside the cryostat, a single long fiber, FT800UMT from Thorlabs, goes down from each optical feedthrough routed along the walls of the cryostat to the bottom of the cryostat where a \num{1}-to-\num{7} fiber bundle, comprising FT200UMT fibers from Thorlabs, is connected to each long fiber. A total of \num{720} of these fibers are guided to the \dwords{pmt} at the bottom of the detector. The end of the fiber is fixed at the \dword{pmt} support structure pointing the photocathode. The fibers and bundles are \num{0.39}\,NA TECS$^\text{TM}$ hard-clad, multimode, step-index fibers with high OH (hydroxyl) content to increase the light transmission at low wavelengths. To optimize the light transmission of the fiber-bundle connection, the inner fibers have a diameter of \SI{800}{\um}, big enough to distribute uniformly the light at the bundle entrance, total diameter \SI{700}{\um}.  From the mechanical point of view, the described approach of bundles attached to fibers is safer than connecting  the bundles directly to the feedthroughs. To have good homogeneity of the light at the fiber-bundle connections, \dword{sma} connectors are chosen. Vacuum-compatible \dword{sma}-to-\dword{sma} mating sleeves are required in order to avoid \dword{lar} freezing inside the connector, which would reduce light transmission.

Three different measurements will be performed during the detector operation:
\begin{itemize}
    \item gain stability: \dword{spe} spectrum at the \dword{pmt} operating \dword{hv} will be measured to obtain the gain. This measurement will be taken every time the \dwords{pmt} are biased up, and regularly, every day at the beginning, and every few days when stable operating conditions are reached. Considering that high \dword{led} repetition rates can be used \cite{Belver:2019lqm}, sufficient \dword{spe} statistics can be acquired in less than one minute. 
    \item gain vs. \dword{hv}: \dword{spe} spectrum at different \dword{pmt} voltages (\SI{1300}{\V} - \SI{1900}{V} in \SI{200}{V} steps) to obtain the gain calibration curve. This measurement will be taken every time the operating conditions change. It can be performed occasionally when stable operating conditions are reached, also considering that a full gain versus high voltage scan can be taken in only a few minutes .  
    \item light response: The \dword{led} voltage can be increased to study the \dword{pmt} performance for different light intensities from the \dword{spe} to several tens of photo-electrons.
\end{itemize}

Alternatives to the calibration system baseline design described here are also being considered. See Appendix~\ref{sec:dp-pds-appendix-calibration}. 



%placeholder
\section{DUNE Calibration Strategy}
\label{sec:phys-calib-strat}


The DUNE \dword{fd} presents a unique challenge for calibration in many ways. It differs from existing \dword{lbl} neutrino detectors and existing \dwords{lartpc} because of its size -- the largest \dword{lartpc} ever constructed -- but also because of its deep underground location. 
The DUNE \dword{nd}, which we expect to include a \dword{lartpc}, will also differ from previous experiments (e.g., MINOS and \nova). In particular, while the ND will be highly capable, pile-up and readout will be different, and this may complicate extrapolation of all relevant detector characteristics.

As for any \lartpc, full exploitation of DUNE's capability for precision tracking and calorimetry requires a detailed understanding of the detector response. The inherently highly convolved detector response model and the strong correlations that exist between various calibration quantities make this challenging. 
For example, the determination of energy associated with an event of interest will depend on the simulation model, associated calibration parameters, non-trivial correlations between the parameters, and spatial and temporal dependence of those parameters caused by the non-static nature of the \dword{fd}. 
Changes can be abrupt (e.g., noise, a broken resistor in the \dword{fc}), or ongoing (e.g., exchange of fluid through volume, ion accumulation).

Convincing physics measurements will require a demonstration that the overall detector response is well understood. The systematic uncertainties for the \dword{lbl} and low-energy (\dword{snb}) program will determine the required precision on dedicated calibration systems.
The calibration program must provide measurements at the few-percent-or-better level stably across an enormous volume and over a long period of time, and provide sufficient redundancy.

This section describes the current calibration strategy for DUNE that uses existing sources of particles, external measurements, and dedicated external calibration hardware systems. Existing calibration sources for DUNE include beam or atmospheric neutrino-induced samples, cosmic rays, argon isotopes, and instrumentation devices such as \lar purity and temperature monitors. Dedicated calibration hardware systems currently include laser and pulsed neutron system (PNS).
%and radioactive source deployment systems. %%KM removed
The responsibility of these hardware systems and assessment of alternative calibration system designs fall under the joint \dword{sp} and \dword{dp} calibration consortium. External measurements by \dword{protodune} and SBN will validate techniques, tools and the design of systems applicable to the DUNE calibration program;  \dword{protodune} will also perform essential measurements of charged particle interactions in \dword{lar}.
%and are essential to the success of the overall calibration program.  %%KM I removed essential SBN 

Under current assumptions, the calibration strategy described in this document is applicable to both \dwords{spmod} and \dwords{dpmod}. %Sections~\ref{sec:phys-calib-req} and \ref{sec:phys-calib-approach} briefly describe the physics-driven calibration requirements, including the calibration sources and the systems required at the different stages of the experiment. 
Section~\ref{sec:phys-calib-req} briefly describes the physics-driven calibration requirements. 
The nominal \dword{dune} \dword{fd} calibration design is described in Section~\ref{sec:phys-calib-sources}. Finally,
%Section~\ref{sec:phys-calib-sum} provides a summary along with future plans for calibration.
Section \ref{sec:phys-calib-approach} describes a staging plan for calibration
%the calibration development plan 
from after the \dword{tdr} through to the operation of the experiment including design validation at \dword{protodune}.
%, to provide context for the systems required at the different stages of the experiment.

%KM: Older text 
\begin{comment}
The DUNE \dword{fd} presents a unique challenge for calibration in many ways. It differs from existing \dword{lbl} neutrino detectors and existing \dwords{lartpc} because of its size -- the largest \dword{lartpc} ever constructed -- but also because of its deep underground location. 
The DUNE \dword{nd}, which we expect to include a \dword{lartpc}, will also differ from previous experiments (e.g., MINOS and \nova). Pile-up and readout issues  may complicate extrapolation of detector characteristics.

As for any \lartpc, full exploitation of DUNE's capability for precision tracking and calorimetry requires a detailed understanding of the detector response. The inherently highly convolved detector response model and the strong correlations that exist between various calibration quantities make this challenging. 
For example, the determination of energy associated with an event of interest will depend on the simulation model, associated calibration parameters, non-trivial correlations between the parameters, and spatial and temporal dependence of those parameters caused by the non-static nature of the \dword{fd}. 
Changes can be abrupt (e.g., noise, a broken resistor in the \dword{fc}), or ongoing (e.g., exchange of fluid through volume, ion accumulation).

Convincing physics measurements will require a demonstration that the overall detector response is well understood. The systematic uncertainties for the \dword{lbl} and low-energy (\dword{snb}) program will determine the required precision on each calibration parameter. 
The calibration program must provide measurements at the few-percent-or-better level stably across an enormous volume and over a long period of time, and provide sufficient redundancy.

This section describes the current calibration strategy for DUNE that uses existing sources of particles, external measurements, and dedicated external calibration hardware systems. Existing calibration sources for DUNE include beam or atmospheric neutrino-induced samples, cosmic rays, argon isotopes, and instrumentation devices such as \lar purity and temperature monitors. Dedicated calibration hardware systems currently include laser, neutron and radioactive source deployment systems. The responsibility of these hardware systems fall under the joint \dword{sp} and \dword{dp} calibration consortium. External measurements by \dword{protodune} and SBND  will validate techniques, tools and the design of systems applicable to the DUNE calibration program and are essential to the success of the overall calibration program. 

Under current assumptions, the calibration strategy described in this document is applicable to both \dwords{spmod} and \dwords{dpmod}. Sections~\ref{sec:phys-calib-req} and \ref{sec:phys-calib-approach} briefly describe the physics-driven calibration requirements, including the calibration sources and the systems required at the different stages of the experiment. The nominal \dword{dune} \dword{fd} calibration design is described in Section~\ref{sec:phys-calib-sources}. Finally, Section~\ref{sec:phys-calib-sum} provides a summary along with future plans for calibration.
\end{comment}

%%%%%%%%%%%%%%%%%%%%%%%%%%%%%%%%%%%%%%%%%%%%%%%%%%%%%%%


\subsection{Physics-driven Calibration Requirements}
\label{sec:phys-calib-req}


To perform adequate calibrations the physics processes that lead to the formation of the signals required for DUNE's broad physics program,
expected (and unexpected) detector effects must be carefully understood, as they ultimately affect the detector's energy, position and particle identification response. 
%and detector effects that influence their propagation must be carefully understood, as they ultimately affect the detector's energy response. 
Other categories of effects, such as the neutrino interaction model or reconstruction pathologies, can impact measurements of physical quantities. These other effects are beyond the scope of the \dword{fd} calibration effort and would only lead to a higher overall error budget.

\subsubsection{\dword{lbl} physics}
\label{sec:phys-calib-lbl}
%In the physics volume of the DUNE \dword{cdr}~\cite{Acciarri:2015uup}, Figure~3.23 shows that increasing the uncertainties on the \nue event rate from \num{2}\% overall to \num{3}\% results in a \num{50}\% longer run period to achieve a 5$\sigma$ determination of \dword{cpv} for 50\% of its possible values. The \dword{cdr} also assumes that the fiducial volume is known to the 1\% level. Thus, c
Calibration information needs to provide an approximately 1-2\% understanding of normalization 
and position resolution within the detector to support \dword{dune} \dword{lbl} physics. 
%Later studies expanded the simple treatment of energy  presented above. In particular, \num{1}\% 
A bias on the lepton energy has a significant impact on the sensitivity to \dword{cpv}. 
%
A \num{3}\% bias in the hadronic state (excluding neutrons) is important, as the inelasticity  distribution for neutrinos and antineutrinos is quite different.  Different fractions of their energies go into the hadronic state. Finally, while studies largely consider a single, absolute energy scale, DUNE will need to monitor and correct relative spatial differences across the enormous DUNE \dword{fd} volume; this is also true for time-dependent changes~\cite{ebias}.

A number of in situ calibration sources will be required to address these broad range of requirements. 
Michel electrons, neutral pions and radioactive sources (both intrinsic and external) are needed for calibrating detector response to electromagnetic activity in the tens-to-hundreds of MeV energy range. Stopping protons and muons from cosmic rays or beam interactions form an important calibration source for calorimetric reconstruction and particle identification. 
\Dword{protodune}, as a dedicated test beam experiment, provides important measurements to characterize and validate particle identification strategies in a \SI{1}{kt}-scale detector and is an essential input to the overall program. Dedicated calibration systems, like lasers, will be useful to provide in situ full-volume measurements of \efield distortions. 
Measuring the strength and uniformity of the \efield is a key aspect of calibration, as  estimates of calorimetric response and \dword{pid} depend on the \efield through recombination. The stringent physics requirements on energy scale and fiducial volume also put similarly stringent requirements on detector physics quantities such as \efield, drift velocity, electron lifetime, and the time dependencies of these quantities; this is discussed in more detail in 
%the calibration chapter of the \dword{sp} volume of the \dword{tdr} \fixme{reference SP TDR; SG: chapters in different volumes cannot be cross linked, so we just need to mention the chapter and the volume of the TDR and that is it. I added this in.} and 
the dedicated laser system discussion under Section~\ref{sec:phys-calib-hardware}.

\subsubsection{\dword{snb} and low-energy neutrino physics}
%The supernova signal will include a electron, photon and neutron (capture) components. 
A combination of 6~MeV (direct neutron capture response), 9~MeV (peak visible $\gamma$-energy of interest to \dword{snb} and $^{8}B$/hep solar neutrinos), 15~MeV (upper visible energy of $^{8}B$/hep solar neutrinos) and $\sim$30~MeV (decay electrons) is needed to map the low energy response. Supernova signal events present specific reconstruction and calibration challenges, and observable energy is shared between different charge clusters and types of energy depositions. In particular, the supernova signal will have a low-energy electron, gamma and neutron capture component, and each needs to be characterized. As discussed further in Section~\ref{ch:snb-lowe}, primary requirements for this physics include 
(1) calibration of absolute energy scale and energy resolution, which is important for resolving spectral features of \dword{snb} events;
(2) calibration of time and light yield response of optical photon detectors;  
(3) absolute timing of events;  
(4) measurement of trigger efficiency at low energies;  and 
(5) understanding of detector response to radiological backgrounds. Further details on the the necessary energy scale, energy resolution and trigger efficiency targets needed can be in Ref~\cite{bib:docdb14068}.
Potential calibration sources in this energy range include Michel electrons from muon decays (successfully utilized by ICARUS and \dword{microboone}~\cite{Acciarri:2017sjy}), which have a well known spectrum up to $\sim\,$\SI{50}{\MeV}.
Photons from neutral pion decay (from atmospheric and beam induced $\pi^0$) will provide an overall energy scale between \SI{50}{\MeV} and \SI{100}{\MeV}, in addition to cosmic ray muon energy loss. 
However, the limited statistical power of those samples (see Table~\ref{tab:cosmic-ray-calib-rates}) mean that it is not possible for these samples to provide the energy scale or resolution at the spatial and temporal granularity needed.
The pulsed neutron system can provides a source of direct neutron capture across the entire DUNE volume, providing a timing and energy calibration.
%removing the laser part here since not relevant?
%the laser system is timed as well. 
The proposed radioactive source system provides an in situ source of the electrons and de-excitation $\gamma$ rays, which are directly relevant for physics signals from \dword{snb} or $^{8}B$ solar neutrinos. These two systems (discussed in more detail in Section~\ref{sec:phys-calib-hardware}) can provide calibrations of photon, electron, and neutron response for energies below \SI{10}{\MeV}, where photons and electrons may have very different characteristics in \dword{lar}.
%It is more challenging to find ``standard candles'' between \SI{50}{Mev} and \SI{100}{\MeV}, beyond cosmic ray muon energy loss.
%SG removing the below sentence since this is mentioned in at least two other places so no need to mention it here under SNB.
%\Dword{protodune} is a test bed for the planned systems and any additional systems or alternative designs being considered.
%, alternative calibration strategies. 

%One can imagine also ancillary studies of detector response using detectors such as \lariat~\cite{Cavanna:2014iqa}, \dword{microboone}~\cite{Acciarri:2016smi}, and SBND~\cite{Antonello:2015lea}.

\subsubsection{Nucleon decay and other exotic physics}
The calibration needs for nucleon decay and other exotic physics are comparable to those for the \dword{lbl} program, as listed in Section~\ref{sec:phys-calib-lbl}. Signal channels for light \dword{dm} and sterile neutrino searches will be \dword{nc} interactions that are background to the \dword{lbl} physics program. 
Based on the widths of $dE/dx$-based metrics of \dword{pid}, qualitatively, we need to calibrate $dE/dx$ across all drift and track orientations at the few-percent level, similar to the \dword{lbl} effort.


%Older text
\begin{comment}

To perform adequate calibrations the physics processes that lead to the formation of the signals required for DUNE's broad physics program, and the detector effects that influence their propagation, must be carefully understood, as they ultimately affect the detector's energy response. 
Other categories of effects, such as the neutrino interaction model or reconstruction pathologies, can impact measurements of physical quantities. These other effects are beyond the scope of the \dword{fd} calibration effort and would only lead to a higher overall error budget.


\subsubsection{Physics Case Studies}

\paragraph{\dword{lbl} physics}
\label{sec:phys-calib-lbl}
In the physics volume of the DUNE \dword{cdr}~\cite{Acciarri:2015uup}, Figure~3.23 shows that increasing the uncertainties on the \nue event rate from \num{2}\% overall to \num{3}\% results in a \num{50}\% longer run period to achieve a 5$\sigma$ determination of \dword{cpv} for 50\% of its possible values. 
The \dword{cdr} also assumes that the fiducial volume is known to the 1\% level. Thus, calibration information needs to provide an approximately 1-2\% understanding of normalization 
and position resolution within the detector. Later studies~\cite{ebias} expanded the simple treatment of energy  presented above. In particular, \num{1}\% bias on the lepton energy has a significant impact on the sensitivity to \dword{cpv}. 
%
A \num{3}\% bias in the hadronic state (excluding neutrons) is important, as the inelasticity  distribution for neutrinos and antineutrinos is quite different.  Different fractions of their energies go into the hadronic state. Finally, while studies largely consider a single, absolute energy scale, DUNE will need to monitor and correct relative spatial differences across the enormous DUNE \dword{fd} volume; this is also true for time-dependent changes. 

A number of in situ calibration sources will be required to address these broad range of requirements. 
Michel electrons, neutral pions and radioactive sources (both intrinsic and external) are needed for calibrating detector response to electromagnetic activity in the tens-to-hundreds of MeV energy range. Stopping protons and muons from cosmic rays or beam interactions form an important calibration source for calorimetric reconstruction and particle identification. 
\Dword{protodune}, as a dedicated test beam experiment, provides critical measurements to characterize and validate particle identification strategies in a \SI{1}{kt}-scale detector and is an essential input to the overall program. Dedicated calibration systems, like lasers, will be useful to provide in situ full-volume measurements of \efield distortions. 
Measuring the strength and uniformity of the \efield is a key aspect of calibration, as  estimates of calorimetric response and \dword{pid} depend on the \efield through recombination. The stringent physics requirements on energy scale and fiducial volume also put similarly stringent requirements on detector physics quantities such as \efield, drift velocity, electron lifetime, and the time dependencies of these quantities. 

\paragraph{\dword{snb} and low-energy neutrino physics}
These signal events present specific reconstruction and calibration challenges, and observable energy is shared between different charge clusters and types of energy depositions. Primary requirements for this physics include 
(1) calibration of absolute energy scale and an understanding and improvement to the nominal 20\% energy resolution, which is important for resolving spectral features of \dword{snb} events;
(2) calibration of time and light yield response of optical photon detectors;  
(3) absolute timing of events;  
(4) measurement of trigger efficiency at low energies;  and 
(5) understanding of detector response to radiological backgrounds. 
Potential calibration sources in this energy range include Michel electrons from muon decays (successfully utilized by ICARUS and \dword{microboone}~\cite{Acciarri:2017sjy}), which have a well known spectrum up to $\sim\,$\SI{50}{Mev}. Radiological sources provide calibrations of photon, electron, and neutron response for energies below \SI{10}{\MeV}. 
It is more challenging to find ``standard candles'' between \SI{50}{Mev} and \SI{100}{\MeV}, beyond cosmic ray muon energy loss. \Dword{protodune} could potentially be a test bed for various calibration strategies. One can imagine also ancillary studies of detector response using detectors such as \lariat~\cite{Cavanna:2014iqa}, \dword{microboone}~\cite{Acciarri:2016smi}, and SBND~\cite{Antonello:2015lea}.


\paragraph{Nucleon decay and other exotic physics}
The calibration needs for nucleon decay and other exotic physics are comparable to those for the \dword{lbl} program, as listed in Section~\ref{sec:phys-calib-lbl}. Signal channels for light \dword{dm} and sterile neutrino searches will be \dword{nc} interactions that are background to the \dword{lbl} physics program. 
Based on the widths of $dE/dx$-based metrics of \dword{pid}, qualitatively, we need to calibrate $dE/dx$ across all drift and track orientations at the few-percent level or better, similar to the \dword{lbl} effort.
\end{comment}


\subsection{Calibration Sources, Systems and External Measurements}
\label{sec:phys-calib-sources}
%\subsubsection{Calibration sources and systems}
Calibration sources and systems provide measurements of the detector response model parameters, or provide tests of the response model itself. Calibration measurements can also provide corrections to data, data-driven efficiencies, systematics and particle responses. 
Figure~\ref{fig:calibneeds} shows the broad range of categories of measurements that calibrations can provide, and lists 
%the critical 
important calibration parameters for DUNE's detector response model applicable to both \dword{sp} or \dword{dp}. Due to the significant interdependencies of many parameters (e.g., recombination, \efield, and \lar purity), a calibration strategy will either need to  measure parameters iteratively, or find sources that break these correlations.

Table~\ref{tab:calibsystem} provides a list of the calibration sources and dedicated calibration systems, along with their primary usage, that will comprise the current
%currently envisioned 
nominal DUNE \dword{fd} calibration design. 
The next sections provide more details on each of them. \Dword{protodune} and previous measurements provide independent tests of the response model, indicating that the choice of parameterization and values correctly reproduces real detector data. 
Not all of the ex situ measurements can be directly extrapolated to DUNE, however, due to other detector effects and conditions -- only those considered to be universal (e.g., argon ionization energy). 

Each of the many existing calibration sources comes with its own challenges. For example, while electrons from muon decay (Michel electrons) are very useful for studying the detector response to low-energy electrons (\SI{50}{\MeV}), these low-energy electrons present reconstruction challenges due to the loss of charge from radiative photons, as demonstrated in \dword{microboone}~\cite{Acciarri:2017sjy}.  
Michel electrons are therefore considered an important, independent, and necessary test of the TPC energy response model, but they will not provide a measurement of a particular response parameter.

\begin{dunefigure}[Categories of measurements provided by calibration]{fig:calibneeds}{Categories of measurements provided by calibration.}
%\includegraphics[width=.9\textwidth]{CalibNeeds_Final.png}
\includegraphics[width=.9\textwidth]{graphics/CalibNeedsFig_TDR_cropped.pdf}
\end{dunefigure}

\begin{dunetable}[Calibration systems and sources of the nominal DUNE FD calibration design]
{p{.4\textwidth}p{.55\textwidth}}{tab:calibsystem}{Primary calibration systems and sources that comprise the nominal DUNE FD calibration design along with their primary usage.}
%{ll}
% &  \\ \textbf{Usage} \toprowrule
System & \textbf{Primary Usage}  \\ \toprowrule 
& \\
\textbf{Existing Sources} & \textbf{Broad range of measurements} \\ \toprowrule
$\mu$, predominantly from cosmic ray & Position (partial), angle (partial), %velocity (timing), 
electron lifetime, wire response, $dE/dx$ calibration etc.\\ \colhline % also timing offsets listed, related to drift velocity based on our discussion?
Decay electrons, $\pi^0$ from beam, cosmic, atm $\nu$ & Test of electromagnetic response model \\ \colhline
$^{39}$Ar beta decays &  electron lifetime (x,y,z,t), diffusion, wire response \\  \colhline
& \\ 
\textbf{External Measurements} & \textbf{Tests of detector model, techniques and systems} \\ \toprowrule
ArgoNeuT~\cite{Acciarri:2013met}, ICARUS~\cite{Amoruso:2004dy, Antonello:2014eha, Cennini:1994ha}, MicroBooNE & Model parameters (e.g., recombination, diffusion) \\ \colhline 
DUNE \dword{35t}~\cite{Warburton:2017ixr} & Alignment and \textit{t0} techniques\\ \colhline 
ArgonTUBE~\cite{Ereditato:2014tya}, MicroBooNE~\cite{Acciarri:2016smi}, SBND, ICARUS~\cite{Auger:2016tjc},  \Dword{protodune}~\cite{Abi:2017aow} & Test of systems (e.g., Laser) \\ \colhline
ArgoNeuT~\cite{Acciarri:2015ncl}, MicroBooNE~\cite{bib:uBlifetime, MICROBOONE-NOTE-1018-PUB, MICROBOONE-NOTE-1028-PUB, Acciarri:2017sjy, Abratenko:2017nki, Acciarri:2013met}, ICARUS~\cite{Ankowski:2008aa,  Ankowski:2006ts,Antonello:2016niy},  \Dword{protodune} & Test of calibration techniques and detector model (e.g., electron lifetime, Michel electrons, $^{39}$Ar beta decays) \\ \colhline
\Dword{protodune}, LArIAT~\cite{Cavanna:2014iqa}, CAPTAIN~\cite{Bhandari:2019rat} & Test of particle response models and fluid flow models \\  \colhline
\dword{lartpc} test stands~\cite{Cancelo:2018dnf, Moss:2016yhb, Moss:2014ota, Li:2015rqa} & Light and LAr properties; signal processing techniques \\ \colhline 
& \\
\textbf{Monitoring Systems} & \textbf{Operation, Commissioning and Monitoring} \\ \toprowrule
Purity monitors & Electron lifetime \\ \colhline
Photon detection monitoring System & \dword{pds} response \\ \colhline
Thermometers & Temperature, velocity; test of fluid flow model \\ \colhline
Charge injection & Electronics response \\ \colhline
& \\
\textbf{Dedicated Calibration Systems} & \textbf{Targeted (near) independent, precision calibration}\\ \toprowrule
Direct ionization via laser & Position, angle, electric field (x,y,z,t) \\ \colhline
Photoelectric ejection via laser & Position, electric field (partial) \\ \colhline
Neutron injection & Test of \dword{snb} signal, neutron capture model \\ \colhline
Proposed Radioactive source deployment & Test of \dword{snb} signal model \\ \colhline
%External Muon Tracker (if deployed) & Position, angle, muon reconstruction efficiency \\ \colhline
\end{dunetable}  



%\fixme{Can't find reference Ereditato:82014tya. Anne}


%%%%%%%%%%%%%%%%%%%%%%%%%%%%%%%%%%%%%%%%%%%%%%%%%%%%%%%

%The nominal calibration design includes existing sources, external measurements, monitors, as well as dedicated calibration systems intended to provide information beyond the reach of the other systems. The dedicated systems include a laser system, a neutron-injection system and a radioactive source deployment system. 


\subsubsection{Existing sources} 
\label{sec:phys-calib-exis}
Cosmic rays and neutrino-induced interactions provide commonly used ``standard candles,'' e.g., electrons from muon decays, and photons from neutral pions, which have characteristic energy spectra. Cosmic ray muons are also used to determine detector element locations (alignment), timing offsets or drift velocity, electron lifetime, and channel-by-channel response, and to help constrain \efield distortions. 
Table~\ref{tab:cosmic-ray-calib-rates} summarizes the rates for cosmic ray events. Certain measurements (e.g., channel-to-channel gain uniformity and cathode panel alignment) are estimated to take several months of data. Table~\ref{tab:atmos_rates} gives the atmospheric $\nu$ interaction rates, which  
 are comparable to beam-induced events -- neither occurs at sufficient rates to provide meaningful spatial or temporal calibration; they will likely provide supplemental measurements only. (The beam will not yet be operational for calibration of the first \dword{detmodule} during early data taking.) Instead, we can use the reconstructed energy spectrum of ${}^{39}$Ar beta decays to make a precise measurement of electron lifetime with spatial and temporal variations. 
 This can also provide other necessary calibrations, such as measurements of wire-to-wire response variations and diffusion measurements using the signal shapes associated with the beta decays. The ${}^{39}$Ar beta decay rate in commercially-provided argon is about \SI{1}{\becquerel\per\kilo\gram}, so $O(\mathrm{50k})$ ${}^{39}$Ar beta decays are expected in a single \SI{5}{\milli\s} event readout in an entire \nominalmodsize \detmodule. 
 The ${}^{39}$Ar beta decay cutoff energy is \SI{565}{\keV},  which is close to the energy deposited on a single wire by a \dword{mip}. However, several factors can impact the observed charge spectrum from ${}^{39}$Ar beta decays, such as electronics noise, electron lifetime and recombination fluctuations; more details can be found in the Appendix~\ref{app:ar39}. MicroBooNE~\cite{MICROBOONE-NOTE-1050-PUB} and ProtoDUNE are actively pursuing this technique, thus providing valuable inputs for DUNE.

\begin{dunetable}
[Annual rates for classes of cosmic-ray events useful for calibration]
{lrl}
{tab:cosmic-ray-calib-rates}
{Annual rates for classes of cosmic-ray events described in this section assuming 100\% reconstruction efficiency.  Energy, angle, and fiducial requirements
have been applied. Rates and geometrical features apply to the single-phase far detector design. }
Sample & Annual Rate & Detector Unit \\ \colhline
Inclusive & $1.3\times 10^6$ & Per \nominalmodsize module \\ \colhline
Vertical-Gap crossing & 3300 & Per gap \\ \colhline
Horizontal-Gap crossing & 3600 & Per gap \\ \colhline
\dword{apa}-piercing & 2200 & Per \dword{apa} \\ \colhline
\dword{apa}-\dword{cpa} piercing & 1800 & Per active \dword{apa} side \\ \colhline
\dword{apa}-\dword{cpa} piercing, \dword{cpa} opposite to \dword{apa} & 360 & Per active \dword{apa} side \\ \colhline
Collection-plane wire hits & 3300 & Per wire \\ \colhline
Stopping Muons & 28600 & Per \nominalmodsize module \\ \colhline
$\pi^0$ Production & 1300 & \nominalmodsize module \\ \colhline
\end{dunetable}

\subsubsection{Monitors} 
 
Chapter~8 of \voltitlesp{} and \voltitledp{} discuss several instrumentation and detector monitoring devices in detail. These devices, including liquid argon temperature monitors, \lar purity monitors, gaseous argon analyzers, cryogenic (cold) and inspection (warm) cameras, and liquid level monitors, will provide valuable information for early calibrations and for tracking the space-time dependence of the \dwords{detmodule}. 
The \dword{cfd} simulations play a key role for calibrations initially in the design of the cryogenics recirculation system, and later for physics studies when the cryogenics instrumentation data can be used to validate the simulations. Chapters~4 and~5 of the \dword{detmodule} volumes discuss other instrumentation devices essential for calibration, such as drift \dword{hv} current monitors and external charge injection systems. 

\subsubsection{External measurements} 

DUNE will use external measurements from past experimental runs (e.g., ArgoNeuT, the DUNE \dword{35t}, ICARUS, and \lariat), from ongoing and future experiments (e.g., \dword{microboone}, \dword{protodune}, and SBND), and from small scale \dword{lartpc} test stands. External measurements provide a test bed for dedicated calibration hardware systems and techniques for the \dword{fd}. In particular, \dword{protodune} will provide validation of the fluid flow model using cryogenic instrumentation data. 
Early calibration for physics in DUNE will utilize \lar physical properties from \Dword{protodune} or SBN  for tuning detector response models in simulation. Table~\ref{tab:calibsystem} provides  references for specific external measurements. The usability of ${}^{39}$Ar has been demonstrated with \microboone data~\cite{MICROBOONE-NOTE-1050-PUB}. 
Use of  ${}^{39}$Ar  and other radiological sources and, in particular, the \dword{daq} readout challenges associated with their use, will be tested on the \dword{protodune} detectors. Dedicated systems for DUNE, 
including the laser system, have been used by previous experiments (ARGONTUBE~\cite{Zeller:2013sva,Ereditato:2014lra}, CAPTAIN, and MicroBooNE experiments) and at SBND in the future, and will provide more information on use of the system and optimization of the design.  The small-scale \lar test stand planned at Brookhaven National Lab, USA, will provide important information on simulation and calibration of field response for DUNE.

External measurements of particle response (e.g. pion interactions in \dword{lar}) are also important inputs to the detector model. These include dedicated measurements made with ProtoDUNE, LArIAT, and CAPTAIN~\cite{Bhandari:2019rat}; the DUNE ND, with both a \dword{lar} and low density gas detector, will also make measurements which characterize the relevant cross sections and outgoing final state particles.


%KM: older text
\begin{comment}
\subsubsection{Calibration sources and systems}

Calibration sources and systems provide measurements of the detector response model parameters, or provide tests of the response model itself. Calibration measurements can also provide corrections to data, data-driven efficiencies, systematics and particle responses. 
Figure~\ref{fig:calibneeds} shows the broad range of categories of measurements that calibrations can provide, and lists the critical calibration parameters for DUNE's detector response model applicable to both \dword{sp} or \dword{dp}. Due to the significant interdependencies of many parameters (e.g., recombination, \efield, and \lar purity), a calibration strategy will either need to  measure parameters iteratively, or find sources that break these correlations.

Table~\ref{tab:calibsystem} provides a list of the calibration sources and proposed calibration systems, along with their primary usage, that will comprise the currently envisioned nominal DUNE \dword{fd} calibration design. 
The next sections provide more details on each of them. \Dword{protodune} and previous measurements provide independent tests of the response model, indicating that the choice of parameterization and values correctly reproduces real detector data. 
Not all of the ex situ measurements can be directly extrapolated to DUNE, however, due to other detector effects and conditions -- only those considered to be universal (e.g., argon ionization energy). 

Each of the many existing calibration sources comes with its own challenges. For example, whereas electrons from muon decay (Michel electrons) are very useful for studying the detector response to low-energy electrons (\SI{50}{\MeV}), these low-energy electrons present reconstruction challenges due to the loss of charge from radiative photons, as demonstrated in \dword{microboone}~\cite{Acciarri:2017sjy}.  
Michel electrons are therefore considered an important, independent, and necessary test of the TPC energy response model, but they will not provide a measurement of a particular response parameter.
\begin{dunefigure}[Categories of measurements provided by calibration]{fig:calibneeds}{Categories of measurements provided by calibration.}
\includegraphics[width=.9\textwidth]{CalibNeeds_Final.png}
\end{dunefigure}

\begin{dunetable}[Calibration systems and sources of the nominal DUNE FD calibration design]
{p{.4\textwidth}p{.55\textwidth}}{tab:calibsystem}{Primary calibration systems and sources that comprise the nominal DUNE FD calibration design along with their primary usage.}
%{ll}
% &  \\ \textbf{Usage} \toprowrule
System & \textbf{Primary Usage}  \\ \toprowrule 
& \\
\textbf{Existing Sources} & \textbf{Broad range of measurements} \\ \toprowrule
$\mu$, predominantly from cosmic ray & Position (partial), angle (partial), %velocity (timing), 
electron lifetime, wire response, $dE/dx$ calibration etc.\\ \colhline % also timing offsets listed, related to drift velocity based on our discussion?
Decay electrons, $\pi^0$ from beam, cosmic, atm $\nu$ & Test of electromagnetic response model \\ \colhline
$^{39}$Ar &  electron lifetime (x,y,z,t), diffusion \\  \colhline
& \\ 
\textbf{External Measurements} & \textbf{Tests of detector model, techniques and systems} \\ \toprowrule
ArgoNeuT~\cite{Acciarri:2013met}, ICARUS~\cite{Amoruso:2004dy, Antonello:2014eha, Cennini:1994ha}, MicroBooNE & Model parameters (e.g., recombination, diffusion) \\ \colhline 
DUNE \dword{35t}~\cite{Warburton:2017ixr} & Alignment and \textit{t0} techniques\\ \colhline 
ArgonTUBE~\cite{Ereditato:2014tya}, MicroBooNE~\cite{Acciarri:2016smi}, SBND, ICARUS~\cite{Auger:2016tjc},  \Dword{protodune}~\cite{Abi:2017aow} & Test of systems (e.g. Laser) \\ \colhline
ArgoNeuT~\cite{Acciarri:2015ncl}, MicroBooNE~\cite{bib:uBlifetime, MICROBOONE-NOTE-1018-PUB, MICROBOONE-NOTE-1028-PUB, Acciarri:2017sjy, Abratenko:2017nki, Acciarri:2013met}, ICARUS~\cite{Ankowski:2008aa,  Ankowski:2006ts,Antonello:2016niy},  \Dword{protodune} & Test of calibration techniques and detector model (e.g., electron lifetime, Michel electrons, $^{39}$Ar beta decays) \\ \colhline
\Dword{protodune}, LArIAT~\cite{Cavanna:2014iqa} & Test of particle response models and fluid flow models \\  \colhline
\dword{lartpc} test stands~\cite{Cancelo:2018dnf, Moss:2016yhb, Moss:2014ota, Li:2015rqa} & Light and LAr properties; signal processing techniques \\ \colhline 
& \\
\textbf{Monitoring Systems} & \textbf{Operation, Commissioning and Monitoring} \\ \toprowrule
Purity monitors & Electron lifetime \\ \colhline
Photon detection monitoring System & \dword{pds} response \\ \colhline
Thermometers & Temperature, velocity; test of fluid flow model \\ \colhline
Charge injection & Electronics response \\ \colhline
& \\
\textbf{Dedicated Calibration Systems} & \textbf{Targeted (near) independent, precision calibration}\\ \toprowrule
Direct ionization via laser & Position, angle, electric field (x,y,z,t) \\ \colhline
Photoelectric ejection via laser & Position, electric field (partial) \\ \colhline
Neutron injection & Test of SN signal, neutron capture model \\ \colhline
Radioactive source deployment (if deployed) & Test of SN signal model \\ \colhline
%External Muon Tracker (if deployed) & Position, angle, muon reconstruction efficiency \\ \colhline
\end{dunetable}  
\end{comment}


%%%%%%%%%%%%%%%%%%%%%%%%%%%%%%%%%%%%%%%%%%%%%%%%%%%%%%%


%KM Older text and also strategy moved to the end.
\begin{comment}
\subsection{A Staged Approach}
\label{sec:phys-calib-approach}

The calibration strategy for DUNE will need to address the evolving operational and physics needs at every stage of the experiment in a timely manner using the primary sources and systems listed in Table~\ref{tab:calibsystem}. 
At the \dword{tdr} stage, a clear and complete calibration strategy with necessary studies must be provided to demonstrate how the existing sources and dedicated systems meet the physics requirements.  
This \dword{tdr} presents the baseline calibration systems and strategy. Post-\dword{tdr}, once the calibration strategy is set, the calibration consortium will need to develop the necessary designs for calibration hardware along with tools and methods to be used with various calibration sources. 
To allow for flexibility in this process, the physical interfaces for calibration such as flanges or ports on the cryostat must be designed to accommodate the calibration hardware. As described in the calibration \dword{sp} detector volume, the calibration task force has provided the necessary feedthrough penetration design updates for the \dword{spmod} and will soon finalize the design for the \dword{dpmod}. 

As DUNE physics turns on at different rates and times, a calibration strategy at each stage for physics and data taking is required. This strategy described in this section assumes that all systems are commissioned and deployed according to the nominal DUNE run plan.


\textit{Commissioning:} When a \dword{detmodule} is filled, data from various instrumentation devices validate the argon fluid flow model and purification system. Once filled and at the desired high voltage, the \dword{detmodule} immediately becomes live for \dword{snb} and proton decay signals (beam and atmospheric neutrino physics will require a few years of data accumulation)
at which point it is critical that early calibration track the space-time dependence of the detector. Noise data 
and pulser data (taken with signal calibration pulses injected into the electronics) are needed to understand the TPC electronics response. Essential systems at this stage include temperature monitors, purity monitors, \dword{hv} monitors, robust \dword{fe} charge injection system for cold electronics, and a \dword{pds} monitoring system. 
In addition, as the $^{39}$Ar data is available immediately, DUNE must be ready (in terms of reconstruction tools and methods) to utilize $^{39}$Ar decays for understanding both low-energy response and space-time uniformity. 
Dedicated calibration systems as listed in Table~\ref{tab:calibsystem} are deployed and commissioned at this stage. Commissioning data from these systems must verify the expected configuration for each system and identify any needed adjustments to tune for data taking.
 
\textit{Early data taking:} Since DUNE will not yet have all in situ measurements of \lar physical properties at this stage, early calibration of the detector will use \lar measurements from \dword{protodune} or SBND, and \efield{}s from calculations tuned to measured \dword{hv} values.
This early data will most likely need to be recalibrated at a later stage with dedicated calibration runs when in situ measurements are available and as data taking progresses. It will be necessary to tune the detector response models in the simulation on \Dword{protodune} or SBND data during this early phase, requiring that the mechanism for performing this tuning be ready. 
This, together with cosmic ray muon analysis, will provide an approximate energy-response model to use for early physics. The early physics will also require analysis of cosmic ray muon data to develop methods and tools for muon reconstruction from MeV to TeV and a well validated cosmic ray event generator with data. 
Dedicated early calibration runs using calibration hardware systems will develop and tune calibration tools to beam data taking and correct for any space-time irregularities observed in the TPC. Given the expected low rate of cosmic ray events at the underground location (see Section~\ref{sec:phys-calib-sources}), calibration with cosmic rays is not possible over short time scales and will proceed from coarse-grained to fine-grained over the course of years, as statistics accumulate. 
The experiment will rely on calibration hardware systems, such as a laser system, for calibrations that require an independent probe with reduced or removed interdependencies, fine-grained measurements (both in space and time), and detector stability monitoring on the time scales required by physics. Some measurements are simply not possible with cosmic rays (e.g., \dword{apa} flatness, global alignment of all \dword{apa}s). Calibration systems such as radioactive sources or neutron-injection sources will provide low-energy electromagnetic response at the precision required for low-energy \dword{snb} physics. 


\textit{Stable operations:} Once the detector is running stably, dedicated calibration runs, ideally before, during and after each run period, will ensure that detector conditions have not significantly changed.
As statistics accumulate, DUNE can use standard-candle data samples (e.g., Michel electrons and neutral pions) from cosmic rays and beam-induced and atmospheric neutrinos to validate and improve the detector response models needed for precision physics. 
As DUNE becomes systematics-limited, dedicated precision-calibration campaigns using the calibration hardware systems will become crucial for meeting the stringent physics requirements on energy scale reconstruction and detector resolution. For example, the very high energy cosmic ray muons that initiate electromagnetic showers  at that depth 
will provide information to study the electromagnetic  response at high energies. 
\Dword{protodune} and the SBND program initially, and later the DUNE \dword{nd}, will determine other calibration needs. Studies described in Section~\ref{sec:phys-calib-devplan} as part of the calibration development plan may also point toward additional systems to deploy in DUNE.
\end{comment}
%%%%%%%%%%%%%%%%%%%%%%%%%%%%%%%%%%%%%%%%%%%%%%%%%%%%%%%




%%%%%%%%%%%%%%%%%%%%%%%%%%%%%%%%%%%%%%%%%%%%%%%%%%%%%%%

%Older text
\begin{comment}


\subsection{Calibration Sources, Systems and External Measurements}
\label{sec:phys-calib-sources}

%The nominal calibration design includes existing sources, external measurements, monitors,  and dedicated calibration systems. The dedicated systems include a laser system, a neutron-injection system and a radioactive source deployment system. The dedicated systems are motivated as they supply necessary information beyond the reach of the external measurements,  existing sources, and monitors.
The nominal calibration design includes existing sources, external measurements, monitors, as well as dedicated calibration systems intended to provide information beyond the reach of the other systems. The dedicated systems include a laser system, a neutron-injection system and a radioactive source deployment system. 


\subsubsection{Existing sources} 
\label{sec:phys-calib-exis}
%Cosmic rays and neutrino-induced interactions provide commonly used ``standard candles'' like electrons from muon and pion decays, and neutral pions, which have characteristic energy spectra. Cosmic ray muons are also used to determine detector element locations (alignment), timing offsets or drift velocity, electron lifetime, and channel-by-channel response, and help constrain electric field distortions. The rates for cosmic-ray events are summarized in Table~\ref{tab:cosmic-ray-calib-rates}, and certain measurements (e.g. channel-to-channel gain uniformity and cathode panel alignment) are estimated to take several months of data. The rates for atmospheric $\nu$ interactions can be found in Table~\ref{tab:atmnu-rates} \todo{this is Table 1.3 in the non-accelerator physics chapter of the physics volume, need to link it} and are comparable to beam-induced events; both atmospheric and beam induced interactions do not have sufficient rates to provide meaningful spatial or temporal calibration and are expected to provide supplemental measurements only. Also, beam neutrinos may not contribute to the first module calibration during early data taking as the beam is expected to arrive later. The reconstructed energy spectrum of ${}^{39}$Ar beta decays can be used to make a precise measurement of electron lifetime with spatial and temporal variations. It can also provide other necessary calibrations, such as measurements of wire-to-wire response variations and diffusion measurements using the signal shapes associated with the beta decays. The ${}^{39}$Ar beta decay rate in commercially-provided argon is about \SI{1}{\becquerel\per\kilo\gram}, so $O(\mathrm{50k})$ ${}^{39}$Ar beta decays are expected in a single \SI{5}{\milli\s} event readout in an entire \SI{10}{\kt} \detmodule. The ${}^{39}$Ar beta decay cut-off energy is \SI{565}{\keV} which is close to the energy deposited on a single wire by a \dword{mip}. However, there are several factors that can impact the observed charge spectrum from ${}^{39}$Ar beta decays such as electronics noise, electron lifetime and recombination fluctuations. This technique is currently being actively pursued in MicroBooNE~\cite{MICROBOONE-NOTE-1050-PUB} and ProtoDUNE thus providing valuable inputs for DUNE.
Cosmic rays and neutrino-induced interactions provide commonly used ``standard candles,'' e.g., electrons from muon and pion decays, and neutral pions, which have characteristic energy spectra. Cosmic ray muons are also used to determine detector element locations (alignment), timing offsets or drift velocity, electron lifetime, and channel-by-channel response, and to help constrain \efield distortions. 
Table~\ref{tab:cosmic-ray-calib-rates} summarizes the rates for cosmic ray events. Certain measurements (e.g., channel-to-channel gain uniformity and cathode panel alignment) are estimated to take several months of data. Table~\ref{tab:atmos_rates} gives the atmospheric $\nu$ interaction rates, which  
%\todo{this is Table 1.3 in the non-accelerator physics chapter of the physics volume, need to link it. Anne put it in.} 
 are comparable to beam-induced events -- neither occurs at sufficient rates to provide meaningful spatial or temporal calibration; they will likely provide supplemental measurements only. (The beam will not yet be operational for calibration of the first \dword{detmodule} during early data taking.) Instead, we can use the reconstructed energy spectrum of ${}^{39}$Ar beta decays to make a precise measurement of electron lifetime with spatial and temporal variations. 
 This can also provide other necessary calibrations, such as measurements of wire-to-wire response variations and diffusion measurements using the signal shapes associated with the beta decays. The ${}^{39}$Ar beta decay rate in commercially-provided argon is about \SI{1}{\becquerel\per\kilo\gram}, so $O(\mathrm{50k})$ ${}^{39}$Ar beta decays are expected in a single \SI{5}{\milli\s} event readout in an entire \nominalmodsize \detmodule. 
 The ${}^{39}$Ar beta decay cutoff energy is \SI{565}{\keV},  which is close to the energy deposited on a single wire by a \dword{mip}. However, several factors can impact the observed charge spectrum from ${}^{39}$Ar beta decays, such as electronics noise, electron lifetime and recombination fluctuations. MicroBooNE~\cite{MICROBOONE-NOTE-1050-PUB} and ProtoDUNE are actively pursuing this technique, thus providing valuable inputs for DUNE.

\begin{dunetable}
[Annual rates for classes of cosmic-ray events useful for calibration]
{lrl}
{tab:cosmic-ray-calib-rates}
{Annual rates for classes of cosmic-ray events described in this section assuming 100\% reconstruction efficiency.  Energy, angle, and fiducial requirements
have been applied. Rates and geometrical features apply to the single-phase far detector design. }
Sample & Annual Rate & Detector Unit \\ \colhline
Inclusive & $1.3\times 10^6$ & Per \nominalmodsize module \\ \colhline
Vertical-Gap crossing & 3300 & Per gap \\ \colhline
Horizontal-Gap crossing & 3600 & Per gap \\ \colhline
\dword{apa}-piercing & 2200 & Per \dword{apa} \\ \colhline
\dword{apa}-\dword{cpa} piercing & 1800 & Per active \dword{apa} side \\ \colhline
\dword{apa}-\dword{cpa} piercing, \dword{cpa} opposite to \dword{apa} & 360 & Per active \dword{apa} side \\ \colhline
Collection-plane wire hits & 3300 & Per wire \\ \colhline
Stopping Muons & 11000 & Per \nominalmodsize module \\ \colhline
$\pi^0$ Production & 1300 & \nominalmodsize module \\ \colhline
\end{dunetable}

\subsubsection{Monitors} 
%Several instrumentation and detector monitoring devices discussed in detail in Chapter 8 of \voltitlesp{} and \voltitledp{} of the \dword{tdr} will provide valuable information for early calibrations and to track the space-time dependence of the detector. The instrumentation devices include liquid argon temperature monitors, \lar purity monitors, gaseous argon analyzers, cryogenic (cold) and inspection (warm) cameras, and liquid level monitors. The computational fluid dynamics (CFD) simulations play a key role for calibrations initially in the design of the cryogenics recirculation system, and later for physics studies when the cryogenics instrumentation data is used to validate the simulations. Other instrumentation devices essential for calibration such as drift \dword{hv} current monitors and external charge injection systems are discussed in detail in Chapters 4 and 5, %\todo{check with Anne the chapter numbers are correct; we want to refer to HV and TPC-CE chapters here} 
%respectively, of \voltitlesp{} and \voltitledp{} of the \dword{tp}, respectively. 
Chapter~8 of \voltitlesp{} and \voltitledp{} discuss several instrumentation and detector monitoring devices in detail. These devices, including liquid argon temperature monitors, \lar purity monitors, gaseous argon analyzers, cryogenic (cold) and inspection (warm) cameras, and liquid level monitors, will provide valuable information for early calibrations and for tracking the space-time dependence of the \dwords{detmodule}. 
The \dword{cfd} simulations play a key role for calibrations initially in the design of the cryogenics recirculation system, and later for physics studies when the cryogenics instrumentation data can be used to validate the simulations. Chapters~4 and~5 of the \dword{detmodule} volumes discuss other instrumentation devices essential for calibration, such as drift \dword{hv} current monitors and external charge injection systems. 
%\todo{check with Anne the chapter numbers are correct for HV and TPC-CE} 

\subsubsection{External measurements} 
%External measurements here include both past measurements (e.g., ArgoNeuT, DUNE \dword{35t}, ICARUS, \lariat), anticipated measurements from ongoing and future experiments (e.g., \dword{microboone}, SBND, \dword{protodune}) as well as from small scale \dword{lartpc} test stands. External measurements provide a test bed for dedicated calibration hardware systems and techniques which are applicable to the DUNE FD. In particular, \dword{protodune} will provide validation of the fluid flow model using instrumentation data. Early calibration for physics in DUNE will utilize liquid argon physical properties from \Dword{protodune} or SBN  for tuning detector response models in simulation. Table~\ref{tab:calibsystem} provides  references for specific external measurements. The usability of ${}^{39}$Ar has been demonstrated with \microboone data~\cite{MICROBOONE-NOTE-1050-PUB}. 
%Use of  ${}^{39}$Ar  and other radiological sources, including the DAQ readout challenges associated with their use, will be tested on the \dword{protodune} detectors. Dedicated systems for DUNE, including the laser system, have been used by previous experiments (ARGONTUBE~\cite{Zeller:2013sva}, \dword{Ereditato:2014aa}, CAPTAIN, and SBND experiments), and will provide increased information of the use of the system and optimization of the design. Measurements from small-scale liquid argon test stands can also provide valuable information for DUNE. The liquid argon test stand planned at Brookhaven National Lab will provide important information for how field response is simulated and calibrated at DUNE.
%\fixme{KM: We should reach out to Chao for any referencing on TDR timescale. SG: contacted him for CPAD on this, unfortunately none yet at this stage}
DUNE will use external measurements from past experimental runs (e.g., ArgoNeuT, the DUNE \dword{35t}, ICARUS, and \lariat), from ongoing and future experiments (e.g., \dword{microboone}, SBND, and \dword{protodune}), and from small scale \dword{lartpc} test stands. External measurements provide a test bed for dedicated calibration hardware systems and techniques for the \dword{fd}. In particular, \dword{protodune} will provide validation of the fluid flow model using instrumentation data. 
Early calibration for physics in DUNE will utilize \lar physical properties from \Dword{protodune} or SBN  for tuning detector response models in simulation. Table~\ref{tab:calibsystem} provides  references for specific external measurements. The usability of ${}^{39}$Ar has been demonstrated with \microboone data~\cite{MICROBOONE-NOTE-1050-PUB}. 
Use of  ${}^{39}$Ar  and other radiological sources and, in particular, the \dword{daq} readout challenges associated with their use, will be tested on the \dword{protodune} detectors. Dedicated systems for DUNE, 
including the laser system, have been used by previous experiments (ARGONTUBE~\cite{Zeller:2013sva}, \cite{Ereditato:2014aa}, CAPTAIN, and SBND experiments), and will provide more information on use of the system and optimization of the design.  The small-scale \lar test stand planned at Brookhaven National Lab, USA, will provide important information on simulation and calibratio of field response  DUNE.
\end{comment}


\subsubsection{Dedicated Calibration Hardware Systems}
\label{sec:phys-calib-hardware}

This section briefly describes the physics motivation and measurement goals for the calibration hardware systems and the designs currently envisioned. The calibration chapters in \voltitlesp{} and \voltitledp{} of the \dword{tdr} provide further details on the design and development plan for these systems. We plan to deploy prototype designs of these systems in 
%a potential 
the phase 2 of \dword{protodune} to demonstrate proof-of-principle.

\textbf{Laser systems} 

The primary purpose of a laser system is to provide an independent, fine-grained estimate of the \efield in space or time, which is a critical parameter for physics signals as it ultimately impacts the spatial resolution and energy response of the detector. External measurements, e.g.,  MicroBooNE's, use both a laser system and cosmic rays to estimate the \efield, however the expected cosmic rate at the deep underground installation of the \dword{fd} will not provide sufficient spatial or temporal granularity to study local distortions.

\efield distortions can arise from multiple sources. Current simulation studies indicate that positive ion accumulation and drift (space charge) due to ionization sources such as cosmic rays or ${}^{39}$Ar are small in the \dword{fd}; however, the fluid flow pattern in the \dword{fd} is not yet sufficiently understood to exclude the possibility of stable eddies that may amplify the effect for both \single and \dual modules. The \dword{dpmod} risks significant further amplification due to  accumulation in the liquid of ions created by the electron multiplication process in the gas phase.
%SG: above sentence is a direct edit from Bo
%due to ion accumulation at the liquid-gas interface. 
Detector imperfections can also cause localized \efield distortions. Examples include \dword{fc} resistor failures, non-uniform resistivity in the voltage dividers, \dword{cpa} misalignment, \dword{cpa} structural deformations, and \dword{apa} and \dword{cpa} offsets and  deviations from flatness. Individual \efield distortions may add in quadrature with other effects, and can reach 4-5\% under certain conditions, which corresponds to a 1-2\% impact on charge, 
%(dQ), 
and a $\sim 2$ cm impact on position (and fiducial volume). Both charge and position distortions affect energy scale. Understanding all these effects requires an in situ calibration of the E field with a precision of about 1\% with a coverage of at least 75\% of the detector volume.
%proper in situ calibration of the \efield{}. 

The laser calibration system offers secondary uses, e.g., alignment (especially modes that are weakly constrained by cosmic rays, see Figure~\ref{fig:apacurtainalign}), stability monitoring, and diagnosing detector failures in systems such as \dword{hv}.  

\begin{dunefigure}[Sample distortion that may be difficult to detect with cosmic rays]{fig:apacurtainalign}
{An example of a distortion that may be difficult to detect with cosmic rays.  The \dword{apa} frames are shown as rotated rectangles, as viewed from the top.}
\includegraphics[width=0.8\textwidth]{apacurtainalign.png}
\end{dunefigure}

Two systems are under consideration to extract the \efield map: \phel{}s from the \lartpc cathode and direct ionization of the \dword{lar}, both driven by a \SI{266}{\nano\m} laser.  The reference design from \dword{microboone}~\cite{bib:uBlaser2019} and SBND uses direct ionization laser light with multiple laser paths. This can provide field map information in $(x, y, z, t)$; a \phel laser only provides an integrated measurement of the \efield along the drift direction.
The ionization-based system can characterize the \efield with fewer dependencies compared to other systems. If two laser tracks enter the same spatial voxel in a \dword{detmodule}, the relative position of the tracks provides an estimate of the local \threed \efield. The deviation from straightness of single ``laser tracks'' can also be used to constrain local \efield{}s. Comparison of the known laser track path against the path reconstructed from cosmic or beam data, assuming uniform \efield, can also be used to estimate local \efield distortions. A schematic of the ionization laser setup and a laser track from \dword{microboone} is shown in Figure~\ref{fig:uB_laser_schematic}.


\begin{dunefigure}[\microboone laser calibration system schematics]{fig:uB_laser_schematic}
{Left: Schematics of the ionization laser system in \dword{microboone}~\cite{Antonello:2015lea}. Right: A UV laser event in the MicroBooNE detector~\cite{bib:uBlaser2019}. The laser track can be identified by the endpoint on the cathode (larger charge visible at the top of the image) and the absence of charge fluctuations along the track. The charge released at the cathode comes from photoelectric effect. Other tracks seen in the display are from cosmic muons.}
\includegraphics[width=0.45\linewidth]{uB_laser_schematic}
\includegraphics[width=0.45\linewidth]{graphics/run1306_ev134-2.png}
\end{dunefigure}
%\fixme{should \cite{ref:ub-laser-event} be \cite{bib:uBlaser2019}? Anne}

A \phel{}-based calibration system was used in the T2K gaseous (predominantly Ar), TPCs~\cite{Abgrall:2010hi}. 
Thin metal surfaces placed at surveyed positions on the cathode provided point-like and line sources of \phel{}s when illuminated by a laser. The T2K \phel system provided measurements 
of adjacent electronics modules' relative timing response, drift velocity with a few \si{\nano\s} resolution over their \SI{870}{\milli\m} drift distance, electronics gain, 
transverse diffusion, and an integrated measurement of the \efield along the drift direction. DUNE would use the system similarly to diagnose electronics or TPC response issues on demand, and to provide an integral field measurement across drift as well as measure relative distortions of $y$, $z$ positions with time, $x$ and/or drift velocity. \microboone has also observed ejection of \phel{}s from the cathode using the direct ionization laser system. 

\textbf{Pulsed neutron source} 

An external neutron generator system would provide a triggered, well defined energy deposition from neutron capture in $^{40}$Ar detectable throughout the \dword{detmodule} volume. Neutron capture is a critical component of signal processes for \dword{snb} and \dword{lbl} physics; this system would enable direct testing of the detector response  spatially and temporally for the low-energy program.  This is important to measure energy scale, energy resolution and detection threshold spatially and temporally across the enormous DUNE volume.

\begin{dunefigure}[Cross sections enabling the PNS concept]{fig:pns_xsec}
{Illustration of interference anti-resonance dip in the cross section of  \isotope{Ar}{40}. Elastic scattering cross section data is obtained from ENDF VIII.0}
\includegraphics[width=8cm]{graphics/Calib_pns_ES_xsec_Ar40.pdf}
\end{dunefigure}

A triggered pulse of neutrons can be generated outside the TPC and injected into the \dword{lar}, where it spreads through the entire volume to produce a mono-energetic cascade of photons via the $^{40}$Ar(n,$\gamma$)$^{41}$Ar capture process. The uniform population of neutrons throughout the \dword{detmodule} volume exploits a remarkable property of argon -- the near transparency to neutrons of energy near \SI{57}{\keV}. 
This is due to a deep minimum in the cross section caused by the destructive interference between two high-level states of the \isotope{Ar}{40} nucleus (see Fig.~\ref{fig:pns_xsec}). This cross section ``anti-resonance'' is approximately  \SI{10}{\keV} wide, and \SI{57}{keV} neutrons consequently have a scattering length of \SI{859}{m}; the scattering length averaged over the isotopic abundance in natural Ar is approximately \SI{30}{m}. 
For neutrons moderated to this energy the DUNE \dword{lartpc} is essentially transparent. The \SI{57}{keV} neutrons that do scatter quickly leave the anti-resonance and thermalize, at which time they capture. Each neutron capture releases exactly the binding energy difference between \isotope{Ar}{40} and \isotope{Ar}{41}, about \SI{6.1}{\MeV}, in the form of gamma rays. 
%In November 2017, the ACED~\cite{aced-svoboda} collaboration took several hundred thousand neutron capture events at the DANCE\cite{Reifarth:2013xny} facility at LANSCE, which are currently being analyzed to characterize the neutron capture gamma spectrum. 
The neutron capture cross-section and the $\gamma$ spectrum have been measured and characterized. Recently, the ACED Collaboration performed a neutron capture experiment using  the Detector  for Advanced  Neutron  Capture  Experiments  at DANCE (ACED)  at the  Los  Alamos  Neutron  Science  Center  (LANSCE). The result of neutron capture cross-section was published~\cite{Fischer:2019qfr} and will be used to prepare a database for the neutron capture studies. The data analysis of the energy spectrum of correlated gamma cascades from neutron captures is underway.

DUNE plans
%would plan 
to place a fixed, shielded deuterium-deuterium ($DD$) neutron generator  above a penetration in the hydrogenous insulation of the \dword{detmodule} cryostat. Between the generator and the cryostat, layers of water or plastic and intermediate fillers would provide sufficient degradation of the neutron energy. 

\textbf{Additional Systems}

There are 
%alternative designs or 
additional systems under consideration for DUNE calibration. Radioactive source deployment provides an in situ source of low energy electrons and de-excitation gamma rays at a known location and with a known activity, which are directly relevant for detection of \dword{snb} or $^{8}B$ solar neutrinos. As shown in Section~\ref{ch:snb-lowe}, the electron and photon response in the TPC is quite different (electrons leave worm-like tracks, photons leave `blips'). The PNS source will provide a 6.1 MeV multi-photon signal; radioactive sources can provide a single photon signal to measure detection threshold and demonstrate sufficient uncertainty on energy resolution at the peak of the \dword{snb} photon signal.  The radioactive source system is under study, and feasibility and safety of deployment would be established with a dedicated run using a prototype system in ProtoDUNE.

The utility of internal source injection (e.g., ${}^{222}$Rn or ${}^{220}$Rn injection) for mapping electron lifetime and fluid flow in the \dword{tpc}, used in dark matter experiments, will also be considered in the future. The major challenge for this system is if the 
coverage of the \dword{pds} is sufficient, and whether or not it will be able to identify a signal and trigger over the massive amount of ${}^{39}$Ar present. Recognizing that the presence of radioactive impurities can also impact such a system, the newly formed DUNE \dword{fd} Background Task Force will address this concern. This system would not require any cryostat penetrations or affect major \dword{daq} requirements.



\begin{comment}
\subsubsection{Dedicated Calibration Hardware Systems}
\label{sec:phys-calib-hardware}
%This section briefly describes the physics motivation and measurement goals for the external calibration hardware systems along with a brief description of the designs currently envisioned. Further details on the design and development plan for these systems can be found in the calibration chapters in the \voltitlesp{} and \voltitledp{} of the \dword{tdr}. For all the systems listed here, the goal is to deploy Prototype designs in a potential phase 2 of ProtoDUNE to demonstrate proof of principle before they are installed in the FD.
This section briefly describes the physics motivation and measurement goals for the calibration hardware systems and the designs currently envisioned. The calibration chapters in \voltitlesp{} and \voltitledp{} of the \dword{tdr} provide further details on the design and development plan for these systems. We plan to deploy prototype designs of these systems in a potential phase 2 of \dword{protodune} to demonstrate proof-of-principle.

\textbf{Laser systems} 

%The primary purpose of a laser system is to provide an independent, fine-grained estimate of the \efield in space or time, which is a critical parameter for physics signals as it ultimately impacts the spatial resolution and energy response of the detector. External measurements, such as MicroBooNE, use both laser and copious cosmic rays to estimate the \efield, but the expected cosmic rate at DUNE does not provide sufficient spatial or temporal granularity to probe distortions of interest or unforeseen ones. 
The primary purpose of a laser system is to provide an independent, fine-grained estimate of the \efield in space or time, which is a critical parameter for physics signals as it ultimately impacts the spatial resolution and energy response of the detector. External measurements, e.g.,  MicroBooNE's, use both a laser system and cosmic rays to estimate the \efield, however the expected cosmic rate at the deep underground installation of the \dword{fd} will not provide sufficient spatial or temporal granularity to be useful.

%There are multiple sources which may distort the electric field temporally or spatially in the detector. Current simulation studies indicate that positive ion accumulation and drift (space charge) due to ionization sources such as cosmic rays or ${}^{39}$Ar is small in the DUNE \dword{fd}; however, not enough is known yet about the fluid flow pattern in the FD to exclude the possibility of stable eddies which may amplify the effect for both \single and \dual modules. This effect can get further amplified significantly in \dword{dpmod} due to ion accumulation at the liquid-gas interface. 
%Additionally, other sources in the detector (especially detector imperfections) can cause \efield distortions. For example, field cage resistor failures, non-uniform resistivity in the voltage dividers, CPA misalignment, CPA structural deformations, and APA and CPA offsets and  deviations from flatness can create localized \efield distortions. Each individual \efield distortion may add in quadrature with other effects, and can reach 4\% under certain conditions. Understanding all these effects require in-situ measurement of \efield for proper calibration. Many useful secondary uses of laser include alignment (especially modes that are weakly constrained by cosmic rays), stability monitoring, and diagnosing detector failures (e.g., \dword{hv}).  
\efield distortions can arise from multiple sources. Current simulation studies indicate that positive ion accumulation and drift (space charge) due to ionization sources such as cosmic rays or ${}^{39}$Ar are small in the \dword{fd}; however, the fluid flow pattern in the \dword{fd} is not yet sufficiently understood to exclude the possibility of stable eddies that may amplify the effect for both \single and \dual modules. The \dword{dpmod} risks significant further amplification due to ion accumulation at the liquid-gas interface. 
Detector imperfections can also cause localized \efield distortions. Examples include \dword{fc} resistor failures, non-uniform resistivity in the voltage dividers, \dword{cpa} misalignment, \dword{cpa} structural deformations, and \dword{apa} and \dword{cpa} offsets and  deviations from flatness. Individual \efield distortions may add in quadrature with other effects, and can reach 4\% under certain conditions. Understanding all these effects requires proper in situ calibration of the \efield{}. 

The laser calibration system offers secondary uses, e.g., alignment (especially modes that are weakly constrained by cosmic rays), stability monitoring, and diagnosing detector failures in systems such as \dword{hv}.  

%\todo{fix figure; Ask Anne for figure, can't find in IDR overleaf link?}
%\begin{dunefigure}[Sample distortion that may be difficult to detect with cosmic rays]{fig:apacurtainalign}{An example of a distortion that may be difficult to detect with cosmic rays.  The \dword{apa} frames are shown as rotated rectangles, as viewed from the top.}
%\includegraphics[width=0.8\textwidth]{apacurtainalign.png}
%\end{dunefigure}

%Two laser-based systems have been considered to extract the electric field map: photoelectrons from the \lartpc cathode and direct ionization of the \dword{lar}, both driven by a \SI{266}{\nano\m} laser.  The reference design from \dword{microboone} and SBND experiments uses direct ionization laser light with multiple laser paths. This can provide field map information in $(x, y, z, t)$; \phel laser only provides integral field across the drift. The ionization-based system can characterize the electric field with fewer dependencies compared to other systems. If two laser tracks enter the same spatial voxel  ($10 \times 10 \times 10~\textrm{cm}^3$ volume)\fixme{Include text connecting the 1\% FV requirement to the required laser measurement granularity? Jose has written up some argument on why 10x10x10cm voxels was enough, and maybe even 20x20x20cm, based on the estimate of maximum 4\% expected field distortions.} in the \dword{detmodule}, the relative position of the tracks provides an estimate of the local \threed \efield. The deviation from straightness of single ``laser tracks'' can also be used to constrain local \efield{}s. Comparison of the known laser track path with the path reconstructed from cosmic or beam data assuming uniform field can also be used to estimate local \efield distortions. 
Two systems are under consideration to extract the \efield map: \phel{}s from the \lartpc cathode and direct ionization of the \dword{lar}, both driven by a \SI{266}{\nano\m} laser.  The reference design from \dword{microboone} and SBND uses direct ionization laser light with multiple laser paths. This can provide field map information in $(x, y, z, t)$; a \phel laser only provides an integrated measurement of the \efield along the drift direction.
%\fixme{integral field across the drift is not clear to me. anne; SG: clarified} 
The ionization-based system can characterize the \efield with fewer dependencies compared to other systems. If two laser tracks enter the same spatial voxel  
%($10 \times 10 \times 10~\textrm{cm}^3$ volume)
%\fixme{Include text connecting the 1\% FV requirement to the required laser measurement granularity? Jose has written up some argument on why 10x10x10cm voxels was enough, and maybe even 20x20x20cm, based on the estimate of maximum 4\% expected field distortions.} 
in a \dword{detmodule}, the relative position of the tracks provides an estimate of the local \threed \efield. The deviation from straightness of single ``laser tracks'' can also be used to constrain local \efield{}s. Comparison of the known laser track path against the path reconstructed from cosmic or beam data, assuming uniform field, can also be used to estimate local \efield distortions. 


%A \phel{}-based calibration system was used in the T2K gaseous (predominantly Ar), TPCs~\cite{Abgrall:2010hi}.
%Thin metal surfaces placed at surveyed positions on the cathode provided point-like and line sources of \phel{}s when illuminated by a laser. The T2K \phel system provided measurements of adjacent electronics modules' relative timing response, drift velocity with few \si{\nano\s} resolution of \SI{870}{\milli\m} drift distance, electronics gain, transverse diffusion, and an integrated measurement of the electric field along the drift direction. For DUNE, the system would be similarly used as on T2K to diagnose electronics or TPC response issues on demand, and provide an integral field measurement and relative distortions of $y$, $z$ positions with time, and of either $x$ or drift velocity. Ejection of \phel{}s from the cathode using the direct ionization laser system has also been observed in MicroBooNE.
%\todo{link uB laser event display showing this here: https://microboone-docdb.fnal.gov/cgi-bin/private/ShowDocument?docid=4687\&version=6}. SG: maybe not, since there is no public reference to use or too much emphasis on not so important thing here?
A \phel{}-based calibration system was used in the T2K gaseous (predominantly Ar), TPCs~\cite{Abgrall:2010hi}. 
Thin metal surfaces placed at surveyed positions on the cathode provided point-like and line sources of \phel{}s when illuminated by a laser. The T2K \phel system provided measurements 
of adjacent electronics modules' relative timing response, drift velocity with a few \si{\nano\s} resolution over their \SI{870}{\milli\m} drift distance, electronics gain, 
%\fixme{but have they used it for anything? If not, this seems irrelevant. anne; SG: they haven't, but most don't know this can be done and it conveys redundancy.}

\textbf{Pulsed neutron source} 

%An external neutron generator system would provide a triggered, well defined energy deposition from neutron capture in $^{40}$Ar that can be detected throughout the volume. Neutron capture is a critical component of signal processes for \dword{snb} and \dword{lbl} physics; this system enables direct testing of the response of the detector spatially and temporally for the low energy program. 
An external neutron generator system would provide a triggered, well defined energy deposition from neutron capture in $^{40}$Ar detectable throughout the \dword{detmodule} volume. Neutron capture is a critical component of signal processes for \dword{snb} and \dword{lbl} physics; this system would enable direct testing of the detector response  spatially and temporally for the low-energy program. 

%A triggered pulse of neutrons can be generated outside the TPC, then injected into the \dword{lar}, where it spreads through the entire volume to produce mono-energetic a cascade of photons via the $^{40}$Ar(n,$\gamma$)$^{41}$Ar capture process. The uniform population of neutrons throughout the detector module volume exploits a remarkable property of argon -- the near transparency to neutrons with an energy near \SI{57}{\keV} due to a deep minimum in the cross section caused by the destructive interference between two high-level states of the \isotope{Ar}{40} nucleus. This cross section ``anti-resonance'' is about \SI{10}{\keV} wide, and 57 keV neutrons consequently have a scattering length of 859 m; the scattering length averaged over the isotopic abundance in natural Ar is approximately 30~m. For neutrons moderated to this energy the DUNE \dword{lartpc} is essentially transparent. The 57 keV neutrons that do scatter quickly leave the anti-resonance and thermalize, at which time they capture. Each neutron capture releases exactly the binding energy difference between \isotope{Ar}{40} and \isotope{Ar}{41}, about \SI{6.1}{\MeV} in the form of gamma rays. In Nov 2017, the ACED~\cite{aced-svoboda} Collaboration took several hundred thousand neutron capture events at the DANCE\cite{Reifarth:2013xny} facility at LANSCE which are currently being analyzed to characterize neutron capture gamma spectrum. 
A triggered pulse of neutrons can be generated outside the TPC and injected into the \dword{lar}, where it spreads through the entire volume to produce a mono-energetic cascade of photons via the $^{40}$Ar(n,$\gamma$)$^{41}$Ar capture process. The uniform population of neutrons throughout the \dword{detmodule} volume exploits a remarkable property of argon -- the near transparency to neutrons of energy near \SI{57}{\keV}. 
This is due to a deep minimum in the cross section caused by the destructive interference between two high-level states of the \isotope{Ar}{40} nucleus. This cross section ``anti-resonance'' is approximately  \SI{10}{\keV} wide, and \SI{57}{keV} neutrons consequently have a scattering length of \SI{859}{m}; the scattering length averaged over the isotopic abundance in natural Ar is approximately \SI{30}{m}. 
For neutrons moderated to this energy the DUNE \dword{lartpc} is essentially transparent. The \SI{57}{keV} neutrons that do scatter quickly leave the anti-resonance and thermalize, at which time they capture. Each neutron capture releases exactly the binding energy difference between \isotope{Ar}{40} and \isotope{Ar}{41}, about \SI{6.1}{\MeV}, in the form of gamma rays. 
In November 2017, the ACED~\cite{aced-svoboda} collaboration took several hundred thousand neutron capture events at the DANCE\cite{Reifarth:2013xny} facility at LANSCE, which are currently being analyzed to characterize the neutron capture gamma spectrum. 


%In terms of design, the fixed, shielded deuterium-deuterium ($DD$) neutron generator would be located above a penetration in the hydrogenous insulation. Between the generator and the cryostat, layers of water or plastic and intermediate fillers will be included for sufficient degradation of the neutron energy. 
DUNE would plan to place a fixed, shielded deuterium-deuterium ($DD$) neutron generator  above a penetration in the hydrogenous insulation of the \dword{detmodule} cryostat. Between the generator and the cryostat, layers of water or plastic and intermediate fillers would provide sufficient degradation of the neutron energy. 


\textbf{Radioactive source system}

%Radioactive source deployment provides an in-situ source of the electrons and de-excitation gamma rays, which are directly relevant for physics signals from supernova or $^{8}B$ solar neutrinos. Secondary measurements from the source deployment include electromagnetic (EM) shower characterization for long-baseline $\nu_e$ CC events, electron lifetime as a function of \dword{detmodule} vertical position, and help determine radiative components of the electron energy spectrum from muon decays.
Radioactive source deployment provides an in situ source of the electrons and de-excitation gamma rays, which are directly relevant for physics signals from \dword{snb} or $^{8}B$ solar neutrinos. 
Secondary measurements from the source deployment include electromagnetic  shower characterization for \dword{lbl} $\nu_e$ \dword{cc} events and electron lifetime as a function of \dword{detmodule} vertical position. Source deployment would also help determine radiative components of the electron energy spectrum from muon decays.

%A composite source can be used that consists of $^{252}$Cf, a strong neutron emitter, and $^{58}$Ni, which, via the $^{58}$Ni(n,$\gamma$)$^{59}$Ni process, converts one of the $^{252}$Cf decay neutrons, suitably moderated, to a mono-energetic 9 MeV photon~\cite{Rogers:1996ks}. The source is envisaged to be inside a cylindrical teflon moderator such that it can be deployed using one of the instrumentation ports. The activity of the radioactive source is chosen such that no more than one \SI{9}{\MeV} capture $\gamma$-event occurs during a single drift period. This allows one to use the arrival time of the measured light as a $t0$ and then measure the average drift time of the corresponding charge signal(s). The sources would be deployed outside the \dword{fc} within the cryostat and would be removed and stored outside the cryostat when not in use. The design and deployment of radioactive source calibration system for the DUNE FD is currently being studied actively by the calibration consortium. 
DUNE could implement a composite source that consists of $^{252}$Cf, a strong neutron emitter, and $^{58}$Ni, which, via the $^{58}$Ni(n,$\gamma$)$^{59}$Ni process, converts one of the $^{252}$Cf decay neutrons, suitably moderated, to a mono-energetic \SI{9}{MeV} photon~\cite{Rogers:1996ks}. We envisage placement of the source inside a cylindrical teflon moderator so as to deploy it using one of the instrumentation ports. The activity of the radioactive source is chosen such that no more than one \SI{9}{\MeV} capture $\gamma$-event occurs during a single drift period. We can then use the arrival time of the measured light as a $t0$ and measure the average drift time of the corresponding charge signals. The sources would be deployed  within the cryostat but outside the \dword{fc}, and would be removed and stored outside the cryostat when not in use. The calibration consortium is actively studying  the design and deployment of a radioactive source calibration system for the\dword{fd}. 
\end{comment}

%%%%%%%%%%%%%%%%%%%%%%%%%%%%%%%%%%%%%%%%%%%%%%%%%%


\subsection{Calibration Staging Plan}
%\subsection{Calibration  Plan}
\label{sec:phys-calib-approach}

%\fixme{KM: I felt the previous summary was not really full of content, and that ending on seeing the entire system working was better.}

The calibration strategy for DUNE will need to address the evolving operational and physics needs at every stage of the experiment in a timely manner using the primary sources and systems listed in Table~\ref{tab:calibsystem}. 
%Here we describe the development plan of calibration during the phases of the experiment: design validation, commissioning, early data taking, and stable operation. 
Here we describe the validation plan for calibration systems at ProtoDUNE and a staging plan to deploy calibration systems during different phases of the experiment: commissioning, early data taking, and stable operations. 

This \dword{tdr} presents the baseline calibration systems and strategy. Post-\dword{tdr}, once the calibration strategy is set, the calibration consortium will need to develop the necessary designs for calibration hardware along with tools and methods to be used with various calibration sources. To allow for flexibility in this process, the physical interfaces for calibration such as flanges or ports on the cryostat will be designed to accommodate the calibration hardware. As described in the calibration \dword{sp} detector volume, the calibration task force has provided the necessary feedthrough penetration design 
%updates 
for the \dword{spmod} and will soon finalize the design for the \dword{dpmod}.  As DUNE physics turns on at different rates and times, a calibration strategy at each stage for physics and data taking is required. The strategy described in this section assumes that all systems are commissioned and deployed according to the nominal DUNE run plan.

\textit{Design Validation:} A second run of ProtoDUNE will be used to validate the designs of dedicated calibration systems, including the laser, PNS, and possibly the proposed radioactive source. In addition, ProtoDUNE data (and the SBN program) will provide data analysis techniques, tools, and detector model simulation improvements in advance of DUNE operation.


\textit{Commissioning:} When a \dword{detmodule} is filled, data from various instrumentation devices validate the argon fluid flow model and purification system. Once filled and at the desired high voltage, the \dword{detmodule} immediately becomes live for \dword{snb} and proton decay signals (beam and atmospheric neutrino physics will require a few years of data accumulation)
at which point it is critical that early calibration track the space-time dependence of the detector. Noise data 
and pulser data (taken with signal calibration pulses injected into the electronics) are needed to understand the TPC electronics response. Essential systems at this stage include temperature monitors, purity monitors, \dword{hv} monitors, robust \dword{fe} charge injection system for cold electronics, and a \dword{pds} monitoring system. 
In addition, as the $^{39}$Ar data is available immediately, DUNE must be ready (in terms of reconstruction tools and methods) to utilize $^{39}$Ar decays for understanding both low-energy response and space-time uniformity. 
Dedicated calibration systems as listed in Table~\ref{tab:calibsystem} are deployed and commissioned at this stage. Commissioning data from these systems must verify the expected configuration for each system and identify any needed adjustments to tune for data taking.

\textit{Early data taking:} Since DUNE will not yet have all in situ measurements of \lar physical properties at this stage, early calibration of the detector will use \lar measurements from \dword{protodune} or SBN, and \efield{}s from calculations tuned to measured \dword{hv} values.
This early data will most likely need to be recalibrated at a later stage with dedicated calibration runs when in situ measurements are available and as data taking progresses.
%It will be necessary to tune the detector response models in the simulation on \Dword{protodune} or SBN data during this early phase, requiring that the mechanism for performing this tuning be ready. 
%This, together with cosmic ray muon analysis, will provide an approximate energy-response model to use for early physics. 
The early physics will also require analysis of cosmic ray muon data to develop methods and tools for muon reconstruction from MeV to TeV and a well validated cosmic ray event generator with data. 
Dedicated early calibration runs using calibration hardware systems will develop and tune calibration tools to beam data taking and correct for any space-time irregularities observed in the TPC. Given the expected low rate of cosmic ray events at the underground location (see Section~\ref{sec:phys-calib-sources}), calibration with cosmic rays is not possible over short time scales and will proceed from coarse-grained to fine-grained over the course of years, as statistics accumulate. 
The experiment will rely on calibration hardware systems, such as a laser system, for calibrations that require an independent probe with reduced or removed interdependencies, fine-grained measurements (both in space and time), and detector stability monitoring on the time scales required by physics. Some measurements are simply not possible with cosmic rays (e.g., \dword{apa} flatness, global alignment of all \dword{apa}s). %Calibration systems such as %radioactive sources or the PNS,  will provide low-energy electromagnetic response at the precision required for low-energy \dword{snb} physics. 

\textit{Stable operations:} Once the detector is running stably, dedicated calibration runs, ideally before, during and after each run period, will ensure that detector conditions have not significantly changed.
As statistics accumulate, DUNE can use standard-candle data samples (e.g., Michel electrons and neutral pions) from cosmic rays and beam-induced and atmospheric neutrinos to validate and improve the detector response models needed for precision physics. 
As DUNE becomes systematics-limited, dedicated precision-calibration campaigns using the calibration hardware systems will become crucial for meeting the stringent physics requirements on energy scale reconstruction and detector resolution. For example, understanding electromagnetic (EM) response in the FD will require both cosmic rays
and external systems. The very high energy muons from cosmic rays at that depth that initiate
EM showers (which would be rare at ProtoDUNE or SBND), will provide information to study EM response at high energies. External systems such as the pulsed neutron source system or the proposed radioactive source system will provide low energy EM response at the precision required for low energy supernovae
physics. Dedicated measurements of charged hadron interactions, initially in \Dword{protodune} and later with DUNE \dword{nd} will also be important in this phase.


%KM Older text:
\begin{comment}
\subsubsection{Calibration Development Plan}
\label{sec:phys-calib-devplan}
\fixme{{\bf placeholder:} Update with studies by the final TDR deadline}

%\fixme{This top pgraph moved up from bottom of section, and split into 3 by anne}
Studies are underway to clarify the physics use limitations of the existing sources presented in Section~\ref{sec:phys-calib-exis}:
\begin{itemize}
\item The relative importance of electromagnetic shower photons below pair production threshold and quantifying what can be achieved for both electron lifetime measurements and the overall energy-scale calibration from cosmic rays, ${}^{39}$Ar beta decays, \dword{lbl} interactions and atmospheric neutrinos is being determined, in terms of spatial and temporal granularity using decay electrons and $\pi^0$ samples. We expect that combining information from cosmic ray events with proposed and existing systems (laser-based, neutrino-induced events, and dedicated muon systems) will reduce the total uncertainties on misalignment.  

\item The impact of misalignments on the physics case is being studied, especially for alignment modes that are weakly constrained due to cosmic ray direction, including global shifts and rotations of all detector elements, as well as crumpling modes where the edges of the \dwords{apa} hold together but angles are slightly different from nominal. Misalignment can be studied in two steps: 1. using a parameterization in the simulation that produces distorted muons which can then be reconstructed using nominal reconstruction to study impact on the multiple-coloumb scattering (MCS) momentum measurement. 2. constrain all of the APA (and CPA) locations with cosmic rays and apply these as alignment constants during reconstruction so that downstream algorithms are minimally impacted by misalignment. The second step is what the end goal is for the operating far detector. The motivation for step 1 is to estimate the impact on the physics of residual uncorrected misalignment once the second step is done.

%The impact of misalignment on physics can be studied in two steps: 1. a parameterization is needed to estimate the impact of misalignment on the physics. This can be done using what is called a charge drift distortion service that the simulation module can use to produce distorted simulation. The plan is to reconstruct distorted muons with nominal reconstruction and see if it distorts the multiple-coloumb scattering (MCS) momentum measurement. 2. The second step is what the end goal is for the operating far detector. We would like to constrain all of the APA (and CPA) locations with cosmic rays and apply these as alignment constants to the reconstruction step so that downstream algorithms are minimally impacted by misalignment. The reason for doing the first step is to estimate the impact on the physics of residual uncorrected misalignment once the second step is done.

\item The impact of the fluid model on physics needs requires quantification through \dword{cfd} simulations (e.g., overall temperature variation in the cryostat and impact on drift velocity, overall impurity variation across the \detmodule{}, and impact on energy scale especially for \dword{dp}, which has a \SI{12}{\m} long single drift path). The \dword{cfd} studies will also be important for understanding how \lar flow can impact space charge from both ionization and non-ionization sources, and ion accumulation (both positive and negative), separately for the \dword{spmod} and \dword{dpmod} designs. 

%\item Simulation tools to quantify \efield distortions from various sources are being developed to study impact on physics. Maps of electric field distortions are produced from expectations for both ionization (e.g. cosmic rays, ${}^{39}$Ar) and non-ionization sources (e.g. deformations of \dword{hv} components) and are propagated in simulation with appropriate knobs to turn on\slash off the spatial and recombination effects~\cite{MICROBOONE-NOTE-1050-PUB}. The former arises as \efield distorts the reconstructed positions of ionization electron clusters detected by the TPC wire planes and the latter arises due to the dependence of electron-ion recombination on \efield.
\end{itemize}

%As part of the calibration development plan, additional systems are being explored for the DUNE FD. One of the primary systems that is currently being investigated by the calibration group is the utility of internal source injection (e.g. ${}^{222}$Rn or ${}^{220}$Rn injection) for mapping electron lifetime and fluid flow in the TPC. Initial considerations and challenges for such a system include  efficiency\slash coverage of PDS impacting the ability to tag t$_{0}$ for such events, and the visibility of signal and triggering relative to the massive amount of ${}^{39}$Ar present in the detector. These are currently being studied by the calibration group. The presence of radioactive impurities can also impact such a system. This concern will be addressed within the newly formed DUNE FD Background Task Force. As this system does not require any cryostat penetrations or major \dword{daq} requirements, no accommodations will need to be made for this system.

Beyond the dedicated calibration systems described in earlier sections, the calibration consortium is exploring additional systems for the \dword{fd}:
\begin{itemize}
\item The consortium is also exploring an external muon tagger (EMT) system. A dedicated fast-tracking EMT system would provide track position, direction, and time information independent of the TPC and \dword{pds}. Rock muons from beam interactions in the rock surrounding the \dwords{detmodule} have similar energy  and angular  spectra as \dword{cc} \numu events and an EMT covering the front face of the detector can tag these muons. 

\item The utility of internal source injection (e.g., ${}^{222}$Rn or ${}^{220}$Rn injection) for mapping electron lifetime and fluid flow in the TPC is currently under investigation.  Initial considerations and challenges for such a system include impacts to the efficiency and coverage of the \dword{pds} and its ability to tag $t_{0}$, and the visibility of signal and triggering relative to the massive amount of ${}^{39}$Ar present. Recognizing that the presence of radioactive impurities can also impact such a system, the newly formed DUNE \dword{fd} Background Task Force will address this concern. This system would not require any cryostat penetrations or affect major \dword{daq} requirements.
\end{itemize}

%An external muon tagger (EMT) system is another system that is currently being explored. A dedicated fast tracking EMT system would provide track position, direction, and time information independent of TPC and PDS systems. Rock muons from beam interactions in the rock surrounding the cryostat have similar energy  and angular  spectrum as CC \numu events. A nominal design of the EMT would cover the front face of the detector (approximately $14\textrm{m} \times 12\textrm{m}$) to provide an estimate of the initial position, and the time for a subset of these events, independent of the TPC and PDS systems. A second, similarly sized panel, \SI{1}{\m} away from the cryostat would provide directional information. Additional measurements are possible elsewhere in the detector if the system is portable; it could be positioned on top of the cryostat to capture (nearly downward-going) cosmic ray particles during commissioning, or positioned along the side for rock muon-induced tracks along the drift direction. Optimization of EMT size and pixelization, possible cost-saving options including re-use of existing scintillators (e.g. MINOS) or counter systems (e.g., \dword{protodune} or SBN), and the available space for the EMT around the cryostat are currently being investigated by the calibration group. 

%The consortium is also exploring an external muon tagger (EMT) system. A dedicated fast-tracking EMT system would provide track position, direction, and time information independent of the TPC and \dword{pds}. Rock muons from beam interactions in the rock surrounding the \dwords{detmodule} have similar energy  and angular  spectra as \dword{cc} \numu events. A nominal design of the EMT would cover the front face of a \dword{detmodule} (approximately \cryostatwdth $\times$ \tpcheight) and provide an estimate of the initial position and the time for a subset of these events, independent of the TPC and \dword{pds}. A second, similarly sized panel, \SI{1}{\m} away from the cryostat would provide directional information. Additional measurements are possible elsewhere around a \dword{detmodule} if the system is portable. Positioned on top of the cryostat, it could capture (nearly downward-going) cosmic ray particles during commissioning, or positioned along the side it could capture rock muon-induced tracks along the drift direction. 

%Optimization of the EMT size and pixelization, possible cost-saving options including re-use of existing scintillators (e.g., MINOS) or counter systems (e.g., \dword{protodune} or SBN), and the available space for the EMT around the cryostat are currently under investigation. 

%Studies are also underway to clarify the physics use limitations of the various existing sources presented in Section~\ref{sec:phys-calib-exis}. For example, determining the relative importance of electromagnetic shower photons below pair production threshold and quantification of what can be achieved for electron lifetime measurements and the overall energy scale calibration from cosmic rays, ${}^{39}$Ar beta decays, long baseline interactions and atmospheric neutrinos, in terms of spatial and temporal granularity using decay electrons, $\pi^0$ samples. It is expected that combinations of information from cosmic-ray events with dedicated and existing systems (laser-based, neutrino-induced events, and dedicated muon systems) will reduce the total uncertainties on misalignment.  The impact of misalignments on the physics case needs to be studied, especially for alignment modes which are weakly constrained due to cosmic ray direction, including global shifts and rotations of all detector elements, and crumpling modes where the edges of the \dwords{apa} hold together but angles are slightly different from nominal. The impact of the fluid model on physics needs require quantification through \dword{cfd} simulations (e.g., overall temperature variation in the cryostat and impact on drift velocity; overall impurity variation across the \detmodule{}, and impact on energy scale especially for \dword{dp} which has a \SI{12}{\m} long single drift path). The CFD studies will also be important in understanding how \lar flow can impact space charge from both ionization and non-ionization sources and ion accumulation (both positive and negative ions), separately for \dword{spmod} and \dword{dpmod} designs. 

%\fixme{anne moved edited pgraph to top of this section}
%%%%%%%%%%%%%%%%%%%%%%%%%%%%%%%%%%%%%%%%%%%%%%%%%%%%%%%
\subsection{Summary and Future Plans}
\label{sec:phys-calib-sum}

\fixme{{\bf placeholder:} Currently very generic; update with specific conclusions once studies converge}

The physics requirements for the broad DUNE physics program places stringent requirements on calibration systems and sources. DUNE has identified the dedicated calibration systems discussed in section~\ref{sec:phys-calib-hardware} as important to its physics program. The calibration consortium will develop and test the full designs of the these systems as well as develop plans for \dword{daq}, costing, risk mitigation, integration and installation efforts. The multi-purpose ports on the \dword{fd} cryostats enable deployment of additional calibrations systems in the future.
%The formation of the calibration consortium in October 2018 was a major step forward in terms of developing and testing the full designs of the external hardware systems discussed in Section~\ref{sec:phys-calib-hardware} as well as developing plans for \dword{daq}, costing, risk mitigation, integration and installation efforts. 

As presented in Section~\ref{sec:phys-calib-devplan} the calibration task force is developing the necessary tools and techniques in collaboration with other physics groups to propagate detector physics effects into the physics analyses and is studying the impact of calibrations on DUNE physics at a preliminary level. The task force will further improve and expand upon
these studies in the near future as tools and techniques continue to mature. 

\end{comment}

%-- Copied from IDR %%
%\fixme{KM: What is our development plan//any other core pieces? Connect to deetector volumes and refernce any specific physics studies there. Global program of protoDUNE. Timescale TBD for when this is finalized, and include milestones for assembling // consortium.}

%Table~\ref{tab:TDRsteps} lists some of the key upcoming milestones for the DUNE calibration effort. 
\begin{comment}
\begin{dunetable}[Key calibration milestones leading to first detector installation]{ll}{tab:TDRsteps}{Key calibration milestones leading to first detector installation.}
Date & \textbf{Milestone}\\ \toprowrule
May 2018 & \dword{tp} \\ \colhline
June 2018 & Finalize process of integrating calibration into consortium structure\\ \colhline
Jan. 2019 & Design validation of calibration systems using \dword{protodune}\slash SBN data  \\
&(where applicable) and incorporate lessons learned into designs \\ \colhline
Apr. 2019 & Technical design report \\ \colhline
Sep. 2022 & Finish construction of calibration systems for Cryostat \#1 \\ \colhline
May 2023 & Cryostat 1 ready for TPC installation \\ \colhline
Oct. 2023 & All calibration systems installed in Cryostat \#1 \\
\end{dunetable} 
\end{comment}

%%%%%%%%%%%%%%%%%%%%%%%%%%%%%%%%


\subsection{$^{39}$Ar beta decays}
\label{app:ar39}

%The reconstructed energy spectrum of $^{39}$Ar beta decays serve as one of the tests of the detector model, i.e. it compares a ``known'' source to the detector simulation. 
%The technique of reconstructing these events has been recently demonstrated with MicroBooNE data~\cite{MICROBOONE-NOTE-1050-PUB}.
Assuming the $^{39}$Ar beta decays are uniformly distributed in the drift direction, one is able to precisely determine the expected reconstructed energy spectrum 
%for 
provided a given set of well measured detector response parameters.  This can be done independently of using timing information (e.g.~from prompt scintillation light). 

A number of factors can impact accurately measuring the end point energy, including noise, wire response, electron lifetime, recombination (and electric field), cosmogenic activity, and other radiological backgrounds.
%In addition to the mentioned detector response effects, cosmogenic activity and other radiological backgrounds can contribute to the high energy tail of the observed signal.  
Many of the detector effects may be determined in-situ.  For instance, measuring the electronics response can be done in situ with pulser data (charge injection on the front-end ASICs); measuring the wire field response can be done with cosmic tracks and other dedicated measurements ex-situ. There are also plans to measure recombination parameters ex-situ (e.g. ProtoDUNE, MicroBooNE). Figure~\ref{fig:ar39} illustrates the different possible reconstructed $^{39}$Ar beta decay electron energy spectra one might see in the SP DUNE far detector after correcting for all other detector effects except for electron lifetime.
%, for $^{39}$Ar beta decays occurring in the single-phase DUNE far detector.  
Also shown in Figure~\ref{fig:ar39} is the impact of varying the 
recombination model.
%true recombination model from the one assumed in energy reconstruction of the $^{39}$Ar beta decay electron, with infinite electron lifetime.  
The impact on the reconstructed energy spectrum is very different for the two detector effects, allowing for simultaneous determination of both quantities.




This method is one foreseeable way to obtain a fine-grained (spatially and temporally) electron lifetime measurement in the DUNE FD.  It can also provide other necessary calibrations, such as measurements of wire-to-wire response variations and diffusion measurements, 
%using the signal shapes associated with the beta decays, 
and could serve as an online monitor of 
%electric field 
\efield distortions in the detector by looking at the relative number of decays 
%in the detector 
near the edges of the 
%LArTPC
detector.  
%Currently, the plan is to study this calibration technique with data from \dword{protodune}.
%, well in advance of first operations with the DUNE far detector.

One important consideration is whether or not the DUNE 
%far detector 
\dword{daq} can provide the necessary rate and type of data 
%in order 
to successfully carry out this calibration at the desired frequency and level of spatial precision.  Knowing that the $^{39}$Ar beta decay rate is about 1~Bq/kg in natural (atmospheric) argon, one finds that $O(\mathrm{50k})$ $^{39}$Ar beta decays are expected in a single 5~ms event readout in an entire 10~kt module.  
From studies at MicroBooNE, $O(\mathrm{250k})$ will be needed 
%to carry out 
for percent-level calibration of electron lifetime which means that for DUNE one would only need roughly five readout events in order to make a single measurement. 
However, to allow for the electron lifetime to spatially vary throughout the entire 10~kt module, it may be necessary to collect much more data in order to obtain a precise electron lifetime measurement throughout the detector.  
Studies of data rates and alternative methods for recording special $^{39}$Ar calibration data are currently in progress.



\begin{dunefigure}[Impact of different detector effects on the reconstructed \Ar39 $\beta$ decay energy spectrum]{fig:ar39}{Illustration of the impact of different detector effects on the reconstructed \Ar39 beta decay electron energy spectrum for decays observed in the SP DUNE far detector.  On the left are examples of the reconstructed energy spectrum for various different electron lifetimes, as well as the nominal 
\Ar39 beta decay spectrum (corresponding to an infinite electron lifetime).  On the right are examples of the reconstructed energy spectrum when the true recombination model is different from the one assumed in energy reconstruction (varying the $\alpha$ parameter of the modified Box model, $\mathcal{R} = \ln(\alpha + \xi)/\xi$, where $\xi = \beta\frac{dE}{dx}/{\rho}E_{\mathrm{drift}}$ and with fixed $\beta = 0.212$) and the electron lifetime is infinite.  All curves have been normalized to have the same maximal value.}
\includegraphics[width=.49\textwidth]{graphics/Ar39_energyPlot_DUNESPFD_lifetime.pdf}
\includegraphics[width=.49\textwidth]{graphics/Ar39_energyPlot_DUNESPFD_recomb.pdf}
\end{dunefigure}
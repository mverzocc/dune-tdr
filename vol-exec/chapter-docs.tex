%%%%%%%%%%%%%%%%%%%%%%%%%%%%%%%
\chapter{The LBNF and DUNE Design Reports, July 2019}

\begin{comment}
The international \dword{dune} experiment is designing a world-class, dual-site  long-baseline neutrino observatory and nucleon decay experiment to answer fundamental questions about the nature of elementary particles and their role in the universe. The \dword{lbnf} project is constructing the facilities and infrastructure at \dword{fnal} (the near site) and \dword{surf}, in Lead, SD, USA (the far site), to house and support the experiment. The \dword{lbnf}/\dword{dune} program is hosted at \dword{fnal}, in Batavia, IL, USA.
\end{comment}

This document serves as a brief introduction to the overall, international \dword{lbnf}/\dword{dune} program. It is considered neither a volume of the \dword{dune} \dword{tdr} \fixme{ref} nor a \dword{lbnf} design report, per se, but summarizes portions of  these documents. 
\begin{itemize}
    \item Chapter~\ref{ch:project-overview} introduces \dword{lbnf} project and the \dword{dune} project and experiment,
    \item Chapter~\ref{v1ch:science} summarizes the goals of the  \dword{dune} science program,
    \item Chapter~\ref{ch:lbnf-tech-designs} briefly describes the \dword{lbnf} facilities and infrastructure, and 
    \item Chapter~\ref{ch:dune-det-tech-ov} briefly describes the \dword{dune} detector technologies, 
    \item Chapter~\ref{ch:dunelbnf-org-mgmt} introduces the management and organizational structures.
\end{itemize}


As of July 2019, the \dword{lbnf} design reports are still \dwords{cdr}, with the exception of the  beamline \dword{pdr},  which has been drafted but is not in final form \cite{lbnfdesignrptslinks}.


The \dword{dune} \dword{tdr} will be finalized at the end of July 2019. It describes in detail the experiment's proposed physics program, the 
technical designs of the two far detector \dword{lartpc} technologies, and the technical coordination required to construct and commission the first two far \dwords{detmodule}. The designs have been prototyped with the two \dword{protodune} detectors at \dword{cern}, and these designs, while largely completed, are undergoing adjustments following lessons learned. Production of detector components is being planned. 
The \dword{dune} \dword{tdr} therefore presents a final technical design for most elements and the few remaining alternative or enhanced designs that are under consideration. The \dword{nd} \dword{cdr} is planned for 2020.

The \dword{dune} \dword{tdr} is composed of five volumes, as follows:

\begin{itemize}
\item Volume~\volnumberexec{} provides the executive summary of the overall  experimental program.
\item Volume~\volnumberphysics{} outlines the scientific objectives and describes the physics studies that the \dword{dune} collaboration will undertake to address them and the methods to be used.
\item Volume~\volnumbersp{} describes the \dword{sp} \dword{fd} technology, the subsystems and components that will comprise the first (and any subsequent) \dword{sp} \dwords{detmodule}, and the installation plan for the first \dword{fd} module, which will be of this type. 
\item Volume~\volnumberdp{} describes the \dword{dp} \dword{fd} technology, the subsystems and components that will comprise the first (and any subsequent) \dword{dp} \dwords{detmodule}, and the installation plan for the first \dword{dp} module. 
\item Volume~\volnumbertc{} describes the organizational structures,  methodologies, procedures, requirements, risks and other technical  coordination aspects of constructing the first two \dword{fd} modules in South Dakota.
\end{itemize}

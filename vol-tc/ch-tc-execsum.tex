\chapter{Executive Summary}

\label{vl:tc-execsum}




This volume describes how the activities required to design,
construct, fabricate, install, and commission the \dword{dune}
\dword{fd} modules are organized and managed. The \dword{fd} modules
are hosted at the \dword{lbnf} site at the \dword{surf}. The
\dword{dune} \dword{fd} construction project is one piece of the
global \dword{lbnf-dune}, which encompasses all of the facilities,
supporting infrastructure, and detector elements required to carry out
the \dword{dune} science program at \dword{surf}.
      
The \dword{dune} collaboration has  responsibility for the design 
and construction of the \dword{dune} detectors.  Groups of collaboration 
institutions, referred to as ``consortia,'' assume responsibility for 
the different detector subsystems.  The activities of the consortia are 
overseen and coordinated through the \dword{dune} \dword{tc} organization 
headed by the \dword{dune} \dword{tcoord}.  The \dword{tc} organization 
provides project support functions such as safety coordination, 
engineering integration, change control, document management, scheduling, 
risk management, and technical review planning.  \dword{dune} \dword{tc} 
manages internal, subsystem-to-subsystem interfaces and is responsible 
for ensuring the proper integration of the different subsystems.   

A \dword{jpo} establishes the global engineering
and documentation requirements adhered to within the \dword{dune} 
\dword{fd} construction project, manages external \dword{dune} detector 
interfaces with \dword{lbnf}, and is responsible for ensuring proper 
integration of the \dword{dune} detector elements within the facilities 
and supporting infrastructure.
\dword{dune} \dword{tc} works closely with the support teams of its 
\dword{lbnf-dune} partners within the framework of the \dword{jpo} to 
ensure coherence in project support functions across the entire global 
enterprise.  To ensure consistency of the \dword{dune} \dword{esh} 
and \dword{qa} programs with those across \dword{lbnf-dune}, the 
\dword{lbnf-dune} \dword{esh} and \dword{qa} managers, who sit within 
the \dword{jpo}, are embedded within the \dword{dune} \dword{tc} 
organization.  

The \dword{lbnf}/\dword{dune} \dword{integoff} under the 
direction of the \dword{ipd} incorporates the onsite team responsible 
for coordinating integration and installation activities at \dword{surf}.
Detector integration and installation activities are supported by the
\dword{dune} consortia, which maintain responsibility for ensuring
proper installation and commissioning of their subsystems.  External
\dword{dune} interfaces with the onsite integration and installation
activites are managed through the \dword{jpo}. Support services are
provided by the \dword{sdsd} and \dword{surf}.

The ordering of the subsequent chapters is chosen to provide first,  
additional detail regarding the organizational structures summarized 
here; second, overviews of the facilities, supporting infrastructure, 
and detectors for context; and third, information on project-related 
functions and methodologies used by \dword{dune} \dword{tcoord} 
focusing on the areas of integration engineering, technical reviews, 
\dword{qa}, and safety oversight.  Due to their more advanced stage 
of development, functional examples presented here focus primarily on 
the \dword{sp} \dword{detmodule}.

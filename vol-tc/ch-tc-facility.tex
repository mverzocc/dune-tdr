\chapter{Facility Description}
\label{vl:tc-facility}

The \dword{dune} detectors are located in the main underground campus
at \dword{surf}. The main campus is located at the 4850 level (4850L) between
the Ross and Yates Shafts. This campus and associated surface
facilities are being developed by \dword{lbnf} through excavation
(EXC) and outfitting (\dword{bsi}) contracts with the civil
contractor. Once the contractor has delivered and \dword{aup} has
concluded, cryostat construction will commence. Further infrastructure
will be delivered by \dword{sdsd}.  The following sections describe
the facilites as they are related to the \dword{dune} detectors.

\section{Underground Facilities and Infrastructure}
\label{sec:fdsp-coord-uderground-excavation}

The \dword{dune} underground campus at the \dword{surf} 4850L is shown in
Figure~\ref{fig:dune-underground}.
\begin{dunefigure}[Underground campus]{fig:dune-underground}
  {Underground campus at the 4850L.}
  \includegraphics[width=0.75\textwidth]{underground_campus_vertical}
\end{dunefigure}
\dword{lbnf} will provide facilities and services, on the surface and
underground, to support the \dword{dune} detectors.  This includes
logistical, cryogenic, electrical, mechanical, cyber, and environmental
facilities and services.  All of these facilities are provided for the
safe and productive operation of the \dwords{detmodule}.

The primary path for both personnel and material access to the
underground excavations is through the Ross Shaft. On the surface, the
Ross Shaft and Ross Headframe are undergoing major upgrades. The
structure of the headframe that supports the conveyances is being
reinforced and renovated to make the work flow more efficient and
safer.  All of the wood timber from the 100+ year old shaft has been
removed and replaced with steel sets to improve the time required to
traverse the 4850L to the underground spaces.  The brakes, drive,
clutches, and controls for the conveyance are being completely
overhauled and updated.  In addition to all of these improvements, a
new cage is being designed and fabricated for the Ross Shaft.  The new
cage incorporates features to allow the transport of larger items and
includes rigging points underneath for slung loads without the removal
of the cage.  Details on the Ross Cage design and size constraints for
material and slung loads can be found in~\citedocdb{3582}.

On the surface, a new compressor building is being constructed
adjacent to the headframe.  This building will house the cryogenics
systems for receiving cryogenic fluids and preparing them for delivery
down the Ross Shaft.  New piping is being installed down the Ross
shaft compartment to transport \dword{gar} and N$_2$ underground.

New transformers are being installed in the Ross substation on the
surface to support the underground power needs.  New power cables are
being installed down the Ross Shaft to transmit the power underground
to a new substation.

A portion of the Ross Dry basement is being refurbished to house the
surface cyber infrastructure (\dword{mcr}) required for data and other
underground information.  Redundant fiber optic cables will leave the
\dword{mcr} to travel down both the Ross and Yates shafts to newly
excavated underground communication distribution room (CDR), which is
located near the west entrance drift of the north cavern.  From the
CDR, the fibers branch out to the \dword{cuc} and detector caverns to
support detector data, cryogenic and detector safety systems and will
be tied into the \dword{bms}.  The \dword{bms} controls the facilities
\dword{fls}.  All \dword{fls} signals from the detector or cryogenic
safety systems will be tied into the facilities \dword{bms} for
communications to personnel underground and on the surface.

\section{Detector Caverns}
\label{sec:fdsp-coord-faci-caverns}


Underground spaces are being excavated to support the four
\dword{dune} detectors and infrastructure.  Two large detector caverns
are being excavated.  Each of these caverns will support two
\larmass cryostats.  These caverns, labeled north and
south, are \SI{144.5}{\meter} long, \SI{19.8}{\meter} wide,  and 
%\SI{27.95}{\meter} 
\SI{28.0}{\meter} high. The tops of the cryostats are approximately
aligned with the 4850L of \dword{surf} with the cryostats resting
at the 4910L.  A \SI{12}{\meter} space between the cryostats will
be used as part of the detector installation process, placement of
cryogenic pumps and valves, and for access to the 4910L.  The
\dword{cuc}, between the north and south caverns, is \SI{190}{\meter}
long, \SI{19.3}{\meter} wide, and \SI{10.95}{\meter} high.  The
\dword{cuc} will house infrastructure items, such as cryogenic
equipment, nitrogen dewars and compressors, data acquisition racks and
electronics, chillers for the underground cooling, and electrical
services to support the underground space and detectors.  These three
main caverns will be connected via a series of drifts at the 4850L as shown in
Figure~\ref{fig:dune-underground}. Access tunnels lie at the east and
west ends of each cavern, the north and south sides of the north cavern, and
the north side of the south cavern. Additionally, the cross section of the
existing drifts where experimental components will be transported is
being increased to allow passage.  Other ancillary spaces being
provided underground include a maintenance shop, electrical
substation, concrete supply chamber, compressor room, and  a spray chamber
to house the cooling tower for heat rejection to the mine air for
exhaust through a new 1200 foot$\times$12 foot diameter bore hole up
to the 3650L.  This bore hole will provide additional ventilation
to support excavation.

The first detector will be built in the east side of the north cavern and
the second detector will be built in the east side of the south cavern, as
shown in Figure~\ref{fig:dune-full_assmebly}.
\begin{dunefigure}[\dshort{dune} cryostat]{fig:dune-full_assmebly}
  {Placement of detectors in the north and south caverns}
  \includegraphics[width=0.85\textwidth]{LBNF_full_assembly}
\end{dunefigure}
The primary reasons for constructing the first detectors in the east
sides of two caverns are:
\begin{itemize}
\item {\bf Personnel safety during construction and filling}: It is
  planned that the first detector will be in the filling phase while the second detector
  is in the construction phase. Therefore, construction of the the
  second detector has to take place in a separate cavern from
  filling of the first detector. In addition, the airflow in both
  caverns is from west to east; therefore, construction
  personnel who are primarily on the west side of the detectors will
  be upstream of the cryogens.
\item{\bf Access during construction}: The main access for bringing
  large items into both caverns is from the west side. It is
  advantageous for construction of the second detector in each cavern
  that items that enter from the west are directly lowered and do not
  have to pass over the detector construction site.
\end{itemize}
Both reasons drive the plan for detectors one and two to be built on
the east side of the two separate caverns.  It is therefore planned
that both caverns are ready for \dword{aup} at the start of respective
detector construction phases.  It should be noted that both detectors
will be built at the 4910L. However, with the exception of the
\dword{lar} circulation pumps, all services are at the 4850L to
facilitate access during the operations phase.

To support the electrical requirements new 35~MVA power feeds will be
routed from the surface substation down the Ross Shaft to the
underground electrical substation and from the substation to the
electrical room in the \dword{cuc}.  The electrical room will have
multiple transformers from which to distribute power for various
functions such as lighting, HVAC, cryogenics equipment, and detector  
power.  Each detector cavern and the \dword{daq} room will have
dedicated feeds from the electrical room.  Details of this are
described in the electrical section.  There are multiple electrical
panels planned in each cavern to provide power for the different
phases of construction.  These phases include the installation
of the cryostat warm structure and cold membrane, detector
installation and detector operations. During detector operations, 
separate panels are available for detector (clean) and building
(dirty) power as is discussed further in
Section~\ref{sec:fdsp-coord-faci-power}.

To support underground cooling requirements, four 400-ton chillers
will be located in the \dword{cuc} adjacent to the electrical room and
cryogenics equipment.  These chillers are designed to provide
\SI{400}{\kilo\watt} of cooling for the various cryogenics systems,
\SI{500}{\kilo\watt} of cooling for the electronics on each of the
detectors and \SI{750}{\kilo\watt} of cooling to the \dword{daq} room
in the \dword{cuc}.  These chillers will also provide chilled water to
the HVAC systems underground to maintain the ambient temperature and
humidity.

Fire protection in the underground spaces will be determined by zone,
hazard type, and requirements.  Details of this can be found on ARUP
drawing U1-FD-E-651 in the Underground Electrical package, which is shown
in Figure~\ref{fig:lbnf-fire}.
\begin{dunefigure}[Underground fire zones]{fig:lbnf-fire}
  {Underground fire zones}
 \includegraphics[width=0.85\textwidth]{lbnf_fire}
\end{dunefigure}
The drifts will be outfitted with normal wet type sprinklers and
broken into three zones.  The north and south caverns will be outfitted
with a pre-action type sprinkler system to protect sensitive detector
electronics from water damage from accidental release.  A pre-action
system requires two signals to activate.  These are the detection of
smoke by one of the sensors and the fusing of the sprinkler
head.  This type of system was the most economical choice to reduce the
risk of unnecessary discharge of water over the detector electronics.
The DDSS will interface with the BMS fire system to turn off power
to the racks before water is introduced and reduce the impact of
the water on the system and reduce the risk of electrical shock.  A
pre-action system will also be installed in the CUC over the cryogenic
equipment as shown in the figure.  In the electrical substation,
electrical room, \dword{daq} room, and CDR, a clean-agent type system will be
employed.  Clean-agent systems use either inert gas or chemical agents
to extinguish a fire and are typically used in areas that contain
sensitive electronics or data/power centers.

In addition to the above services, systems are being installed to
provide compressed air, industrial water, internet, and a configurable
security access system.

\section{Cryostat}
\label{sec:fdsp-coord-cryostat}

Each detector will be housed inside a cryostat designed to hold the
liquid argon (\dword{lar}), cryogenic piping, and the detector as shown in
Figure~\ref{fig:dune-cryostat}.
\begin{dunefigure}[\dshort{dune} cryostat]{fig:dune-cryostat}
  {Overall construction and dimensions of the \dword{dune} cryostat.}
  \includegraphics[width=0.85\textwidth]{cryostat}
\end{dunefigure}
Each cryostat consists of a warm structure that provides structural
support, insulating layers that maintain the temperature of the fluid
inside, and the inner membrane that provides a compliant inner surface
that helps to maintain the \dword{lar} purity and serves as primary
containment.

The external warm structure measures \SI{65.84}{\meter} long,
\SI{18.94}{\meter} wide, and \SI{17.84}{\meter} high, and its internal dimensions are 
\SI{63.6}{\meter} long, \SI{16.7}{\meter} wide, and \SI{15.6}{\meter}
high.  It is a welded and bolted structure constructed of
I-beam elements and a \SI{12}{\mm} inner steel plate reinforced with
ribs.  It is designed to support the
weight of the structure itself, 
insulation, inner membrane, liquid, detector, and any equipment placed
on the structure.  It also supports the hydrostatic pressure of the
fluid inside and resists the small gas overpressure in the ullage.
The structure is positioned on the concrete slab provided at the 4910L
of the north and south caverns.  The \SI{12}{\mm} inner steel plate of the
warm structure also serves as the tertiary layer of containment.

The insulation layer and inner membrane will be installed inside the
warm structure.  The technology for this comes from the shipping
vessels used for transporting liquid natural gas. The insulating
layers are made from prefabricated panels of reinforced polyurethane
foam.  There are two layers of the foam panels that have a flexible
membrane in between that serves as a secondary containment membrane.
The two foam layers are installed in an overlapping pattern to reduce
heat introduced into the fluid.  The total thickness of the two foam
layers is \SI{0.8}{\meter} with a density of approximately
\SI{90}{mg/cm$^3$} and a residual heat input of
\SI{6.3}{W/\meter$^2$}.  A stainless steel corrugated membrane will be
installed inside the insulating layers to create the primary
containment membrane for the \dword{lar}.  This is constructed from \SI{1.2}{\mm}
thick stainless steel panels that overlap and are welded together.
The entire inner surface of the cryostat will be tiled with these
panels to create the inner surface.  These panels need to be
corrugated to accommodate the shrinkage of the stainless steel from
ambient temperature to \dword{lar} temperature.  With the addition of the
insulation and inner membrane, the internal dimensions of the finished
vessel are \SI{62}{\meter} long, \SI{15.1}{\meter} wide, and
\SI{14}{\meter} high.

The top of the cryostat will have penetrations provided based on
drawings developed by LBNF and \dword{dune} to accommodate detector support,
electronic and data cables, cryogenic pipes, and connections and other
devices.

Each cryostat will have a vertical \dword{tco} on one short end.
These openings will be approximately \SI{13.43}{\meter} high and
\SI{2.68}{\meter} wide and will be used to move the detector elements
into the vessel during installation.  Once most of the detector
material is inside the cryostat, the \dword{tco} will be closed and
leak tested.  After the \dword{tco} closing, the last of the detector
installation will be completed. After final detector installation and
equipment removal through the roof openings, the cryostat will be
closed for purge, \cooldown{}, and filling.

At the \dword{tco} end of the cryostat, there will be four penetrations for
the cryogenic fluid pumps to be connected.  A normally closed valve
will be installed at each penetration to prevent any loss of fluid
with pump maintenance or damage.

\section{Cryogenics}
\label{sec:fdsp-coord-cryogenics}


The detector cryogenics system supplies \dword{lar} and provides
circulation, re-condensation, and purification. The cryogenic system
components are housed inside the \dword{cuc}, on top of each detector module
and between the two detector modules in each cavern. The cryogenics system comprises 
\begin{itemize}
\item {\bf Infrastructure Cryogenics}: This includes \dword{lar} and \dword{ln} receiving
  facilities on the surface, nitrogen refrigeration systems (both
  above ground and underground), \dword{ln} buffer storage
  underground, piping to interconnect equipment (\dword{ln}, GN$_2$, and \dword{gar}),
  components in the detector cavern, and the \dword{cuc} and process control/support
  equipment.
\item {\bf Proximity cryogenics}: This includes reliquefaction 
  and purification subsystems for the argon (both gas and liquid), associated
  instrumentation and monitoring equipment and \dword{lar} piping to
  interconnect equipment and components in the detector cavern and the
  \dword{cuc}. The proximity cryogenics are split into three areas: in the
  \dword{cuc}; on top of the mezzanine, as shown in Figure~\ref{fig:detector_mezzanines};
  and on the side of the cryostat where \dword{lar} circulation pumps are installed.
\item {\bf Internal cryogenics}: This includes \dword{lar} and \dword{gar} distribution
  systems inside the cryostat, as well as features to cool the
  cryostat and the detector uniformly.
\end{itemize}

Figure~\ref{fig:dune-cryogenics} shows the process flow diagram of the
\dword{lbnf} cryogenic system. For convenience, only one cryostat is shown. The three main areas are 
\begin{enumerate}
  \item Surface (on the left), with the receiving facilities and the
    recycle compressors of the nitrogen system.  All these items are
    part of the infrastructure cryogenics.
  \item
  \dword{cuc} (in the center), with the \dword{lar}/\dword{gar}
    purification and regeneration systems (part of the proximity
    cryogenics) and the \coldbox{}es, expanders, and \dword{ln} storage (part of
    the infrastructure cryogenics).
\item Detector cavern (on the right), with the argon condensing system
  and the distribution of argon from the purification system in the
  \dword{cuc} to the cryostat and vice versa.
\end{enumerate}
Argon and nitrogen are received and stored on the surface in the
liquid phase.  They are vaporized and transferred underground in the
gas phase, the nitrogen as part of the nitrogen system, the argon separately.
\begin{dunefigure}[Cryogenics system]{fig:dune-cryogenics}
  {Overall process flow diagram of the cryogenic system showing one
    cryostat only; other cryostats are the same.}
  \includegraphics[width=0.85\textwidth]{LBNF_PFD_180909}
\end{dunefigure}


The cryogenics system fulfills the following modes of operations:
\begin{itemize}
  \item {\bf Gaseous argon purge}: Initially, each cryostat is filled
    with air, which must be removed by means of a slow \dword{gar} piston
    purge.  A slow flow of argon is introduced from the bottom and
    displaces the air by pushing it to the top of the cryostat where
    it is vented.  Once the impurities, primarily nitrogen, oxygen, and 
    water, drop below the parts per million (ppm) level, the argon
    exhausted at the top of the cryostat is circulated in closed loop
    through the gas purification system and re-injected at the
    bottom. Once the contaminants drop below the ppm level, the
    cool-down can commence. The \dword{gar} for the purge comes directly from
    above ground, passes through the \dword{gar} purification and then it is
    injected at the bottom of the cryostat by means of a \dword{gar}
    distribution system.
  \item {\bf Cryostat and detector cool-down}: The detector elements
    must be cooled in a controlled and uniform manner. Purified \dword{lar}
    flows into sprayers that atomize it. A second set of sprayers
    flowing purified \dword{gar} moves the mist of argon around to achieve a
    uniform cooling. The cooling power required to recondense argon in
    the condensers outside the cryostat is supplied by the
    vaporization of nitrogen from the nitrogen system. Once the
    detector elements reach about 90 K, the filling can commence. The
    \dword{gar} for the cool-down comes directly from above ground, passes
    through the \dword{gar} purification,  then is condensed in the
    condensers before being injected into the sprayers. The \dword{gar} moving
    the mist of argon around only goes through the \dword{gar} purification
    and not the condensers.
  \item {\bf Cryostat filling}:Argon is vaporized and transferred
    underground as a gas from the receiving facilities on the surface.
    It first flows through the \dword{gar} purification system and is
    recondensed in the argon condensers by means of vaporization of
    \dword{ln}.  It then flows through the \dword{lar} purification and is introduced
    in the cryostat. The filling of each cryostat varies in duration,
    from 8 to 15 months, depending on the available cooling power at
    each stage. With the full refrigeration system available, the filling of the fourth
cryostat  will take approximately 15 months. 
  \item{\bf Steady state operations}: The \dword{lar} contained inside each
    cryostat is continuously purified through the \dword{lar} purification
    system using the main external \dword{lar} circulation pumps. The boil-off
    \dword{gar} is recondensed in the argon condensers and purified as liquid
    in the same \dword{lar} purification system as the bulk of the \dword{lar}.
  \item{\bf Cryostat emptying}: At the conclusion of the experiment,
    each cryostat is emptied and the \dword{lar} is removed from the system.
\end{itemize}
Each detector has its own stand-alone process controls system, which is
redundant and independent of the others. It resides locally in each
cavern.  PLC racks are located on the mezzanine, in the pit over the
protective structure of the main \dword{lar} circulation pump, in the \dword{cuc} and
on the surface. %A workspace and a desk with two stations are available
%on top of the mezzanine and in the CUC respectively for installation
%and commissioning.
A workspace on the mezzanine and a desk with two stations in the \dword{cuc}  are available
during installation and commissioning.

Before argon is offloaded from each truck into the receiving tanks, a
sample of the \dword{lar} is analyzed locally to ensure compliance with the
requirements. If the specifications are met, the truck driver is given permission
 to offload the truck. The process is automated to reduce
human error. The purity is measured before and after the purification
system by custom-made purity monitors to verify correct
functioning of the system.


\section{Detector and Cavern Integration}
\label{sec:fdsp-coord-det-cav-integ}


Figure~\ref{fig:detector_ew_elevation} shows the north
elevation view of the detector in the cavern. The services from the
\dword{cuc} enter the cavern through a passage visible on the left.
\begin{dunefigure}[North elevation view of detector]{fig:detector_ew_elevation}
  {North elevation view of one detector module in the cavern.}
  \includegraphics[width=0.75\textwidth]{LBNF-Cryostat-South_Elevation_in_Cavern_c}
\end{dunefigure}

Figure~\ref{fig:dune-cryostat} shows one detector module in the
cavern. In this figure, the cryogenics equipment and racks on top of
the detector are visible. The \dword{lar} recirculation pumps can also be seen
on the lower level.
Figure~\ref{fig:detector_ns_elevation} shows the west
elevation view of the detector in the cavern. The services entering
from the \dword{cuc} are visible on the right.
\begin{dunefigure}[West elevation view of detector]{fig:detector_ns_elevation}
  {West elevation view of one detector in the cavern with overall dimensions.}
  \includegraphics[width=0.9\textwidth]{LBNF-Cryostat-West_Elevation_in_Cavern_c_with_Dimensions}
\end{dunefigure}
Figure~\ref{fig:detector_mezzanines} shows the elevation view of the
top of cryostat showing mezzanines, cryogenic equipment, and
electronic racks.
The cryogenics are installed on a mezzanine supported from
the cavern roof and cavern wall. Cryogenic distribution lines are
routed under the mezzanine. Local control rooms for the
cryogenic equipment are on the mezzanine.
\begin{dunefigure}[Elevation view of top of cryostat]{fig:detector_mezzanines}
  {Elevation view of top of cryostat showing mezzanines, cryogenics
    equipment, and electronic racks.}
  \includegraphics[width=0.85\textwidth]{LBNF-Cryostat-West_Elevation-Cryo_c_with_Dimensions}
\end{dunefigure}

Detector electronics are installed in short racks close to
feedthroughs and in taller racks installed on a separate electronics
mezzanine shown on the left of Figure~\ref{fig:detector_mezzanines}.
This will allow easy access for maintenance and reduce complexity on
top of the detector.

\section{Detector Grounding}
\label{sec:fdsp-coord-faci-grounding}


The grounding strategy provides the detectors with independent
isolated grounds to minimize any environmental electrical noise that
could couple into detector readout electronics either conductively or
through emitted electromagnetic interference.

The detectors will be placed at the 4910L of \surf. The
electrical characteristics of the various rock masses are unknown, but
should have extremely poor and inconsistent conductive
properties. Ensuring adequate sensitivity of the detectors requires a
special ground system that will isolate the detectors from all other
electrical systems and equipment, minimize the influence of inductive
and capacitive coupling, and eliminate ground loops. The grounding
infrastructure should reduce or eliminate ground currents through the
detector that would affect detector sensitivity, maintain a low
impedance current path for equipment short circuit and ground fault
currents, and ensure personnel safety by limiting any potential for
equipment-to-equipment and equipment-to-ground contact.

The infrastructure grounding plan of the underground facilities is
fully described elsewhere~\cite{bib:cernedms2095975}. %EDMS-2095975 
We have planned a separate detector ground, isolated from the rest of the facility, for each of the four \dwords{detmodule}.    
The detector ground will primarily 
comprise the steel containment vessel, cryostat membrane, and
connected readout electronics.  The facility ground is constructed out
of two interconnected grounding structures; these are the cavern
ground and the \dword{ufer} grounds which are described below.  For safety
reasons, a saturable inductor will connect the detector ground to the
facility ground.

Figure~\ref{fig:dune-grounding} shows the areas of construction for
the cavern and \dword{ufer} grounds.
\begin{dunefigure}[Overall \dshort{dune} grounding structure]{fig:dune-grounding}
  {Overall \dshort{dune} facility grounding structure incorporated in cavern.}
  \includegraphics[width=0.85\textwidth]{SURF_Grounding}
\end{dunefigure}
Grounding structure definitions include 
\begin{enumerate}
 \item Cavern ground consisting of overlapping welded wire mesh
   supported by rock bolts and covered with shotcrete. The
   \dword{lbnf}/\dword{dune} cavern ground includes all walls and
   crown areas above the 4850L in the north and south detector
   caverns and their associated central access drifts, as well as tin-plated
   copper bus bars that run the length of the detector vessels
   on each side along the cavern walls and are mounted external to the
   shotcrete.  The cavern ground structure
\begin{enumerate}
 \item spans the full length of the cavern from the west end access
   drift entrance through the mid-chamber to the east end access drift
   entrance;
 \item spans the full width of the cavern from the 4850L sill
   (top of the detector vessels and mid-chamber floor) on both sides
   up and across the crown of the cavern;
 \item includes mid chamber walls to the 4910L; and 
 \item includes the east and west end walls of the cavern, from the
   4850L to the crown.
\end{enumerate}
 \item \dword{ufer} ground consisting of metal rebar embedded in
   concrete floors. The \dword{lbnf}/\dword{dune} \dword{ufer} ground
   system includes the concrete floors in the cavern mid-chambers,
   center access drifts, and \dword{cuc}. The cavern and
   \dword{ufer} grounds will be well bonded electrically to construct
   a single facility ground isolated from detector ground.
 \item Detector ground consisting of the steel containment vessel
   enclosing the cryostat and all metal structures attached to or
   supported by the detector vessel.
\end{enumerate}


To ensure safety, a safety ground with one or more saturable inductors
will be installed between the detector ground and the electrically
bonded \dword{ufer} and cavern grounds that form the facility ground.
Figure~\ref{fig:dune-grounding_figure} illustrates the use of the
safety ground. The safety ground inductors saturate with flux under
low-frequency high currents, presenting minimal impedance to these
currents.  Thus, an AC power fault current would be shunted to the
facility ground and provide a safe grounding design. At higher
frequencies and lower currents, such as coupled noise currents, the
inductor provides high impedance, restricting current flow
between grounded metal structures. The desired total impedance between
the detector ground structure and the cavern/\dword{ufer} ground
structure should be a minimum of \SI{10}{Ohms} at \SI{10}{MHz}.
\begin{dunefigure}[Simplified detector grounding]{fig:dune-grounding_figure}
  {Simplified detector grounding scheme.}
  \includegraphics[width=0.5\textwidth]{Simplified_Grounding_Figure2}
\end{dunefigure}

As stated above, the detector ground exists only in the area of the
steel containment vessel enclosing the cryostat and all metal
structures attached to or supported by the detector vessel.  All
signal cables that run between the detector and the \dword{daq} underground
processing room in the \dword{cuc} will be fiber optic.
All connections to the cryogenics plant on the facility
ground will be isolated from the cryostat with dielectric breaks.  A
conceptual drawing showing the isolation of the cryostat is presented
in Figure~\ref{fig:dune-grounding_scheme}.
\begin{dunefigure}[Detector grounding schematic]{fig:dune-grounding_scheme}
  {Schematic of detector grounding system.}
  \includegraphics[width=0.99\textheight,angle=90]{DUNE_FD_GNDing_diagram_revH1}
\end{dunefigure}

The construction of the facility ground provides a low impedance path
for return currents of the facility services, such as cryogenic pumps,
and noise coupling from facility services will be greatly reduced or
eliminated.  The experiment has also been carefully designed such that
facility return currents will not flow under the cryostats.

The cryostat itself is treated as a Faraday cage.  Any connections
coming from facilities outside of actual detector electronics are
electrically isolated from the cryostat.  For detector electronics,
specific rules for signal and power cables penetrating the cryostat
exist~\cite{bib:cernedms2095958}. % EDMS-2095958





\section{Detector Power}
\label{sec:fdsp-coord-faci-power}

%The \dword{dune} detectors are %constrained in 
%limited by both size and power
%consumption.  
After requirements were given to \dword{fscf}, 
limits were established on the size of the cavern excavation, cooling
capabilities, and electronics power consumption.   
The  \dword{dune} \dwords{detmodule} must stay within these limits. Of the available 360~kW per module, the \dword{spmod} (\dword{dpmod}) will only use
216(253)~kW,  %implying 
leaving a margin of 40\%(30\%).

The \dword{fscf} will supply a 1000~kVA transformer for
each cavern.  Each cavern will host two \dword{dune}  \dwords{detmodule}.  Power from
this initial transformer will be de-rated with no more than 75\%
of total power available at the electrical distribution panels.  We
plan for a maximum consumption of 360~kW per  \dword{detmodule}.

Figure~\ref{fig:power} summarizes estimated detector loads of the 
\dword{dune} electronics located in the detector caverns.  The \dword{daq} power is 
described at the end of this section. 
\begin{dunefigure}[\dshort{dune} estimated power consumption.]
{fig:power}
{\dword{dune} estimated power consumption.}
  \includegraphics[width=0.8\textwidth]{DUNE_PowerTable_v1}
\end{dunefigure}


The TPC \dword{ce} readout dissipates an estimated maximum of 360~W
per \dword{apa}. An additional 20~W will be consumed due to power loss in the
cables.  The low-voltage power supplies have a controller that adds
approximately 35~W per \dword{apa}, and supplies have an efficiency of
approximately 85\%. This adds up to about 488~W per \dword{apa}, or a
total load of 73~kW per detector module. The \dword{apa} wire-bias power
supplies have a maximum load of 465~W per set of six \dword{apa}s, for a total
budget of around 12~kW. Cooling fans and heaters near the feedthroughs
will use a %nominal 
small amount of power, so the overall power budget for
\dword{ce} is expected to be less than 90~kW.

The \dword{pds} electronics is based on the \dword{mu2e}
    (\dword{daphne}) electronics, from which we estimate a total power
    budget of approximately 6~kW. \dword{dune} plans a slightly higher
    power budget of 8~kW to account for cable and power supply
    inefficiencies.  The \dword{pds} electronics presents a significantly
    lower power load than the \dword{ssp} alternative solution used in 
    \dword{pdsp}, which requires a power budget of approximately 72~kW per 
    detector module.


Each of the approximately 80 detector racks will have fan units,
Ethernet switches, rack protection, and slow controls modules, adding
a load of about 500~W per rack, bringing the total to 40~kW.

Twenty-five racks are reserved for cryogenics instrumentation with a
per-rack load conservatively estimated at 2~kW, for a total of 50~kW.

The SP detector module will thus use an estimated 216~kW of power. The
higher-load SSP alternative for the PDS would increase this to
280~kW. This higher estimate represents approximately 60--78\% of
our available power.

The \dword{dp} electronics estimate is
approximately 253~kW and also fits well within the planned maximum of
360~kW.

For the \dword{spmod}, the power will be largely distributed to a number
of detector racks that will sit on a detector rack mezzanine above
the cryostat.  Each of the racks will receive a \SI{30}{A} \SI{120}{V} 
service, with a maximum of 80 racks.

The other area where \dword{dune} requires power underground is in the \dword{daq}
room.  There the power budget is determined by the available 750~kVA
transformer.  The available power must be de-rated to 80\% at
the electrical distribution panel and another 80\% for equipment
efficiency.  Thus, 480~kW of power will be distributed to a maximum of
60 racks.  Each water-cooled rack will have approximately 8~kW
available for computing power.

A minimal level of UPS power will be provided to the \dword{daq} equipment to
allow for powering down servers.  Cryogenic controls and any
critical safety interlocks will have access to long term UPS back-up.
At this time, no UPS power is being proposed for detector readout
electronics.


\section{Data Fibers}
\label{sec:fdsp-coord-faci-fibers}


The \dword{dune} experiment requires a number of fiber optic pairs to
run between the surface and the 4850L.  A total of 96 fiber
pairs, which accommodates both \dword{dune} and \dword{lbnf} needs, will
be supplied through redundant paths with bundles of 96 pairs coming
down both the Ross and Yates shafts.  The individual fibers are
specified to allow for transmission of 100 Gbps.  A schematic view of
the fiber paths from surface to underground is shown in
Figure~\ref{fig:dune-fiber_path}. From the surface main communications room (MCR, the surface \dword{daq}
room), we will connect to the WAN and ESnet as described in
Section~\ref{sec:fdsp-coord-surf-rooms}, which is being designed by
\dword{fscf} and the \dword{fnal} networking groups.

\begin{dunefigure}[Detector fiber path]{fig:dune-fiber_path}
  {Schematic of detector fiber path.}
  \includegraphics[width=0.95\textwidth]{Fiber_path}
\end{dunefigure}

The redundant fiber cable runs of 96 fiber pairs are received in
existing communications enclosures at the 4850L entrances of the Ross
and Yates shafts.  The two 96 fiber pair bundles are next routed to
the CDR.  The fibers will be received and terminated in an optical
fiber rack. Two additional network racks will be used to route the
fiber data between the surface and underground.

The plan is to use only the Ross Shaft set of 96 fiber pairs with the
Yates set being redundant.  The network switches will allow for
switchover in case of catastrophic failure of fibers in the Ross
shaft. %Catastrophic failure carries a very low risk, 
There is a very low risk of catastrophic failure, but it could occur
if, for instance, rock fell and damaged the fibers.  The Yates path is
viewed as a hot spare. Plans are being formed to periodically test
the redundant Yates path and verify its viability.

From the CDR, fibers designated for \dword{fscf} are routed to provide
general network connections to the detector caverns,  \dword{cuc}, and the underground \dword{daq} room.

A total of 96 fiber pairs are routed to the underground \dword{daq} room for
use by the \dword{dune} experiment and LBNF. The fibers are reserved as
follows:
\begin{itemize}
  \item 15 pairs for \dword{dune} data per detector-total 60 pairs,
\item 1 pair for slow controls per detector-total 4 pairs,
\item 2 pairs reserved for \dword{gps},
\item 6 pairs for \dword{fscf},
\item 4 pairs for \dword{lar} cryogenics,
\item 4 pairs for \dword{ln} cryogenics, and
  \item 16 pairs reserved as spares.
\end{itemize}
The set of reserved fiber pairs total to 80.



\section{Central Utility Cavern Control and \dshort{daq} Rooms}
\label{sec:fdsp-coord-cuc-daq}

The \dword{cuc} contains various cryogenic equipment and the
\dword{daq} and Control Room for the \dword{dune} experiment.  The
cryogenic system and areas are described in
Section~\ref{sec:fdsp-coord-cryogenics}. Both the control and \dword{daq}
rooms are at the west end of the \dword{cuc} (see
Figure~\ref{fig:dune-cuc}).
\begin{dunefigure}[\dshort{daq} and Control Room in CUC]{fig:dune-cuc}
  {Location of underground \dword{daq} and Control Room in the CUC.}
  \includegraphics[width=0.85\textwidth]{Location_Underground_DAQ_Control_Rooms}
\end{dunefigure}
\begin{dunefigure}[\dshort{daq} and Control Room]{fig:dune-daq}
  {Underground \dword{daq} and Control Room layout.}
  \includegraphics[width=0.85\textwidth]{Preliminary_Layout_DAQ_Control_Rooms}
\end{dunefigure}


The control room is an underground space of approximately 18 x 48 feet
and serves multiple purposes.  It provides a meeting or work space
during system commissioning. It provides easy access to \dword{daq}
equipment for debugging and service during commissioning.  During
experimental operations, the \dword{dune} experiment will have a remote
control room located at Fermilab's main campus.

Figure~\ref{fig:dune-daq} shows the layout and suggested outfitting of
the rooms. Additionally, the control room provides the required
workstation for monitoring of Fire and Life Safety and the building
management system.  These are facility services for which \dword{dune} is not
directly responsible, but the experiment will need to
interface with these systems.  One example of this interface would be
the reporting of any smoke detected within a detector rack.

Lastly, the cryogenic team requires a space allocation within the
Control Room of two racks and two work benches for the technicians who
monitor the cryogenic systems. This space is needed only during
commissioning with remote operation to follow. Additional space for
cryogenics commissioning may be available on the mezzanines.
       
The \dword{daq} room is approximately 26 $\times$ 56 square feet and will
contain 52--60 racks that will be used for fiber optic cable
distribution, networking, \dword{dune} \dword{daq} and two or three
racks for conventional facilities.  The current design, shown in
Figure~\ref{fig:dune-DAQ_layout}, shows a total of 60 racks 
possible.
\begin{dunefigure}[\dshort{daq} room layout]{fig:dune-DAQ_layout}
  {Proposed rack layout in \dword{daq} room.}
  \includegraphics[width=0.85\textwidth]{Prop_DAQ_room_layout}
\end{dunefigure}
  

A quantity of 48 racks are reserved exclusively for \dword{daq}.  Two
additional racks are required for optical fiber distribution and
network connection to the surface.

The \dword{fscf} will supply the \dword{daq} room with cooling
water, a \SI{46}{cm} (18~inch) raised floor, lighting, HVAC, dry fire protection
and a dedicated 750~kVA transformer.  
The \dword{integoff} has responsibility for installing the remaining infrastructure which includes water cooled racks, the piping required to distribute water to the racks,  the electrical distribution system required to provide AC power for the racks, and supporting cable trays.


\section{Surface Rooms}
\label{sec:fdsp-coord-surf-rooms}

The \dword{dune} experiment requires space on the surface for a small
number of \dword{daq}, networking, and fiber optic distribution racks.  Space
is also allocated for cryogenics.  The surface cryogen building and
operations is described in 
Section~\ref{sec:fdsp-coord-cryogenics}.

The \dword{daq} consortium requires a surface computer room with eight
racks and a minimum of 50~kVA of power.  \dword{daq} also requires connection
to the optical fibers running to the 4850L via the Ross and Yates
shafts as well as to the Energy Sciences Network (ESnet).

The surface \dword{daq}, networking equipment, and fiber distribution racks
will be placed in a new main communications room (MCR) in the Ross Dry
building.  The MCR is approximately 628 square feet and will be
completed as part of the LBNF project, with seven racks installed for
the conventional facilities and space allocated for eight racks
provided by the experiment.  The seven racks allocated for
conventional facilities will include networking and fiber optic
distribution.  The eight racks allocated to the experiment will
contain computer servers, disk buffer, and some network connections.
Power and cooling will be provided as part of the \dword{lbnf} project.
\begin{dunefigure}[Surface rooms layout]{fig:dune-surface_layout}
  {Ross area surface rooms layout.}
  \includegraphics[width=0.95\textwidth]{Surface_rooms_layout}
\end{dunefigure}

\section{DUNE Detector Safety System}
\label{sec:fdsp-coord-det-safety}


The \dword{ddss} functions to protect experimental equipment.  The
system must detect abnormal and potentially harmful operating
conditions.  It must recognize when conditions are not within the
bounds of normal operating parameters and automatically take
pre-defined protective actions to protect equipment. Protective
actions are hardware or \dword{plc} driven.

\dword{dune} \dword{tc} works with the consortia to identify equipment
hazards and ensure that harmful operating conditions can be detected
and mitigated.  These hazards include smoke detected in racks, leak(s)
detected in water cooling areas, \dword{odh} detection, a drop in the
cryostat \dword{lar} liquid level, laser or radiation hazards in
calibration systems, over or under voltage conditions, and others to
be determined.  Some hazards will be unique to the actual detector
being implemented.

The slow controls system plays an active role and collects, archives,
and displays data from a broad variety of sources and provides
real-time status, alarms, warnings, and hardware interlock status for
the \dword{ddss} and detector operators. Slow controls %provide monitoring of operating 
monitor operating parameters for items such as HV systems, TPC electronics,
and \dword{pd} systems. Data is acquired via network interfaces, and status and
alarm levels will be sent to the \dword{ddss}.  Safety-critical issues that
require a hardware interlock, such as smoke detection or a drop in the
\dword{lar} level, which could cause \dword{hv} damage to components, are monitored by
slow controls; interlock status is provided to the \dword{ddss}.  The
protective action of a safety critical issue is done through hardware
interlocks and does not require the action of an operator, software, or
\dword{plc}.

%It is also noted that t
The \dword{ddss} will provide input to the 4850L
fire alarm system.  The 4850L fire alarm system will provide life
safety and play an integral role in detecting and responding to an
event as well as notifying occupants and emergency responders.  This
system is the responsibility of the host laboratory and \dword{sdsta}.  The
fire alarm system is described in the BSI design~\cite{bib:cernedms2093229}.  % (Support of LBNF \dword{fscf} --- 90\% Fire and Safety Report and can be found in EDMS-2093229~\cite{bib:cernedms2093229}). 
The 4850L fire alarm system is connected to the surface incident command vault in the
second floor of the Yates Administration building. Limits on the
number of occupants and egress paths are discussed. The table of
triggering inputs and Fire Alarm Sequence of Operation is documented
in the set of BSI underground electrical drawings, sheet U1-FD-E-308,
which is also found in~\cite{bib:cernedms2093229}. % EDMS-2093229~\cite{bib:cernedms2093229}. 
The
experiment will be adding a list of initiating inputs, such as smoke
detected in electronic racks or water leaks detected in the
\dword{daq} room to this sheet as designs reach a higher level of
maturity.  Additional information regarding facility fire and life
safety can be found in~\cite{bib:cernedms2156770}. % the BSI ``Support of \dword{lbnf} Far Site Conventional Facilities 100\% Fire and Life Safety Report'' (EDMS-2156770).  
\fixme{Citation added by Anne 11/4. this doc is not public in EDMS}

 
The \dword{ddss} must communicate to the \dword{dune} slow controls
system as well as the 4850L fire alarm system.  The \dword{dune} slow
controls system monitors and records detector
status.  Working together through communication links, the three
systems will (1) monitor the status of the experiment (slow controls),
(2) protect equipment (\dword{ddss}),  and (3) and provide life safety (4850L fire alarm system). Figure~\ref{fig:dune-DDSS} indicates how these systems
interact. %, with a few illustrated examples.
\begin{dunefigure}[Examples of \dshort{dune} detector safety system information flow]{fig:dune-DDSS}
  {Sample \dword{ddss} information flow.}
  \includegraphics[width=0.85\textwidth]{DDSS_Block_Diagram}
\end{dunefigure}


The \dword{ddss} will be implemented through robust sensors feeding
information to redundant \dwords{plc} that activate hardware
interlocks. The selection of the \dword{plc} hardware platform is
still an open decision.  Listed below are some of the general
\dword{dune} experimental conditions that require intervention of the
\dword{ddss}:
\begin{enumerate}
 \item A drop in the \dword{lar} level.  This condition requires a hardware
   interlock on the liquid level.  If the level drops below a
   pre-determined level, the drift \dword{hv}  must automatically be 
   shut off to prevent equipment damage due to \dword{hv} discharge.  Slow controls would be
   alerted through normal monitoring and record the status of the detector.
 \item Smoke or a temperature/humidity increase above normal operating
   levels. This could be detected inside a rack or near an instrumented
   feedthrough.  If any of these conditions are detected, local
   power must be automatically switched off. If smoke is detected, a
   dedicated line will alert 4850L fire alarm system.
 \item A water leak detected near energized equipment in the \dword{daq}
   underground data processing room.  Water leak detectors 
   report to the \dword{ddss} \dword{plc} and a decision will be made to either
   issue an alert or immediately shut down power to the room, depending
   on the detected magnitude of the leak.  This condition would also be reported
   to the 4850L fire alarm system.
\end{enumerate}



%\section{LBNF/SURF Safety System}
%\label{sec:fdsp-coord-surf-safety}


\chapter{Detector Installation and Commissioning Organization}
\label{ch:tc-jpo}


As discussed in Chapter~\ref{vl:tc-global}, the \dword{ipd} has
responsibility for coordinating the planning and execution of 
the \dword{lbnf-dune} installation activities both 
in the underground detector caverns at \dword{surf} and 
nearby surface facilities.  The \dword{dune} consortia maintain 
responsibility for their subsystems over the course of these 
activities and provide the expert personnel and specialized 
equipment necessary to integrate, install, and commission their 
detector components.  Likewise, \dword{lbnf} has responsibility 
for activities associated with the installation 
of supporting infrastructure items, which are coordinated under 
the direction of the \dword{ipd}.       

The \dword{lbnf-dune} \dword{integoff} will evolve over 
time to incorporate the team in South Dakota responsible for the 
overall coordination of onsite installation activities.  In the 
meantime, the installation planning team within the \dword{integoff} works with 
the \dword{dune} consortia and \dword{lbnf} project team members 
to plan these activities.  
%\dword{jpo} installation team members are concurrently embedded within the \dword{dune} \dword{tc} organization during this period to facilitate required interactions with the \dword{dune} consortia. 

The \dword{integoff} installation planning team is responsible for specification 
and procurement of common infrastructure items associated with 
installation of the detectors, which are not included within 
the scope of the \dword{dune} consortia.  Some of these items 
are detector pieces such as racks, cable trays, cryostat flanges 
and mechanical structures for supporting the detectors within 
the cryostats.  Others are general items required for detector 
installation such as clean rooms, cranes, scaffolding and 
personnel lifts.
%Within its support role, \dword{dune} \dword{tc} works closely with the \dword{jpo} team to characterize these items and contributes engineering resources for required design efforts.

The onsite \dword{integoff} team includes rigging teams responsible for moving 
materials in and out of the shaft, through the underground drifts, 
and within the detector caverns.  It includes personnel responsible 
for overseeing safety and logistics planning.  These team members 
are anticpated to sit within the \dword{sdsd}, an organization 
formed to provide \dword{fnal} support services in South Dakota.    

\section{Far Site Safety}
\label{sec:far_site_safety}

The foundation of a credible installation plan is an \dword{esh}
program that ensures the safety of team members and
equipment supporting the program, and protection of the environment at
the \dword{surf} site.  The \dword{ipd} has responsibility for
implementing the \dword{lbnf-dune} \dword{esh} program for 
installation activities in South Dakota.  The \dword{lbnf-dune}
\dword{esh} manager heads the onsite safety organization and reports
to the \dword{ipd} to support the execution of this
responsibility.

The far site \dword{esh} coordinators sitting under the
\dword{lbnf-dune} \dword{esh} manager oversee the day-to-day execution
of the installation work as shown in
Figure~\ref{fig:dune_esh_installation} and is described further in
Chapter~\ref{vl:tc-ESH}.
\begin{dunefigure}[\dshort{dune} installation \dshort{esh}]{fig:dune_esh_installation}
  {High level \dword{dune} installation \dword{esh} organization.}
  \includegraphics[width=0.85\textwidth]{DUNE_Installation_Safety_OrgChart}
\end{dunefigure}
As we move to 2-shift installation activity, additional far site \dword{esh} 
coordinators will be assigned to each work shift.  The safety coordinator 
assigned to a particular shift is responsible for leading safety discussions 
during the toolbox meeting and for ensuring that all workers on that shift, 
including those from the consortia or contractors, are properly trained.  The 
reporting chain for safety incidents goes through the onsite safety team to 
the \dword{lbnf-dune} \dword{esh} manager to minimize any potential conflicts 
of interest.  All \dword{integoff} installation team members as well as \dword{dune} 
consortia personnel and \dword{lbnf} project team members have the right to 
stop work for any safety issues.

Operation of all equipment used for installation activities such as
cranes, power tools, and personnel lifts is restricted to team members
who have been properly trained and certified for use of that
equipment.  The safety coordinator for each shift is responsible for
ensuring that all team personnel are properly trained and that safety
documentation and work procedures are up-to-date and stored within the
\dword{edms}.

Documentation, including accident reports, near misses, weekly
reports, equipment inspection, and training records is an important
component of the \dword{lbnf-dune} \dword{esh} program. The work
planning and \dword{ha} program utilizes detailed work plan documents,
\dword{ha} reports, equipment documentation, safety data sheets,
\dword{ppe} and job task training to mimimize work place hazards and
maximize efficiency.  Sample documentation is developed through the
\dword{ashriver} trial assembly process, which maps out the step by
step procedures and brings together the documenation needed for
approving the work plan.  The sample documentation is modified to
account for differences required for performing work underground and
the updated procedures are provided to the review process (as
discussed in Chapter~\ref{vl:tc-review}) for \dwords{irr} and
\dwords{orr}.

\section{Integration Office Management}
\label{vl:tc-facility_mgmt}

The \dword{ipd} is responsible for coordinating all installation
activities at \dword{surf} including those that fall under the %direct 
responsibility of \dword{lbnf} and the \dword{dune} consortia.  
The coordinators of this activity and crucial technical support
staff sit within the \dword{integoff}.  The organization of this onsite team is 
shown in Figure~\ref{fig:ash_river}.
\begin{dunefigure}[\dshort{lbnf-dune} IO organization chart]{fig:ash_river}
  {\dword{integoff} organization chart}
  \includegraphics[width=0.95\textwidth]{Org-Far-Site-TDR-10-9-19}
\end{dunefigure}
 
As discussed in Section~\ref{sec:far_site_safety} the onsite safety
organization including the far site \dword{esh} coordinators
working under the direction of the \dword{lbnf-dune} \dword{esh}
manager oversee all onsite activites and have a %direct 
reporting line
to the \dword{ipd}.

Due to the lack of surface space at \dword{surf}, a separate
warehousing facility in the vicinity of \dword{surf} is required to
receive and store materials in advance of their delivery to the
undergound area as discussed in Section~\ref{sec:sdwf}.  Warehouse
operations are coordinated by the \dword{lbnf-dune} logistics manager
who is tasked with determining the exact sequence in which materials
are delivered into the underground areas.

The underground cavern coordinator is responsible for managing all 
activities in the two undergound detector caverns as well as the
\dword{cuc}, this includes contracted workers.  Work within the
detector caverns follows a time ordered sequence that includes
installation of the cryostats (warm and cold), cryogenic systems, and
the detectors themselves.  Work in the \dword{cuc} includes
installation of major cryogenic system pieces and the detector
\dword{daq} electronics.  The underground cavern coordinator relies on
separate installation teams focusing on cryostats, cryogenic systems,
and the detectors.  The cryostat and cryogenics system installation
teams are contracted resources provided by \dword{lbnf}.  For this
reason, coordinators of these activities are jointly placed within 
both the \dword{lbnf} project team and the \dword{integoff}.  The detector
installation teams incoporate a substantial number of scientific and
technical personnel from the \dword{dune} consortia.  \dword{integoff} coordinators 
of the detector installation effort are jointly placed within 
\dword{dune} \dword{tc} to facilitate consortia involvement in the 
detector installation activities.  Any modifications to the facilities 
occuring after \dword{aup} are managed by the underground cavern 
coordinator under the direction of the \dword{ipd}.

The \dword{ipd} manages common technical and engineering resources to
support installation activities.  Technical resources include the
support crews needed for rigging materials on and off the hoist at the
top and bottom of the shaft, transporting materials to the underground
caverns from the bottom of the shaft, and rigging the detector and
infrastructure pieces within the underground caverns during the
installation process.  Additional, common technical resources used
across the installation efforts are welders and survey teams. Access
to on-call electricians and plumbers is required to support the
installation effort and deal with operational issues as they arise.
Required technical resources described here are provided through 
\dword{sdsd}, which will employ full-time staff members and contracted
support staff to provide the necessary functions.

Engineering resources for installation activities sit within the \dword{integoff}.  
The engineering team, which includes both mechanical and electrical 
engineers, resolves last minute issues associated with component 
handling and detector grounding that arise over the course of the 
installation process.  Other required engineering functions include 
procurement support, configuration management, and particpation in 
the safety review process.

\subsection{South Dakota Warehouse Facility}
\label{sec:sdwf}

The \dword{sdwf} is a leased 5000m$^2$ facility hosted by 
\dword{sdsd}.  Approximately six months before \dword{aup}
of the underground detector caverns is received, the \dword{sdwf} 
is required to be in place for receiving cryostat and detector 
components.  Laydown space near the Ross headframe is extremely 
limited.  For this reason, the transportation of materials from 
the \dword{sdwf} to the top of the Ross shaft requires careful 
coordination. The \dword{lbnf-dune} logistics manager works 
with the \dword{cmgc}, through the end of excavation activities, 
and the other members of the \dword{integoff} team to coordinate transport 
of materials into the underground areas.  Since no materials or 
equipment can be shipped directly to the Ross or Yates headframes, 
the \dword{sdwf} is used for both short and long-term storage, as 
well as for any re-packaging of items required prior to transport 
into the underground areas. 

A small number of \dword{dune} consortia members work at the
\dword{sdwf} to check received components for potential damage
incurred during shipment and track all materials coming in and out of
the facility, using the inventory management system.  In some cases
re-packaging of materials is required for lowering them down the shaft
into the underground areas.  The \dword{dune} consortia take
responsibility for these efforts.

\subsection{Underground Caverns}

The installation 
process in the underground detector caverns and \dword{cuc} can 
be broken into a time-sequenced set of activities, coordinated 
through the \dword{integoff}.  In the detector caverns, installation 
of the warm and cold cryostat strucutres is followed by (with 
some overlap) installation of the cryogenic infrastructure and 
detectors.  In the \dword{cuc} installation of the \dword{daq}  
infrastructure and detector readout components proceeds in 
parallel with that of the cryogenic infrastructure.  A high-level 
schedule showing the inter-dependencies between these activities 
is shown in Figure~\ref{fig:underground_schedule}.
\begin{dunefigure}[Underground summary schedule]{fig:underground_schedule}
  {Summary schedule of the different phases of work underground}
  \includegraphics[width=0.95\textwidth]{Overall_schedule-T0}
\end{dunefigure}

The ability of \dword{lbnf-dune} to meet this schedule depends
critically on its ability to work within limitations on the total
number of people allowed in the underground areas at a given time
(144) as well as occupancy limits on work in the cryostat.  In order
to satisfy these limitations, careful balancing of the numbers of
workers assigned to different concurrent tasks taking place within the
different underground caverns is required.  This is a particular
challenge during the excavation period for the second detector cavern,
which runs in parallel with cryostat installation in the first
detector cavern.  The \dword{integoff} works with its \dword{lbnf-dune} project 
partners to manage and optimize the underground work schedule so that 
interferences between concurrent work efforts are minimized.

The underground cavern coordinator manages the contributions of 
the technical team supporting the installation activities.
The size of the technical support team is anticipated to evolve  
over time to meet the needs of the specific installation tasks 
taking place.  The functions provided by the technical team 
supporting the work in the underground caverns include the
following:
\begin{itemize}
  \item {material transport:} The transport team shown in
    Figure~\ref{fig:ctr_orgchart} is responsible for unloading of
    materials from the trucks arriving from the \dword{sdwf}, loading
    or rigging of materials at top of Ross Shaft, unloading or rigging
    of materials at bottom of Ross Shaft, and delivery of materials
    from the bottom of shaft to the underground caverns. This does not
    include operation of the hoists which is performed by
    \dword{surf}.
  \item {cavern rigging operations:}  Storage of 
        components within the available spaces 
        in the cavern and movement of materials as required to 
        execute the installation process (three rigging stages
        for detector installation are moving components into 
        clean room, integrating components within clean room, 
        and installing integrated elements inside cryostat).
  \item {installation technicians:}  General technician 
        support for specific installation activities. 
  \item {welders and survey crews:}  Perform 
        specific tasks incorporated within each of the different 
        installation efforts.
  \item {electrical technicians and plumbers:} On-call support staff
    to modify systems as work transitions from one stage to the next
    and to address issues as they arise.
\end{itemize}   
    
The organization responsible for managing contributions of 
the technical support team to the installation 
activities taking place in the underground caverns is shown 
in Figure~\ref{fig:ctr_orgchart}.  The structure is illustrated
for the case of the largest anticipated workforce (approaching 
roughly sixty team members in total covering multiple shifts) 
for the periods with ongoing detector installation efforts.
These personnel support two 10-hour shifts on Mondays through 
Thursdays and a day shift on the remaining days to cover 
activities occuring over weekends. 
\begin{dunefigure}[Common Technical Resources]{fig:ctr_orgchart}
  {Summary of the \dshort{integoff}/\dword{sdsd} Common Technical Resources}
  \includegraphics[width=0.95\textwidth]{Org_Technical_Resources-JPO-TDR-10-9-19}
\end{dunefigure}

On the surface at the start of each shift, there is a toolbox safety
meeting and work assignment update. An hour separates the two shifts,
in which the lead workers, safety coordinators, and other management
team members, are paid overtime to overlap with each other and
transfer information from one shift to the next. The safety
coordinator for each shift is responsible for conducting the safety
discussion at the meeting and ensuring that all workers assigned to
that shift have the proper trainings.

The team responsible for detector installation incorporates 
members of the technical support team described above 
and includes scientific and technical personnel from 
the \dword{dune} consortia.  The team is led by the detector
installation manager who has three shift supervisors working 
with him or her to provide onsite coverage for every shift.
The management team works with the underground cavern
coordinator to ensure that required technical support team 
members are available as needed and that required materials 
are delivered to the detector caverns on a schedule to keep
the installation effort moving forward.         

The management team supervises technical resources assigned to 
the detector installation effort and works with consortia 
team members to maximize the overall efficiency of the installation 
process.  The organizational structure to manage the detector 
installation activities is shown in Figure~\ref{fig:uit_orgchart}.
\begin{dunefigure}[Underground detector installation team]{fig:uit_orgchart}
  {The \dword{uit}}
  \includegraphics[width=0.95\textwidth]{Org_UIT-TDR-10-9-19}
\end{dunefigure}

The detector installation manager oversees all shifts and serves as
the supervisor for specific shifts as needed.  They serve as the
contact with the underground cavern coordinator for obtaining
required technical support team members and organizing the delivery of
needed materials into the detector caverns.  The detector installation
manager attends all high-level meetings with the underground cavern
coordinator and is tasked with submitting weekly progress reports.
They work with the \dword{dune} consortia to manage the overall work
schedule and ensure that the correct resources are in the right place
at the right time.
    
The installation supervisors are working managers, trained as riggers
and equipment operators to fill in as needed on their shifts.  They
are fully trained in all installation procedures and work with the
consortia shift team members to keep the installation effort on
schedule.  Installation supervisors fill in for their lead workers as
needed and are the primary points of contact for information exchange
between shifts.  Lead workers direct the technical support personnel
assigned for their shift.  The lead workers are trained in all
installation procedures and provide assistance to the consortia work
teams as needed.

\subsection{Trial Assembly at \dshort{ashriver}}

The trial assembly work at \dword{ashriver}, site of the \dword{nova}
far detector, focuses on mechanical tests of the installation
process for \dword{dune}. This effort is critical to confirm final
detector component designs, including modifications originating from
\dword{protodune}. It confirms and practices installation techniques
for both the cleanroom and cryostat.  The \dword{nova} far site
detector hall in Ash River, Minnesota has facilities that
match \dword{dune} needs, including a \SI{16.75}{m} deep pit with
$\sim$\SI{300}{m$^2$} of floor space available for testing full-scale
\dword{dune} detector components and a capable workforce that is
needed for \dword{nova} operations and can be leveraged in a cost effective
manner for \dword{dune}.  The \dword{nova} far detector Laboratory is
managed by the University of Minnesota (UMN) and is partially funded
through an operations contract from \dword{fnal}.  Work performed at
the \dword{ashriver} site follows university safety regulations and
any \dword{dune} safety requirements. University code officials approve
all building permits, which include engineered drawings signed by an
engineer registered in Minnesota. All hazard analyses and work
procedure documents are approved by the joint \dword{dune}/UMN safety
committee with members drawn from both the University of Minnesota
(UMN) and \dword{dune} that includes specialists as needed.

The work at \dword{ashriver} has five main goals:
\begin{itemize}
  \item using prototype \dword{dune} components, verify that the
    \dword{dune} detector can be installed in a safe and efficient
    manner,
    \item test installation equipment needed to install the
      \dword{dune} detector at \dword{surf},
  \item validate mechanical design changes made to the detector
    elements subsequent to \dword{protodune} operation,
  \item complete a set of reviewed engineering and procedural
    documents that will serve as the basis for work to be performed
    underground at \dword{surf}, and
  \item serve as a training center for personnel who will 
    contribute to \dword{dune}  installation at \dword{surf}.
\end{itemize}

The full time staff of five people at \dword{ashriver}
includes a manager, deputy manager, and three experienced technicians
that all participated in \dword{pdsp} installation at \dword{cern} and
the \dword{pdsp} trial assembly at \dword{ashriver}.  The staff
oversees operations of the \dword{nova} detector and performs trial
assembly studies of the \dword{dune} detector components.  One of the
three technicians also serves as the site safety officer and
chairperson of the joint \dword{dune}/UMN safety committee.  Two additional staff
members will be added in the near future to handle the additional
workload associated with preparations for the \dword{protodune2}
installation effort.
\begin{dunefigure}
  [Phase 1 \dshort{apa} installation frame being installed on
    the \dshort{apa} assembly tower at \dshort{ashriver}]{fig:ashriver1}
  {Phase 1 \dword{apa} installation frame (in red) being installed on the
  \dword{apa} assembly tower at \dword{ashriver}. In the foreground is
  the \dword{pdsp} trial assembly structure.}
   \includegraphics[width=0.65\textwidth]{ashriver_tower}
\end{dunefigure}

The work at \dword{ashriver} is divided in three major phases.
The three phases are the following:
\begin{itemize}
  \item {Phase 0:} A vertical cabling test using two full-scale 
         \dword{apa} side tubes connected top to bottom and mounted 
         against a vertical column in the detector hall.  Using this 
         setup the proposed cable bundles have been run through the 
         tubes to see how well the designed conduit system functions.
         This work has led to several proposed modifications to the 
         designs which are currently being considered.  The older 
         \dword{protodune} trial assembly structure is concurrently 
         being used to perform mechanical tests of \dword{protodune2} 
         components. 
  \item {Phase 1:} A prototype of the \dword{dune} \dword{apa} 
         assembly tower using a steel frame large enough to hold a 
         commercial stair scaffold within its mid-section, as shown 
         in Figure~\ref{fig:ashriver1}, was constructed and %.  The 
        % tower was 
        used to test the process for connecting top 
         and bottom \dword{apa} pairs together and installing the 
         required cable bundles.  The next step will be to add a
         \dword{cpa} assembly station and test assembly procedures 
         for the updated \dword{cpa} designs.  A prototype 
         \dword{apa} shipping frame is also being constructed to 
         test the mechanical features of the shipping container 
         design.  
  \item {Phase 2:} A more complex steel structure will be 
         designed and fabricated to mock up the network of rails 
         and support structures used to install the \dword{dune}
         \dword{fd} modules including pieces of the \dword{dss}, 
         which sits inside the cryostat.  This structure as 
         illustrated in Figure~\ref{fig:ashriver2} will provide 
         a platform for performing more detailed tests of the 
         proposed detector installation plan.  Installation steps 
         to be tested include \dword{dss} installation, transfer 
         of \dword{tpc} components through the \dword{tco}, 
         installation of the \dword{tpc} end walls, cabling 
         through the cryostat penetrations, movement of the 
         \dword{apa} and  \dword{cpa} pairs into their final 
         positions, and deployment of the top and bottom field 
         cage modules.
\end{itemize}
\begin{dunefigure}[Phase 2 trial assembly at \dshort{ashriver}]{fig:ashriver2}
  {Phase 2 trial assembly at \dword{ashriver}.}
  \includegraphics[width=0.65\textwidth]{Phase2_Trial_Assembly.pdf}
\end{dunefigure}

\section{South Dakota Services Division}
\label{sec:fdsp-coord-host_facility_services}

\dword{surf} is operated by the \dword{sdsta} through a cooperative 
agreement with the \dword{doe}.  Prior to this agreement going forward, 
\dword{sdsta} signed onto a \dword{mou} with the Fermi Research Alliance 
(FRA) detailing facility support services to be provided by \dword{sdsta}
in support of \dword{lbnf-dune}.  The \dword{mou} establishes a Joint 
Coordination Team with regularly scheduled meetings to ensure that all  
\dword{surf} support functions necessary for achieving \dword{lbnf-dune} 
objectives are provided.             

\dword{fnal} has established the \dword{sdsd} to support integration
and installation activities in South Dakota. \dword{sdsd} will support
these activities, which are the responsibility of the \dword{ipd}, by
providing access to the required technical resources.  These resources
include dedicated \dword{fnal} personnel sitting within the division
and contracted labor provided through the division.  Some examples of
\dword{sdsd} support staff include rigging teams to support activities 
in underground caverns and at the headframe and bottom of the shaft,
transport crews for moving materials between the warehouse and site 
and from the bottom of the shaft into the underground caverns, and a 
core group of technicians for performing maintenance and installation
activities.  \dword{sdsd} will also assist \dword{lbnf-dune} partners 
in understanding work requirements at \dword{surf} and ensuring that 
appropriate provisions are incorporated into partner contracts with 
external contractors.  The \dword{sdsd} will have its own procurement 
team to assist the \dword{ipd} in acquiring the common infrastrucutre 
items required for the installation effort.  This team will also be 
responsible for handling any contracts associated with further work 
on the facilties subsequent to departure of the \dword{lbnf} \dword{cmgc}.         

In analogy with the Ash River site, where University of Minesota 
officials are responsible for any building permits, \dword{sdsd} 
is responsible for any electrical or building permits required 
for the leased spaces at \dword{surf}.  \dword{sdsd} also takes 
responsibility for badging personnel requiring access to the 
leased areas at \dword{surf} in coordination with the \dword{fnal} 
Global Services Office and the \dword{surf} Administrative Services 
Office. This includes providing and coordinating the trainings 
required to access surface and underground areas.  To maintain safe 
working conditions within the leased areas, \dword{sdsd} performs 
regular inspections and maintenance of all \dword{lbnf-dune} 
equipment operating at \dword{surf} including lifts, conveyances, 
networking equipment, cooling and ventillation equipment, rigging 
equipment, electrical power installations, life safety systems, 
and controlled access equipment.

%%%%%%%%%%%%%%%%%%%%%%%%%%%%%%%%
\section{Interface Documents}
\label{sec:fdsp-coord-interface}

A set of interface documents defines the scope of each subsystem and
with progressively more detail defines the detailed interfaces between
subsystems. There are three sets of interface documents. One set of
documents includes all of the consortia-consortia interfaces. A second
set includes the interfaces between the consortia and the facilities
(provided either by \dword{tc}, \dword{lbnf} or \dword{jpo}). The
third set is between the consortia and the installation team. All
three sets are managed by \dword{tc}/\dword{jpo}.

The \dword{dune} interface documents are actively maintained in the
\dword{edms}, but a copy has been archived in DocDB that captures the
status of these documents at the time of this \dword{tdr}.  A matrix
with links to the interface documents in DocDB between consortia for
the single phase detector are shown in
Table~\ref{tab:interface_sp_consortia}.
\begin{dunetable}
  [\dshort{spmod} inter-consortium interface document matrix]
  {|p{0.08\linewidth}|rp{0.08\linewidth}||rp{0.08\linewidth}||rp{0.08\linewidth}||rp{0.08\linewidth}|rp{0.08\linewidth}||rp{0.08\linewidth}|rp{0.08\linewidth}|||}
  {tab:interface_sp_consortia}
  {\dword{spmod} consortium-to-consortium interface document matrix. All entries point to documents in the DUNE DocDB.}
       & PDS  & CE   & HV   & DAQ  & CISC & CAL  & COMP \\ \toprowrule
  APA  & \cite{bib:docdb6667} &  \cite{bib:docdb6670} & \cite{bib:docdb6673} & \cite{bib:docdb6676} &  \cite{bib:docdb6679} &  \cite{bib:docdb7048} &  \cite{bib:docdb7102} \\ \colhline
  PDS  &      &  \cite{bib:docdb6718} &  \cite{bib:docdb6721} & \cite{bib:docdb6727} &  \cite{bib:docdb6730} &  \cite{bib:docdb7051} & \cite{bib:docdb7105} \\ \colhline
  CE   &      &      & \cite{bib:docdb6739} & \cite{bib:docdb6742} & \cite{bib:docdb6745} & \cite{bib:docdb7054} & \cite{bib:docdb7108} \\ \colhline
  HV   &      &      &      & \cite{bib:docdb6736} & \cite{bib:docdb6787} & \cite{bib:docdb7066} & \cite{bib:docdb7120} \\ \colhline
  DAQ  &      &      &      &      & \cite{bib:docdb6790} & \cite{bib:docdb7069} & \cite{bib:docdb7123} \\ \colhline
  CISC &      &      &      &      &      & \cite{bib:docdb7072} & \cite{bib:docdb7126} \\ \colhline
  CAL  &      &      &      &      &      &      & \cite{bib:docdb6868} \\ 
\end{dunetable}

A matrix with links to the interface documents in DocDB between
consortia and \dword{tc}/\dword{jpo} for the single phase detector are
shown in Table~\ref{tab:interface_sp_tc}.
\begin{dunetable}
  [\dshort{spmod} consortium-TC interface document matrix]
  {|p{0.15\linewidth}||rp{0.08\linewidth}||rp{0.08\linewidth}||rp{0.08\linewidth}||rp{0.08\linewidth}||rp{0.08\linewidth}||rp{0.08\linewidth}||rp{0.08\linewidth}||rp{0.08\linewidth}|}
  {tab:interface_sp_tc}
 {\dword{spmod} consortium-to-TC interface document matrix. All entries point to documents in the DUNE DocDB.}
                &  APA & PDS  & CE   & HV   & DAQ  & CISC & CAL  & COMP \\ \toprowrule
  Facility      & \cite{bib:docdb6967} & \cite{bib:docdb6970} & \cite{bib:docdb6973} & \cite{bib:docdb6985} & \cite{bib:docdb6988} & \cite{bib:docdb6991} & \cite{bib:docdb6829} & \cite{bib:docdb6841} \\ \colhline
  Installation  & \cite{bib:docdb6994} & \cite{bib:docdb6997} & \cite{bib:docdb7000} & \cite{bib:docdb7012} & \cite{bib:docdb7015} & \cite{bib:docdb7018} & \cite{bib:docdb6847} & \cite{bib:docdb6853} \\ \colhline
  Physics       & \cite{bib:docdb7075} & \cite{bib:docdb7078} & \cite{bib:docdb7081} & \cite{bib:docdb7093} & \cite{bib:docdb7096} & \cite{bib:docdb7099} & \cite{bib:docdb6865} &   \cite{bib:docdb6871}   \\ 
\end{dunetable}


%%%%%%%%%%%%%%%%%%%%%%%%%%%%%%%%
%\section{Cost}
%\label{sec:fdsp-coord-cost}
%
%\forlbnc{This section will discuss the cost estimate.}

%%%%%%%%%%%%%%%%%%%%%%%%%%%%%%%%
\section{Schedule Milestones}
\label{sec:fdsp-coord-controls}

A series of tiered milestones have been developed for the \dword{dune}
project. The spokespersons and host laboratory director are
responsible for the tier 0 milestones. Three tier 0 milestones have
been defined and the dates set:
\begin{enumerate}
\item Start main cavern excavation \hspace{2.58in} 2020
\item Start \dword{detmodule}~1 installation \hspace{2.1in} 2024
\item Start operations of \dwords{detmodule} \#1--2 with beam \hspace{0.8in} 2028
\end{enumerate}
These dates will be revisited after the U.S. \dword{lbnf} project is
reviewed. The \dword{tcoord}, \dword{ipd} and \dword{lbnf} project
manager hold the Tier 1 milestones.  The consortia hold Tier 2
milestones. Table~\ref{tab:DUNE_schedule} provides a high level version of the
\dword{dune} milestones from the \dword{lbnf-dune} schedule.
\begin{dunetable}
[\dshort{dune} schedule milestones]
{p{0.65\textwidth}p{0.25\textwidth}}
{tab:DUNE_schedule}
{\dword{dune} schedule milestones for first two far detector modules. Key DUNE dates and milestones, defined for planning purposes in this TDR, are shown in orange.  Dates will be finalized following establishment of the international project baseline.}
Milestone & Date   \\ \toprowrule
Technology Decision  &   April 2020   \\ \colhline
Final Design Reviews &   2020   \\ \colhline
South Dakota Logistics Warehouse available& \sdlwavailable      \\ \colhline
 \dwords{prr} &  2022    \\ \colhline
\rowcolor{dunepeach} Beneficial occupancy of cavern 1 and \dword{cuc}& \cucbenocc      \\ \colhline
\dshort{cuc} \dword{daq} room available& \accesscuccountrm      \\ \colhline
Top of \dshort{detmodule} \#1 cryostat accessible& \accesstopfirstcryo      \\ \colhline
\rowcolor{dunepeach}Start of \dshort{detmodule} \#1 \dshort{tpc} installation& \startfirsttpcinstall      \\ \colhline
Top of \dshort{detmodule} \#2 cryostat accessible& \accesstopsecondcryo      \\ \colhline
End of \dshort{detmodule} \#1 \dshort{tpc} installation& \firsttpcinstallend      \\ \colhline
\rowcolor{dunepeach}Start of \dshort{detmodule} \#2 \dshort{tpc} installation& \startsecondtpcinstall      \\ \colhline
End of \dshort{detmodule} \#2 \dshort{tpc} installation& \secondtpcinstallend      \\  \colhline
\rowcolor{dunepeach}\dshort{detmodule} \#1 operations & July 2026 \\
\end{dunetable}
To monitor progress, \dword{tc}/\dword{jpo} will maintain the
\dword{lbnf-dune} schedule that links all consortium schedules and
contains milestones for each consortia.  The schedules will go under
change control after each consortium agrees to the milestone dates,
the \dword{tdr} is approved and the \dword{lbnf} project is baselined.

To ensure that the \dword{dune} detector remains on schedule,
\dword{tc}/\dword{jpo} will monitor schedule status from each
consortium and organize reviews of schedules and risks as appropriate.
As schedule problems arise, \dword{tc}/\dword{jpo} will work with affected
consortium to resolve the problems. If problems cannot be solved, the
\dword{tcoord} will take the issue to the \dword{tb} and \dword{exb}.

A monthly report with input from all consortia will be published by
\dword{tc}/\dword{jpo} and provided to the \dword{lbnc}. This will
include updates on consortium and \dword{tc} technical progress
against the schedule.


%%%%%%%%%%%%%%%%%%%%%%%%%%%%%%%%
\section{Requirements}
\label{sec:fdsp-coord-requirements}

The scientific goals of \dword{dune} as described in \dword{dune}
\dword{tdr} Volume~\volnumberexec:~\voltitleexec include
\begin{itemize}
\item a comprehensive program of neutrino oscillation measurements
  including the search for CP violation
\item measurement of $\nu_{e}$ flux from a core-collapse supernova within our
  galaxy should one occur during \dword{dune} operations
\item searching for baryon number violation
\end{itemize}
These goals motivate a number of key detector requirements: drift
field, electron lifetime, system noise, photon detector light yield
and time resolution. The \dword{exb} has approved a list of high level
detector specifications, including those listed above. These will be
maintained in \dword{edms}, and the high level requirements with
significant impact on physics (applying to both \dword{sp} and
\dword{dp} detectors) are highlighted in Table~\ref{tab:dunephysicsreqs}.
\begin{dunetable}
  [DUNE physics-related specifications owned by EB]
  {p{0.025\textwidth}p{0.06\textwidth}p{0.2\textwidth}p{0.35\textwidth}p{0.15\textwidth}p{0.1\textwidth}}
  {tab:dunephysicsreqs}
  {\dword{dune} physics-related specifications owned by \dword{exb}}
  ID & System & Parameter & Physics Requirement Driver & Requirement & Goal \\ \toprowrule
  1   & HVS    & Minimum drift field &  Limit recombination, diffusion and space charge impacts on particle ID. Establish adequate \dword{s/n} for tracking. & >\SI{250}{V/cm} & \spmaxfield \\ \colhline
  2   & CE     & System noise & The noise specification is driven by pattern recognition and two-track separation.  & <\SI{1000}{enc} & ALARA \\ \colhline
  3   & PDS    & Light yield  & The light yield shall be sufficient to measure time of events with visible energy above 200 MeV.  Goal is 10\% energy measurement for visible energy of 10 MeV.  & >\SI{0.5}{pe/MeV} & >\SI{5}{pe/MeV}  \\ \colhline
  4   & PDS    & Time resolution  & The time resolution of the photon detection system shall be sufficient to assign a unique event time.  & $<\,\SI{1}{\micro\second}$ & $<\,\SI{100}{\nano\second}$  \\ \colhline
  5   & all    & liquid argon purity & The LAr purity shall be sufficient to enable drift e- lifetime of 3 (10)ms & $<$\,\SI{100}{ppt} & $<$\,\SI{30}{ppt} \\ 
\end{dunetable}
Eleven other significant specifications for the \dword{sp} detector
owned by the \dword{exb} are listed in Table~\ref{tab:dunephysicsspecs}
along with another twelve high level engineering specifications.
\begin{dunetable}
  [DUNE high-level system specifications owned by EB]
  {p{0.025\textwidth}p{0.06\textwidth}p{0.2\textwidth}p{0.35\textwidth}p{0.15\textwidth}p{0.1\textwidth}}
  {tab:dunephysicsspecs}
  {\dword{dune} high-level system specifications owned by \dword{exb}}
  ID & System & Parameter & Physics Requirement Driver & Requirement & Goal \\ \toprowrule
  6   & APA & Gaps between APAs  & minimize events lost due to vertex in gaps between APAs (15mm on same support beam, 30mm on adjacent beams) & <\SI{30}{mm} & <\SI{15}{mm} \\ \colhline
  7   & DSS & Drift field uniformity & tolerance on drift field due to component location & $<\,\SI{1}{\%}$  &   \\ \colhline
  8   & APA & wire angles  & 0$^\circ$ collection, $\pm$35.7$^\circ$ induction &  &  \\ \colhline
  9   & APA & wire spacing  & \SI{4.669}{mm} for U,V; \SI{4.790}{mm} for X,G &  &  \\ \colhline
  10  & APA & wire position tolerance  & & $\pm\,\SI{0.5}{mm}$  &  \\ \colhline
  11  & HVS & Drift field uniformity & tolerance on drift field due to HVS system & $<\,\SI{1}{\%}$  &  \\ \colhline
  12  & HVS & Cathode power supply ripple & very small compared to intrinsic electronics noise & $<\,\SI{100}{enc}$ &   \\ \colhline
  13  & CE & Frontend peaking time  & optimize vertex resolution & \SI{1}{\micro\second} &  \\ \colhline
  14  & CE & Signal saturation  & largest signals occur with multiple protons in the primary vertex & 500k $e^-$ &  \\ \colhline
  15  & cryo & LAr N$_2$ contamination  & optical attenuation length in liquid argon with 50~ppm of N$_2$ contamination is roughly 3~m & $<\,\SI{25}{ppm}$ &  \\ \colhline
  16  & all & Detector dead time  & risk of missing a supernova burst if all operating cryostats are offline & $<\,\SI{0.5}{\%}$ &  \\ 
\end{dunetable}
The high level \dword{dune} requirements that drive the \dword{lbnf} design are
maintained in~\citedocdb{112} and under change control. These are owned by
the \dword{dune} \dword{tcoord} and \dword{lbnf} project manager.

Lower level detector specifications are held by the consortia and
described in the \dword{dune} \dword{tdr} \dword{sp}
Volume~\volnumbersp\ and \dword{dp} Volume~\volnumberdp\ chapters for
each consortium. A complete list of detector specifications is
provided in Section~\ref{sec:fdsp-app-requirements}.

%%%%%%%%%%%%%%%%%%%%%%%%%%% Anne moved up to keep reqs with reqs
\section{Full DUNE Requirements}
\label{sec:fdsp-app-requirements}


%%%%%%%%%%%%%%%%%%%%%%%%%%%%%%%
\subsection{Single-phase}
\label{sec:tc-req-sp}

\input{generated/req-longtable-SP-FD}

\input{generated/req-longtable-SP-APA}
\input{generated/req-longtable-SP-HV}
\input{generated/req-longtable-SP-ELEC} 
\input{generated/req-longtable-SP-PDS}
\input{generated/req-longtable-SP-CALIB}
\input{generated/req-longtable-SP-DAQ} 
\input{generated/req-longtable-SP-CISC} 
\input{generated/req-longtable-SP-INST}

%%%%%%%%%%%%%%%%%%%%%%%%%%%%%%%
\subsection{Dual-phase}
\label{sec:tc-req-dp}


\input{generated/req-longtable-DP-FD}

\input{generated/req-longtable-DP-CRP}
\input{generated/req-longtable-DP-HV}
\input{generated/req-longtable-DP-ELEC}
\input{generated/req-longtable-DP-PDS}
\input{generated/req-longtable-DP-CALIB}
\input{generated/req-longtable-DP-DAQ}
\input{generated/req-longtable-DP-CISC}
%% This file is generated, any edits may be lost.
\begin{footnotesize}
%\begin{longtable}{p{0.14\textwidth}p{0.13\textwidth}p{0.18\textwidth}p{0.22\textwidth}p{0.20\textwidth}}
\begin{longtable}{p{0.12\textwidth}p{0.18\textwidth}p{0.17\textwidth}p{0.25\textwidth}p{0.16\textwidth}}
\caption{Specifications for DP-INST \fixmehl{ref \texttt{tab:spec:DP-INST}}} \\
  \rowcolor{dunesky}
       Label & Description  & Specification \newline (Goal) & Rationale & Validation \\  \colhline


   
  \newtag{DP-INST-1}{ spec:handling-specs }  & Compliance with the material handling specification for all material transported underground  &  Fulfill material handling specification &  Being able to tranport undergound the equipment &  Material documentation and inspection \\ \colhline
    
   
  \newtag{DP-INST-2}{ spec:material-buffer }  & Material buffer at the warehouse facility  &  >1 month &  Limit the impact of delivery delays on the installation schedule &  Regular report from the construction sites \\ \colhline
    
   
  \newtag{DP-INST-3}{ spec:underground-storage }  & Material storage available underground  &  >4~\dshorts{crp} &  Limit the impact of rejected detector components on the installation schedule &  Dedicated space to store the detector component available undergorund in the vicinity of the clean room \\ \colhline
    
   
  \newtag{DP-INST-4}{ spec:clean-room-specs }  & Cryostat and clean room environment  &  >ISO-8 &  Safe handling of the detector components and limit the radioactive background &  Regular air quality checks, constant air filtering, regular cleaning of the clean room and the cryostat, use of clean room equipment \\ \colhline
    
   
  \newtag{DP-INST-5}{ spec:cold-box-cryo }  & Cold box cryogenics system  &  Independent operation of the four cold boxes &  Guarantee the needed flexibility in the tests &  Simultaneous operation of the cold boxes \\ \colhline
    


\label{tab:specs:DP-INST}
\end{longtable}
\end{footnotesize}  Not there yet 10/29


%%%%%%%%%%%%%%%%%%%%%%%%%%%%%%%%
\section{Risks}
\label{sec:fdsp-coord-risks}

\dword{dune} initiated a risk registry in 2018 (available
in~\citedocdb{6443}). This document includes consortia risks and
\dword{tc} risks. It includes a summary of the most significant
overall \dword{dune} risks.  This registry has been updated for the
\dword{tdr} and the full listing can be found in
Appendix~\ref{sec:fdsp-app-risk}. It is expected to be updated
approximately yearly. The previous update was in early 2018
before ProtoDUNE was completed and the most recent update is for the
\dword{tdr}. Another update is planned for 2020. Several risks
associated with ProtoDUNE have been retired. \dword{lbnf} and
\dword{dune}-US would like \dword{dune} to update and expand this risk
register to allow a Monte Carlo analysis of cost and schedule risks to
the \dword{us} project resulting from international \dword{dune}
risks. This request is under consideration as it may be useful for
other national projects as well.  Successfully operating
\dword{protodune} retired many \dword{dune} risks in
\dword{dune}. This includes most risks associated with the technical
design, production processes, \dword{qa}, integration and
installation. Residual risks remain relating to design and production
modifications associated with scaling to \dword{dune}, mitigations to
known installation and performance issues in \dword{protodune},
underground installation at \surf and organizational growth.

The highest technical risks include development of a system to
deliver \SI{600}{kV} to the \dual cathode; general delivery of the
required \dword{hv}; cathode and \dword{fc} discharge to the cryostat
membrane; noise levels, particularly for the \dword{ce}; 
number of dead channels; lifetime of components surpassing \dunelifetime{}; 
\dword{qc} of all components; verification of improved \dword{lem}
performance; verification of new cold  \dword{adc} and  \dword{coldata} performance;
argon purity; electron drift lifetime; \phel light yield;
incomplete calibration plan; and incomplete connection of design to
physics. Other major risks include insufficient funding, optimistic
production schedules, incomplete plans for integration, testing and installation.

One update to the risk registry since 2018 has been for \dword{tc}, in which some
risks  associated with  DUNE  integration and  installation have  been
added (see Volume~\volnumbersp:~\voltitlesp, Chapter~9, Table~9.2)

In addition to installation related risks \dword{tc} is developing its
own set of overall project risks not captured by conortia.  Key risks
for \dword{tc} to manage include the following:
\begin{enumerate}
\item Consortia leave too much scope unaccounted for and too much falls
  to  the \dword{comfund}.
\item Insufficient organizational systems are put into place to
  ensure that this complex international mega-science project,
  including \dword{tc}, \fnal as host laboratory, \surf, DOE, and all international
  partners continue to work together successfully to ensure
  appropriate processes and services are provided for the success of
  the project.
\item Inability of \dword{tc} to obtain sufficient personnel resources to
  ensure that \dword{tc} can oversee and coordinate all project tasks.  While the \dword{us}, 
  as host country, has a special responsibility to \dword{tc}, personnel resources should
  be directed to \dword{tc} from each collaborating country. 
\end{enumerate}

The consortia have provided preliminary versions of risk analyses that
have been collected on the \dword{tc} webpage (\citedocdb{6443}). These have
been developed into an overall risk register that will be monitored
and maintained by \dword{tc} in coordination with the consortia. This
full set of risks can be found in 
Section~\ref{sec:fdsp-app-risk}.


%%%%%%%%%%%%%%%%%%%%%%%%%%%%%%%%
\section{Full DUNE Risks}
\label{sec:fdsp-app-risk}

\input{vol-tc/ch-tc-risks}

%%%%%%%%%%%%%%%%%%%%%%%%%%%%%%%%
\section{Hazard Analysis Report (HAR)}
\label{sec:fdsp-har}

A key element of an effective \dword{esh} program is the hazard
identification process. Hazard identification allows production of a
list of hazards within a facility, so these hazards can be screened
and managed through a suitable set of controls.

The \dword{lbnf-dune} project completed a \dword{har}
to ensure that identified hazards are mitigated early in the
the design process.  The focus of the report is on process hazards,
not activity hazards that are typically covered in a job hazard
analysis.  The \dword{har} has been completed, identifying
hazards anticipated in the project's construction and operational
phases.

The hazard \dword{har} looks at the consequences of a hazard
to establish a pre-mitigation risk category. Proposed mitigation is
applied to hazards of concern to reduce risk and then establishes a
post-mitigation risk category.

As the \dword{dune} design matures, the \dword{har} will be
updated to ensure that all hazards are properly identified and
controlled through design and safety management system programs.  In
addition, some sections of the \dword{har} are used to meet
the safety requirements as defined in 10 CFR 851 and \dword{doe} Order
420.2C, Safety of Accelerator Facilities.  Table~\ref{tab:hazards}
summarizes these hazards.  The sections following the table describe
in more detail the hazards that are most applicable to \dword{dune}
activities and the design and operational controls used to mitigate
these hazards. The results of these evaluations confirm that the
potential risks from construction, operations, and maintenance are
acceptable. Individual activity-based \dword{ha} will be
developed for each work \dword{lbnf-dune} activity at
\dword{surf}.

\begin{longtable}{|p{0.35\textwidth}|p{0.28\textwidth}|p{0.28\textwidth}|}
  \caption[List of identified hazards]{List of identified hazards}
  \label{tab:hazards} \\  \toprowrule
  \rowtitlestyle   HA-1 (Construction) & HA-2 (Natural Phenomena) & HA-3 (Environmental)   \\ \toprowrule
  Site Clearing, Excavation, Mining, Tunneling (explosives), Vertical/Horizontal Conveyance Systems,
  Confined space, Heavy Equipment, Work at Elevations (steel, roofing), Material Handling (rigging)
  Utility interfaces, (electrical, steam, chilled water), Slips/trips/falls, Weather related conditions
  Scaffolding, Transition to Operations, Radiation Generating Devices &
  Seismic, Flooding, Wind, Lightning, Tornado &
  Construction impacts,
  Storm water discharge (construction and operations), Operations impacts, Soil and groundwater activation/contamination,
  Tritium contamination, Air activation, Cooling water activation (HVAC and Machine),
  Oils/chemical leaks or spills, Discharge/emission points (atmospheric/ground)\\ \colhline
  \rowtitlestyle HA-4 (Waste) & HA-5 (Fire) & HA-6 (Electrical)   \\ \toprowrule
  Construction Phase, Facility maintenance, Experimental Operations, Industrial, Hazardous, Radiological &
  Facility Occupancy Classification, Construction Materials, Storage, Flammable/combustible liquids,
  Flammable gasses, Egress/access, Electrical, Lightning, Welding/cutting/brazing work, Smoking  &
  Facility, Experimental, Job built Equipment, Low Voltage/High Current, High Voltage/High Power,
  Maintenance, Arc flash, Electrical shock, Cable tray overloading/mixed utilities, Exposed 110V,
  Stored energy (capacitors \& inductors), Be in contactors   \\ \colhline
  \rowtitlestyle   HA-7 (Mechanical) & HA-8 (Cryo/ODH) & HA-9 (Confined Space)   \\ \toprowrule
  Construction Tools, Machine Shop Tools, Industrial Vehicles, Drilling, Cutting, Grinding,
  Pressure/Vacuum Vessels and Lines, High Temp Equipment (Bakeouts) &
  Thermal, Cryogenic systems, Pressure, Handling and Storage,
  Liquid argon/nitrogen spill/leak, Use of inert gases (argon, nitrogen, helium), Specialty gases &
  Sumps, Utility Chases        \\ \colhline
  \rowtitlestyle   HA-11 (Chemical) & HA-14 (Laser) & HA-15 (Material Handling)   \\ \toprowrule
  Toxic, Compressed gas, Combustibles, Explosives, Flammable gases, Lead (shielding), Cryogenic &
  Alignment Laser, Testing and Calibration, Magnetic Fields, Calibration \& Testing &
  Overhead cranes/hoists, Fork trucks, Manual material handling, Delivery area distribution,
  Manual movement of materials, Hoisting \& Rigging, Lead, Beryllium Windows,Oils, Solvents, Acids,
  Cryogens, Compressed Gases   \\ \colhline
  \rowtitlestyle   HA-16 (Experimental Ops) &  &    \\ \toprowrule
  Electrical equipment, Water Hazard, Working from heights (scaffolding/lifts), Transportation of hazardous materials,
  Liquid Argon/Nitrogen, Chemicals (Corrosive, Reactive, Flammable), Elevations, Ionizing radiation,
  Ozone production, Slips, trips, falls, Machine tools/hand tools, Stray static magnetic fields, Research gasses (Inert, Flammable) &
  &   \\  \colhline
\end{longtable}

\subsection{Construction Hazards (LBNF-DUNE HA-1)}

The project will use the existing work planning and
control process for the laboratories along with a construction project safety and health
plan to communicate these policies and procedures as required by \dword{doe}
Order 413.3b. The installation and construction hazards
anticipated for the \dword{lbnf-dune} project include the following:
\begin{itemize}
\item Site clearing
\item Excavation
\item Installing vertical/horizontal conveyance systems
\item Confined space
\item Heavy equipment operation
\item Work at elevation (erecting steel, roofing)
\item Material handling (rigging)
\item Utility interfaces (electrical, chilled water, ICW, natural gas)
\item Slips/trips/falls
\item Weather related conditions
\item Scaffolding
\item Transition to operations
\item Devices generating radiation
\end{itemize}

To reduce risks from construction hazards, \fnal will use engineered
and approved excavation and fall protection systems.  Heavy equipment
will use required safety controls. The \fnal construction safety
oversight program includes periodic evaluation of the construction
site and construction activities, hazard analysis for all
subcontractor activities and frequent \dword{esh} communications at
the daily tool box meetings of subcontractors.

\subsection{Natural Phenomena (LBNF-DUNE HA-2)}

The \dword{lbnf-dune} design will be governed by the
International Building Code, 2015 edition; \dword{doe} Standard
(STD)-1020, 2016 edition, Natural Phenomena Hazard Analysis and Design
Criteria for \dword{doe} Facilities, guided the design in meeting the
natural phenomena hazard requirements.  The International Building
Code specifies design criteria for wind loading, snow loading, and
seismic events.

\dword{lbnf-dune} was determined to be a low hazard,
performance category 1 facility according to the \dword{doe}
STD-1021-93. \dword{lbnf-dune} areas will contain small
quantities of activated, radioactive, and hazardous chemical
materials. Should a natural phenomenon hazard cause significant
damage, the impact will be mission related and will not pose a hazard
to the public or the environment.

\subsection{Environmental Hazards (LBNF-DUNE HA-3)}

Environmental hazards from \dword{dune} include potentially releasing
chemicals to soil, groundwater, surface water, air, or sanitary sewer
systems that could, if not controlled, exceed regulatory limits.

\fnal maintains an environmental management system equivalent to ISO
14001, consisting of programs for protecting the environment, assuring
compliance with applicable environmental regulations and standards,
and avoiding adverse environmental impact through continual
improvement.  These programs are documented in the 8000 and 11000
series of chapters in the \dword{feshm}.  The environmental mitigation
plan also meets federal and state regulations.


\subsection{Waste Hazards (LBNF-DUNE HA-4)}

Waste related hazards from \dword{dune} include the potential for
releasing waste materials (oils, solvents, chemicals, and radioactive
material) to the environment, injury to personnel, and reactive or
explosive event. Typical initiators will be transportation accidents,
incompatible materials, insufficient packaging/labeling, failure of
packaging, and a natural phenomenon.

During installation and \dword{dune} operation, we anticipate few
hazardous materials will be used. Such materials include paints,
epoxies, solvents, oils, and lead in the form of shielding. No current
or anticipated activities at \dword{dune} would expose workers to
levels of contaminants (dust, mists, or fumes) above regulatory
limits.

The \dword{esh}\&Q section industrial hygiene group and hazard control
technology team manage the program and guide collaborators subject to
waste-related hazards.  Their staff identify workplace hazards, help
identify controls, and monitor implementation. Industrial hygiene
hazards will be evaluated, identified, and mitigated as part of the
work planning and control hazard assessment process.

\subsection{Fire Hazards (LBNF-DUNE HA-5)}

Fire hazards have been evaluated and addressed to comply with
\dword{doe} Order 420.1C, Facility Safety, Chapter II and
\dword{doe}-STD 1066, Fire Protection Design Criteria.  The intent of
these documents is to meet \dword{doe}'s highly protected risk (HPR)
approach to fire protection.  In addition, the National Fire
Protection Association Standard 520, Standard on Subterranean Spaces,
was used in developing the basis for design related to fire
protection/life safety.

The combustible loads and the use of flammable and/or reactive
materials in the \dword{lbnf-dune} facility are controlled
following the International Building Code building occupancy
classification. Certain ancillary buildings outside the main structure
may be classified as higher hazard areas (Use Group H occupancy),
including the gas cylinder and chemical storage rooms, because they
hold more concentrated quantities of flammable or combustible
materials.  The control area concept used in the International
Building Code and National Fire Protection Association standards will
be followed for hazardous chemical use and storage areas to provide
the most flexibility and control of materials by allowing inventory
thresholds per control area.  The \dword{lbnf-dune} facility
will be equipped with fire detection systems and alarm systems that
will monitor water flow in case of fire suppression activation as well
as monitor control valves and detection systems.

Audible/visual alarm notification devices will alert building
occupants.  Manual pull stations for the fire alarms will be installed
at all building exits.  Following National Fire Protection Association
90A, air handling systems will have photoelectric smoke detectors.
Smoke detection will be provided in areas with highly sensitive
electronic equipment.  Combinations of audible/visual alarm
notification devices will be set up throughout the underground
enclosures and service buildings to alert occupants. All fire alarm
signals will report through a centralized system at \dword{surf}.
Fire alarm and supervisory signals will be transmitted to internal and
external emergency responders using the campus reporting system.

While fixed fire protection systems afford an excellent level
of protection, additional strategies include operational controls
that minimize combustible materials, adequately
fused power supplies, fire safety inspections, and operational
readiness reviews will be used to further reduce fire hazards
within the facility following \dword{doe} highly protected risk methods.

Experimental cabling will meet the requirements of the National Fire
Protection Association 70 and National Electrical Code, 2015 edition.
Preferred cables should be fire resistant, using appropriately
designated cable types for plenum or general-purpose cables.  When
there is a large investment in equipment for experiment power or
computer rack systems or when equipment is custom made (as opposed to
off-the-shelf commercial electronics), a device to detect faults or
smoke in the system should be provided.  This device should also shut
down the individual rack or racks when smoke or faults are detected.


\subsection{Electrical Hazards (LBNF-DUNE HA-6)}

\dword{lbnf-dune} will have significant facility-related
systems and subsystems that produce or use high voltage, high current,
or high levels of stored energy, all of which can present electrical
hazards to personnel. Electrical hazards include electric shock and
arc flash from exposed conductors, defective and substandard
equipment, lack of training, or improper procedures.

\fnal has a well-established electrical safety program that
incorporates deenergizing equipment, isolation barriers, personal
protective equipment, and training. The cornerstone of the program is
the lockout/tagout following the \dword{feshm} Chapter 2100, \fnal
Energy Control Program (Lockout/Tagout).

Design, installation, and operation of electrical equipment will
comply with the National Electrical code (NFPA 70), applicable parts
of Title 29 Code of Federal Regulations, Parts 1910 and 1926, National
Fire Protection Association 70E, and \fnal electrical safety policies
documented in the \dword{feshm} 9000 series chapters. Equipment
procured from outside vendors or international in-kind partners will
be either certified by a nationally recognized testing laboratory,
conform to international standards previously evaluated and deemed
equivalent to US standards, or inspected and accepted using \fnal's
electrical equipment inspection policies outlined in \dword{feshm}
9110, Electrical Utilization Equipment Safety.


\subsection{Noise/Vibration/Thermal/Mechanical (LBNF-DUNE HA-7)}

Hazards include overexposure of personnel noise and vibrations as
specified by the American Conference of Governmental Industrial
Hygienists and US Occupational Safety and Health Administration
(OSHA), which set noise limits to avoid permanent hearing loss, also
known as permanent threshold shift . Vibration of equipment can
contribute to noise levels and could damage or interfere with
sensitive equipment.

\dword{lbnf-dune} will use a wide variety of equipment that
will produce a wide range of noise and vibration. Support equipment,
such as pumps, motors, fans, machine shops, and general HVAC, all
contribute to point source and overall ambient noise levels. While
noise will typically be below the ACGIH and OSHA 8-hour time weighted
average, certain areas with mechanical equipment could exceed that
criterion and will require periodic monitoring, posting, and use of
protective equipment. Ambient background noise is more a concern for
collaborator comfort, stress levels, and fatigue.

The detector facilities use a wide variety of noisy equipment. Pumps,
fans, and machine shop devices, among others, are possible sources of
noise levels that might exceed the \fnal noise action
levels. \dword{feshm} Chapter 4140, Hearing Conservation, details
requirements for reducing noise and protecting personnel exposed to
excessive noise levels. Warning signs are posted wherever hazardous
noise levels may occur, and hearing protection devices are readily
available. Ways to reduce noise and vibration will be incorporated
into the \dword{lbnf-dune} design. These techniques include
using low noise/vibration producing equipment, especially for fans in
the HVAC equipment, isolating noise producing equipment by segregating
or enclosing it, and using sound deadening materials on walls and
ceilings.

\subsection{Cryogenic/Oxygen Deficiency Hazard (LBNF-DUNE HA-8)}

The \dword{lbnf-dune} project will use large volumes of liquid
argon, nitrogen, and helium within the Far Site facilities. Cryogenic
hazards could include oxygen deficient atmospheres due to failure of
the cryogenic systems, thermal (cold burn) hazards from cryogenic
components, and pressure hazards. Initiators could include the failure
or rupture of cryogenic systems from overpressure, failure of
insulating vacuum jackets, mechanical damage or failure, deficient
maintenance, or improper procedures.

Cryogenic liquids and gasses are extremely dangerous to humans,
destroying tissue, and can damage materials and equipment past repair
by altering characteristics and properties (like size, strength, and
flexibility) of metals and other materials.

Although cryogens are used extensively at \fnal, quantities that may
be used within a facility are strictly limited. Uses beyond defined
limits require oxygen deficiency hazard analyses and using
ventilation, oxygen deficiency monitoring, or other controls.

Cryogenic systems are subject to formal project review, which includes
independent reviews by a subpanel of the Cryogenic Safety Subcommittee
following National Fire Protection Association Chapter 5032, Cryogenic
System Review. The members of this panel have relevant knowledge in
appropriate areas. They review the system safety documentation,
\dword{odh} analysis documentation, and the equipment before new
systems are permitted to begin the cool down process.

\fnal has developed and successfully deployed \dword{odh} monitoring
systems throughout the laboratory to support its current cryogenic
operations. The systems provide both local and remote alarms when
atmospheres contain less than 19.5\% oxygen by volume.

\fnal has a mature training program to address cryogenic safety
hazards. Key program elements include \dword{odh} training,
pressurized gas safety, and general cryogenic safety.


\subsection{Confined Space Hazards (LBNF-DUNE HA-9)}

Hazards from confined spaces could result in death or injury from
asphyxiation, compressive asphyxiation, smoke inhalation, or impact
with mechanical systems. Initiators would include failure of cryogenic
systems that are releasing liquid, gas, fire, or failure of mechanical
systems.

The \fnal confined space program is outlined in \fnal Environmental
and Safety Manual Chapter 4230, Confined
Spaces. \dword{lbnf-dune} facilities will be incorporated into
this program. The emphasis at the \dword{lbnf-dune} design
phase will be to create the minimum number of confined spaces by
clearly articulating the definition of confined spaces to facility
designers to assure that such spaces have adequate egress, mechanical
spaces are adequately sized, and, wherever possible, no confined space
created at all. During facility operations, the existing campus
confined space program, along with appropriate labeling of confined
spaces, work planning and control, and entry permits will be used to
control access to these spaces.

\subsection{Chemical/Hazardous Materials Hazards (LBNF-DUNE HA-11)}

The \dword{dune} facility anticipates minimal use of chemical and hazardous
materials. Materials like paints, epoxies, solvents, oils, and lead
shielding may be used during construction and operation of the
facility. Exposure to these materials could result in injury;
exposure could also exceed regulatory limits. Initiators could be
experimental operations, transfer of material, failure of packaging,
improper marking/labeling, reactive or explosive event, improper
selection of or lack of personal protective equipment, or a
natural phenomenon.

\fnal maintains an database of hazardous chemicals in compliance with
the requirements imposed by 10 CFR 851 and \dword{doe} orders. In
addition to an inventory of chemicals at the facility, copies of each
manufacturer's safety data sheets are maintained. Reviews of
conventional safety measures at the facilities show that using these
chemicals does not warrant special controls other than appropriate
signs, procedures, appropriate use of personal protective equipment,
and hazard communication training. \dword{dune} will also supply
safety data sheet documentation to the \dword{surf} \dword{esh}
department for all chemicals and hazardous materials that arrive on
site.

The industrial hygiene program, detailed in the \dword{feshm} 4000
series chapters, addresses potential hazards to workers using such
materials. The program identifies how to evaluate workplace hazards
when planning work and the controls necessary to either eliminate or
mitigate these hazards to an acceptable level.

Specific procedures are also in place for safe handling, storing,
transporting, inspecting, and disposing of hazardous materials. These
are contained in the \dword{feshm} 8000 and 10000 series chapters,
Environmental Protection and Material Handling and Transportation,
which describe how to comply with the standards set by the Code of
Federal Regulations, Occupational Safety and Health Standards, Hazard
Communication, Title 29 CFR, Part 1910.1200.


\subsection{Lasers \& Other Non-Ionizing Radiation Hazards (LBNF-DUNE HA-14)}
\label{sec:tc-esh-lasers}

Production and delivery of Class 3B and Class 4, near-infrared, UV,
and visible lasers must be completely contained in transport pipes or
designated enclosures for the Class 3b and Class 4 lasers, thus
creating a laser controlled area. (This will be in accordance with
\fnal \dword{feshm} chapter 4260.)  Establishing the laser controlled
area prevents areas around it from exceeding the maximum permissible
exposure as set by the \fnal laser safety officer.

\subsection{Material Handling Hazards (LBNF-DUNE HA-15)}

\dword{dune} will require a significant amount of manual and mechanical
material handling during the construction, installation, and operations
phases.  The consequences of these hazards include serious injury or
death to equipment operators and bystanders, damage to equipment and
structures, and interruption of the program.  Additional material
handling hazards from forklift and tow cart operations include injury
to the operator or personnel in the area and contact with equipment or
structures. Cranes and hoists will be used during fabrication,
testing, removal, and installation of equipment. The error precursors
associated with this type of work may include irregular shaped loads,
awkward load attachments, limited space, obscured sight lines, and
poor communication.  The material or equipment being moved is
typically one of a kind, expensive or of considerable programmatic
value, and without dedicated lifting points or an obvious center
of gravity.

Lessons learned from across the \dword{doe} complex and OSHA have been
evaluated and incorporated into the \fnal material handling programs
documented in the \dword{feshm} 10000 series chapters.  The laboratory
limits personnel with access to mechanical material handling equipment
like cranes and forklifts to those who have successfully completed the
laboratory's training programs and demonstrated competence in
operating this equipment.


\subsection{Experimental Operations (LBNF-DUNE HA-16)}
Experimental activity undertaken at \dword{lbnf-dune} will be
fully reviewed under the operational readiness clearance (ORC) process
and by other experts as needed (e.g., representatives from electrical
safety, fire safety, environmental compliance, industrial hygiene,
cryogenic safety, and industrial safety), to identify and manage the
hazards of each experimental operation. The shift leader will ensure
that all safety reviews take place for each activity and that any
issues are appropriately addressed. The ORC process will document
these reviews, covering the necessary controls and management approval
to proceed.

Typically, the ORC process evaluates the scope of the proposed
experimental activity and identifies the hazards and controls to
mitigate them. The process ensures that collaborators are properly
trained, that qualified, hazardous material is kept to a minimum, that
engineering controls are deployed as a preferred mitigation, and that
personnel protective equipment is appropriate for the hazard.


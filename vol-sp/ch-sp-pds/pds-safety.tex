%%%%%%%%%%%%%%%%%%%%%%%%%%%%%%%%%%%%%%%%%%%%%%%%%%%%%%%%%%%%%%%
\section{Safety}
\label{sec:fdsp-pd-safety}
%\metainfo{\color{blue} Content: Warner}
%\metainfo{(Length: \dword{tdr}=5 pages, TP=1 pages)}

Safety management practices will be critical for all phases of the photon system assembly, and testing.  Planning for safety in all phases of the project, including fabrication, testing, and installation will be part of the design process.  The initial safety planning for all phases will be reviewed and approved by safety experts as part of the initial design review.  All component cleaning, assembly, testing,  and installation procedure documentation will include a section on safety concerns relevant to that procedure and will be reviewed during the appropriate pre-production reviews.

Areas of particular importance to the \dword{pds} include
\begin{itemize}
\item Hazardous chemicals (particularly WLS chemicals such as \dword{ptp} used in filter plate coating) and cleaning compounds:  All potentially hazardous chemicals used will be documented at the consortium management level, with MSDS and approved handling and disposal plans in place.

\item Liquid and gaseous cryogens used in module testing:  Full hazard analysis plans will be in place at the consortium management level for all module or module component testing involving cryogenic hazards, and these safety plans will be reviewed in the appropriate pre-production and production reviews.

\item High voltage safety:  Some of the candidate \dwords{sipm} require bias voltages above \SI{50}{VDC} during warm testing (although not during cryogenic operation), which may be a regulated voltage as determined by specific laboratories and institutions.  Fabrication and testing plans will demonstrate compliance with local HV safety requirements at the particular institution or laboratory where the testing or operation is performed, and this compliance will be reviewed as part of the standard review process.

\item UV and VUV light exposure:  Some \dword{qa} and \dword{qc} procedures used for module testing and qualification may require use of UV and/or VUV light sources, which can be hazardous  to unprotected operators.  Full safety plans must be in place and reviewed by consortium management prior to beginning such testing.

\item Working at heights, underground:  Some aspects of \dword{pd}S module fabrication, testing and installation may require working at heights or deep underground. Personnel safety will be an important factor in the design and planning for these operations, all procedures will be reviewed prior to implementation, and all applicable safety requirements at the relevant institutions will be observed at all times.
%

\end{itemize}

%%%%%%%%%%%%%%%%%%%%%%%%%%%%%%%%%%%%%%%%%%%%%%%%%%%%%%%%%%%%%%%%
\section{Quality Assurance and Quality Control}
\label{sec:fdsp-pd-qaqc}
%\metainfo{Content: Warner}

The quality assurance and quality control programs for the \dword{fd} are based on our experience with the \dword{pdsp}.  Our design-phase quality management system is based upon that experience.  Following completion of the 60-percent design review, we will develop a quality final assurance program focused on final specifications and drawings, and developing a formal set of fabrication procedures along with detailed QC and test plans.

During fabrication, integration into the detector, and detector installation into the cryostat, our QC plan will be carefully followed, including incoming materials and other inspection reports, fabrication travelers, and formal test result reports entered into the DUNE QA/QC database.

Particular steps in this process are detailed below.

\subsection{Design Quality Assurance}
\label{sec:fdsp-pd-designqa}

\dword{pd} design \dword{qa} focuses on ensuring that the detector modules meet the following goals:
\begin{itemize}
\item physics goals as specified in the DUNE requirements document;
\item interfaces with other detector subsystems as specified by the subsystem interface documents; and
\item materials selection and testing to ensure non-contamination of the \lar volume.
\end{itemize}

The \dword{pds} consortium will perform the design and fabrication of the components in accordance with the applicable requirements of the LBNF/DUNE \dword{qa} plan. If the institute (working under the supervision of the consortium) performing the work has a documented \dword{qa} program, the work may be performed in accordance with their own program.

Upon completion of the \dword{pds} design and \dword{qa}/\dword{qc} plan, there will be a pre-production review process, with the reviewers charged to ensure that the design demonstrates compliance with the goals above.

\subsection{Production and Assembly Quality Assurance}
\label{sec:fdsp-pd-prodqa}

The \dword{pds} will undergo a \dword{qa} review for all components prior to completion of the design and development phase of the project.  The \dword{pdsp} test will represent the most significant test of near-final \dword{pd} components in a near-DUNE configuration, but additional tests will also be performed.  The \dword{qa} plan will include, but not be limited to, the following areas:

\begin{itemize}
\item materials certification (in the \dword{fnal} materials test stand and other facilities) to ensure materials compliance with cleanliness requirements;
\item cryogenic testing of all materials to be immersed in \lar, to ensure satisfactory performance through repeated and long-term exposure to \lar{}.  Special attention will be paid to cryogenic behavior of fused silica and plastic materials (such as filter plates and wavelength-shifters), \dwords{sipm}, cables and connectors.  Testing will be conducted both on small-scale test assemblies (such as the small test cryostat at CSU) and full-scale prototypes (such as the full-scale CDDF cryostat at CSU). 
%(Add pictures?)
\item mechanical interface testing, beginning with simple mechanical go/no-go gauge tests, followed by installation into the \dword{pdsp2} system, and finally full-scale interface testing of the \dword{pds} into the final pre-production TPC system models; and
\item full-system readout tests of the \dword{pd} readout electronics, including trigger generation and timing, including tests for electrical interference between the TPC and \dword{pd} signals.
\end{itemize}


Prior to beginning construction, the \dword{pds} will undergo a final design review, where these and other \dword{qa} tests will be reviewed and the system declared ready to move to the pre-production phase.


\subsection{Production and Assembly Quality Control}
\label{sec:fdsp-pd-prodqc}

%\fixme{Modified this to clarify production testing plan, per Mont comment  DWW:  DONE}

Prior to the start of fabrication, a manufacturing and \dword{qc} plan will be developed detailing the key manufacturing, inspection, and test steps.  The fabrication, inspection, and testing of the components will be performed in accordance with documented procedures. This work will be documented on travelers and applicable test or inspection reports. Records of the fabrication, inspection and testing will be maintained. When a component has been identified as being in noncompliance to the design, the nonconforming condition shall be documented, evaluated, and dispositioned as: {\it use-as-is} (does not meet design but can meet functionality as it is), {\it rework} (bring into compliance with design), {\it repair} (will be brought to meet functionality but will not meet design), and {\it scrap}. For products with a disposition of accept, as is, or repair, the nonconformance documentation shall be submitted to the design authority for approval.   

All \dword{qc} data  (from assembly and pre- and post-installation into the \dword{apa}) will be directly stored to the DUNE database for ready access of all \dword{qc} data.  Monthly summaries of key performance metrics (to be defined) will be generated and inspected to check for quality trends.

Based on the \dword{pdsp} model, we expect to conduct the following production testing:

Prior to shipping from assembly site:
\begin{itemize}
\item dimensional checks of critical components and completed assemblies to insure satisfactory system interfaces;
\item post-assembly cryogenic checkouts of \dword{sipm} mounting PCBs (prior to assembly into \dword{pd} modules);
\item module dimensional tolerances using go/no-go gauge set; and
\item warm scan of complete module using motor-driven \dword{led} scanner (or UV \dword{led}  array).
\end{itemize}

Following shipping to the US reception and checkout facility but prior to storage at \dword{sdwf}:
\begin{itemize}
\item mechanical inspection;
\item warm scan (using identical scanner to initial scan); and
\item cryogenic testing of completed modules (in CSU CDDF or similar facility).
\end{itemize}

Following delivery to integration clean room underground, prior to and during integration and installation:
\begin{itemize}
\item warm scan (using identical scanner to initial scan);
\item complete visual inspection of module against a standard set of inspection points, with photographic records kept for each module;
\item end-to-end cable continuity and short circuit tests of assembled cables; and
\item an \dword{fe} electronics functionality check.
\end{itemize}

\subsection{Installation Quality Control}
\label{sec:fdsp-pd-installqc}

\dword{pds} pre-installation testing will follow the model established for \dword{pdsp}.  Prior to installation in the \dword{apa}, the \dword{pd} modules will undergo a warm scan in a scanner identical to the one at the \dword{pd} module assembly facility and the results compared.  In addition, the module will undergo a complete visual inspection for defects and a set of photographs of selected critical optical surfaces taken and entered into the \dword{qc} record database.  Following installation into the \dword{apa} and cabling, an immediate check for electrical continuity to the \dwords{sipm} will be conducted.

Following the mounting of the TPC cold electronics and the photon detectors, the entire \dword{apa} will undergo a cold system test in a gaseous argon cold box, similar to that performed during \dword{pdsp}.  During this test, the \dword{pds} system will undergo a final integrated system check prior to installation, checking dark and \dword{led}-stimulated \dword{sipm} performance for all channels, checking for electrical interference with the cold electronics, and confirming compliance with the detector grounding scheme.
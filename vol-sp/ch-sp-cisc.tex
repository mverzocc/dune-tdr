\chapter{Cryogenics Instrumentation and Slow Controls}
\label{ch:sp-cisc}


%%%%%%%%
\section{Introduction} 

The \dfirst{cisc} consortium provides comprehensive monitoring for all detector  components and for \dword{lar} quality and behavior as well as a control system for many detector components.
The \dword{sp} and \dword{dp} modules both use the same control
system and have nearly identical cryogenics instrumentation except
for differences in location due to the different \dword{tpc}
geometries and the addition of dedicated instrumentation for
monitoring temperature and pressure in the gas phase for the
\dword{dpmod}.  \dpchcisc %The corresponding chapter of Volume~4 
of this
\dword{tdr} is virtually the same as this chapter apart from
those few differences.


The consortium responsibilities are split into 
two main branches: cryogenics instrumentation and slow controls, as illustrated in  Figure~\ref{fig:cisc-subsystem-chart}. % provides a subsystem chart for the \dword{cisc} system. 

\begin{dunefigure}[CISC subsystem chart]{fig:cisc-subsystem-chart}
  {\dword{cisc} subsystem chart}
  \includegraphics[width=0.8\textwidth]{graphics/CISC_scope_SP_2019Apr15.png}
\end{dunefigure}

Each element of \dword{cisc} contributes to the DUNE physics program primarily through the maintenance of high detector live time.  As described in Volume~\volnumberphysics{}, \voltitlephysics{}, of this \dword{tdr}, neutrino \dword{cpv} and resolution of the neutrino mass hierarchy over the full range of possible neutrino oscillation parameters will require at least a decade of running the \dword{fd}.  Similar requirements apply to searches for nucleon decay and \dword{snb} events from within our galaxy.  Throughout this long run-time the interior of any DUNE cryostat remains completely inaccessible.  No possibility exists for repairs to any components that could be damaged within the \dword{tpc} structure; hence environmental conditions that present risks must be detected and reported quickly and reliably. 
 
Detector damage risks peak during the initial fill of a module with \dword{lar}, as temperature gradients take on their highest values during this phase.  Thermal contractions outside of the range of design expectations could result in broken \dword{apa} wires, \dwords{sipm} in \dwords{pd} that detach from the \dword{xarapu} light detectors, or poor connections at the cathode \dword{hv} feedthrough point that could lead to unstable \efield{}s.  These considerations lead to the need for %design of 
a robust temperature monitoring system for the detector, supplemented with liquid level monitors, and a high-performance camera system to enable visual inspection of the interior of the cryostat %through 
during the filling process.  These systems are fully described in Section~\ref{sec:fdsp-cryo-therm} of this chapter.
 
Argon purity must be established as early as possible in the filling process, a period in which gas analyzers are most useful, and must maintain an acceptable value, corresponding to a minimum electron drift lifetime of \SI{3}{ms}, throughout the data-taking period.  Dedicated  purity monitors (Section~\ref{sec:fdgen-slow-cryo-purity-mon}) 
provide precise lifetime measurements up to values of \SI{10}{ms}, the range over which electron attenuation most affects \dword{s/n} in the \dword{tpc}.  The purity monitors and gas analyzers remain important even after high lifetime has been achieved as periodic detector ``top-off''  fills occur; the new \dword{lar} must be of very high quality as it is introduced into the cryostat.
 
The \dword{cisc} system must recognize and prevent fault conditions that could develop in the \dword{detmodule} over long periods of running.  For example, the liquid level monitors must register any drop in liquid level; a drop in the level could place top sections of the \dword{fc} or bias \dword{hv} points  for the \dword{apa}s close enough to the gas-liquid boundary to trigger sparking events. %; the liquid level monitors must head off this possibility.  
Very slow-developing outgassing phenomena could conceivably occur, with associated bubble generation creating another source of \dword{hv} breakdown events.  The cold camera system enables detection and identification of bubbling sites, and the development of mitigation strategies such as lower \dword{hv} operation for some period of time.  A more subtle possibility is the formation of quasi-stable eddies in argon fluid flow that could prevent positive argon ions from being cleared from the \dword{tpc} volume, resulting in space charge build up that would not otherwise be expected at the depth of the \dword{fd}.  The space charge could in turn produce distortions in the \dword{tpc} drift field that degrade tracking and calorimetry performance.  The high-performance thermometry  of the DUNE \dword{cisc} system creates input for well developed complex fluid flow models described in Section~\ref{sec:fdgen-cryo-cfd} that should enable detection of  conditions associated with these eddies.
 
Finally, a high detector live-time fraction over multi-year operation cannot be achieved without an extensive system to monitor all aspects of detector performance, report this information in an intelligent fashion to detector operators, and archive the data for deeper offline studies.  Section~\ref{sec:sp-cisc-slowctrl} details the DUNE slow controls system designed for this task.
 
The baseline designs for all the \dword{cisc} systems have been used in \dfirst{pdsp}, % designs arethe baseline for all \lar instrumentation devices, and %requirements for 
and most design
parameters are extrapolated from these designs. The \dword{pdsp} data (and in some cases \dword{pddp} data) will therefore be used to validate the instrumentation designs and to understand their performance.

%\subsection{Components}
\subsection{Scope}
%\fixme{I am merging this with scope. Anne}

\subsubsection{Cryogenics Instrumentation}
%The devices included under c
Cryogenics instrumentation includes purity monitors,  various types of temperature monitors, and cameras with their associated light emitting systems. Also included are %components like 
gas analyzers and \dword{lar} level monitors that are directly related to the external cryogenics system, which have substantial interfaces with the \dword{lbnf}. \dword{lbnf} provides the needed expertise  for these systems and is responsible for the design, installation, and commissioning, while the \dword{cisc} consortium provides the resources and supplements labor as needed. 

A \dword{citf} for the instrumentation devices is also part of the cryogenics instrumentation.
 \dword{cisc} is responsible for design through commissioning in the \dword{spmod} %\dfirst{fd}.
of \dword{lar} instrumentation devices: purity monitors, thermometers, capacitive level meters, cameras, and light-emitting system, and their associated feedthroughs.
%and the \dword{citf}.

Cryogenics instrumentation %also 
requires significant engineering, physics, and
simulation work, such as \efield simulations and cryogenics modeling
studies using \dfirst{cfd}. \efield simulations
%are required to 
identify desirable locations for instrumentation
devices in the cryostat, away from %so they are not in 
regions of high \efield, so that %and 
their presence does not induce large field distortions. 
\dword{cfd} simulations help identify %are needed to understand 
expected temperature, impurity, and velocity flow distributions and guide the placement and distribution of instrumentation devices inside the cryostat.

%%%%%%%%%%%%
\subsubsection{Slow Controls}
\label{sec:sp-cisc-slowctrl}
The slow controls portion of \dword{cisc} consists of three main components: 
hardware, infrastructure, and software. The slow controls hardware and infrastructure comprises networking hardware, signal processing hardware, computing hardware, and associated rack infrastructure. The slow controls software provides, for every slow control quantity, the central slow controls processing architecture, databases, alarms, archiving, and control room displays.

\dword{cisc} provides software and infrastructure for controlling and monitoring all detector elements that provide data on the health of the \dword{detmodule} or conditions important to the experiment, as well as  some related hardware. 
%Slow controls includes the systems detailed below.

%\textbf{Slow controls base software and databases:}
Slow controls base software and databases are the central tools needed to develop
control and monitoring for various detector systems and interfaces. These include:
\begin{itemize}
\item base input/output software;
\item alarms, archiving, display panels, and similar operator interface tools; and 
\item slow controls system documentation and operations guidelines.
\end{itemize}

%\textbf{Slow controls for external systems:} 
Slow controls for external systems collect data from systems
external to the \dword{detmodule} and provide status monitoring for operators
and archiving. %These systems include 
They %will 
collect data on beam status, cryogenics status,
\dword{daq} status, facilities systems status, interlock
status bit monitoring (but not the actual interlock mechanism), ground
impedance monitoring, and possibly building and detector hall
monitoring, as needed.

The \dword{ddss} can provide inputs to \dword{cisc} on safety interlock status, and \dword{cisc} will monitor and make that information available to the experiment operators and experts as needed. However, \dword{ddss} and \dword{cisc} are separate monitors, and the slow controls portion of \dword{cisc} does not provide any inputs to \dword{ddss}. A related question is whether \dword{cisc} can provide software intervention before a hardware safety interlock. In principle such intervention can be implemented in \dword{cisc}, presumably by (or as specified by) the hardware experts. For example, at \dword{pdsp}, the automatic lowering of \dword{hv} to clear streamers was implemented in the software for the \dword{hv} control using \dword{cisc}-level software. 


Slow controls %for detector hardware systems %develop 
covers software interfaces for detector hardware devices, including:
\begin{itemize}
\item monitoring and control of all power supplies,
\item full rack monitoring (rack fans, thermometers and rack protection system),
\item instrumentation and calibration device monitoring (and control to the extent needed),
\item power distribution unit and computer hardware monitoring,
\item \dword{hv} system monitoring through cold cameras, and
\item detector components inspection %through 
using warm cameras.
\end{itemize}
%
%\textbf{Slow controls hardware:} 
\dword{cisc} will develop, install, and commission any hardware related to rack monitoring and control. Most power supplies may only need a cable from the device
to an Ethernet switch, but some power supplies might need special cables (e.g., GPIB or RS232) for communication. The \dword{cisc} consortium is responsible for providing these control cables.

%In addition to the listed activities, 
\dword{cisc} %also has 
has additional activities outside the scope of the consortium that require coordination with other groups. This is discussed in Section~\ref{sec:interfaces}.

%\subsection{Requirements}
\subsection{Design Considerations}

%Some common 
Important design considerations for instrumentation devices include stability, reliability, and longevity, so that devices can survive for at least \dunelifetime.
Such longevity is uncommon for any device, so the overall design allows replacement of devices where possible.
Some devices are critical for filling and commissioning but less critical for later operations; for these devices we specify a minimum lifetime of 18 months and 20 years as a desirable goal.
DUNE requires the \efield  on any instrumentation devices inside the cryostat to be less than \localefield to minimize the risk of dielectric breakdown in \dword{lar}. 

A consideration important for event reconstruction is the maximum noise level induced by instrumentation devices that the readout electronics  can tolerate. \dword{pdsp} is evaluating this. 
Table~\ref{tab:specs:SP-CISC} shows the top-level specifications that determine the requirements for \dword{cisc} together with selected high-level specifications for \dword{cisc} subsystems. The physics-driven rationale for each requirement and the proposed validation are also included in the table.
Tables~\ref{tab:fdgen-slow-cryo-requirements-1} and \ref{tab:fdgen-slow-cryo-requirements-2} show the full set of specifications 
for the \dword{cisc} subsystems. In all these tables two values are quoted for most of the design parameters: 
 1) specification, which is the intended value or limits for the parameter set by physics and engineering needs, and 2) goal, an improved value offering a benefit which the collaboration aims to achieve where it is cost-effective to do so.
% i) {\it specification}, which is the minimum requirement to guarantee baseline performance, and ii) {\it goal},  
% an ideal requirement enabling more detailed studies and for achieving improved precision.
%an improved version enabling more detailed studies which could lead to an improved design of the cryogenics system for subsequent detectors. 

Data from purity monitors and different types of thermometers will be used to validate the \dword{lar} fluid flow model. 
A number of requirements drive the design parameters for the precision and granularity of monitor distribution across the cryostat. 
For example, the electron lifetime measurement precision must be \SI{1.4}{\%} to keep the bias on the charge readout in the \dword{tpc} below \SI{0.5}{\%} at \SI{3}{ms} lifetime. For thermometers, the %requirements 
parameters are driven by the \dword{cfd} simulations based on \dword{pdsp} design.
The temperature measurement resolution must be less than \SI{2}{mK}, and the relative precision of those measurements must be less than \SI{5}{mK}. The resolution is defined as the temperature \dword{rms}  for individual measurements and is driven by the electronics. The relative precision also includes the effect of reproducibility for successive immersions in \dword{lar}. 
The relative precision is particularly important in order to characterize %because 
gradients below \SI{20}{mK}. % should be characterized.  
As will be described below, the laboratory calibration data and the recent analysis of thermometer instrumentation data from \dword{pdsp} shows that a \SI{2.5}{mK} relative precision is achievable. 

% Different from dp here
The level meters must have a precision of 0.1\% over \SI{14}{m} (i.e., \SI{14}{mm}) for measurement accuracy during filling. This precision is also sufficient to ensure that the \dword{lar} level stays above the \dwords{gp} of a \dfirst{sp} module. As shown in Table~\ref{tab:fdgen-slow-cryo-requirements-2}, several requirements drive the design of cold and warm cameras and the associated light emitting system. The components of the camera systems must not contaminate the \dword{lar} or produce bubbles % when the \dword{hv}
so as  to avoid  increasing the risk of \dword{hv} discharge. Both cold and warm cameras must provide coverage of at least \SI{80}{\%} of the \dword{tpc} volume 
with a resolution of \SI{1}{cm} for cold cameras and \SI{2}{mm} for warm cameras on the \dword{tpc}.

For the \dword{citf}, a cryostat with a capacity of only \num{0.5} to approximately \SI{3}{m^3} %cubic meter is reasonable because 
will suffice and will keep turn-around times and filling costs %are 
lower. % for smaller cryostats. 
For gas analyzers, the operating range must allow establishment of useful electron lifetimes;
% is an important requirement; 
details  are in Table~\ref{tab:fdgen-slow-cryo-requirements-1}.

For slow controls, the system must
be sufficiently robust to monitor 
a minimum of 150,000 variables per \dword{detmodule}, and 
support a broad range of monitoring and archiving rates;
the estimated variable count, data rate, and archive storage needs
are discussed in Section \ref{sec:fdgen-slow-cryo-quant}.
The system must also
interface with a large number of detector subsystems and
establish two-way communication with them for control and monitoring. 
For the alarm rate, 150 alarms/day is used as the specification as it is the maximum to which humans can be expected to respond. The goal for the alarm rate is less than 50 alarms/day. The alarm logic system will need to include features for managing ``alarm storms'' using alarm group acknowledgment, summaries, delays, and other aids.

%\fixme{get ITF out of spec table!}
\input{generated/req-longtable-SP-CISC}


\begin{dunetable}
[Specifications for CISC subsystems (1)]
{p{0.45\linewidth}p{0.25\linewidth}p{0.25\linewidth}}
{tab:fdgen-slow-cryo-requirements-1}
{List of specifications for the different CISC subsystems (1).}   
Quantity/Parameter				                             & Specification			                                        & Goal		                                              \\ \toprowrule                     
Noise from Instrumentation devices				             & $\ll$ \elecnoisefe                                      & 
\\ \colhline                     
Max. \efield near instrumentation devices				     & <\localefield			                                                & <15 kV/cm		                                          \\ \colhline                     
\textbf{Purity Monitors}	                                             &                                                                      &                                                         \\ \colhline                      
Precision in electron lifetime				                 & <1.4\% at 3~ms,  <4\% at 9~ms,  relative differences <2.5\%			                                            & < 1\%		                                              \\ \colhline                     
Range in electron lifetime				                     & 0.04 - 10 ms  			                    & (0.04 - 30 ms inline)       
\\ \colhline                         
Longevity				                                     & \dunelifetime			                                                    & > \dunelifetime		                                      \\ \colhline                     
Stability				                                     & Match precision requirement at all places/times			    & %Match precision requirement at all places/times  
\\ \colhline  	                   
Reliability				                                     & Daily Measurements			                                        & Measurements %capable of being 
as needed	  \\ \colhline                         
\textbf{Thermometers}	                                             &                                                                      &                                                         \\ \colhline                      
Vertical density of sensors for T-gradient monitors			 & > 2 sensor/m			                                                & > 4 sensors/m		                                      \\ \colhline                 
2D horizontal density for top/bottom individual sensors		 &  1 sensor/5(10) m 			                                        &  1 sensor/3(5) m 		                          
        \\ \colhline 
Swinging/deflection of T-Gradient monitors                   &  < 5 cm 			                                                    &  < 2 cm 		          
        \\ \colhline                     
Resolution of temperature measurements				         & < 2 mK			                                                    & <0.5 mK		                                          \\ \colhline                         
Precision: temperature reproducibility 				         & < 5 mK			                                                    & 2 mK		                                              \\ \colhline                     
Reliability				                                     & 80\% (in 18 months)			                                        & 50\% (during 20 years)		                              \\ \colhline                     
Longevity				                                     & > 18 months			                                                & > 20 years		                                      \\ \colhline                         
Stability 	  &  < \SI{2}{mK} at all places and times	 &   Match precision requirement at all places/times \\ \colhline                 
Discrepancy between lab and  in situ calibrations for temperature sensors			             & < 5 mK			                                                    & < 3 mK		                                          \\ \colhline                           
Discrepancy between measured temperature map and CFD simulations in \dword{pdsp}	 & < 5 mK & %< 5 mK		     
\\ \colhline                             
\textbf{Gas Analyzers}	   &   &  \\ \colhline            
Operating Range O2	 & 0.2 (air) to 0.1 ppt  & %Air (0.2) to 0.1 ppt
\\ \colhline    
Operating Range H2O				                             & %Nominally Air to sub ppb levels, depending on species of contaminant	
Nom. air to sub ppb; contaminant-dependent & %Air to sub ppb.  levels, depending on species of contaminant	          
\\ \colhline           
Operating Range N2				                             & Nominally Air Nom. air to sub ppb; contaminant-dependent	& %Air to sub ppm.
\\ \colhline             
Precision: 1 sigma at zero				                     & %depends on the range of the gas analyzer
per gas analyzer range
& %depends on the range of the gas analyzer		                     
\\ \colhline     
Detection limit: 3 sigma & Different analyzer modules needed to cover entire range	& %Different Gas analyzer modules are needed to cover the entire range 
\\ \colhline           
Stability   & <\% of full scale range.		 & %<\% of full scale range.
\\ \colhline         
Longevity		 & >10 years	  & %10 years  
\\   \colhline
\textbf{Pressure Meters (GAr)}	          &    &          \\ \colhline            
Relative precision (DUNE side)		   & 0.1~mbar	& %0.1\% over 14 m (14 mm)
\\ \colhline  
Absolute precision (DUNE side)		   & <5~mbar	& %0.1\% over 14 m (14 mm)
\\  
\end{dunetable}


\begin{dunetable}
[Specifications for CISC subsystems (2)]
{p{0.45\linewidth}p{0.25\linewidth}p{0.25\linewidth}}
{tab:fdgen-slow-cryo-requirements-2}
{List of specifications for the different CISC subsystems (2)}   
Quantity/Parameter		     & Specification	  & Goal   \\ \toprowrule   
\textbf{Level Meters}	          &    &          \\ \colhline            
Precision (LBNF scope)		   & 0.1\% over 14 m (14 mm)			                                    & %0.1\% over 14 m (14 mm)
\\ \colhline           
Precision (capacitive level meters, \dword{dune} scope) & 1~cm  &  <5 mm
\\ \colhline         
Longevity (all)		  & 20 years	   & > 20 years		                                                  \\ \colhline     
\textbf{Cold cameras}	                                             &                                                                      &                                                                     \\ \colhline        
Coverage				                                     & 80\% of the exterior of HV surfaces			                        & 100\% 	                                                          \\ \colhline         
Frames per second	   & yet to be defined	  & %yet to be defined
\\ \colhline             
Resolution 	 & 1 cm on the \dword{tpc}	 & yet to be defined
\\ \colhline           
Duty cycle	  & yet to be defined	 & %yet to be defined
\\ \colhline         
longevity			 & > 18 months			                                                & > 20 years		                                                  \\ 
\textbf{Inspection cameras}	     &                                                                      &                                                                     \\ \colhline        
Coverage	 & 80\% of the \dword{tpc}		  & yet to be defined		                                              \\ \colhline         
Frames per second		   & yet to be defined	   & %yet to be defined	
\\ \colhline             
Resolution 	  & 2 mm on the \dword{tpc}			                                            & yet to be defined		                                              \\ \colhline           
heat transfer	  & no generation of bubbles			                                & 	%no generation of bubbles	
\\ \colhline         
longevity			  & > 18 months			                                                & > 20 years		                                                  \\ \colhline         
\textbf{Light emitting system}	                                     &                                                                      &                                                                     \\ \colhline        
radiant flux     & > 10 mW/sr		 & 100 mW/sr \\ \colhline         
power				     & < 125 mW/LED			                                        & %\dword{alara}	
\\ \colhline           
wavelength	   & red/green		 & IR/white	   \\ \colhline         
longevity	  & > 18 months (for cold cameras) 			                            & > 20 years		                                              \\ \colhline         
\textbf{\dfirst{citf}}	                 &                                                                      &                                                                     \\ \colhline            
Dimensions		  & 0.5 to 3  cubic meters 			                                    & %0.5-3 cubic meters
\\ \colhline             
Temperature stability	 & $\pm$1K	 & %+- 1K
\\ \colhline                                       
Turn-Around time	 & $\sim\,$9 days   & 9 days 	  \\ \colhline                                       
LAr purity		   & O2, H2O: low enough  to measure drifting electrons of devices under test, $\sim\,\SI{0.5}{ms}$.    N2: ppm for scintillation light tests. 	        &  >1.0 ms                                                            \\ \colhline
\textbf{Slow Controls}		                                         &                                                                      &                                                                     \\ \colhline
Alarm rate	  & <150/day			                                                    &  < 50/day                                                           \\ \colhline
Total No. of variables per \dword{detmodule}				                         & 150,000			                                                    &  150,000 - 200,000                                                   \\ \colhline
Server rack space				                             & 2 racks			                                                    &  3 racks                                                            \\ \colhline
Archiving rate 				                                 & 0.02 Hz			                                                    &  Broad range 1 Hz  to 1 per few min.                                \\ \colhline
Near Detector Status & Beam conditions and detector status	                                &  Full beam and detector status                                      \\          
\end{dunetable}                                  

%%%%%%%%
\subsection{Fluid Dynamics Simulation}
\label{sec:fdgen-cryo-cfd}

Proper placement of purity monitors, thermometers, and liquid level monitors within the \dword{detmodule} requires knowing how \lar flows within the cryostat, given its fluid dynamics, heat and mass transfer, and distribution of impurity concentrations. Fluid flow is also important in understanding how the positive and negative ion excess created by various sources (e.g., ionization from cosmic rays and $^{36}$Ar; ion feedback at the liquid-gas interface in a \dword{dp} detector) is distributed across the detector as it affects \efield uniformity. 
Finally, \dword{cfd} simulations are crucial to predict the purity of the argon in regions where experimental data is unavailable. The overall goal of the \dword{cfd} simulations
%\dword{dune} \dword{fd} 
%for the \dword{spmod} 
is to better understand and predict the fluid (in either liquid or vapor state) motions and the implications for detector performance. % of the detectors. 

Fluid motion within the cryostat is driven primarily by small changes in density caused by thermal gradients within the fluid although pump flow rates and inlet and outlet locations also contribute. Heat sources include exterior heat from the surroundings, interior heat from electronics, and heat flow through the pump inlet. In principle, purity monitors can be placed throughout the cryostat to determine if the argon is pure enough for experimentation. However, some areas inside the cryostat are off limits for such monitors. 
% \fixme{anne moved prev pgraph up --- looks good [Glenn, Carmen, Sowjanya]}


The fluid flow behavior can be determined by simulating \dword{lar} flow within a \dword{detmodule} %the detector 
using Siemens Star-CCM$+$\footnote{https://mdx.plm.automation.siemens.com/star-ccm-plus}, a commercially available \dword{cfd} code.  Such a model must properly define the fluid characteristics, solid bodies, and fluid-solid interfaces, as well as provide a way to measure contamination, while still maintaining reasonable computation times. In addition, these fluid dynamics simulations can be compared to available experimental data to assess simulation accuracy and credibility.



Although simulation of the \dword{detmodule} presents challenges, %there are 
acceptable simplifications can %to 
accurately represent the fluid, the interfacing solid bodies, and variations of contaminant concentrations. Because of the magnitude of thermal variation within the cryostat, modeling of the \dword{lar} is simplified by using constant thermophysical properties, calculating buoyant force with the Boussinesq Model (using a constant density for the fluid with application of a temperature-dependent buoyant force), and a standard shear stress transport turbulence model. Solid bodies that touch the \dword{lar} include the cryostat wall, cathode planes, anode planes, \dword{gp}, and \dword{fc}. As in previous \dword{cfd} models of the \dword{dune} 35-ton prototype and \dword{pdsp}
\cite{bib:docdb5915}, the \dword{fc} planes, anode planes, and \dword{gp} can be represented by porous bodies. Because impurity concentration and electron lifetime do not affect fluid flow, these variables can be simulated as passive scalars, as is commonly done for smoke released \cite{cfd-1} 
in air or dyes released in liquids.


Discrepancies between real data and simulations may affect detector performance. % because 
Simulation results contribute to decisions about where to place sensors and monitors, and to %as well as 
the definitions of various calibration quantities. Methods of mitigating such risks include well established convergence criteria, sensitivity studies, and comparison to results of previous \dword{cfd} simulation work. Moreover, the simulation will be improved with input from \lar temperature and purity measurements and validation tests from \dword{pdsp}\footnote{Because \dword{pddp} was not instrumented with high-precision thermometers in the liquid phase and because the cryogenics design is the same for \dword{sp} and \dword{dp} modules of the \dword{dune} \dword{fd}, \dword{pdsp} data will be used to validate the liquid \dword{cfd} model.}. 

%Taking into account that the \dword{cfd} model can predict both temperature and impurity levels, the procedure for validating and tuning the \dword{cfd} model will be the following: 1) temperature predictions will be constrained with temperature measurements in numerous locations in the cryostat to improve the \dword{cfd} model, 2) the improved model is then used to predict the \lar impurity level at the location of purity monitors, 3) this prediction is compared with the actual measurement from purity monitors to further constrain the \dword{cfd} model.  

Taking into account that the \dword{cfd} model can predict both temperature and impurity levels, the procedure for validating and tuning the \dword{cfd} model will be the following: 1) use temperature measurements in numerous locations in the cryostat to constrain temperature predictions and  improve the \dword{cfd} model, 2) use the improved model  to predict the \lar impurity level at the purity monitor locations, and 3) compare the predictions to the actual  purity monitor measurements to further constrain the \dword{cfd} model.  

Figure~\ref{fig:cfd-example} shows an example of the temperature
distribution on a plane intersecting a \dword{lar} inlet and at a
plane halfway between an inlet and an outlet; 
the geometry used for
this simulation is shown in Figure~\ref{fig:cfd-example-geometry}\footnote{The inlet and outlet map has recently changed; it now consists of two rows of 64 inlets each at each longer side of the cryostat and four outlets along the shorter sides (drift direction) of the cryostat.}. Note the plume of higher temperature \dword{lar} between the walls and
the outer \dword{apa} on the inlet plane. The current placement of instrumentation in
the cryostat as shown in Figure~\ref{fig:cisc-tsensor-map} was determined using temperature and impurity distributions from previous simulations.

\begin{dunefigure}[\dshort{cfd} example]{fig:cfd-example}
  {Distribution of temperature on a plane intersecting an inlet (left) and halfway between an inlet and an outlet (right), as predicted by previous \dword{cfd} simulations (from~\cite{bib:docdb5915}). (See Figure~\ref{fig:cfd-example-geometry} for geometry.)}
  \includegraphics[height=0.4\textwidth]{cisc_cfd_inlet_z52.png}
  \includegraphics[height=0.4\textwidth]{cisc_cfd_outlet_z0.png}
\end{dunefigure}

\begin{dunefigure}[\single CISC geometry layout]{fig:cfd-example-geometry}
  {Layout of the \single \dword{tpc} within the cryostat (top) and positions of \dword{lar} inlets and outlets (bottom) as modeled in the \dword{cfd} simulations~\cite{bib:docdb5915}. The $y$ axis is vertical and the $x$ axis is parallel to the \dword{tpc} drift direction. Inlets are shown in green and outlets are shown in red.}
  \includegraphics[width=0.7\textwidth]{cisc_cfd_cryostat-layout.png}
  \includegraphics[width=0.7\textwidth]{cisc_cfd_inlet-outlet-layout.png}
\end{dunefigure}

The strategy for  future \dword{cfd} simulations begins with understanding the performance of the \dword{pdsp} cryogenics system and modeling the \dwords{detmodule} to derive specifications %requirements 
for instrumentation. We are pursuing a prioritized set of studies to help determine the requirements for other systems. We plan to 
\begin{itemize}
\item review the \dword{dune} \dword{fd} cryogenics system design and verify the current implementation in simulation %to ensure that the model represents what will be built.
to ensure that the simulation represents the actual design.
\item 
model the \dword{pdsp} liquid and gas regions with the same precision as the \dword{fd}. Presently, we have only the liquid model, which is needed to interpret the thermometer data. The gas model is needed to see how to place thermometers in the ullage and verify the design of the gaseous argon purge system.
\item verify the \dword{cfd} model for the \dword{spmod} in a simulation performed by LBNF; this defines the requirements for instrumentation devices (e.g., thermometry).
\end{itemize}


%%%%%%%%%%%%%%%%%%%%
\subsubsection{Validation in ProtoDUNE}
\label{sec:cfdvalid}
\dword{pdsp} has collected data to validate the \dword{cfd} using: % already installed instrumentation.
\begin{itemize}
\item static and dynamic T-gradient thermometers, 
\item individual temperature sensors placed in the return \dword{lar} inlets, 
\item two \twod grids of individual temperature sensors installed below the bottom ground planes and above the top ground planes, 
\item a string of three purity monitors vertically spaced from near the bottom of the cryostat to just below the \dword{lar} surface,
\item two pressure sensors (relative and absolute) in the argon gas,
\item H$_{2}$O, N$_{2}$, and O$_{2}$ gas analyzers, 
\item \dword{lar} level monitors, and
\item standard cryogenic sensors including pressure transducers, individual temperature sensors placed around
the cryostat on the membrane walls, and recirculation flow rates transducers.
\end{itemize}


The data, which has been logged through the \dword{pdsp} slow control system \cite{pdspdcs_proc}, is available for offline analysis. %\fixme{reference? -- reference added} 

In parallel, \dword{cisc} has produced a \dword{pdsp} \dword{cfd} model %has been produced 
with input from \dword{pdsp} measurements (see Table~\ref{tab:fdgen-cisc-CFDparam}). Streamlines\footnote{In fluid mechanics, a streamline is a line that is everywhere tangent to the local velocity vector. For steady flows, a streamline also represents the path that a single particle of the fluid will take from inlet to exit.} from the current  simulation (Figure~\ref{fig:cisc-cfd-larflow-inlets}) show the flow paths from the four cryostat inlets to the outlet. The validation of this model consists of an iterative process in which several versions of the \dword{cfd} simulation, using different input parameters, eventually %will %be produced to 
converge %on 
to a reasonable agreement with data from instrumentation devices. Those comparisons will be shown in Section~\ref{sec:fdgen-slow-cryo-temp-ana}.

\begin{dunetable}
[CFD parameters for ProtoDUNE]
{p{0.24\textwidth}p{0.17\textwidth}p{0.49\textwidth}}
{tab:fdgen-cisc-CFDparam}
{\dword{cfd} input parameters for \dword{protodune}-SP}   


Parameter  &	Value &	Comments \\ \colhline
Cryostat height
&
7.878 m
&
Measured with laser (1 cm error approx.)
\\  \colhline
LAr surface height
&
7.406 m
&
Measured by capacitive level meter ($<1$ cm error)
\\  \colhline	
Ullage pressure		
&
1.045 bar
&
Measured by pressure gauges
\\  \colhline
\lar surface temperature
&
87.596 K
&
Computed using ullage pressure and \cite{larpropertiesbnl}%(Luke to add to bib) \linebreak
%https://lar.bnl.gov/properties/basic.html\#phase
\\  \colhline
\lar inlet temperature
&
bulk LAr + 0.2 K
&
Estimated from pressure settings in cryo-system
\\  \colhline
\lar flow rate per pipe
&
0.417 kg/s
& Estimated from cryostat filling rate 
\\
\end{dunetable}

\begin{dunefigure}[Streamlines for \lar flow inside  ProtoDUNE-SP]{fig:cisc-cfd-larflow-inlets}
  {Streamlines for \dword{lar} flow inside  \dword{pdsp}}
  \includegraphics[width=0.8\textwidth]{cisc_cfd_larflow-inlets.png}
\end{dunefigure}



Once the \dword{pdsp} \dword{cfd} model %is able to 
predicts the fluid temperature in the entire cryostat to a reasonable level under different conditions, we will use it %this model will be used 
to produce maps of impurity levels in the \dword{detmodule}. These can be easily converted into electron lifetime maps, which we will %be 
compare to the %data provided by the 
\dword{pdsp} purity monitor data. 


%%%%%%%%
\section{Cryogenics Instrumentation}
\label{sec:fdgen-cryo-instr}
%Instrumentation inside the cryostat must ensure that the condition of the \dword{lar} is adequate to operate the \dshort{tpc}.
Instrumentation inside the cryostat must accurately report the condition of the \dword{lar} so that we can ensure that it is adequate to operate the \dshort{tpc}.
This instrumentation includes %devices (the 
purity monitors %)
to check the level of impurity in the argon and %providing 
to provide high-precision electron lifetime measurements,
as well as gas analyzers to verify that the levels of atmospheric contamination do not rise above %drop below 
certain limits during the cryostat purging, cooling, and filling. 
Temperature sensors deployed in vertical arrays and at the top and bottom of the \dword{detmodule} monitor the cryogenics system operation, providing a 
detailed \threed temperature map that helps predict the \dword{lar} purity across the entire cryostat. The cryogenics instrumentation also includes \dword{lar} level monitors and
a system of internal cameras to help find sparks in the cryostat and %for overall 
to monitor the overall cryostat interior. 

The proper placement of purity monitors, thermometers, and liquid-level monitors in the \dword{detmodule} requires %knowing how \dword{lar} behaves 
understanding the \dword{lar} fluid dynamics, heat and mass transfer, and the distribution of impurity concentrations within the cryostat. % in terms of its . 
Both this and % Besides that, 
coherent analysis of the instrumentation data require \dword{cfd} simulation results.

%Something on \dword{protodune} Validation 
\dword{pdsp} is testing the performance of 
purity monitors, thermometers, level monitors and cameras
%all cryogenic instrumentation 
for the \dword{spmod}, validating the baseline  %\dword{fd} 
design.





%%%%%%%%%%%%%%%%%%%%%%%%%%%%%%%
\subsection{Thermometers}
\label{sec:fdsp-cryo-therm}
As discussed in Section~\ref{sec:fdgen-cryo-cfd}, a detailed \threed temperature map is important for monitoring 
the cryogenics system for correct functioning and the \dword{lar} for uniformity.
Given the complexity and size of purity monitors, they can only be installed on the sides of the cryostat to provide a local measurement of
\dword{lar} purity.  
A direct measurement of the \dword{lar} purity across the entire cryostat is not feasible, but a sufficiently detailed \threed temperature map based on \dword{cfd} simulations can predict it. The vertical coordinate is especially important because it will relate closely to the 
\dword{lar} recirculation and uniformity. 

The baseline sensor distribution and the cryostat ports used to extract cables (with indication of number of cables per port) are shown in Figure~\ref{fig:cisc-tsensor-map}. The baseline distribution will evolve as more information becomes available (precise \dword{cfd} simulations, better understanding of \dword{dss} ports, installation interfaces with other groups), but the baseline suffices to establish the overall strategy.

\begin{dunefigure}[Distribution of temperature sensors inside the cryostat]{fig:cisc-tsensor-map}
  {Distribution of temperature sensors inside the cryostat}
  \includegraphics[width=0.95\textwidth]{cisc_tsensor_map_crop.png}
  \includegraphics[width=0.85\textwidth]{cisc_tsensor_map_legend.png}
\end{dunefigure}
%\fixme{legend is hard to read in figure (anne)}

High-precision temperature sensors will be distributed near the \dword{tpc} walls in two ways:
(1) forming high density (\(>2\) sensors/\si{m}) vertical arrays %(the so-
(called T-gradient monitors) and (2) in coarser ($\sim$ 1 sensor/\SI{5}{m}) 2D arrays %(the so-
(called individual sensors) at the top and bottom of the \dword{detmodule}, where it is most crucial to know the temperature.

Expected temperature variations inside the cryostat are very small ($\SI{0.02}{K}$; see Figure~\ref{fig:cfd-example}),
so sensors must be cross-calibrated to better than $\SI{0.005}{K}$. Most sensors will be calibrated in the laboratory before installation
(installation is described in Section \ref{sec:fdgen-slow-cryo-install-th}).
Calibration before installation is the only option for sensors installed on the long sides of the detector and the top and bottom of the cryostat, where space is limited.
Given the precision required and the unknown longevity of the sensors -- possibly requiring another  calibration after some time -- an additional method
will be used for T-gradient monitors installed on the short ends of the detector in the space between the field cage end walls and the cryostat walls. There is sufficient space in this area for a movable system, which can be used to cross calibrate
the temperature sensors in situ, as described in \ref{sec:fdgen-slow-cryo-dynamic-therm}.

The baseline design for all thermometer systems have three elements in
common: sensors, cables, and readout system. We plan to use Lake Shore
PT100-series\footnote{Lake Shore Cryotronics\texttrademark{} platinum RTD series,
  \url{https://www.lakeshore.com/}.} %products/cryogenic-temperature-sensors/platinum-rtds/models/Pages/Overview.aspx}.} 
platinum sensors with \SI{100}{\ohm} resistance %are used 
because in
the temperature range \SIrange{83}{92}{K} %these particular sensors
they 
show high reproducibility of $\sim\SI{5}{mK}$ and absolute temperature
accuracy of \SI{100}{mK}.  Using a four-wire readout greatly reduces
issues related to lead resistance, any parasitic resistances,
connections through the flange, and general electromagnetic noise
pick-up. Lakeshore PT102 sensors (see
Figure~\ref{fig:sensor-support}, right) were used in the \dword{35t} and \dword{pdsp}, % detector, 
giving excellent results. For the inner
readout cables, a custom cable made by Axon\footnote{Axon\texttrademark{} Cable, \url{http://www.axon-cable.com}.}
is the baseline. It
consists of four teflon-jacketed copper wires (\dword{awg} 28), forming two
twisted pairs, with a metallic external shield and an outer teflon
jacket.
The readout system is described in Section \ref{sec:fdgen-slow-cryo-therm-readout}.

Another set of lower-precision sensors epoxied into the bottom membrane of the cryostat will monitor  the cryostat filling in the initial stage.   
Finally, the inner walls and roof of the cryostat will have the same types of sensors to monitor the temperature during \cooldown and filling (``W'' sensors in Figure~\ref{fig:cisc-tsensor-map}).
 

% % % %
\subsubsection{Dynamic T-gradient monitors}
\label{sec:fdgen-slow-cryo-dynamic-therm}

 To address concerns about potential differences in sensor readings prior to and after installation in a \dword{detmodule}, and potential drifts over the lifetime of the %detector 
 module that may affect accuracy of the vertical temperature gradient measurement, % in a \dword{detmodule}, 
 a dynamic temperature monitor allows cross-calibration of sensor readings %{\em in situ}. 
 in situ.
Namely, this T-gradient monitor is motorized, allowing vertical motion of the temperature sensor array %in vertical direction while installed 
in the \dword{detmodule}, %. This ability enables 
enabling precise cross-calibration between the sensors, %{\em in situ
as illustrated in Figure~\ref{fig:sensor-cross-calibration}.  

\begin{dunefigure}[Principle of cross-calibration with dynamic T-gradient monitor]{fig:sensor-cross-calibration}
  {In step 1, sensor temperature measurements are taken with the T-gradient monitor in the home position. In step 2, the entire system is moved up \SI{25}{cm} and another set of temperature readings is taken by all sensors. Then, the offsets between pairs of sensors are calculated for each position. In step 3, offsets are linked together, providing cross-calibration of all sensors, to obtain the entire vertical temperature gradient measurement with respect to a single sensor (number 1 in this case). }
  \includegraphics[width=1.0\textwidth]{cisc_cross_calibration_illustration.png}%
\end{dunefigure}

The procedure for cross-calibrations is the following: in step 1, the temperature reading  of all sensors is taken at the home (lowest) position of the carrier rod. In  step 2, the stepper motor moves the carrier rod up \SI{25}{cm}. Since the sensors along the entire  carrier rod are positioned \SI{25}{cm} apart, when the system is moved up \SI{25}{cm}, each sensor is positioned at the height that was occupied by another sensor in step 1. Then a second temperature reading is taken. In this manner, except for the lowest position, two temperature measurements are taken at each location with %two 
different sensors. Assuming that the temperature at each location is stable over the few minutes required to make the measurements, %the 
any difference in the temperature readings between the two different sensors is due to their relative measurement offset. This %temperature readings 
difference is then calculated for all locations.  In step 3, readout differences between pairs of sensors at each location are linked to one another, expressing temperature measurements at all heights with respect to a single sensor. In this way, temperature readings from all sensors are cross-calibrated %{\em in situ}
in situ, canceling all possible offsets due to electromagnetic noise or any parasitic resistances that may have prevailed despite the four-point connection to the sensors that should cancel most of the offsets. These measurements are taken with a very stable current source, which ensures high precision of repeated temperature measurements over time. The motion of the dynamic T-monitor is stepper-motor operated, delivering measurements with high spatial resolution. 


A total of \num{72} sensors will be installed with \SI{25}{cm} spacing, decreased to \SI{10}{cm} spacing for the top and bottom \SI{1}{m} of the carrier rod.  
 The vertical displacement of the system is such that every sensor can be moved to the nominal position of at least five other sensors, minimizing the risks associated with sensor failure and allowing for several points of comparison. The total expected motion range of the carrier rod is \SI{1.35}{m}.



This procedure was tested in \dword{pdsp}, where the system was successfully moved up by a maximum of \SI{51}{cm}, allowing cross-calibration of all sensors (22 sensors with \SI{10.2}{cm} spacing at top and bottom and \SI{51}{cm} in the middle). 

Figure~\ref{fig:dynamic_t_pumps-off} shows the temperature profile after calibration when the recirculation pumps are off. Under these conditions the  temperature should be very homogeneous except near the surface. This is indeed what is observed in that figure, demonstrating the reliability of the method.  

\begin{dunefigure}[Temperature profile for dynamic T-gradient with pumps-off]{fig:dynamic_t_pumps-off}
  {Temperature profile as measured by the dynamic T-gradient monitor after cross-calibration, when the recirculation pumps are off. Temperature variation is of the order of \SI{3}{mK} except close to the top and the gas phase interface, as expected.}
  \includegraphics[width=0.6\textwidth]{cisc_dynamic_t_pumps-off.png}%
\end{dunefigure}


A dynamic T-gradient monitor has three parts: a carrier rod on which sensors are mounted; an enclosure above the cryostat housing space that allows the carrier rod to move vertically  \SI{1.5}{m} over its lowest location; and the motion mechanism. The motion mechanism consists of a stepper motor connected through a ferrofluidic dynamic seal to a gear and pinion motion mechanism. The sensors have two pins soldered to a printed circuit board (PCB). 
%\fixme{maybe add gloss term}
Two wires are individually soldered to the common soldering pad for each pin.  A cutout in the PCB around the sensor allows free flow of argon for more accurate temperature readings.  Stepper motors typically have very fine steps that allow highly precise positioning of the sensors.  Figure~\ref{fig:fd-slow-cryo-dt-monitor-overview} shows the overall design of the dynamic T-gradient monitor. % with the sensor carrier rod, enclosure above the cryostat, and stepper motor mounted on the side of the enclosure. 
The enclosure has two parts connected by a six-cross flange. One side of this flange will be used for signal wires, another will be used as a viewing window, and the two other ports will be spares. Figure~\ref{fig:fd-slow-cryo-sensor-mount}, left shows the PCB mounted on the carrier rod and the sensor mounted on the PCB along with the four point connection to the signal readout wires. %Finally, 
Figure~\ref{fig:fd-slow-cryo-sensor-mount}, right shows the stepper motor mounted on the side of the rod enclosure. The motor remains outside the enclosure, at room temperature, %and 
as do its power and control cables. % also remain outside.

\begin{dunefigure}[Dynamic T-gradient monitor overview]{fig:fd-slow-cryo-dt-monitor-overview}
  {%An overview 
  A schematic of the dynamic T-gradient monitor.}
% \includegraphics[width=0.11\textwidth,angle=90]{cisc_DTOverview.png}
 \includegraphics[width=0.95\textwidth,angle=0]{cisc_DynamicProfiler.png}
\end{dunefigure}
\begin{dunefigure}[Sensor-cable assembly for dynamic T-gradient monitor]{fig:fd-slow-cryo-sensor-mount}
  {Left: Sensor mounted on a PCB board and PCB board mounted on the rod. Right:
    The driving mechanism of the dynamic T-gradient monitor, consisting of a stepper motor driving the pinion and gear linear motion mechanism. }
  \includegraphics[width=0.40\textwidth]{cisc_DTSensorMount.png}
  \hspace{3cm}%
  \includegraphics[width=0.12\textwidth]{cisc_DTMotor.png}
\end{dunefigure}


% % % %
\subsubsection{Static T-gradient monitors}
\label{sec:fdgen-slow-cryo-static-therm}

Several vertical arrays of high-precision temperature sensors cross-calibrated in the laboratory will be installed behind the \dword{apa}s.  
The baseline design assumes six arrays with \num{48} sensors each. Spacing between sensors
is \SI{20}{cm} at the top and bottom and \SI{40}{cm} in the middle area. This configuration is similar to the one used in \dword{pdsp} but with nearly double the spacing. 
As shown in Figure~\ref{fig:pd_static_t_results} a configuration with \num{48} sensors was appropriate in \dword{pdsp}, as it should be in the \dword{spmod} where the expected total gradient is no larger than in \dword{pdsp} (see Figure~\ref{fig:cfd-example}). 

\begin{dunefigure}[ProtoDUNE-SP static T-gradient results]{fig:pd_static_t_results}{
 Left: Temperature profile as measured by the static T-gradient monitor for two different calibration methods. Right: Distribution of the difference between both methods.}
  \includegraphics[height=0.34\textwidth]{cisc_static_t_profiles.png}%
  \includegraphics[height=0.34\textwidth]{cisc_static_t_calib_diff.png}%
\end{dunefigure}

Sensors will be cross-calibrated in the laboratory using a controlled environment and a high-precision readout system, described in Section \ref{sec:fdgen-slow-cryo-therm-readout}.
The accuracy of the calibration for \dword{pdsp} was estimated to be $\SI{2.6}{mK}$, as shown in Figure~\ref{fig:Trepro}. Preliminary results for the analysis of \dword{pdsp} static T-gradient monitor data are shown in Figure~\ref{fig:pd_static_t_results}. The temperature profile has been computed using both the laboratory calibration and the so-called ``in-situ pump-off calibration,'' which consists %in 
of estimating the offsets between sensors assuming the temperature of \dword{lar} in the cryostat is homogeneous when the re-circulation pumps are off (the validity of this method is demonstrated in Section~\ref{sec:fdgen-slow-cryo-dynamic-therm}).  
The \dword{rms} of the difference between both methods is $\SI{4.6}{mK}$, slightly larger than the value quoted above for the accuracy of the laboratory calibration, due to the presence of few outliers (under investigation) and to the imperfect assumption of homogeneous temperature when pumps are off (see Figure~\ref{fig:dynamic_t_pumps-off}).  


\begin{dunefigure}[Temperature sensor resolution and reproducibility]{fig:Trepro}{
 Left:   Temperature offset between two sensors as a function of time for four independent immersions in LAr. The reproducibility of those sensors, defined as the RMS of the mean offset in the flat region, is $\sim\,\SI{1}{mK}$,
    The resolution for individual measurements, defined as the RMS of one offset in the flat region, is better than \SI{0.5}{mK}. Right: Difference between the mean offset obtained with two independent calibration methods for the 51 calibrated sensors. The standard deviation of this distribution is interpreted as precision of the calibration.}
  \includegraphics[height=0.44\textwidth]{cisc_tsensor_calib.png}%
  \includegraphics[height=0.45\textwidth]{cisc_tsensor_calib2.png}%
\end{dunefigure}


\begin{dunefigure}[Conceptual design of the static T-gradient monitor]
{fig:cisc-static-tgradient}
  {Conceptual design of the static T-gradient monitor.}
  \includegraphics[width=0.8\textwidth]{cisc_static_tgradient_new.png}
\end{dunefigure}

Figure~\ref{fig:cisc-static-tgradient} shows the baseline mechanical design of
the static T-gradient monitor. Two strings (stainless steel or carbon fibre) are anchored at the top and bottom corners of the cryostat using the available M10 bolts (see Figure~\ref{fig:sensor-support}, left). One string routes the cables while the other supports the temperature sensors.
Given the height of the cryostat, an intermediate anchoring point to reduce swinging is under consideration. 
A prototype is being built at \dfirst{ific}, Spain, where the full system will be mounted using two dummy cryostat corners.   


Figure~\ref{fig:sensor-support} (right) shows the baseline design of the ($52\times \SI{15}{mm^2}$) PCB support for temperature sensors with an IDC-4 male connector. A narrower connector (with two rows of two pins each) is being studied. This alternative design would reduce the width of the PCB assembly and allow more sensors to be calibrated simultaneously. Each four-wire cable from the sensor to the flange will have an IDC-4 female connector on the sensor end; the flange end of the cable will be soldered or crimped to the appropriate connector, whose type and number of pins  depend on the final design of the \dword{dss} ports that will be used to extract the cables. SUBD-25 connectors were used in \dword{pdsp}.

\begin{dunefigure}[Cryostat bolts and temperature sensor support]{fig:sensor-support}
  {Left: bolts at the bottom corner of the cryostat. Right: Lakeshore PT102 sensor mounted on a PCB with an IDC-4 connector.}
  \includegraphics[height=0.2\textwidth]{cisc_cryostat_bolts.png}%
    \hspace{1cm}%
  \includegraphics[height=0.2\textwidth]{cisc_tsensor.png}%
\end{dunefigure}


% % % %
\subsubsection{Individual Temperature Sensors}
\label{sec:fdgen-slow-cryo-individual-therm}

T-gradient monitors will be complemented by a coarser 2D array (every \SI{5}{m}) of precision sensors at the top and bottom of the \dword{detmodule}, as shown in Figure~\ref{fig:cisc-tsensor-map}. Following the \dword{pdsp} design, bottom sensors will use the cryogenic pipes as a support structure, while top sensors will be anchored to the \dwords{gp}. Although sensors at the top will have a similar distribution to those at the bottom, %some differences will occur because 
suitable anchoring points at the top and bottom will differ. 

As in \dword{pdsp}, another set of standard sensors will be evenly distributed and epoxied to the bottom membrane. They will detect the presence of \dword{lar} when cryostat filling starts. Finally, two vertical arrays of standard sensors will be epoxied to the lateral walls in two opposite vertical corners, with a spacing of \SI{102}{cm} (every three corrugations), to monitor the cryostat membrane temperature during the \cooldown and filling processes. 

%While 
Whereas in \dword{pdsp} cables were routed individually (without touching neighboring cables or any metallic elements) to prevent grounding loops in case the outer Teflon jacket broke, such a failure has been proved to be very unlikely. Thus, in the \dwords{detmodule}, cables will be routed in bundles, simplifying the design enormously. As Figure~\ref{fig:cisc-tsensor-map} shows, up to 20 sensors will use the same \dword{dss} port, large enough for a cable bundle \SI{16}{mm} in diameter.

Cable bundles of several sizes will be configured using custom made Teflon 
pieces %,  
%which 
that will be anchored to different cryostat and detector elements to route cables from sensors to \dword{dss} ports. For sensors at the bottom (on pipes and floor), cables will be routed towards the cryostat bottom horizontal corner using stainless steel split clamps anchored to pipes (successfully prototyped in \dword{pdsp}), and from there, to the top of the cryostat using vertical strings (as with static T-gradient monitors). For sensors on the top \dwords{gp}, cables bundles will be routed to the corresponding \dword{dss} port using Teflon supports attached to both the \frfour threaded rods in the union between two \dword{gp} modules and to the \dword{dss} I-beams (both successfully prototyped in \dword{pdsp}). Sensors on the walls will use bolts in the vertical corners for cable routing. 

For all individual sensors, PCB sensor support, cables, and connection to the flanges will be the same as for the T-gradient monitors. 
  

% % % %
\subsubsection{Analysis of temperature data in \dword{pdsp}}
\label{sec:fdgen-slow-cryo-temp-ana}
\fixme{This whole section could use a bit more editing; but no time. Anne}

Temperature data from \dword{pdsp} has been recorded since \dword{lar} filling %started 
in August 2018. The analysis of this data and the comparison with \dword{cfd} simulations is actively underway, but interesting preliminary results are available and are described below. Figure~\ref{fig:pd_inst_map} shows the distribution of temperature sensors in the \dword{pdsp} cryostat.  

\begin{dunefigure}[ProtoDUNE-SP instrumentation map]{fig:pd_inst_map}{Distribution of temperature sensors in the \dword{pdsp} cryostat. Notice that four of the bottom sensors are located right above the LAr inlets. Purity monitors and level meters are also indicated. }
  \includegraphics[width=0.7\textwidth]{cisc_pdsp_instrumentation.png}%
\end{dunefigure}


All precision temperature sensors (for the static and dynamic T-gradient monitors,  and the \twod arrays at top and bottom) were calibrated in the laboratory prior to installation. However, since the calibration method was still under development when those sensors 
%had to be 
were installed, this calibration was only satisfactory (\SI{2.6}{mK} precision) for the static T-gradient monitor, for which the sensors were calibrated last and  plugged in just few days prior to installation in the cryostat. 
\fixme{where do you say what this in-the-lab calib method was? anne}
In Section~\ref{sec:fdgen-slow-cryo-static-therm}, an additional calibration method, the so-called ``pumps-off calibration,'' is described and the agreement with the laboratory calibration was demonstrated %proved 
(see Figure~\ref{fig:pd_static_t_results}). Since this is the only reliable calibration for individual sensors, this method is used for the data analysis presented in this section, for all sensors except for the dynamic T-gradient monitor, for which the calibration based on the movable system is more precise (see Section~\ref{sec:fdgen-slow-cryo-dynamic-therm}). 

Figure~\ref{fig:pd_tgradient_results} shows the vertical temperature profiles as measured by the dynamic and static T-gradient monitors at a given moment in time (a 10 minute period in May 2019). The stability of these profiles has been carefully studied: the relative variation between any two sensors on the same profiler remained below \SI{3}{mK} during the entire data taking period, demonstrating that the shape of the profiles is nearly constant in time. In Figure~\ref{fig:pd_tgradient_results} it is clear  %one immediately observes 
that the shapes of the two profiles are similar, with a bump at \SI{6.2}{m}, but the magnitude of %the 
variation of the static profile almost doubles %the one of 
compared to the dynamic profile. This effect is attributed to the fact that the dynamic T-gradient monitor is in the path of the \lar flow, which makes the temperature more uniform, while the static profiler is on the side.  \fixme{check my edits above. anne}

\begin{dunefigure}[ProtoDUNE-SP T-gradient results]{fig:pd_tgradient_results}{Temperature profiles measured by the T-gradient monitors and comparison to the CFD model with different boundary conditions. Left: dynamic T-gradient monitor; Right: Static T-gradient monitor.}
  \includegraphics[width=0.5\textwidth]{cisc_dynamic_cfd_data.png}%
%  \hspace*{0.3cm}
  \includegraphics[width=0.5\textwidth]{cisc_static_cfd_data.png}%
\end{dunefigure}

We can use temperature measurements by the bottom sensor grid to connect the two different 
regions covered by the T-gradient monitors. Figure~\ref{fig:pd_t_bottom_results} shows the temperature difference between bottom sensors and the second sensor of the static T-gradient monitor (the one \SI{40}{cm} from the floor), which is used as reference. %It is also 
Also shown in the figure is the  dynamic T-gradient monitor's third sensor, located at a similar height. Three different periods are shown in the figure: two periods with pumps-on and one period with pumps-off. It is observed that when the pumps are working, the temperature decreases towards the \dword{lar} pump, and is %being  
higher in the sensors below the cathode. The horizontal gradient observed in this situation is of the order of \SI{20}{mK} -- larger than the vertical gradient. When the pumps are off the horizontal gradient decreases, although a residual gradient of  \SI{5}{mK}  is observed. This gradient is attributed to the inertia of the liquid once the pumps are switched off: it takes more than one day to recover the horizontal homogeneity.    

\begin{dunefigure}[\dshort{pdsp} bottom sensor results ]{fig:pd_t_bottom_results}{Temperature difference between bottom sensors at  \SI{40}{cm} from the floor and static T-gradient sensor at same height. The third dynamic T-gradient sensor, at the same height, is also shown. Two pumps-on periods (left and middle panels) and one pump-off period (right panel) are shown.}
  \includegraphics[width=0.95\textwidth]{cisc_t_bottom.png}%
\end{dunefigure}

\dword{cfd} simulations have been produced using different inputs. We have identified two  parameters %have been identified 
as potential drivers of the convection regime: (1) the incoming \dword{lar} flow rate, and (2) the incoming \dword{lar} temperature. Figure~\ref{fig:pd_tgradient_results} shows the result of varying these parameters. 
The \dword{cfd} model reasonably predicts the main features of the data, but the details still need to be understood: the bump at \SI{6.2}{m} or \fixme{and?} the lower measured temperature at the bottom. It is worth noting that the simulation depends minimally on the \dword{lar} temperature while the flow rate has more impact, especially in the regions where discrepancies are larger. All simulations use the nominal \dword{lar} flow rate, \SI{0.42}{kg/s}, except the one explicitly indicated 
in the plots that uses half that rate. More simulations with other \dword{lar} flow rates %are needed to conclude.  
are still in progress.

The \dword{cfd} reassuringly predicts a reasonable response for more than one set of initial conditions. It is still important to %, but 
measure the 
instrumentation response to help establish the validity of the \dword{cfd} model. We did not run tests with differing initial conditions during the beam run because even controlled changes of the cryostat environment could have undesirable effects. However, recently we ran dedicated tests to validate the \dword{cfd} under various deliberate changes of the cryostat. % were performed recently. 
These  additional tests included pump and recirculation manipulations (such as pump on-off, change of pumping speed, and bypassing of filtration), and changing the cryostat pressure set point to a higher (or lower) value\footnote{The HV 
was ramped down for this exercise because dropping the pressure too fast might provoke boiling of the \dword{lar} near the surface.} to induce changes in the pressure for a specified time while monitoring the instrumentation. Any change in pressure could change the temperatures everywhere in the cryostat. Studying the rate of this change, as detected at various heights in the cryostat, might provide interesting constraints on the \dword{cfd} model.

\subsubsubsection{Comparison of calibration methods}

%Three different calibration methods have been described above, each of them having a different purpose. 
Each of the three calibration methods described above has a different purpose. 
\fixme{there should be an itemized list of them - I have to dig! I think it's ``in the lab'' (but I don't know what the method was), and in situ pumps on, and in situ pumps off. Am I right? anne}
The underlying assumption is that reliable temperature monitoring at the few \si{mK} level is desirable during the entire lifetime of the experiment, both to guarantee the correct functioning of the cryogenics system and to compute offline corrections based on temperature measurements and \dword{cfd} simulations. This is only possible if an in situ calibration method is available, since relative calibration between sensors is expected to diverge with time. % with respect to the one performed in the laboratory. 
Two in situ methods have been used. The pumps-off calibration method is very powerful since it is the only way of cross-calibrating all sensors in the cryostat at any point in time. However, it relies on the assumption that the temperature is uniform when the recirculation pumps are off. The validity of this assumption has to be bench-marked with real data, and this is done in \dword{pdsp} using the calibration based on the movable system (see Figure~\ref{fig:dynamic_t_pumps-off}). The last 
\fixme{I have to dig to figure out which one the ``last'' is. Anne}
is the most precise calibration method and the one that sustains all other methods, providing a reliable reference during the entire lifetime of the experiment. This method is even more crucial for the \dword{fd}. Indeed, recirculation pumps will be located on one side of the cryostat, very far (almost \SI{60}{m}) from some regions of the \dword{lar} volume, where the inertia will be more pronounced and the homogeneous temperature assumption becomes less valid.   

% % % %
\subsubsection{Readout system for thermometers}
\label{sec:fdgen-slow-cryo-therm-readout}

A %highly precise 
high-precision and very stable system is required to achieve a readout level of $<\,\SI{5}{mK}$.
The proposed readout system was used in \dword{pdsp} and relies on a variant of an existing mass PT100 temperature readout system developed at
\dword{cern} for an \dword{lhc} experiment; it has already been tested and validated in \dword{pdsp}.
%; it has already been tested and validated by the collaboration experts. 
The system has an electronic circuit that includes
\begin{itemize}
\item a precise and accurate current source\footnote{The actual current delivered is monitored with high-precision resistors such that the effect of ambient temperature can be disentangled.} for the excitation of the temperature sensors measured using the four-wires method, 
\item a multiplexing circuit connecting the temperature sensor signals and forwarding the selected signal to a single line, and 
\item a readout system  with a high-accuracy voltage signal readout module\footnote{National Instrument CompactRIO\texttrademark{} device  with a signal readout NI9238\texttrademark{} module.} with 24 bit resolution over a \SI{1}{V} range.
\end{itemize}
This readout system also drives the multiplexing circuit and calculates temperature values. The CompactRIO device is connected to the detector Ethernet network, sending temperature values to the \dword{dcs} software through a standard \dword{opc-ua} driver.


The current mode of operation averages more than \num{2000} samples taken every second for each sensor. 
Figure~\ref{fig:Trepro} shows the system has a resolution higher than 
\SI{1}{mK}, the \dword{rms} of one of the offsets in the stable region.

%%%%%%%%%%%%%%%%%%%%%%%%%%%%%%%%%%%%%%%%%%%%%%%%%%%%%%
%%%%%%%%%%%%%%%%%  PURITY MONITORS %%%%%%%%%%%%%%%%%%%

%%%%%%%%%%%%%%%%%%%%%%%%%%%%%%%%%%%

\subsection{Purity Monitors}
\label{sec:fdgen-slow-cryo-purity-mon}

%Laura, Jianming
A fundamental requirement of a \dword{lar} \dshort{tpc} is that ionization electrons drift over long distances in \dword{lar}. Part of the charge is inevitably lost due to electronegative impurities in the liquid. To keep this loss to a minimum, monitoring impurities and purifying the \dword{lar} during operation is essential.


A purity monitor is a small ionization chamber used to infer independently  the effective free electron lifetime in the \dword{lartpc}.  
It illuminates a cathode with UV light to generate a known electron current, then collects the drifted current at an anode a known distance away.  Attenuation of the current is related to the electron lifetime.
Electron loss can be parameterized as
%
\(N(t) = N(0)e^{-t/\tau},\)
%
where $N(0)$ is the number of electrons generated by ionization, $N(t)$ is the number of electrons after drift time $t$, and $\tau$ is the electron lifetime. 


For the \dword{spmod}, the drift distance is \spmaxdrift, and the \efield is \SI{500}{\volt\per\centi\meter}. Given the drift velocity of approximately \SI{1.5}{\milli\meter\per\micro\second} in this field, the time to go from cathode to anode is roughly \SI{\sim2.4}{\milli\second} \cite{Walkowiak:2000wf}.
The \dword{lartpc} signal attenuation, \([N(0)-N(t)]/N(0)\), must remain %be kept 
less than \SI{20}{\percent} over the entire drift distance~\cite{bib:docdb3384}.  %\cite{fdtf-final-report}
 The corresponding electron lifetime is $\SI{2.4}{ms}/[-\ln(0.8)] \simeq \SI{11}{ms}$.


Residual gas analyzers can be used to monitor the gas in the ullage of the tank and would be an obvious choice for analyzing argon gas. 
Unfortunately, suitable and commercially available mass
spectrometers have a detection limit of \SI{\sim 10}{\dword{ppb}},
whereas \dword{dune} requires a sensitivity of \dword{ppt}. Therefore,
specially constructed and distributed purity monitors will measure \dword{lar} purity in all %the
phases of operation. % at several known positions.  
These measurements,
along with an accurate \dword{cfd} model, enable the
determination of purity at all positions throughout the \dword{detmodule}.

The large scale of the DUNE \dwords{detmodule} increases the risk of failing to notice unexpected cryogenics and circulation failures. If these conditions were to persist, it could cause irreversible contamination to the \dword{lar} and terminate useful data taking.  Well scheduled purity monitor runs can catch sudden drops in e-lifetime, therefore mitigate this risk, as demonstrated in \dword{pdsp}. Purity monitors' role to mitigate this risk is unique because the cosmic-ray rate in DUNE's deep-underground \dword{fd} is too low for the \dword{tpc} to measure e-lifetime frequently with its data.  \fixme{This pgraph is not in dp; not wanted there? anne}


Purity monitors are placed inside the cryostat but outside of the %detector 
\dshort{tpc}. They are also placed  within the recirculation system outside the cryostat, both in front of and behind the filtration system. % before and after filtration. 
Continuous monitoring of  the \dword{lar} supply lines to the \dword{detmodule} provides a strong line of defense against %contaminated \lar. 
contamination from sources both in the \dword{lar} volume and from components in the recirculation system. 
Similarly, gas analyzers (described in Section~\ref{sec:fdgen-slow-cryo-gas-anlyz}) %provide a first line of defense 
protect against contaminated gas.  

Furthermore, using several purity monitors to measure lifetime with high precision at carefully chosen points provides key input, e.g.,  vertical gradients in impurity concentrations, for \dword{cfd} models of the \dword{detmodule}.

Purity monitors were deployed in previous \dword{lartpc} experiments, e.g., \dword{icarus}, \microboone, and the \dword{35t}. During the first run of the \dword{35t}, two of the four purity monitors stopped working during \cooldown, and a third operated intermittently. We later found that this was due to poor electrical contacts between the resistor chain and the purity monitor. A new design was implemented and successfully tested in the second run. 


\dword{pdsp} and \dword{pddp} %use purity monitoring systems consisting of several 
use purity monitors to %that 
measure electron lifetime at different heights. Both use a similar design. 

Figure~\ref{fig-pdsp-prmassembly} shows the assembly of the \dword{pdsp} purity monitors. The design reflects improvements to ensure electric connectivity and improve signals. \dword{pdsp} uses a string of purity monitors connected with stainless steel tubes to protect the optical fibers. 

\begin{dunefigure}[The ProtoDUNE-SP purity monitoring system]{fig-pdsp-prmassembly}
  {The \dword{pdsp} purity monitoring system}
  \includegraphics[width=0.9\textwidth]{PrMon_pdsp-PrMAssembly}
\end{dunefigure}


\dword{pdsp} implements three purity monitors. The purity monitors continuously monitored \dword{lar} purity during all commissioning, beam test and operation periods of \dword{pdsp}. Figure~\ref{fig-pdsp-prm} shows the \dword{pdsp} data taken using %them 
its three purity monitors from commissioning of \dword{pdsp} starting in September 2018, through the entire beam running period, to July 2019.

\begin{dunefigure}[Measured electron lifetimes in the %four 
three purity monitors at ProtoDUNE-SP]{fig-pdsp-prm}
  {The electron lifetimes measured by three purity monitors in \dword{pdsp} as a function of time, September 2018 through July 2019. The purity is low prior to start of circulation in October. Reasons for later dips include recirculation studies and recirculation pump stoppages.}
  \includegraphics[width=0.9\textwidth]{cisc_pdsp_Lifetime_till_July17_without_error}
\end{dunefigure}

Although \dword{pdsp} receives ample cosmic ray data to perform electron lifetime measurements, the purity monitor system was found to be essential for providing quick, reliable real-time monitoring of purity in the detector and to catch purity-related changes in time due to \dword{lar} recirculation issues. 
Each purity monitor electron lifetime measurement is based on purity monitor anode-to-cathode signal ratios from 200 UV flashes within 40 seconds at the same location. The statistical error on this lifetime measurement is less than \SI{0.03}{ms} when the purity is \SI{6}{ms}. 

The purity monitor system at \dword{pdsp} measured electron lifetime every hour during commissioning and daily during the beam test. During 
this time, it alerted the experiment to serious problems several times. 
\fixme{I'd remove `serious'. Anne} The first time was for filter saturation during \dword{lar} filling, and the others were for recirculation pump stoppages, false alarms, and problems from the cryostat-level gauges.  The dips in Figure~\ref{fig-pdsp-prm} show these sudden changes in purity caught by purity monitors. 
Given the high sensitivity, the \dword{pdsp} purity monitors  were immediately able to alert the experiment to the purity drops, 
preventing situations which otherwise would have continued unnoticed for some time, with potentially serious consequences for the ability to take any data. 

During the commissioning and beam test of \dword{pdsp}, the purity monitors operated with different high voltages to change electron drift time, ranging from \SI{150}{\micro\second} to \SI{3}{\milli\second}. This allowed the \dword{pdsp} purity monitors to measure electron lifetime from \SI{35}{\micro\second} to about \SI{10}{\milli\second} with high precision, a dynamic range greater than \num{300}. 
This measurement was also valuable for the \dword{pdsp} lifetime calibration. Because purity monitors have much smaller drift volumes than the \dshort{tpc}, they are less affected by the space charge caused by cosmic rays. 

  A similar %purity monitoring system design and 
  operation plan is planned for %are exploited in 
  the \dword{spmod}, with modifications to accommodate the relative positions of the instrumentation port %placement relative to 
  and the purity monitor system, and the %requirements and constraints of 
  different geometric relationships between the \dshort{tpc} and cryostat.

%%%%%%%%%%%%%%%%%%%%%%%%%%%%%%%%%%%%%%%%%
\subsubsection{Purity Monitor Design}

The \dword{spmod} baseline purity monitor design follows that of  the \dword{icarus} experiment (Figure~\ref{fig:prm})\cite{Adamowski:2014daa}.  It consists of a double-gridded ion chamber immersed in the \dword{lar} volume with four parallel, circular electrodes, a disk holding a photocathode, two grid rings (anode and cathode), and an anode disk. The cathode grid is held at ground potential. The cathode, anode grid, and anode 
each hold separate bias voltages and are electrically accessible via modified vacuum-grade \dword{hv} \fdth{}s. % and separate bias voltages held at each one.   
The anode grid and the field-shaping rings are connected to the cathode grid by an internal chain of \SI{50}{\mega\ohm} resistors to ensure the uniformity of the \efield{}s in the drift regions. A stainless mesh cylinder is used as a Faraday cage to isolate the purity monitor from external electrostatic background. 

The purity monitor measures the electron drift lifetime between its anode and cathode. The purity monitor's UV-illuminated %gold 
photocathode generates the electrons via the photoelectric effect. Because the electron lifetime in \dword{lar} is inversely proportional to the electronegative impurity concentration, the fraction of electrons generated at the cathode that arrives at the anode ($Q_A/Q_C$) after the electron drift time $t$ gives a measure of the electron lifetime $\tau$:
%
\( Q_A/Q_C \sim e^{-t/\tau}.\)



\begin{dunefigure}[Schematic diagram of the %basic 
baseline purity monitor design]{fig:prm}
  {Schematic diagram of the basic purity monitor design \cite{Adamowski:2014daa}.}
  \includegraphics[width=0.9\textwidth]{PrMon_prm.pdf}
\end{dunefigure}


Once the electron lifetime becomes much larger than the drift time $t$ the purity monitor reaches its sensitivity limit.  For $\tau >> t$, the anode-to-cathode charge ratio becomes $\sim\,1$. Because the drift time is inversely proportional to the \efield, in principle, lowering the %drift 
field should make it possible to measure lifetimes of any length, regardless of the length of the purity monitor.
On the other hand, increasing the voltage will shorten the drift time, allowing measurement of a short lifetime when purity is low. 

The electron lifetime of the purest commercial \dword{lar}, after the first filtering and during the filling process, is typically higher than \SI{40}{\micro\second}. However, when the filter starts to saturate, the lifetime decreases to less than \SI{30}{\micro\second}.  To %achieve less 
reduce the energy loss due to impurities,  the \dword{spmod} requires an electron lifetime greater than \SI{3}{\milli\second}.

Varying the operational \dword{hv} on the anode from \SI{250}{V} to \SI{4000}{V} in the \dword{pdsp}'s \SI{24}{cm} (9.5 inch) long purity monitor allowed us to make the \SI{35}{\micro\second} to \SI{10}{\milli\second} electron lifetime measurements. 
Purity monitors with different lengths (drift distances) are needed to extend the measurable range to below \SI{35}{\micro\second} and above  \SI{10}{\milli\second}.

The photocathode that produces the \phel{}s is an aluminum plate coated first with \SI{50}{\angstrom} of titanium followed by \SI{1000}{\angstrom} of gold, and is attached to the cathode disk.
A xenon flash lamp is the light source in the baseline design, although %this could be replaced by 
a more reliable and possibly submersible light source, perhaps \dword{led}-driven, could replace this in the future. The UV output of the lamp is quite good, approximately $\lambda=$ \SI{225}{\nano\meter}, which corresponds closely to the work function of gold (\SIrange{4.9}{5.1}{\eV}). 
Several UV quartz fibers carry the xenon UV light into the cryostat to illuminate the %gold 
photocathode.   Another quartz fiber delivers the light into a properly biased photodiode outside the cryostat to provide the trigger signal when the lamp flashes. 



\subsubsection{Electronics, DAQ, and Slow Controls Interfacing}
%Jianming
The purity monitor electronics and \dword{daq} system consist of \dword{fe} electronics, waveform digitizers, and a \dword{daq} PC.  Figure~\ref{fig:cryo-purity-mon-diag} %shows the block diagram of the 
illustrates the system.

\begin{dunefigure}[Block diagram of the purity monitor system]{fig:cryo-purity-mon-diag}
  {Block diagram of the purity monitor system.}
  \includegraphics[width=0.95\textwidth]{graphics/PrMon_BlockDiagram_v2.pdf}
\end{dunefigure}


The baseline design of the \dword{fe} electronics follows that used in %is the one used for the purity monitors for 
the \dword{35t},  \dword{lapd}, and \microboone. The cathode and anode signals are fed into two charge amplifiers contained within the purity monitor electronics module.
This electronics module includes a \dword{hv} filter circuit and an amplifier circuit, both shielded by copper plates, to allow the signal and \dword{hv} to be carried on the same cable and decoupled inside the purity monitor electronics module. 
A waveform digitizer that interfaces with a \dword{daq} PC records the amplified anode and cathode outputs. 
The signal and \dword{hv} cable shields connect to the grounding points of the cryostat and are separated from the electronic ground with a resistor and a capacitor connected in parallel, mitigating ground loops between the cryostat and the electronics racks. Amplified output is transmitted to an AlazarTech ATS310 waveform digitizer\footnote{AlazarTech ATS310\texttrademark{} - 12 bit, 20 MS/s,  https://www.alazartech.com/Product/ATS310.} containing two input channels, each with 12 bit resolution. Each channel can sample a signal at a rate of \SI{20}{\mega\samples\per\second} to \SI{1}{\kilo\samples\per\second} and store up to \SI{8}{\mega\samples} in memory. One digitizer is used for each purity monitor, and each digitizer interfaces with the \dword{daq} PC across the \dword{pci} bus. 

A custom \dword{labview}\footnote{National Instruments, LabVIEW\texttrademark{}, http://www.ni.com/en-us.html} application running on the \dword{daq} PC has two functions: it controls the waveform digitizers and the power supplies, and it monitors the signals and key parameters. The application configures the digitizers to set the sampling rate, the number of waveforms to be stored in memory, the pre-trigger data, and a trigger mode. A signal from a photodiode triggered by the xenon flash lamp is fed directly into the digitizer as an external trigger to initiate data acquisition.   \dword{labview} automatically turns on the xenon flash lamp by powering a relay when data taking begins and then turns it off when finished.
The waveforms stored in the digitizers are transferred to the \dword{daq} PC and used to obtain averaged waveforms to reduce the electronic noise in them. % waveforms. 

The baseline  is estimated by averaging the waveform samples prior to the trigger. This baseline is subtracted from the waveforms prior to calculating peak voltages of the cathode and anode signals. %These processes are performed in real time within the application and are then used to estimate the electron lifetime.
The application performs these processes in real time. % within and uses the results to estimate the electron lifetime. 
 The application continuously displays the waveforms and important parameters like measured electron lifetime, peak voltages, and drift time % of electrons 
in the purity monitors, and shows the variation in these parameters over time, thus pointing out %.This allows validating the impurity of the \dword{lar} and seeing 
effects that might otherwise be missed. %may not be spotted at that instant. 
Instead of storing the measured parameters, the waveforms and the digitizer configurations are recorded in binary form for offline analysis.  \dword{hv} modules\footnote{iseg Spezialelektronik GmbH\texttrademark{} high voltage supply systems, https://iseg-hv.com/en.} in a WIENER MPOD mini crate\footnote{W-IE-NE-R MPOD\texttrademark{} Universal Low and High Voltage Power Supply System, http://www.wiener-d.com/.} supply negative voltages to the cathode and positive voltages to the anode. The  \dword{labview} application controls and monitors the \dword{hv} systems through an Ethernet interface.

The xenon flash lamp and the \dword{fe} electronics are installed close to the purity monitor flange, to reduce light loss through the optical fiber and prevent signal loss. Other pieces of equipment that distribute power to these items and collect data from the electronics are mounted in a rack separate from the cryostat. The slow control system communicates with the purity monitor \dword{daq} software and controls  the \dword{hv} and \dword{lv} power supplies of the purity monitor system. The optical fiber must be placed within  \SI{0.5}{\milli\meter} of the photocathode for efficient \phel extraction, so little interference with the \dword{pds} is expected. The purity monitors could induce noise in the \dword{tpc} electronics, in particular via the current surge through a xenon lamp when it is flashed.  This source of noise can be controlled by placing the lamp inside its own Faraday cage with %, which allows 
proper grounding and shielding. %According to the operation of purity monitors 
At \dword{pdsp}, after careful checks of the grounding, this noise has remained well below the noise generated by other sources.

In the \dword{spmod} we can make use of triggering to prevent any potential noise from the purity monitor's flash lamp from affecting \dword{tpc} and \dword{pds} signals. The \dword{lartpc} trigger rate is a few hertz, and each trigger window is one or a few milliseconds. %For a purity monitor's flash lamp,  the flash light 
The pulse from a flash lamp is very short (a microsecond or so, much shorter than the gaps between \dword{lartpc} trigger windows). 
Thus, a \dword{lartpc} trigger signal may be sent to a programmable pulse generator, %and the pulse generator then 
which generates a trigger pulse that does not overlap with \dword{lartpc} trigger windows. This trigger pulse %will then be 
is then sent to the external trigger port on the flash lamp \dword{hv} controller so that the lamp flashes between \dword{lartpc} trigger windows. In this way, the electronic and light noises from the flash lamp do %will 
not affect %LArTPC 
data taking at all.



%%%%%%%%%%%%%%%%%%%%%%%%%%%%%%%%%%%%%%%%%%
\subsubsection{Production and Assembly}
\label{sec:PrMon-Production-Assembly}

The \dword{cisc} consortium will produce the individual purity monitors, test them in a test stand, and confirm that each monitor operates at the required level before assembling them into the strings of three monitors each that will be mounted in the \dword{spmod} cryostat using support tubes. The assembly process will follow the methodology developed for \dword{protodune}.



A short version of the %assembly
string
%\fixme{string?}
with all purity monitors will be tested at the \dword{citf}.
%in the \dword{lar} test facility. 
The full string will be assembled and shipped to \dword{surf}. 
 A vacuum test in a long vacuum tube will be performed on-site before inserting the full assembly into the \dword{spmod} cryostat. 



%%%%%%%%%%%%%%%%%%%% LIQUID LEVEL MONITORING %%%%%%%%%%%%%%%%%%%%
\subsection{Liquid Level Meters}
% john L, anselmo
% SP

The goals for the \dword{lar} level monitoring system are basic level sensing when filling, and precise level sensing during static operations. 

Filling the cryostat with \dword{lar} will take several months. During this operation several systems will be use to monitor the \dword{lar} level. 
The first \SI{5.5}{m} will be covered by cameras and by the vertical arrays of \dwords{rtd} at known heights, since temperature will change drastically 
when the cold liquid reaches each \dword{rtd}. Once the liquid reaches the level of the cryogenics pipes going out of the cryostat, 
the differential pressure between that point and the bottom of the cryostat
can be converted to depth using
the known density of \dword{lar}.   Fine tuning of the final \dword{lar} level will be done using several capacitive level meters at the top of the cryostat. 

During operation, liquid level monitoring has two purposes:
the \dword{lbnf} cryogenics system uses monitoring to tune the \dword{lar} flow, and 
DUNE uses monitoring to guarantee that the top \dwords{gp} are always
submerged 
at least \SI{20}{cm} below the \dword{lar} surface to mitigate the risk of dielectric breakdown. This was the value used for the \dword{hv} interlock in \dword{pdsp}. 

The \dword{lar} flow is tuned using two differential pressure level meters, installed as part of the cryogenics system, one on each side of the \dword{detmodule}.  They have a precision of \SI{0.1}{\%}, which corresponds to \SI{14}{mm} at thenominal \dword{lar} surface. Cryogenic pressure sensors will be purchased from commercial sources. Installation methods and positions will be determined as part of the
cryogenics internal piping plan.  

For \dword{hv} integrity, multiple \SI{4}{m} long capacitive level sensors (with a precision of less than \SI{5}{mm}) will be deployed along the top of the fluid %, used 
for use during stable operations, and checked against each other.
One capacitive level sensor at each of the four corners of the cryostat will provide sufficient redundancy to ensure that no single point of failure compromises the %capacitive level sensor 
measurement.


Figure~\ref{fig:cisc_pdsp_level} shows the evolution of the \dword{pdsp} \dword{lar} level over two months as measured by the differential pressure and capacitive level meters. 

\begin{dunefigure}[\lar level measurements]{fig:cisc_pdsp_level}{Evolution of the \dword{pdsp} \dword{lar} level over two months. Left: Measured by the capacitive level meter. Right: Measured by the differential pressure level meter. The units in the vertical axis are percentages of the cryostat height (\SI{7878}{mm}).}
  \includegraphics[width=0.4\textwidth]{cisc_level_cap.png}%
  \hspace*{1cm}
  \includegraphics[width=0.4\textwidth]{cisc_level_diffp.png}%
\end{dunefigure}

\dword{pdsp}  uses the same design for differential pressure level meters %which will also be used in the far detector 
as the \dword{spmod}. In the case of capacitive level meters, \dword{pdsp} is using commercially bought \SI{1.5}{m} long level meters while \dword{pddp} is using  \SI{4}{m} long level meters that are custom-built by \dword{cern}. %For the DUNE \dword{fd}, w
We plan to use the longer capacitive level meters custom-built by  \dword{cern} for both \dword{sp} and \dword{dp} modules.

\subsection{Pressure Meters}
\label{sec:fdgen-slow-cryo-press-meter}

The absolute temperature in the liquid varies with the pressure in the argon gas, therefore, precise measurements of pressure inside the cryostat allows for a better understanding of temperature gradients and \dword{cfd} simulations. In \dword{protodune}, pressure values were also be used to understand the strain gauge signals installed in the cryostat frame.

Standard industrial pressure sensors can be used to measure the pressure of the argon gas in the ullage in the cryostat. For the DUNE \dword{fd}, the plan is to follow the same design and configuration used in \dword{pdsp}. \dword{protodune} uses two types of pressure sensors and a pressure switch, % as described below:
\begin{itemize}
    \item a relative pressure sensor (range: 0-400~mbar, precision: 0.01~mbar),
    \item an absolute pressure sensor (range: 0-1600~mbar, precision: 0.05~mbar), and
    \item a mechanical relative pressure switch adjustable from 50 to 250~mbar. 
\end{itemize}

Both sensors and the pressure switch are installed in a dedicated flange as shown in Figure~\ref{fig:pdsp-pressure} and are connected directly to a slow controls system \dword{plc} circuit. Dedicated analog inputs are used to read the current values (\SIrange{4}{20}{mA}) which are then converted to pressure according to the sensors range. Given the much larger size of the  \dword{dune} \dwords{detmodule}, the system described above will be doubled for redundancy: two flanges in opposite cryostat sides will be instrumented with three sensors each. 

\begin{dunefigure}[Pressure sensors installed on a flange in ProtoDUNE-SP]{fig:pdsp-pressure}
  {Photograph of the pressure sensors installed on a flange in \dword{pdsp}.}
  \includegraphics[width=0.5\textwidth]{graphics/cisc-pdsp-pressure-meters}
\end{dunefigure}

Further, relative and absolute pressure sensors (with comparatively lower precision) are installed by \dword{lbnf} that are also recorded by the slow controls system. The availability of two types of sensors from \dword{lbnf} and \dword{cisc} provides redundancy, independent measurements, and cross checks.

%%%%%%%%%%%%%%%%%%% GAS ANALYZERS %%%%%%%%%%%%%%%%%%%%
\subsection{Gas Analyzers}
\label{sec:fdgen-slow-cryo-gas-anlyz}
% alan h

 Gas analyzers are commercially produced modules that measure trace quantities of specific gases contained within a stream of carrier gas. The carrier gas for \dword{dune} is argon, and the trace gases of interest are oxygen ($\text{O}_2$), water ($\text{H}_2\text{O}$), and nitrogen ($\text{N}_2$). $\text{O}_2$ and $\text{H}_2\text{O}$ affect the electron lifetime in \dword{lar} and must be kept below \SI{0.1}{ppb} ($\text{O}_2$ equivalent) while $\text{N}_2$ affects the efficiency of scintillation light production at levels higher than \SI{1}{ppm}.
The argon is sampled from either the argon vapor in the ullage or from the \dword{lar} by using small diameter tubing run from the sampling point to the gas analyzer. Typically, the tubing from the sampling points are connected to a switchyard valve assembly used to route the sample points to the desired gas analyzers (see Figure~\ref{fig:GA-switchyard}).


\begin{dunefigure}[A gas analyzer switchyard valve assembly]{fig:GA-switchyard}
  {A gas analyzer switchyard that routes sample points to the different gas analyzers.}
  \includegraphics[width=0.9\textwidth]{cisc_GasAnalyzerSwitchyard.png}
\end{dunefigure}

%The following lists three times 
%The gas analyzer would be used three times:
The gas analyzer would be predominantly used during three periods:

\begin{enumerate}
\item Once the detector is installed and after the air atmosphere is eliminated from the cryostat to levels low enough to begin \cooldown. This purge and gas recirculation process is detailed in Section~\ref{sec:fdgen-slow-cryo-install-ga}. Figure~\ref{fig:cisc_Phase1_purge_gas_recirculation} shows the evolution of the $\text{N}_2$, $\text{O}_2$, and $\text{H}_2\text{O}$ levels from gas analyzer data taken during the purge and recirculation stages of the \dword{dune} \dword{35t} %Prototype P
phase 1 run.


\item Before other means of monitoring impurity levels (e.g., purity monitors, or \dshort{tpc} tracks) are sensitive, to track trace $\text{O}_2$ and $\text{H}_2\text{O}$ contaminants from $\>$tens of ppb to hundreds of ppt.  Figure~\ref{fig:cisc_O2AnalyzerPrM2_HVRun1} shows an example plot of $\text{O}_2$ levels at the beginning of \dword{lar} purification from one of the later \dword{35t} \dword{hv} runs.

\item During cryostat filling to monitor the tanker \dword{lar} delivery purity. This tracks the impurity load on the filtration system and rejects any deliveries that do not meet specifications. %Likely 
Specifications for the delivered \dword{lar} are in the \SI{10}{ppm} range per contaminant.

\end{enumerate}

\begin{dunefigure}[Impurity levels during the pre-fill stages for \dshort{35t} phase 1]{fig:cisc_Phase1_purge_gas_recirculation}
  {Plot of the $\text{O}_2$, $\text{H}_2\text{O}$, and $\text{N}_2$ levels during the piston purge and gas recirculation stages of the \dword{35t} Phase 1 run}
  \includegraphics[width=0.7\textwidth]{cisc_Phase1_purge_gas_recirculation}
\end{dunefigure}

\begin{dunefigure}[$\text{O}_2$ just after the \dshort{35t} was filled with \lar]{fig:cisc_O2AnalyzerPrM2_HVRun1}
  {$\text{O}_2$ as measured by a precision $\text{O}_2$ analyzer just after the \dword{35t} cryostat was filled with \dword{lar}, continuing with the \dword{lar} pump start and beginning of \dword{lar} recirculation through the filtration system. As the gas analyzer loses sensitivity, the purity monitor can pick up the impurity measurement. Note that the purity monitor is sensitive to both $\text{O}_2$ and $\text{H}_2\text{O}$ impurities giving rise to its higher levels of impurity.}
  \includegraphics[width=0.7\textwidth]{cisc_O2AnalyzerPrM2_HVRun1.png}
\end{dunefigure}

%Because 
Since any one gas analyzer covers only one contaminant species and a range of \numrange{3}{4} orders of magnitude, several units are needed both for the three contaminant gases and to cover the ranges seen between  cryostat closure and the beginning of \dshort{tpc} operations:
\SI{20}{\percent} to $\lesssim 100$~ppt for $\text{O}_2$,
\SI{80}{\percent} to $\lesssim 1$~ppm for $\text{N}_2$, and
$\sim \SI{1}{\percent}$ to $\lesssim 1$~ppb for $\text{H}_2\text{O}$.
Because the total cost of these analyzers exceeds $\SI{100}[\mathdollar]{k}$, we want to be able to  sample more than a single location or cryostat with the same gas analyzers. At the same time, the tubing run lengths from the sample point should be as short as possible to maintain a timely response for the gas analyzer.  This puts some constraints on sharing devices because, for example, argon is delivered at the surface, so a separate gas analyzer may be required there. %at the surface.


%%%%%%%%%%%%%%%%%%%%%55 CAMERAS %%%%%%%%%%%%%%%%%%%%%%%%5
\subsection{Cameras}
% glenn, jim s, chuck
% same text in single and dual phase

Cameras provide direct visual information about the state of the
\dword{detmodule} during critical operations and when damage or unusual
conditions are suspected.  Cameras in the \dword{wa105} showed spray from \cooldown
nozzles and the level and state of the \dword{lar} as it covered the \dword{crp} \cite{Murphy:20170516}.  A camera was
used in the \dword{lapd} 
cryostat\cite{Adamowski:2014daa} to study \dword{hv} discharges in
\dword{lar} and in EXO-100 while a \dword{tpc} was operating
\cite{Delaquis:2013hva}.  Warm cameras viewing \dword{lar} from a distance
have been used to observe \dword{hv} discharges in \dword{lar} in
fine detail \cite{Auger:2015xlo}.  Cameras are commonly used during
calibration source deployment in many experiments (e.g., the
\kamland ultra-clean system \cite{Banks:2014hra}).

In \dword{dune}, cameras will verify the stability, straightness,
and alignment of the hanging \dword{tpc} structures during \cooldown and
filling; ensure that no bubbling occurs near the \dwords{gp}
(\single) or \dwords{crp} (\dual);  inspect the
state of movable parts in the \dword{detmodule} (calibration devices, dynamic
thermometers); and  closely inspect parts of the \dword{tpc} after any seismic activity or other unanticipated
event.  For these functions, a set of fixed
cold cameras are used; they are permanently mounted at fixed points in the cryostat
for use during filling and commissioning, and a movable, replaceable
warm inspection camera can be deployed through any free
instrumentation flange at any time during the life of the
experiment. 

Eleven cameras were deployed in \dword{pdsp} at the locations shown in Figure~\ref{fig:pdsp-camera-locations}. They successfully provided views of the detector during filling and throughout %the 
its operation. % of the detector.

\begin{dunefigure}[Camera locations in ProtoDUNE-SP]{fig:pdsp-camera-locations}
  {A \threed view showing the locations of the 11 cameras in \dword{pdsp}.}
  \includegraphics[width=0.6\textwidth]{pdsp-camera-locations-3d}%
\end{dunefigure}

The following sections describe the design considerations for both cold
and warm cameras and the associated lighting system. \dword{pdsp} camera system designs and performance are also discussed.  
The same basic
designs can be used for both the \dword{spmod} and the \dword{dpmod}. %\single and \dual \dwords{detmodule}.



%%%%%%%%%%%%%%%%%%%%%%%%%%%%%%%%%%%%%%%
\subsubsection{Cryogenic Cameras (Cold)}

The fixed cameras
monitor the following items during filling:
\begin{itemize}
\item positions of corners of \dword{apa}, \dword{cpa}, \dwords{fc}, \dwords{gp} (\SI{1}{mm} resolution)
\item relative straightness and alignment of \dword{apa}, \dword{cpa}, and \dword{fc} (\(\lesssim\SI{1}{mm}\))
\item relative positions of profiles and endcaps (\SI{0.5}{mm} resolution); and 
\item the \dword{lar} surface, specifically, the presence of bubbling or debris.
\end{itemize}




One design for the \dword{dune} fixed cameras uses an enclosure similar to
the successful EXO-100 design \cite{Delaquis:2013hva}, which was also
successfully used in the \dword{lapd}
and \dword{pdsp} (see Figure~\ref{fig:gen-fdgen-cameras-enclosure}). Cameras 101, 102, 104, and 105, shown in Figure~\ref{fig:pdsp-camera-locations}, use this enclosure.
A thermocouple in the enclosure allows temperature monitoring, and a heating element provides temperature control.  
SUB-D connectors are used at the cryostat flanges and the camera enclosure for signal, power, and control connections.

\begin{dunefigure}[A camera enclosure]{fig:gen-fdgen-cameras-enclosure}
  {Top left: a CAD exploded view of a vacuum-tight camera enclosure suitable for cryogenic applications \cite{Delaquis:2013hva}.
    (1) quartz window, (2 and 7) copper gasket, (3 and 6) flanges, (4) indium wires, (5) body piece, (8) signal \fdth.
    Top right: two of the \dword{pdsp} cameras using a stainless steel enclosure. 
    Bottom left: one of the \dword{pdsp} cameras using acrylic enclosure.
    Bottom right: a portion of an image taken with \dword{pdsp} camera 105 showing a purity monitor mounted outside the \dword{apa} on the beam left side. This photo was taken with \dword{pdsp} completely filled.
  }
  \includegraphics[width=0.4\textwidth]{exo100-camera-case}%
  \includegraphics[width=0.4\textwidth]{edgar-cameras}\\
  \hfill \includegraphics[width=0.22\textwidth]{bo-camera}%
  \hfill \includegraphics[width=0.22\textwidth]{camera-105-purmon-orig-rot-crop}%
  \hfill
\end{dunefigure}

An alternative design uses an acrylic enclosure.
This design was used successfully in \dword{pdsp} (see Figure~\ref{fig:gen-fdgen-cameras-enclosure}, bottom left). Cameras 001, 002, 004, and 005, shown in Figure~\ref{fig:pdsp-camera-locations}, use acrylic enclosures. 
All operate successfully, including those at the bottom of the cryostat. % It must be noted that the \dword{dune} \dword{fd} 
The \dword{fd} modules will be twice as deep as \dword{protodune}, and therefore cameras observing the lowest surfaces of the \dword{fc} must withstand twice the pressure.

Improved designs for the cold cameras will be tested in \dword{pddp} and \dword{citf} for improved imaging including focus adjustment, and in \dword{citf} for pressure resistance, during 2020. 



%%%%%%%%%%%%%%%%%%%%%%%%%%%%%%%%%%%%%%%%%%%%%%%5
\subsubsection{Inspection Cameras (Warm)}

The inspection cameras are intended to be as versatile as possible.
The following %locations 
inspections have been identified as likely uses:
%to be of interest:
\begin{itemize}
\item status of \dword{hv} \fdth and cup,
\item status of \dword{fc} profiles, endcaps (\SI{0.5}{mm} resolution),
\item vertical deployment of calibration sources,
\item status of thermometers, especially dynamic thermometers,
\item \dword{hv} discharge, corona, or streamers on \dword{hv} \fdth, cup, or \dword{fc},
\item relative straightness and alignment of \dword{apa}, \dword{cpa}, and \dword{fc} (\SI{1}{mm} resolution),
\item gaps between \dword{cpa} frames (\SI{1}{mm} resolution),
\item relative position of profiles and endcaps (\SI{0.5}{mm} resolution), and 
\item sense wires at the top of outer wire planes in \single \dword{apa} (\SI{0.5}{mm} resolution).
\end{itemize}

Unlike the fixed cameras, the inspection cameras must operate only as
long as any inspection; the cameras can be replaced in case of failure.  It
is also more practical to keep the cameras continuously warmer than
 \SI{-150}{\celsius} during deployment; this allows use of  %and therefore we willhave 
commercial cameras, %more easily, e.g., %.  For example, we could deploy 
e.g., cameras of the same model were used successfully to observe discharges
in \dword{lar} from \SI{120}{cm} away \cite{Auger:2015xlo}.


The inspection cameras use the same basic
enclosure design as for cold cameras, but the cameras are mounted on a movable
fork so that each camera can be inserted and removed from the cryostat,
using a design similar to the dynamic temperature probes: see
 Figure~\ref{fig:gen-fdgen-cameras-movable} (left) and
 Figure~\ref{fig:fd-slow-cryo-sensor-mount}.  To avoid contaminating the
\dword{lar} with air, the entire system is sealed, and the
camera can only be deployed through a \fdth equipped with a gate
valve and a purging system, similar to the one used in the vertical axis
calibration system at \kamland~\cite{Banks:2014hra}. The entire system
is  purged with pure argon gas before the gate valve is opened.

\begin{dunefigure}[Inspection camera design]{fig:gen-fdgen-cameras-movable}
  {Left: An overview of the inspection camera design using a sealed deployment system opening directly into the cryostat. Right: A photo of the \dword{pdsp} warm inspection camera acrylic tube, immediately before installation; the acrylic tube is sealed with an acrylic dome at the bottom and can be opened at the top.}
  \includegraphics[height=0.3\textheight]{Camera-Sketch}%
  \includegraphics[height=0.3\textheight]{cisc-pdsp-camera-tube}%
\end{dunefigure}

Motors above the flange allow the fork to be rotated and moved vertically to position the camera. 
 A chain drive system with a motor
mounted on the end of the fork allows the camera assembly to tilt, 
creating a point-tilt mount that can be moved vertically.
With the space above the cryostat flanges and the
thickness of the cryostat insulation, cameras can be moved vertically up to
\SI{1}{m} inside the cryostat.
The motors for rotation and vertical motion are outside the sealed
volume, coupled mechanically using ferrofluidic seals, thus reducing any risk of
contamination and allowing manual rotation of the vertical
drive if the motor fails.  

An alternative design was demonstrated in \dword{pdsp}. In this design, the warm camera is contained inside a gas-tight acrylic tube inserted into the feedthrough, so a gate valve or a gas-tight rotatable stage is not needed, and the warm cameras can be removed for servicing or upgrade at any time. Figure~\ref{fig:gen-fdgen-cameras-movable} (right) shows an acrylic tube enclosure and camera immediately before deployment. 
These acrylic tube enclosures for removable cameras were deployed in \dword{pdsp} at the positions marked 201, 202, and 203 in Figure~\ref{fig:pdsp-camera-locations}; they operated successfully. Cameras with fisheye lenses were used in these tubes during initial operation.  One camera was removed without any evidence of contamination of the \dword{lar}. We plan to use other cameras during post-beam running.

Improved designs for the inspection cameras will be tested in the \dword{citf} and \dword{pdsp} during 2020 and 2021, focusing particularly on longevity, camera replaceability, and protection of the \dword{lar}.


%%%%%%%%%%%%%%%%%%%%%%%%%%%%%%%%%%%%%%%
\subsubsection{Light-emitting system}
%%% same text as dual-phase
The light-emitting system uses \dwords{led} to illuminate
the parts of the %\dword{fd} 
\dword{detmodule} in the camera's field of view with selected
wavelengths (IR and visible) that cameras can detect.  Performance criteria for the light-emission system include the efficiency with which the cameras can detect the light and the need to avoid
adding heat to the cryostat. Very high-efficiency
\dwords{led}   
help reduce heat generation; one \SI{750}{nm} \dword{led} \cite{lumileds-DS144-pdf}
has a specification equivalent to
\SI{33}{\%} conversion of electrical input power to light.

While data on how well \dwords{led} perform at cryogenic temperatures
is sparse, some studies of NASA projects~\cite{Carron:2017zzz}
indicate that \dwords{led} are more efficient at low temperatures and
that emitted wavelengths may change, particularly for blue
\dwords{led}.  In \dword{pdsp}, amber \dwords{led} were observed  to
emit green light at \dword{lar} temperature (see bottom right photo
in \ref{fig:gen-fdgen-cameras-enclosure}).  To avoid degradation of
wavelength-shifting materials in the \dword{pds}, short wavelength
\dwords{led} are not used in the \dword{fd}; \dwords{led} will be tested
in \dword{ln} to ensure their wavelength is long enough.


\dwords{led} are placed in a ring around the outside of each
camera, pointing in the same direction as the lens, to 
illuminate nearby parts of the \dword{detmodule} within the camera's field of
view. Commercially available \dwords{led} exist with
a range of angular spreads that can be matched to the needs of the
cameras without additional optics.

Additionally, chains of \dwords{led} connected in series and driven with a
constant-current circuit are used for broad illumination, with each
\dword{led} paired in parallel with an opposite polarity \dword{led} and a resistor
(see Figure~\ref{fig:cisc-LED}).
This allows two different wavelengths of illumination using a single chain simply by changing the direction of the drive current, and allows continued use of an \dword{led} chain even if individual \dwords{led} fail.

\begin{dunefigure}[Example schematic for LED chain]{fig:cisc-LED}
  {Example schematic for LED chain, allowing failure tolerance and two LED illumination spectra.}
  \includegraphics[width=0.6\textwidth]{cisc_led.png}
\end{dunefigure}


%%%%%%%%%%%%%%%%%% CRYOGENICS Instrumentation TEST FACILITY %%%%%%%%%%%%%%%%%%%%
\subsection{Cryogenics Instrumentation Test Facility}
% same for SP and DP
% alan h
The \dword{cisc} consortium plans to build a \dfirst{citf} at \dword{fnal} to facilitate testing of various cryogenics instrumentation devices and small-scale assemblies of \dword{cisc} systems. 
%The \dfirst{citf} at 
In the past and recent times, various test facilities at \dword{fnal} have provided access to small ($<\,\SI{1}{ton}$) to intermediate ($\sim\,\SI{1}{ton}$) volumes of purified \dword{tpc}-grade \dword{lar}, required for %. Hardware that requires this high-purity liquid includes 
any device intended for drifting electrons for millisecond periods. 


The \dword{pab} facility at \dword{fnal} houses the \dword{iceberg} \SI {3000} {liter} cryostat which enables fast turnaround testing for the \dword{dune} \dword{ce}. 

The \dword{pab} facility also includes  \dword{tallbo} (\SI {450} {liter}), Blanche (\SI {500} {liter}), and Luke (\SI {250} {liter}) cryostats. %All these cryostats are available for outside use.
In the recent past, Blanche has been used for \dword{hv} studies,  \dword{tallbo} for \dword{pd} studies, and Luke for the material test stand work. These studies have contributed to the design and testing of  \dword{pdsp} components.

\subsection{Validation in ProtoDUNE}
\label{sec:pdsp-cryo-valid}

Design validation and testing of many planned \dword{cisc} systems for
the \dword{spmod} will be done using the data from
\dword{pdsp} and \dword{pddp} as discussed below.

\begin{itemize}
	\item Level Meters: The same differential pressure level meters
	      which are already validated in \dword{pdsp} will be used in the \dword{spmod}. The same capacitive level meters currently used in
	      \dword{pddp} will be used in the \dword{spmod}. These will be
	      validated in the upcoming \dword{pddp} run.
	      
	\item Pressure Meters (GAr): The same high-precision pressure sensors that are already validated in \dword{pdsp} will be used in \dword{sp} \dword{fd}.
	
	\item Gas Analyzers: The same gas analyzers currently used in
	      \dword{pdsp} will be used in the \dword{spmod}, so they have already
	      been validated.
	      
	\item High-precision thermometer arrays in \dword{lar}: The static
	      and dynamic T-gradient thermometers discussed in the previous sections are validated using \dword{pdsp} data.
	      
	\item Purity Monitors: The same purity monitor basic design used in \dword{pdsp} will be used in the \dword{spmod}. \dword{protodune2} at \dword{cern} provides opportunities to
	      test any improvements to the design.
	\item Cameras: various types of cameras are being actively
	      developed in both \dword{pdsp} and \dword{pddp} so these detectors will perform validation of the 
	      designs. Future
	      improvements can be tested in \dword{protodune2} at \dword{cern}. % \dword{pdsp} and \dword{pddp} run 2 phaseat CERN.
\end{itemize}


%%%%%%%%
\section{Slow Controls}
% same for SP and DP

The slow controls system collects, archives, and displays data from
a broad variety of sources and provides real-time status, alarms, and warnings for detector operators. The slow control system also provides control for %some detector components of the detector systems 
items such as \dword{hv} systems, \dword{tpc} electronics, and \dword{pd} systems. Data is acquired via network interfaces.  Figure~\ref{fig:gen-slow-controls-diagram} shows connections between major parts of the slow controls system and other systems. %Hardware, infrastructure, and software are the three main components of the slow controls system. %, each of which are described in this section.

\begin{dunefigure}[Slow controls connections and data]{fig:gen-slow-controls-diagram}
{Typical slow controls system connections and data flow}
\includegraphics[width=0.7\textwidth]{cisc_slow-controls-diagram}
\end{dunefigure}

The \dword{pdsp} detector control system\cite{pdspdcs_proc} fully met its operational requirements. %the requirements for operating \dword{pdsp}. 
Section~\ref{sec:cisc-slow-control-pdsp} provides a short description of the \dword{pdsp} slow controls and its performance.

%%%%%%%%%%%%%%%%%%%%%%%%%%%%%%%%%%%
 \subsection{Slow Controls Hardware}
\label{sec:fdgen-slow-cryo-hdwr}

%A modest amount of dedicated hardware is envisioned for 
Slow controls is expected to need a modest amount of dedicated hardware, largely for rack monitoring,  %Additionally, slow controls will always 
and a small amount of dedicated network and
computing hardware. % as described below. 
Slow controls also relies on common
infrastructure as described in
Section~\ref{sec:fdgen-slow-cryo-slow-infra}.

%%%%%%%%%%%%%%%%%
\subsubsection{Dedicated Monitoring Hardware}

Every rack (including those in the \dword{cuc}) should have dedicated hardware to monitor rack parameters like rack protection system, rack fans, rack air temperatures, thermal interlocks with power supplies, and any interlock bit status monitoring needed for the racks. For the racks in the \dword{cuc} server room, this functionality is built into the proposed water cooled racks, as already in place at \dword{protodune}.  For the racks on the detector itself, the current plan is to design and install a custom-built 1U rack-mount enclosure containing a single-board computer to control and monitor various rack parameters. Such a system has been successfully used in \dword{microboone}. The design is being improved for the \dword{sbnd} experiment (see Figure~\ref{fig:slow-controls-rack-box}). Other slow controls hardware includes interfacing cables like adapters for communication and debugging and other specialized cables like  \dword{gpib} or National Instruments cables. The cable requirements must be determined in consultation with other groups once hardware choices for various systems are finalized.

\begin{dunefigure}[Rack monitoring box prototype for the SBND; based on MicroBooNE design]{fig:slow-controls-rack-box}
{Rack monitoring box prototype in development for the \dshort{sbnd} experiment based on the original design from \dshort{microboone}.}
\includegraphics[width=0.6\textwidth]{cisc-slow-controls-rackbox}
\end{dunefigure}


% % % % Alec
\subsubsection{Slow Controls Network Hardware}
\label{sec:fdgen-slow-cryo-slow-network}
The slow controls data originates from the cryogenics instrumentation and from other systems as listed in Table~\ref{tab:gen-slow-quant}. This data is collected by software running on servers
(Section~\ref{sec:fdgen-slow-cryo-slow-compute})
housed in the underground data room in the \dword{cuc},
where data is archived in a central \dword{cisc} database.
The instrumentation data is transported over
conventional network hardware from any sensors located in the cryogenics
plant.  However, the readouts that are in the racks on top of the
cryostats must be cautious about grounding and noise.  Therefore, each
rack on the cryostat has a small network switch that sends
any network traffic from that rack to the \dword{cuc} via a fiber transponder.
This is the only network hardware specific to slow controls and will be provided by %the laboratory's
\dword{surf}'s  
general computing infrastructure. % managed by project \dword{tc}. 
The network infrastructure requirements are described in
Section~\ref{sec:fdgen-slow-cryo-slow-infra}.

% % % % Alec
\subsubsection{Slow Controls Computing Hardware}
\label{sec:fdgen-slow-cryo-slow-compute}
Two servers (a primary server and a replicated backup) suitable for the relational database discussed
in Section~\ref{sec:fdgen-slow-cryo-sw} are located in the \dword{cuc} data
room, with an additional
two servers to service the \dword{fe} monitoring interface.  These additional servers would cover assembling dynamic \dword{cisc} monitoring web pages from adjacent
databases. Yet another server will be needed to run back-end I/O.  Any special purpose software, such as iFix used by the cryogenics experts, would
also run here. One or two additional servers should accommodate these programs.
% (The exact number of iFix machines will be determined based
% on input from the LBNF cryogenics experts.)
Replicating this setup on a per-module basis would make commissioning and independent operation easier, accommodate different module
design (and the resulting differences in database tables), and ensure
sufficient capacity.  These four sets of networking hardware would fit tightly into one rack or very comfortably into two. \fixme{No cost info in TDR: Using the requirements from \dword{cisc}, the \dword{daq} consortium will provide the needed cost estimates for servers and racks. }




%%%%%%%%%%%%%%%%%%%%%%%%%%%%%%%%%%%
% Alec
\subsection{Slow Controls Infrastructure}
\label{sec:fdgen-slow-cryo-slow-infra}

The data rate will be in the range of tens of kilobytes per second, given the total number of slow controls quantities and the update rate  
(see Section~\ref{sec:fdgen-slow-cryo-quant}), placing minimal demands
on local network infrastructure.
Network traffic out of \dword{surf} to \dword{fnal} will primarily be database calls
to the central \dword{cisc} database, either from monitoring applications or from
database replication to the offline version of the \dword{cisc} database.  This
traffic requires little bandwidth, so the proposed general purpose
links both out of the %mine 
underground area at  \dword{surf} and back to \dword{fnal} can accommodate the traffic.

Up to two racks of space and appropriate power and cooling are
available in the \dshort{cuc}'s \dword{daq} server room for \dword{cisc} use. This is ample space as described in Section~\ref{sec:fdgen-slow-cryo-slow-compute}.


%%%%%%%%%%%%%%%%%%%%%%%%%%%%%%%%%%
% Sowjanya
\subsection{Slow Controls Software}
\label{sec:fdgen-slow-cryo-sw}

To provide complete monitoring and control of detector subsystems, the slow controls software includes
%
\begin{itemize}
 \item the control systems base for input and output operations
  and defining processing logic, scan conditions, and alarm conditions;
 \item an alarm server to monitor all channels and send alarm
  messages to operators;
 \item a data archiver for automatic sampling and storing values for history tracking; and 
 \item an integrated operator interface providing display panels for
  controls and monitoring.
\end{itemize}

In addition, the software must be able to 
interface indirectly with external systems (e.g., cryogenics control
system) and databases (e.g., beam database) to export data into
slow controls process variables (or channels) for archiving and status
displays. This allows us to integrate displays and warnings into one
system for the experiment operators and %to provide 
provides integrated
archiving for sampled data in the archived database. As one possibility, an input output controller running on a central \dword{daq}
server could provide soft channels for these data.
Figure~\ref{fig:gen-slow-controls-diagram} shows a typical workflow of a
slow controls system.

The key features of the software require highly evolved software designed to manage real-time data exchange, scalable
to hundreds of thousands of channels sampled at intervals of hours to seconds as needed. The software
must be well documented, supported, and reliable. The base
software must also allow easy access to any channel by name. The
archiver software must allow data storage in a database with
adjustable rates and thresholds so data
for any channel can be easily retrieved using channel name and time range. Among other key
features, the alarm server software must remember the state, support an
arbitrary number of clients, and provide logic for delayed alarms and
acknowledging alarms. A standard naming
convention for channels will be part of the software to help handle large
numbers of channels and subsystems.

%\fixme{CP: Paragraph below modified to explain the baseline software here}
The \dword{pdsp} detector control system software \cite{pdspdcs_proc} provides a prototype for %this required 
the \dword{fd} slow controls software.
In \dword{pdsp}, the unified control system base is WinCC OA \cite{winccoa}, a
commercial toolkit used extensively at \dword{cern}, with device interfaces
supported using several standardized interface protocols. A more detailed description is in Section~\ref{sec:cisc-slow-control-pdsp} below.
WinCC OA is our baseline for the \dword{fd} slow control software.
EPICS~\cite{epics7} is an alternative controls system which also meets the specifications; it is used in other neutrino experiments including \dword{microboone}~\cite{microboone} and \dword{nova}~\cite{Lukhanin:2012fp}. 


%%%%%%%%%%%%%%%%%%%%%%%%%%%%%%%%%%
% Ed T
\subsection{Slow Controls Quantities}
\label{sec:fdgen-slow-cryo-quant}

% starter text from Glenn

The final set of quantities to monitor will ultimately be determined
by the subsystems being monitored, documented in
appropriate  interface control documents (ICDs), and continually revised based on operational
experience.  The total number of quantities to monitor has been roughly estimated by taking the total number monitored
in \dword{pdsp}\cite{pdspdcs_proc}, 7595 as of Nov. 19, 2018, and scaling by the detector length and the number of planes, giving approximately 150,000 per \dword{detmodule}.
Quantities should update on average no more than once per minute.
Transmitting a single update for each channel at that rate translates to a few thousand updates per second, or a few tens of thousands of bytes per second. While this is not a significant load on a network with an efficient slow controls protocol, it would correspond to approximately 1~TB per year per \dword{detmodule} if every timestamp and value were stored.
The actual data volume will be lower %less 
because values are stored only if they vary from previous values by more than an amount that is adjustable channel-by-channel.
Database storage also allows data to be sparsified later.
No slow controls data is planned to be written to the \dword{daq} stream.
With careful management of archiving thresholds for each quantity monitored and yearly reduction of stored sample time density, the retained data volume can be reduced to a few TB over the life of the experiment.

The subsystems
to be monitored include the  
cryogenics instrumentation
described in this chapter, the other detector systems, and relevant
infrastructure and external devices. Table~\ref{tab:gen-slow-quant}
lists the quantities expected from each system.

\begin{dunetable}
[Slow controls quantities]
{p{0.3\textwidth}p{0.6\textwidth}}
{tab:gen-slow-quant}
{Slow controls quantities}
System & Quantities \\ \toprowrule
\multicolumn{2}{l}{\bf Detector cryogenics instrumentation } \\ \specialrule{1.5pt}{1pt}{1pt}
Purity monitors & anode and cathode charge, bias voltage and current, flash lamp status, calculated electron lifetime \\ \colhline
Thermometers & temperature, position of dynamic thermometers \\ \colhline
Liquid level & liquid level \\ \colhline
Gas analyzers & purity level readings \\ \colhline
Pressure meters & pressure readings \\ \colhline
Cameras & camera voltage and current draw, temperature, heater current and voltage, lighting current and voltage \\ \toprowrule
\multicolumn{2}{l}{\bf Other detector systems } \\ \specialrule{1.5pt}{1pt}{1pt}
Cryogenic internal piping & \fdth gas purge flow and temperature \\ \colhline
\dword{hv} systems & drift \dword{hv} voltage and current, end-of-field cage current and bias voltage, electron diverter bias, ground plane currents \\ \colhline
\dword{tpc} electronics & voltage and current to electronics \\ \colhline
\dword{pd} & voltage and current for photodetectors and electronics \\ \colhline
\dword{daq} & warm electronics currents and voltages, run status, \dword{daq} buffer sizes, trigger rates, data rates, GPS status, computer and disk health status, other health metrics as defined by \dword{daq} group \\ \colhline
\dword{crp} / \dword{apa} & bias voltages and currents \\ \toprowrule
\multicolumn{2}{l}{\bf Infrastructure and external systems } \\ \specialrule{1.5pt}{1pt}{1pt}
Cryogenics (external) & status of pumps, flow rates, inlet and return temperature and pressure (via OPC or similar SCADA interface) \\ \colhline
Beam status & protons on target, rate, target steering, beam pulse timing (via \dword{ifbeam}) \\ \colhline
Near detector & near detector run status (through common slow controls database) \\ \colhline
Rack power and status & power distribution unit current and voltage, air temperature, fan status if applicable, interlock status \\ \colhline
\multicolumn{2}{l}{\bf Detector calibration systems } \\ \specialrule{1.5pt}{1pt}{1pt}
Laser & laser power, temperature, operation modes, other system status as defined by calibration group\\ \colhline
External neutron source  & safety interlock status, power supply monitoring, other system status as defined by calibration group \\ \colhline
External radioactive source & system status as defined by calibration group\\
\end{dunetable}

%%%%%%%%%%%%%%%%%%%%%%%%%%%%%%%%%%
% Sowjanya and Anselmo
\subsection{Local Integration}
\label{sec:fdgen-slow-cryo-slow-loc-integ}

\fixme{check my edits here. anne}
The local integration of the slow controls consists entirely of software
and network interfaces with systems that are outside of the scope of the \dword{detmodule}. 
This includes the following:
\begin{itemize}
\item readings from the \dword{lbnf}-managed external cryogenics systems, for status of pumps, flow rates, inlet, and return temperature and pressure, which are implemented via \dword{opc-ua} or a similar \dword{scada} interface; 
\item beam status, such as protons-on-target, rate, target steering, and beam pulse timing, which are retrieved via \dword{ifbeam}; 
and \item near detector status, which can be retrieved from a common slow controls database.
\end{itemize}
%
Integration occurs after both the slow controls and non-detector
systems are in place.  The \dword{lbnf}-\dword{cisc} interface is managed by the
cryogenics systems working group in \dword{cisc} (see Section~\ref{sec:cisc-slow-controls-org}), which includes members from both \dword{cisc} and \dword{lbnf}. 
The \dword{ifbeam} DB interface for slow controls is an already well established method used in \dword{microboone}, \dword{nova}, and other \dword{fnal} experiments. An internal \dword{nd}/\dword{fd} working group can be established 
to coordinate detector status exchange between the near and far sites.

%%%%%%%%%%%%%
\subsection{Validation in ProtoDUNE}
\label{sec:cisc-slow-control-pdsp}

The \dword{pdsp} detector control system has met
all requirements for operation of \dword{pdsp}~\cite{pdspdcs_proc} and will be used for \dword{pddp}. The requirements for \dword{protodune} are
nearly identical to those for the \dword{spmod} other than
total channel count. Of particular note, the \dword{protodune} slow control system unified a heterogenous set of devices and data sources
through several protocols into a
single control system, as illustrated in
Figure~\ref{fig:cisc-NP04-DCS-topology}. In addition to what
the figure shows, data were also acquired from external cryogenics and beam
systems.  The topology and data flow of the system matches the general
shape shown in Figure~\ref{fig:gen-slow-controls-diagram}.

\begin{dunefigure}[Diagram of the ProtoDUNE-SP control system topology]{fig:cisc-NP04-DCS-topology}
{Diagram of the \dword{pdsp} control system topology, from \cite{pdspdcs_proc}.}
\includegraphics[height=0.9\textheight,width=0.95\textwidth,keepaspectratio]{NP04-DCS_arch_vert}
\end{dunefigure}

In this control system, the unified control system base is WinCC-OA~\cite{winccoa}, a commercial \dword{scada} system for visualizing and operating of processes, production flows, machines, and plants, used
in many businesses. It was chosen at \dword{cern} as a basis for
developing the control systems of the \dword{lhc} experiments, the
accelerators and the laboratory infrastructure for its flexibility and
scalability, as well as for the openness of the architecture, allowing
it to interface with many different types of hardware devices and
communication protocols. Additional software developed at \dword{cern}
is also used, including Joint COntrols Projects~\cite{jcop} and UNified
Industrial COntrol System (UNICOS)~\cite{unicos}. WinCC-OA and the
additional software developed on top of it in the past 20 years, have
grown into a fairly complex ecosystem. While multiple collaboration
members have experience using the \dword{pdsp} control system,
customizing and using WinCC-OA in an effective way for developing the
control system of \dword{dune} requires proper training and a
non-negligible learning effort.

As noted in Sections~\ref{sec:fdgen-slow-cryo-sw} and~\ref{sec:fdgen-slow-cryo-quant},
the slow control archiver will gradually accumulate terabytes of
data, requiring a sizable database to store the value history and
allow efficient data retrieval. Individually adjustable rates and
thresholds for each channel are key to keeping this database
manageable. The \dword{pdsp} operations provided not only a test of
these features as implemented in the \dword{protodune} slow control system, but also insight into
reasonable values for these archiving parameters for each system.


%%%%%%%%
\section{Organization and Management}
\label{sec:cisc-slow-controls-org}

The organization of the \dword{cisc} consortium is shown in
 Figure~\ref{fig:gen-slow-cryo-org}. The \dword{cisc} consortium board currently comprises institutional representatives from 19 institutes as shown in Table~\ref{tab:gen-slow-cryo-org}. The consortium leader is the spokesperson for the consortium and responsible for the overall scientific program and managing the group. The technical leader of the consortium is responsible for managing the project for the group. Currently, the
consortium has five working groups:
\begin{description}
 \item[Cryogenics Systems] gas analyzers and liquid level
  monitors; \dword{cfd} simulations;
  
 \item[Argon Instrumentation] purity monitors, thermometers, pressure meters, capacitive level meters, cameras and light emitting system, and \dword{citf}, also feedthroughs, \efield simulations, instrumentation precision studies, \dword{protodune} data analysis coordination and validation; 
 
 \item [Slow Controls Base Software and Databases]  base I/O software, alarms and archiving databases, and monitoring tools, also 
   variable naming conventions, and slow controls quantities;
 \item [Slow Controls Detector System Interfaces] signal processing software and hardware interfaces (e.g., power supplies), firmware, rack hardware and infrastructure   
 \item [Slow Controls External Interfaces] interfaces with external detector systems (e.g., cryogenics system, beam, facilities, \dword{daq}, and near detector status).
\end{description}

\begin{dunefigure}[CISC consortium organization]{fig:gen-slow-cryo-org}
{CISC Consortium organizational chart}
\includegraphics[width=0.8\textwidth]{graphics/cisc_org_20190716_zoomedin.png}
\end{dunefigure}

\begin{dunetable}
[CISC consortium institutions]
%{p{0.43\textwidth}p{0.12\textwidth}p{0.22\textwidth}}
{lc}
{tab:gen-slow-cryo-org}
{Current \dword{cisc} consortium board members and their institutional affiliations}
Member Institute                         &  Country       \\%  &  Consortium Board Representative \\ \toprowrule
CIEMAT                                   &  Spain         %  &  Ines Gil Botella 
\\ \colhline
Instituto de Fisica Corpuscular (IFIC)          &  Spain          % &  Anselmo Cervera 
\\ \colhline
University of Warwick                    &  UK % &  Gary Barker 
\\ \colhline
University College London (UCL)             &  UK  %&  Mario Campanelli 
\\ \colhline
Argonne National Lab (ANL)                     &  USA             %&  Jim Grudzinski 
\\ \colhline
Brookhaven National Lab (BNL)                  &  USA            % &  Jim Stewart
\\ \colhline
University of California, Irvine (UCI)        &  USA            % &  Jianming Bian 
\\ \colhline
Drexel University                        &  USA           %  &  Charles Lane 
\\ \colhline
Fermi National Accelerator Lab (\fnal)           &  USA           %  &  Alan Hahn
\\ \colhline
University of Hawaii                     &  USA            % &  Jelena Maricic 
\\ \colhline
University of Houston                    &  USA           %  &  Andrew Renshaw 
\\ \colhline
Idaho State University (ISU)                   &  USA           %  &  Ed Tatar 
\\ \colhline
Kansas State University (KSU)                  &  USA            % &  Glenn Horton-Smith 
\\ \colhline
University of Minnesota, Duluth (UMD)         &  USA            % &  Alec Habig 
\\ \colhline
Notre Dame University                    &  USA            % &  John LoSecco 
\\ \colhline
South Dakota State University (SDSU)           &  USA            % &  Stephen Gent
\\ \colhline
University of Tennessee at Knoxville (UTK)     &  USA           %  &  Sowjanya Gollapinni 
\\ \colhline
Virginia Tech (VT)                            &	USA	           % &  Camillo Mariani
\\	\colhline
	Boston University (BU)			      & USA % & Edward Kearns
\\
\end{dunetable}

Moreover, because the \dword{cisc} consortium broadly interacts with other groups, liaisons have been named as shown in Figure~\ref{fig:gen-slow-cryo-org}. 
A short-term task force was recently formed to explore the need for cryogenics modeling for the consortium. A work plan for \dword{cfd} simulations for both \dword{protodune} and \dword{fd} was developed based on input from the task force. 


\subsection{Institutional Responsibilities}

The \dword{cisc} consortium  
will be a joint  effort for \dword{sp} and \dword{dp}. %\single and \dual.  
A single slow controls system will be implemented to serve both the \dword{spmod} and the \dword{dpmod}.

Design and installation of cryogenics systems (e.g., gas analyzers, liquid level monitoring) 
will be coordinated with \dword{lbnf}, with the consortium providing resources and effort, and expertise provided by \dword{lbnf}. \dword{protodune} designs for \dword{lar} instrumentation (e.g., purity monitors, thermometers, cameras) will be the basis for \dword{detmodule} designs. Design validation, testing, calibration, and performance will be evaluated through \dword{protodune} data.

Following the conceptual funding model envisioned for the consortium, various responsibilities have been distributed across institutions within the consortium pending final funding decisions.
Table~\ref{tab:cisc-inst-resp} shows the current institutional responsibilities for primary \dword{cisc} subsystems. Only lead institutes are listed in the table for a given effort. For physics and simulations studies and for validation using \dword{protodune}, a number of institutes are involved. A detailed list of tasks and institutional responsibilities are presented in~\cite{bib:docdb5609}.

\begin{dunetable}
[Institutional responsibilities in the CISC consortium ]
{p{0.4\textwidth}p{0.45\textwidth}}
{tab:cisc-inst-resp}
{Institutional responsibilities in the \dword{cisc} consortium}
CISC Sub-system     &  Institutional Responsibility \\ \toprowrule
Purity Monitors          &  UCI, Houston \\ \colhline
Static T-gradient monitors     &  IFIC \\ \colhline
Dynamic T-gradient monitors & Hawaii \\ \colhline
Individual Sensors & IFIC \\ \colhline
Readout System for Thermometers & IFIC, Hawaii, CIEMAT \\ \colhline
Pressure Meters & UTK \\ \colhline
Cold Cameras & KSU, BNL \\ \colhline
Warm Cameras & KSU, BNL \\ \colhline
Light-emitting System (for cameras) & Drexel \\ \colhline
Gas Analyzers & FNAL, \dshort{lbnf} \\ \colhline
Differential Pressure Level Meters & \dword{lbnf} \\ \colhline
Capacitive Level Meters & Notre Dame \\ \colhline
\dshort{citf} & FNAL, ANL \\ \colhline
\dshort{cfd} Simulations & SDSU, ANL \\ \colhline
Other Simulation \& Validation Studies & Number of Institutes \\ \colhline
Slow Controls Hardware & UMD, UTK, Drexel\\ \colhline
Slow Controls Infrastructure & UMD, UTK\\ \colhline
Slow Controls Base Software & KSU, UTK, BU, Drexel, Warwick, ANL, IFIC\\ \colhline 
Slow Controls Signal Processing & A number of institutes \\ \colhline
Slow Controls External Interfaces & VT, UTK, UMD \\
\end{dunetable}



\subsection{Schedule}
\label{sec:fdgen-cisc-schedule}


Table \ref{tab:sp-cisc-sched} shows key construction milestones for the \dword{cisc} consortium leading to commissioning of the first \dword{fd} module. \dword{cisc} construction milestones align with the overall construction milestones of the first \dword{fd} module (highlighted in orange in the table). The technology design decisions for \dword{cisc} systems should be made by April 2020 followed by final design reviews in June 2020. Design decisions will largely be based on how a given system performed (technically and physics-wise) in \dword{protodune}. This is currently actively ongoing with the \dword{pdsp} instrumentation data.  
As noted in Section~\ref{sec:pdsp-cryo-valid}, the current plan is to deploy improved designs of static and dynamic T-gradient thermometers, purity monitors, long (\dual-style) level meters and cameras to be validated in \dword{protodune2}. The production of systems aimed for \dword{protodune2} \single should be finished by January 2021 followed by assembly and deployment in March 2021. 

\begin{dunetable}
[SP CISC schedule and milestones]
{p{0.65\textwidth}p{0.25\textwidth}}
{tab:sp-cisc-sched}
{\dword{cisc} construction schedule milestones leading to commissioning of the first FD module. Key DUNE dates and milestones, defined for planning purposes in this TDR, are shown in orange.  Dates will be finalized following establishment of the international project baseline.}   
Milestone & Date (Month YYYY)   \\ \toprowrule
Technology Decision Dates &   April 2020   \\ \colhline
Final Design Review Dates &   June 2020   \\ \colhline
Start of module 0 component production for \dword{protodune2} & August 2020  \\ \colhline
End of module 0 component production for \dword{protodune2} & January 2021  \\ \colhline
\rowcolor{dunepeach} Start of \dword{pdsp}-II installation& \startpduneiispinstall      \\ \colhline
\rowcolor{dunepeach} Start of \dword{pddp}-II installation& \startpduneiidpinstall      \\ \colhline
\rowcolor{dunepeach}South Dakota Logistics Warehouse available& \sdlwavailable      \\ \colhline
 \dword{prr} dates &  September 2022    \\ \colhline
\rowcolor{dunepeach}Beneficial occupancy of cavern 1 and \dword{cuc}& \cucbenocc      \\ \colhline
Start procurement of \dword{cisc} hardware & December 2022 \\ \colhline
\rowcolor{dunepeach} \dword{cuc} counting room accessible& \accesscuccountrm      \\ \colhline
Start of production of \dword{cisc} hardware & April 2023 \\ \colhline
\rowcolor{dunepeach}Top of \dword{detmodule} \#1 cryostat accessible& \accesstopfirstcryo      \\ \colhline
End of \dword{cisc} hardware production  &   April 2024   \\ \colhline

Start integration of \dword{cisc} hardware in the cavern & July 2024   \\ \colhline
\rowcolor{dunepeach}Start of \dword{detmodule} \#1 \dword{tpc} installation& \startfirsttpcinstall      \\ \colhline
Installation of gas analyzers and support structure for all instrumentation devices &  September 2024 \\ \colhline
Installation of individual sensors, static T-gradient thermometers, and level meters & November 2024\\ \colhline
\rowcolor{dunepeach}Top of \dword{detmodule} \#2 cryostat accessible& \accesstopsecondcryo      \\ \colhline
All slow controls hardware, infrastructure, \& networking installed & February 2025\\ \colhline
Slow controls software for I/O, alarms, archiving, displays installed on production systems & May 2025 \\ \colhline
\rowcolor{dunepeach}End of \dword{detmodule} \#1 \dword{tpc} installation& \firsttpcinstallend      \\ \colhline
Install dynamic T-gradient monitors, cameras, purity monitors, and pressure meters & June 2025 \\\colhline
Install all feedthroughs for instrumentation devices & July 2025 \\ \colhline
 \rowcolor{dunepeach}Start of \dword{detmodule} \#2 \dword{tpc} installation& \startsecondtpcinstall      \\ \colhline
Install slow control expert interfaces for all systems in time for testing & September 2025 \\ \colhline
\rowcolor{dunepeach}End of \dword{detmodule} \#2 \dword{tpc} installation& \secondtpcinstallend      \\ 
Full slow controls systems commissioned and integrated into remote operations & July 2026 \\ 
\end{dunetable}

Designs may need review based on performance in \dword{protodune2} and any modifications will be incorporated into the final design before the start of production of \dword{cisc} systems for the \dword{fd} 
\fixme{\dword{spmod}?} in April 2023. This will be followed by assembly of the systems underground in the detector cavern in July 2024. Installation of instrumentation devices will start in September 2024 following the beneficial occupancy of the interior of the cryostat. Installing gas analyzers, level meters, individual temperature sensors, static T-gradient thermometers, and support structure for all instrumentation devices will be finished before installing \dword{tpc}, but installation of dynamic T-gradient thermometers, purity monitors, pressure meters and cameras will occur afterward. \dword{cisc} will work closely with \dword{lbnf} to coordinate installation of the cryogenics systems and instrumentation devices. For slow controls, the goal is to have the full slow controls system commissioned and integrated into remote operations at least three months before the \dword{spmod} is ready for operations.  


 

\subsection{Risks}
%\fixme{SG: Table reformatted and updated. New text added. This is now ready for review.}
%\fixme{Risk tables to be automated; Anne to send template via email. 3/25}

Table~\ref{tab:risks:SP-FD-CISC}  %{tab:fdgen-slow-cryo-risk1}, \ref{tab:fdgen-slow-cryo-risk2}, and \ref{tab:fdgen-slow-cryo-risk3} 
lists the possible risks identified by the \dword{cisc} consortium along with corresponding mitigation strategies. %The table lists 18 risks at low and medium level. f
A more detailed list of risks with additional descriptions can be found in \cite{bib:docdb7192}. The table shows 18 risks, all at medium or low level, mitigated with necessary steps and precautions. More discussion on all medium-level risks are provided in the text below. 
\begin{itemize}
    \item Risk 01: The risk associated with \dword{pdsp}-based designs being inadequate for \dword{fd}, is important because this requires early validation from \dword{protodune} data so R\&D of alternate designs can be timely. With \dword{pdsp} data now available, the consortium is focused on validating instrumentation designs. 
    \item Risk 06: Temperature sensors in the dynamic T-gradient monitor are calibrated using two methods: lab calibration to \SI{0.002}{K}  (as in the static T-gradient monitor case) and in situ cross-calibration moving the system vertically. Disagreement between the two methods can occur. In order to mitigate this we need to investigate and improve both methods, specifically the laboratory calibration since this is the only one possible for sensors behind \dword{apa}s, and top/bottom of the detector.  
    \item Risk 10: This risk involves an inability to build a working prototype for cold cameras during R\&D phase that meets all the requirements \& safety, e.g., that cold camera prototypes fail longevity tests or show low performance (e.g. bad resolution). This risk originates from past experience with cold cameras that became non-operational after a period of time in \dword{lar} or showed low performance. In order to address this, we plan to pursue further R\&D to improve thermal insulation and heaters, develop alternative camera models, etc. If problems persist one can use the cameras in the ullage (cold or inspection) with the appropriate field of view and lighting such that elements inside \dword{lar} can be inspected during filling.
    \item Risk 12: Cameras are delicate devices and some of them located near \dword{hv} devices can be destroyed by \dword{hv} discharges. This can be mitigated by ensuring that most important cold cameras have enough redundancy such that the loss of one camera does not compromise the overall performance. In the case of inspection cameras, % since they are replaceable, one 
    we can simply replace them.
    \item Risk 17: The gas analyzers and level meters may fail as these are commercial devices purchased at some point in their product cycle and cannot be required to last 20 years. Typical warranties are $\sim$1 year from date of purchase. The active electronics parts of both gas analyzers and level meters are external to the cryostat so they can be replaced. To mitigate this, provisions will be made for future replacement in case of failure or loss of sensitivity. Also, the risk is not high since we have purity monitors in the filtration system that can cover the experiment during the time gas analyzers are being replaced or repaired. 
\end{itemize}

Related to risks 12, 16 and 18, ageing is an important aspect for several monitors, especially for those that are inaccessible. The \dword{protodune} tests demonstrate that the devices survive the commissioning phase, and we continue to learn from \dword{protodune} experience. In addition to \dword{protodune}, other tests are planned. For example, in the case of purity monitors, photocathodes are expected to survive the first five years and if we prevent running them with high frequency at low purity (lifetime~$<$~\SI{3}{\milli\second}), ageing can be prevented for a longer time. To understand long-term aeging, R\&D is planned at \dword{citf} and at member institute sites for many of the devices. %Other s
Systems that are replaceable, such as inline purity monitors, gas analyzers, and inspection cameras, can be replaced when failures occur and maintained for the lifetime of the experiment.
 

% risk table values for subsystem SP-FD-CISC
\begin{footnotesize}
%\begin{longtable}{p{0.18\textwidth}p{0.20\textwidth}p{0.32\textwidth}p{0.02\textwidth}p{0.02\textwidth}p{0.02\textwidth}}
\begin{longtable}{P{0.18\textwidth}P{0.20\textwidth}P{0.32\textwidth}P{0.02\textwidth}P{0.02\textwidth}P{0.02\textwidth}} 
\caption[Risks for SP-FD-CISC]{Risks for SP-FD-CISC (P=probability, C=cost, S=schedule) More information at \dshort{riskprob}. \fixmehl{ref \texttt{tab:risks:SP-FD-CISC}}} \\
\rowcolor{dunesky}
ID & Risk & Mitigation & P & C & S  \\  \colhline
RT-SP-CISC-01 & Baseline design from ProtoDUNEs for an instrumentation device is not adequate for DUNE far detectors & Focus on early problem discovery in ProtoDUNE so any needed redesigns can start as soon as possible. & L & M & L \\  \colhline
RT-SP-CISC-02 & Swinging of long instrumentation devices (T-gradient monitors or PrM system) & Add additional intermediate constraints to prevent swinging. & L & L & L \\  \colhline
RT-SP-CISC-03 & High E-fields near instrumentation devices cause dielectric breakdowns in \dshort{lar} & CISC systems placed as far from cathode and FC as possible. & L & L & L \\  \colhline
RT-SP-CISC-04 & Light pollution from purity monitors and camera light emitting system & Use PrM lamp and camera lights outside PDS trigger window; cover PrM cathode to reduce light leakage. & L & L & L \\  \colhline
RT-SP-CISC-05 & Temperature sensors can induce noise in cold electronics & Check for noise before filling and remediate, repeat after filling. Filter or ground noisy sensors. & L & L  & L \\  \colhline
RT-SP-CISC-06 & Disagreement between lab and \em{in situ} calibrations for ProtoDUNE-SP dynamic T-gradient monitor & Investigate and improve both methods, particularly laboratory calibration. & M & L & L \\  \colhline
RT-SP-CISC-07 & Purity monitor electronics induce noise in TPC and PDS electronics. & Operate lamp outside TPC+PDS trigger window. Surround and ground light source with Faraday cage. & L & L & L \\  \colhline
RT-SP-CISC-08 & Discrepancies between measured temperature map and CFD simulations in ProtoDUNE-SP & Improve simulations with additional measurements inputs; use fraction of sensors to predict others   & L & L & L \\  \colhline
RT-SP-CISC-09 & Difficulty correlating purity and temperature in ProtoDUNE-SP impairs understanding cryo system. & Identify causes of discrepancy, modify design. Calibrate PrM differences, correlate with RTDs. & L & L & L \\  \colhline
RT-SP-CISC-10 & Cold camera R\&D fails to produce prototype meeting specifications \& safety requirements & Improve insulation and heaters. Use cameras in ullage or inspection cameras instead. & M & M & L \\  \colhline
RT-SP-CISC-11 & HV discharge caused by inspection cameras & Study E-field in and on housing and anchoring system. Test in HV facility. & L & L & L \\  \colhline
RT-SP-CISC-12 & HV discharge destroying the cameras & Ensure sufficient redundancy of cold cameras. Warm cameras are replaceable. & L & M & L \\  \colhline
RT-SP-CISC-13 & Insufficient light for cameras to acquire useful images & Test cameras with illumination similar to actual detector. & L & L & L \\  \colhline
RT-SP-CISC-14 & Cameras may induce noise in cold electronics & Continued R\&D work with grounding and shielding in realistic conditions. & L & L & L \\  \colhline
RT-SP-CISC-15 & Light attenuation in long optic fibers for purity monitors  & Test the max.\ length of usable fiber, optimize the depth of bottom PrM, number of fibers. & L & L & L \\  \colhline
RT-SP-CISC-16 & Longevity of purity monitors & Optimize PrM operation to avoid long running in low purity. Technique to protect/recover cathode. & L & L & L \\  \colhline
RT-SP-CISC-17 & Longevity: Gas analyzers and level meters may fail. & Plan for future replacement in case of failure or loss of sensitivity.  & M & M & L \\  \colhline
RT-SP-CISC-18 & Problems in interfacing  hardware devices (e.g. power supplies) with slow controls & Involve slow control experts in choice of hardware needing control/monitoring.
 & L & L & L \\  \colhline

\label{tab:risks:SP-FD-CISC}
\end{longtable}
\end{footnotesize}





\subsection{Interfaces} 
\label{sec:interfaces}

\dword{cisc} subsystems interface with all other detector subsystems and potentially impact the work of all detector consortia, as well as some
working groups (e.g., physics, software and computing, beam instrumentation), and technical coordination, requiring interactions with all of these entities.  We also interact heavily with \dword{lbnf} beam and cryogenics groups.  
Detailed descriptions of \dword{cisc} interfaces are maintained in the \dword{dune} DocDB. A brief summary is provided in this section. Table~\ref{tab:fdgen-cisc-interfaces} lists the IDs of the different DocDB documents as well as their highlights. Descriptions of the interfaces and interactions that affect many systems are given below. 

\dword{cisc} interacts with the detector consortia because \dword{cisc} will provide status monitoring of all important detector subsystems along with controls for some components of the detector.
%full rack monitoring (rack fans, thermometers and rack protection system), interlock status bit monitoring (not the actual interlock mechanism) and monitoring and control for all power supplies (PS).
\dword{cisc} will also consult on selecting different power supplies to ensure monitoring and control can be established with preferred types of communication. 
Rack space distribution and interaction between slow controls and other modules from other consortia will be managed by \dword{tc} in consultation with those consortia. 

\dword{cisc} will work with \dword{lbnf} to determine whether heaters and \dwords{rtd} are needed on flanges. If so, \dword{cisc} will specify the heaters and  \dwords{rtd}, and will provide the readout and control, while the responsibility for the actual hardware will be discussed with the different groups.

Installing instrumentation devices will interfere with other devices and must be coordinated with the appropriate consortia.  
On the software side, \dword{cisc} must define, in coordination with other consortia/groups, the quantities to be monitored/controlled by slow controls and the corresponding alarms,
archiving, and GUIs. 



%\fixme{reduced the text in the description column so that we don't have to move to longtable format. anne}
\begin{dunetable}
[CISC system interface links]
{p{0.15\textwidth}p{0.45\textwidth}p{0.14\textwidth}}
{tab:fdgen-cisc-interfaces}
{\dword{cisc} system interface links}   % These are called Interface Documents in Doc-DB, also in Work Breakdown Structure documents.
\small
Interfacing System & Description & Linked Reference \\ \toprowrule
%test & & \citedocdb{6628} \\ \colhline
\dword{apa}	           &
static T-gradient monitors, cameras, and lights
& \citedocdb{6679} 
\\ \colhline

\dword{pds}	     & 


PrMs, light emitting system for cameras
& \citedocdb{6730}  
\\ \colhline

\dword{tpc} Electronics	         &  
noise, power supply monitoring
& \citedocdb{6745}  \\ \colhline


\dword{hv} Systems	           &
shielding, bubble generation by inspection camera, cold camera locations, ground planes
& \citedocdb{6787}   
\\ \colhline

\dword{daq}	                      &
description of \dword{cisc} data storage, 
allowing bi-directional communications between \dword{daq} and \dword{cisc}.      & \citedocdb{6790}
\\ \colhline
Calibration          &
multifunctional \dword{cisc}/\dword{citf} ports; space sharing around ports 
& \citedocdb{7072} 

\\ \colhline
Physics	          &

indirect interfaces through calibration, tools to extract data from the slow controls database 
& \citedocdb{7099} 
\\ \colhline

Software \& Computing	  &


slow controls database design and maintenance
& \citedocdb{7126}  
\\ \colhline

Cryogenics             &  
must be designed and implemented.       
purity monitors, gas analyzers, interlock mechanisms to prevent contamination of LAr
&  -   

\\ \colhline

Beam                      &   %At least the status of this system will be monitored  
beam status &  -     
\\ \colhline
TC Facility              &   
significant interfaces at multiple levels   
& \citedocdb{6991}   \\ \colhline
TC Installation     	  &     
significant interfaces at multiple levels
& \citedocdb{7018}   \\ \colhline
TC Integration Facility    &    
significant interfaces at multiple levels
& \citedocdb{7045}   \\ 
\end{dunetable}







\subsection{Installation, Integration, and Commissioning}


\subsubsection{Purity Monitors}
\label{sec:fdgen-slow-cryo-install-pm}


The purity monitor system will be built in modules, so it can be assembled outside the %\dword{dune} \dword{fd} 
cryostat  
leaving few steps to complete inside the cryostat.  The assembly itself %should
 comes into the cryostat with the individual purity monitors mounted to support tubes, with no \dword{hv} cables or optical fibers yet installed.  The support tube at the top and bottom of the assembly %would then be 
 is then mounted to the brackets inside the cryostat, and  %; the brackets could be 
 the brackets attached to the cables trays and/or the detector support structure.  At much the same time, the \dword{fe} electronics and light source can be installed on the top of the cryostat, and the electronics and power supplies can be installed in the electronics rack.  

Integration %would 
begins by running the  \dword{hv} cables and optical fibers to the purity monitors, through the top of the cryostat.  These cables %would be 
are attached to the  \dword{hv}  feedthroughs with sufficient length to reach each purity monitor inside the cryostat.  
The cables %would be 
are run along cable trays through the port reserved for the purity monitor system. 
Each purity monitor will have three  \dword{hv} cables that connect it to the feedthrough and then further along to the  \dword{fe} electronics.  The optical fibers %would then be 
are then run through the special optical fiber feedthrough, into the cryostat, and %would be 
guided to the purity monitor system either using the cables trays or guide tubes, depending on which solution is adopted. 
%for running the optical fibers from the feedthrough to the purity monitor system, 
This should protect fibers from breaking accidentally as the rest of the detector and instrumentation installation continues.  The optical fibers %would then be 
are then run inside the purity monitor support tube and to the appropriate purity monitor, terminating the fibers at the photocathode of each monitor while protecting them from breaking near the purity monitor system itself.

Integration %would 
continues as the  \dword{hv} cables are connected through the feedthrough to the system  \dword{fe} electronics; then optical fibers are connected to the light source.  The cables connecting the \dword{fe} electronics and the light source to the electronics rack %would also be 
are also run and connected at this time.  This allows the system to be turned on and the software to begin testing the various components and connections.  Once all connections are confirmed successful, integration with the slow controls system begins, first by establishing communication between the two systems and then transferring data between them to ensure successful exchange of important system parameters and measurements.  

Commissioning the purity monitor system % is considered  official
begins once the cryostat is purged and a gaseous argon atmosphere is present.  At this time, the \dword{hv} for the purity monitors %could be ramped up without fear of discharge
  is ramped up without risk of discharge through the air, and the light source turned on.  Although the drift electron lifetime in the gaseous argon would be very large and therefore not %really 
  measurable with the purity monitors themselves, the signal strength at both the cathode and anode will %would 
  give a good indication of how well the light source generates drift electrons from the photocathode.  Comparing the signal strengths at the anode and cathode will indicate whether the electrons successfully drift to the anode.
  Although \dword{qa} and \dword{qc} should make it unlikely for a purity monitor to fail this final test, if that does happen then the electric and optical connections can be fixed before filling.

\subsubsection{Thermometers}
\label{sec:fdgen-slow-cryo-install-th}

Static T-gradient monitors %should 
must be installed before the outer \dword{apa}s, ideally %. Thus, the best moment would be 
right after the pipes are installed. The profilers %will be 
are preassembled before they are delivered to \surf. 
Installation will follow these steps:
\begin{enumerate}
\item anchor the stainless steel bottom plates to the four bolts on the bottom corner of the cryostat,
\item anchor the stainless steel support holding the two strings to the four bolts on the top corner of the cryostat,
\item unroll the array with the help of the scissor lift,
\item anchor the strings to the bottom stainless steel support,   
\item check and adjust tension and verticality,
\item review all cable and sensor supports, 
\item route cable from the top anchoring point to the two \dword{dss} ports, and 
\item plan to plug sensors into IDC-4 connectors later, just before moving the corresponding \dword{apa} into its final position. 
\end{enumerate}

Individual temperature sensors on pipes and cryostat floor %will be 
are installed immediately after installing the static T-gradient monitors. First, vertical stainless steel strings for cable routing %will be 
are installed following a procedure similar to the one described above for the static T-gradient monitors. Next, we anchor all cable supports to pipes. Then each cable %will be 
is routed individually starting from the sensor end (with IDC-4 female connector but without the sensor)
to the corresponding cryostat port. Once all cables going through the same port have been routed, we cut the cables to the same length, so they can be properly assembled into the corresponding connector(s). To avoid damaging the sensors, they are installed later (by plugging the IDC-4 male connector on the sensor PCB to the IDC-4 female connector on the cable end), just before unfolding the bottom \dwords{gp}.

For the \dword{sp}, individual sensors on the top \dword{gp} must be integrated with the \dwords{gp}. For each \dshort{cpa} (with its corresponding four  \dword{gp} modules)
going inside the cryostat, cable and sensor supports will be anchored to the  \dword{gp} threaded rods as soon as possible.
Once the \dshort{cpa} is moved into its final position and its top \dwords{gp} are ready to be unfolded, sensors on these \dwords{gp} %will be 
are installed. Once unfolded, cables 
exceeding the \dword{gp} limits can be routed to the corresponding cryostat port using either neighboring \dwords{gp} or \dshort{dss} I-beams. 



Dynamic T-gradient monitors %will be 
are installed
after the \dword{tpc} components are in place.
Figure~\ref{fig:fd-slow-cryo-dt-monitor-overview} shows the
design of the dynamic T-gradient monitor with its sensor carrier rod,
enclosure above the cryostat, and stepper motor and 
 Figure~\ref{fig:fd-slow-cryo-sensor-mount} shows detailed views of key
components.  Each monitor %will 
comes in several segments with sensors
and cabling already in place. Additional slack will be provided at
segment joints to make installation easier. Segments of the sensor
carrier rod with preattached sensors %will be 
are fed into the flange one
at a time. Each segment, as it is fed into the %detector
cryostat, is % will be
held at the top with a pin that prevents the segment from sliding all
the way in. %to the detector. 
The next segment %will be 
is connected at that
time to the previous segment. Then the pin %will be 
is removed, the first
segment %will be 
is pushed down, and the next segment top %will be 
is held
with the pin at the flange. This process %will be 
is repeated for each
segment %to be installed 
until the entire sensor carrier rod is in
place.  Next, the enclosure %will be 
is installed on top of the flange,
starting with the six-way cross at the bottom of the enclosure.  (See
 Figure~\ref{fig:fd-slow-cryo-sensor-mount}, right.)  Again, extra cable
slack at the top will be provided to ease connection to the D-sub
flange and to allow the entire system to move vertically.  The wires
%will be 
are connected to a D-sub connector on the \fdth on one side port
of the cross. Finally, a crane %will be used to 
positions the remainder
of the enclosure above the top of the cross.  This enclosure includes
the mechanism used to move the sensor rod, which %will be 
is preassembled
with the motor in place on the side of the enclosure, and the pinion
and gear used to move the sensor inside the enclosure.  The pinion
%will be 
gets connected to the top of the rod. The enclosure is then %will then be
connected to top part of the cross, which finishes the installation of
the dynamic T-gradient monitor.

Commissioning all thermometers will occur in several steps. In the first stage, only cables %will be 
are installed, so
the readout performance and the noise level inside the cryostat %will be
is tested with precision resistors. Once sensors are installed, the entire chain %will be 
is checked again at room temperature.
Spare cables, connectors and sensors are available for replacement at \dword{surf} if needed. 
The final commissioning phase %will be done 
takes place during and after cryostat filling.  


\subsubsection{Gas Analyzers}
\label{sec:fdgen-slow-cryo-install-ga}
%\fixme{IIC Gas Analyzers: To be reviewed}

The gas analyzers %should be 
are installed before the piston purge and gas recirculation phases of the cryostat commissioning. They %should be 
are installed near the tubing switchyard to minimize tubing run length and for convenience when switching the sampling points and gas analyzers. Because each is a standalone module, a single rack with shelves %should be 
is adequate to house the modules.

For integration, the gas analyzers typically have an analog output (\SIrange{4}{20}{mA} or \SIrange{0}{10}{V}), which maps to the input range of the analyzers. They also usually have several relays to indicate the scale they are currently running. These outputs can be connected to the slow controls for readout. However, using a digital readout is preferable because this gives a direct analyzer reading at any scale. Currently, the digital output connections are RS-232, RS-485, USB, and Ethernet. The preferred option is chosen at the time of purchase. 
The readout usually responds to a simple set of text query commands. Because of the natural time scales of the gas analyzers and lags in gas delivery times (which depend on the length of the tubing run), sampling every minute is adequate. Our current plan is to record both analog and digital signals to have both outputs available.  

The analyzers %should 
must be brought online and calibrated before beginning the gas phase of the cryostat commissioning.  Calibration varies by module because they are different, but calibration often requires using argon gas with zero contaminants, and argon gas with a known level of the contaminant to check the scale. Contaminants are usually removed with a local inline filter for the first gas sample. %The gas phase of the cryostat 
This gas phase usually begins with normal air, with the more sensitive analyzers valved off at the switchyard to prevent overloading their inputs (and potentially saturating their detectors). As the argon purge and gas recirculation progress, the various analyzers %will be 
are valved back in when the contaminant levels reach the upper limits of the analyzer ranges. 

\subsubsection{Liquid Level Monitoring}
\label{sec:fdgen-slow-cryo-install-llm}

Installing differential pressure level meters is the responsibility of \dword{lbnf}, but the capacitive level meters fall within \dword{cisc}'s scope. The exact number of capacitive level meters must still be decided. There will be at least four, located at the four corners of the cryostat. 
They will be attached to the M10 bolts in the cryostat corners after the detector is installed. Cables will be routed to the appropriate \dword{dss} port. If additional capacitive level meters are needed in the central part of the cryostat, those will be installed before the nearby \dword{apa}s. 

\subsubsection{Pressure Meters}
\label{sec:fdgen-slow-cryo-install-press}
%\fixme{SG: to be reviewed by experts; waiting for response}
Installing pressure meters is the responsibility of \dword{cisc}. A total of six sensors will be mechanically installed in two dedicated flanges (three sensors each) at opposite sides of the cryostat after the detector is installed. Cables will be routed through the same dedicated port assigned for these devices. The pressure signals (absolute and relative) are read and converted to
%electrical signals corresponding to 4...20~mA range.
standard 4--20 mA current loop signals.
A twisted pair shielded cable connects the sensors to the slow controls \dword{plc} controller using software to convert electrical signals to pressure values.

\subsubsection{Cameras and light emitting system}
\label{sec:fdgen-slow-cryo-install-c}
%\fixme{IIC Cameras: To be reviewed}

Installing fixed cameras is simple in principle, but involves a
considerable number of interfaces. The enclosure of each camera has
exterior threaded holes to facilitate mounting on the cryostat wall,
cryogenic internal piping, or \dword{dss}. Each
enclosure %will be 
is attached to a gas line to maintain appropriate
underpressure in the fill gas, %and so 
therefore an interface with cryogenic
internal piping will be necessary. Each camera has a cable (coaxial or
optical) for the video signal and a multiconductor cable for power and
control. These %will be 
get run through cable trays to flanges on assigned
instrumentation feedthroughs.

The inspection camera is designed to be inserted and removed on any
instrumentation feedthrough equipped with a gate valve at any time
during operation.  Installing the gate valves and purge system
for instrumentation feedthroughs falls under cryogenic internal
piping.

Installing fixed lighting sources separate from the cameras
requires mounting on cryostat wall, cryogenic internal piping, or
\dword{dss}, and multiconductor cables for power run through cable
trays to flanges on assigned instrumentation feedthroughs.



\subsubsection{Slow Controls Hardware}
\label{sec:fdgen-slow-cryo-install-sc-hard}
%\fixme{IIC Slow Controls Hardware: To be reviewed}

Slow controls hardware installation %will 
includes installing multiple
servers, network cables, any specialized cables needed
for device communication, and possibly some custom-built hardware to monitor racks. The installation sequence will be 
planned with the facilities group and other consortia. The network
cables and rack monitoring hardware will be common across many racks
and will be installed first as part of the basic rack installation, %which will 
to be led by the facilities group. Specialized cables needed for slow controls and servers %will be 
are installed
after the common rack hardware. The selection and installation of these cables will be coordinated
with other consortia, and servers will be coordinated with the \dword{daq} group.

\subsubsection{Transport, handling, and storage}
\label{sec:fdgen-slow-cryo-install-transport}


Most instrumentation devices will be shipped in pieces to \surf via the \dword{sdwf} and mounted on-site. 
Instrumentation devices are in general small, except for  %other than 
the support structures for purity monitors and dynamic T-gradient monitors,
which will cover the entire height of the cryostat. The load on those structures is relatively small (\(<\SI{100}{kg}\)), so they can be fabricated in sections of less than \SI{3}{m},
which can be easily transported to \surf, down the shaft, and through the tunnels.
All instrumention devices except the dynamic T-gradient monitors can be moved into the cryostat without the crane. The dynamic T-gradient monitors, which %will be 
are introduced into the cryostat from above, %will 
require a crane for the installation of the external enclosure (with preassembled motor, pinion and gear). % on top of the cryostat. 


\subsection{Quality Control}
\label{sec:fdsp-slow-cryo-qc}
%\fixme{SG (4/15): we are missing pressure meters under QC?}
The manufacturer and the institution in charge of device assembly will conduct a series of tests to ensure the equipment can perform its intended function as part of \dword{qc}. \dword{qc} also includes post-fabrication tests and tests run after shipping and installation. For complex systems, the entire system will be tested before shipping. 
Additional \dword{qc} procedures can be performed %at the \dword{itf} and 
underground after installation. % as much as possible. 

The planned tests for each subsystem are described below.  

%\fixme{QC: To be reviewed}

\subsubsection{Purity Monitors}
\label{sec:fdgen-slow-cryo-qc-pm}


The purity monitor system will undergo a series of tests to ensure the
system performs as intended. These tests 
%reflect the \dword{pdsp} purity monitor QC tests, which included
include electronic tests
with a pulse generator, mechanical and electrical connectivity tests
at cryogenic temperatures in a cryostat, and vacuum tests for short
and full assemblies in a dewar and in a long vacuum tube.

The \dword{qc} tests for \dword{fd} purity monitors begin with testing
individual purity monitors in vacuum after each is fabricated and
assembled.  This test checks the amplitude of the signal generated by
the drift electrons at the cathode and the anode to ensure the
photocathode can provide sufficient numbers of photoelectrons to
measure the signal attenuation
with the required precision, and that the field gradient resistors all work properly to maintain the drift field. %and hence transport the drift electrons to the anode.  
A smaller version of the assembly with all purity monitors installed will be  tested at the \dword{citf} %in the \lar test facility
to ensure the full system performs as expected in \dword{lar}.  

Next, %after individual testing would be 
the entire system %will be 
is assembled on the full-length mounting tubes to check the connections along the way.  Ensuring that all electric and optical connections are operating properly during this test reduces the risk of problems once the full system is assembled and ready for the final test in vacuum.  %With the full system assembled it would be 
The fully assembled system %can be 
is placed in the shipping tube, which %can 
serves as a vacuum chamber, and tested at \surf %on the \dword{fd} site
 before the system is inserted into the  %\dword{fd}  
 cryostat. During insertion, electrical connections %will be 
 are tested continuously with multimeters and electrometers.



\subsubsection{Thermometers}
\label{sec:fdgen-slow-cryo-qc-th}

\paragraph{Static T-gradient thermometers}
\label{sec:fdgen-slow-cryo-qc-thst}

Static T-gradient monitors undergo three type of tests at the production site before %they are shipped 
shipment to \surf: a mechanical rigidity test, a calibration of all sensors, and a test of all electrical cables and connectors.
The mechanical rigidity is tested by mounting the static T-gradient monitor between two dummy cryostat corners mounted \SI{15}{m} apart. The tension of the strings is set to match the tension that would occur in a vertical deployment in \dword{lar}, and the deflection of the sensor and electrical cable strings is measured and compared to the expected value; this is to ensure any swinging or deflection of the deployed static T-gradient monitor will be < \SI{5}{cm}, mitigating any risk of touching the \dwords{apa}.
The laboratory calibration of sensors will be performed 
as explained in Section~\ref{sec:fdsp-cryo-therm}. The main concern is reproducibility of results because sensors could change resistance and hence their temperature scale when undergoing successive immersions in \dword{lar}. In this case, the calibration procedure itself provides \dword{qc} because each set of sensors goes through five independent measurements. Sensors with \rms variation outside the requirement (\SI{2}{mK} for \dword{pdsp}) are discarded. This calibration also serves as \dword{qc} for the readout system (similar to the final one) and of the PCB-sensor-connector assembly.
Finally, the cable-connector assemblies are tested; sensors must measure the expected values with no additional noise introduced by either cable or connector. 

An integrated system test is conducted at a \dword{lar} test facility at the production site, which has sufficient linear dimension (>\SI{2}{m}) to test a good portion of the system. This %, which 
ensures that  the system
operates in \dword{lar} at the required level of performance.
The laboratory sensor calibration %will be 
is compared with the in situ calibration
of the dynamic T-gradient monitors by operating both dynamic and static T-gradient monitors simultaneously.   

The last phase of \dword{qc} takes place after installation. 
The verticality of each array %will be 
is checked, and the tensions in the stainless steel strings adjusted as necessary.
Before closing the flange, the entire readout chain is %will be 
tested.  
This allows a test of the sensor-connector assembly, the cable-connector assemblies at both ends, and the noise level inside the cryostat.
If any sensor presents a problem, it is replaced. If the problem persists, the cable is checked and replaced as needed.

\paragraph{Dynamic T-gradient thermometers}
\label{sec:fdgen-slow-cryo-qc-thdy}

The dynamic T-gradient monitor consists of an array of high-precision temperature sensors mounted on a vertical rod. The rod can move vertically to cross-calibrate the temperature sensors in situ. %{\em in situ}. 
We will use the following %several 
tests to ensure that the dynamic T-gradient monitor delivers vertical temperature gradient measurements with a precision of a few \si{mK}.

\begin{itemize}
\item
Before installation, temperature sensors are tested in  \dword{ln}  to verify correct operation and to set the baseline calibration for each sensor with respect to the absolutely calibrated reference sensor. 
\item
Warm and cold temperature readings are taken with each sensor after it is mounted on the PCB board and the readout cables are soldered %of .
\item
The sensor readout is taken for all sensors after the cold cables are connected to electric \fdth{}s on the flange and the warm cables outside of the cryostat are connected to the temperature readout system.
\item 
The stepper motor is tested before and after connecting to the gear and pinion system.
\item
The fully assembled rod is connected to the pinion and gear and moved with the stepper motor on a high platform many times to verify repeatability, possible offsets, and any uncertainty in the positioning. Finally, repeating this test so many times will verify the sturdiness of the system.
\item
The full system is tested after it is installed in the cryostat; both motion and sensor operation are tested by checking % readout for sensors and motion of the system vertically. 
sensor readout and vertical motion of the system.
\end{itemize} 

\paragraph{Individual Sensors}
\label{sec:fdgen-slow-cryo-qc-is}

To address the quality of individual precision sensors, the same method as for the static T-gradient monitors %will be 
is used.
The \dword{qc} of the sensors is part of the laboratory calibration. After mounting six sensors with their corresponding cables, a
SUBD-25 connector %will be 
is added, and the six sensors tested at room temperature. All sensors %should 
must %work and 
give values within specifications.  
If any of the sensors present problems, they are replaced.  If the problem persists, the cable is checked and replaced as needed.

For standard \dwords{rtd} to be installed on the cryostat walls, floor, and roof, calibration is not an issue. Any \dword{qc} required for associated cables and connectors %will be 
is performed following the same procedure as for precision sensors. 

\subsubsection{Gas Analyzers}
\label{sec:fdgen-slow-cryo-qc-ga}


The gas analyzers will be guaranteed by the manufacturer. However, once received, the gas analyzer modules %will be 
are checked for both \textit{zero} and the \textit{span} values using a gas-mixing instrument and two gas cylinders, one having a zero level of the gas analyzer contaminant species and the other cylinder with a known percentage of the contaminant gas. This %should
 verifies the proper operation of the gas analyzers. When they are installed at \surf, this process %will be 
 is repeated before commissioning the cryostat. Calibrations will need to be repeated %at times recommended by the 
 per manufacturer recommendations over the gas analyzer lifetime.


\subsubsection{Liquid Level Monitoring}
\label{sec:fdgen-slow-cryo-qc-llm}

The manufacturer will provide the \dword{qc} for the differential pressure level meters; further \dword{qc} during and after installation %will be 
is the responsibility of \dword{lbnf}.

The capacitive sensors will be tested with a modest sample of \dword{lar} in the laboratory before they are installed. After installation, they are tested in situ %{\em in situ} 
using a suitable dielectric in contact with the sensor.

\subsubsection{Pressure Meters}
\label{sec:fdgen-slow-cryo-qc-press}
%\fixme{SG: to be reviewed by experts; waiting for response}
The manufacturer will provide the \dword{qc} for the pressure meters; further \dword{qc} during and after installation is the responsibility of \dword{cisc}.

The pressure sensors will be tested with a modest sample of gaseous argon in the laboratory before they are installed. After installation, they are tested in situ at atmospheric pressure. The whole pressure readout chain, (including slow controls \dword{plc} and software protocol) 
%WINCC conversion) 
will also be tested and cross-checked with \dword{lbnf} pressure sensors.

\subsubsection{Cameras}
\label{sec:fdgen-slow-cryo-qc-c}

Before %being 
transport to \surf, each cryogenic camera unit (comprising the enclosure, camera, and internal thermal control and monitoring) %will be 
is checked for correct operation of all features, for recovery from \SI{87}{K} non-operating mode, for leakage, and for physical defects. Lighting systems %will be 
are similarly checked. Operations tests will verify correct current draw, image quality, and temperature readback and control. The movable inspection camera apparatus %will be 
are inspected for physical defects and checked for proper mechanical operation before shipping. A checklist %will be 
is created for each unit, filed electronically in the \dword{dune} logbook, and a hard copy sent with each unit. 

Before installation, each fixed cryogenic camera unit is inspected for physical damage or defects and checked at the \dword{citf}
%in a cryogenics test facility  
for correct operation of all features, for recovery from \SI{87}{K} non-operating mode, and for contamination of the \dword{lar}. Lighting systems are similarly checked. Operations tests verify correct current draw, image quality, and temperature readback and control. After installation and connection of wiring, fixed cameras and lighting are again  checked for operation. The movable inspection camera apparatus is inspected for physical defects and, after integration with a camera unit, tested in the facility for proper mechanical and electronic operation and cleanliness before being installed or stored. A checklist will be completed for each \dword{qc} check and filed electronically in the \dword{dune} logbook. 

\subsubsection{Light-emitting System}
\label{sec:fdgen-slow-cryo-qc-les}

The entire light-emitting system is checked before installation to ensure functionality of light emission. 
Initial testing of the system (see Figure~\ref{fig:cisc-LED}) begins with
measuring the current when low voltage (\SI{1}{V}) is applied, to check
that the resistive \dword{led} failover path is correct. Next, the forward voltage is measured using nominal forward current to
check that it is within \SI{10}{\%} of the nominal forward voltage drop of
the \dword{led}, that all of the \dwords{led} are illuminated, and that each \dword{led} is visible over the nominal angular range. If the \dwords{led} are
infrared, a video camera with the IR filter removed is used for a
visual check. This procedure is then duplicated with the current
reversed for \dwords{led} oriented in the opposite direction. Initial tests are performed at room temperature and then repeated in \dword{ln}. Color shifts in the \dwords{led} are expected and will be noted. A checklist is completed for each \dword{qc} check and filed electronically in the \dword{dune} logbook.

Room temperature tests are repeated during and immediately after installation to ensure that the system has not been damaged during transportation or installation. Functionality checks of the \dwords{led} are repeated after the cameras are installed in the cryostat.

\subsubsection{Slow Controls Hardware}
\label{sec:fdsp-slow-cryo-qc-sc-hard}

Networking and computing systems will be purchased commercially, requiring manufacturer's \dword{qa}. However, the new servers %will be 
are tested after delivery to confirm they suffered no damage during shipping. The new system is allowed to burn in overnight or for a few days, 
running a diagnostics suite on a loop in order to validate %. This should turn up anything that escaped 
the manufacturer's \dword{qa} process.

The system %can be 
is shipped directly to \surf
where an on-site
expert %will 
boots the systems and does basic
configuration. %Then s
Specific configuration information %will be 
is pulled over
the network, after which others may log in remotely to do the final
setup, minimizing the number of people underground.


\subsection{Safety}
Safety %should be taken into account 
is of critical importance during %the different 
all phases of the \dword{cisc} project, including R\&D, laboratory calibration and testing, mounting tests, and installation. 
Safety experts %will 
review and approve the initial safety planning for all phases as part of the initial design review, as well as %always 
before implementation. 
All documentation of component cleaning, assembly, testing, and installation will include a section on relevant safety concerns and will be reviewed during appropriate pre-production reviews.

Several areas are of particular importance to \dword{cisc}.
\begin{itemize}
\item Hazardous chemicals (e.g., epoxy compounds used to attach sensors to cryostat inner membrane) and cleaning compounds:
  All chemicals used will be documented at the consortium management level, with a material safety data sheet and approved handling and disposal plans in place.

\item Liquid and gaseous cryogens used in calibrating and testing: \dword{ln} and \dword{lar} %will be 
are used to calibrate and test instrumentation devices.
  Full hazard analysis plans will be in place at the consortium management level for full module or
  module component testing that involves  
  cryogens. These safety plans will be reviewed in appropriate pre-production and production reviews.

\item High voltage safety:  Purity monitors %will 
have a voltage of $\sim\,$\SI{2}{kV}. Fabrication and testing plans will show compliance with local
  \dword{hv} safety requirements at %whichever 
  any institution or laboratory that conducts testing or operation, and this compliance will be reviewed as part of the standard review process.


\item Working at heights: Some fabrication, testing, and installation of \dword{cisc} devices require working at heights.
  Both T-gradient monitors and purity monitors, which span the height of the detector, require working at heights exceeding \SI{10}{m}.
  Temperature sensors installed near the top cryostat membrane and cable routing for all instrumentation devices
  %will 
  also require working at heights. % as well. 
  The appropriate safety procedures including lift and harness training will be designed and reviewed. 
  
\item Falling objects: all work involving heights have associated risks of falling objects. The corresponding safety procedures, including proper helmet use 
and a well restricted safety area, will be included in the safety plan. 
\end{itemize}
